%29.09.16
\documentclass[12pt]{article}
\usepackage{amsmath, amssymb, amsthm, epsfig}

\newenvironment{definition}{\vspace{2 ex}{\noindent{\bf Definition}}}
        {\vspace{2 ex}}
\newenvironment{ques}{\vspace{2 ex}}{\vspace{2 ex}}


\theoremstyle{definition}

\newenvironment{Proof}{\noindent {\sc Proof.}}{$\Box$ \vspace{2 ex}}
\newtheorem{Wp}{Writing Problem}
\newtheorem{Ep}{Extra Credit Problem}

\oddsidemargin-1mm
\evensidemargin-0mm
\textwidth6.5in
\topmargin-15mm
%\headsep25pt
\textheight8.75in
\footskip27pt

\pagestyle{empty}
\renewcommand{\theenumi}{\alph{enumi}}
\begin{document}

\noindent \textit{\textbf{Math 461, Fall 2016}} \hspace{1.3cm} \textit{\textbf{HOMEWORK $\#$7}} \hspace{1.3cm} \textit{\textbf{Peter Gylys-Colwell}} 

\vspace{1cm}

\begin{ques}
	\textbf{1.}
		If we consider the spanning tree $T$ of the graph $G$ defined with
		verticies being towns and edges being weather or
		not it is possible to fly between the two towns. We know that
		$T$ has $100$ verticies, and since it is a tree, it has $99$ edges. If we
		double all edges of $T$, this new graph $T'$ will have $198$
		edges, and also every vertex in $T'$ will have even
		degree since doubling every edge in $T$ would double the degree
		of each vertex. Therefore $T'$ is eulerian.  An eulerian path of $T'$
		corresponds to a flight around the country visiting
		all the towns in $198$ flights since every vertex must be
		visited in an eulerian path of a connected graph and there are
		exactly $198$ edges in a eulerian path of $T'$ and we know
		every edge and vertex of $T'$ is contained in $G$. Therefore there
		is always such a trip with exacly $198$ flights.
\end{ques}

\begin{ques}
	\textbf{2.}
		We can consider a spanning tree of $G$, $T$. We know that $T$
		has a leaf $v$. Removing $v$ from $G$ would yield a connected
		graph $G'$. The reason being that $T$ with $v$ removed is still a
		tree $T'$ (since $T$ being a tree $\Leftrightarrow$ $|V(T)| = |E(T)|
		+ 1$ and removing $v$ removes $1$ from both $|V(T)|$ and
		$|E(T)|$ so the equality still holds for the resulting graph).
		And $T'$ is a spanning tree of $G'$ since all verticies in $G$
		are present in $T'$ and vice-versa. Therefore since $G'$ has a
		spanning tree, it is connected.
\end{ques}

\begin{ques}
	\textbf{3.}
		\begin{enumerate}
			\item
				If $(7,8) \notin E$ then there are $6$
				verticies with degree $1$ that will accept $6$
				edges total, but $7$ and $8$ must pair up with
				$8$ edges total which cannot happen\\
				Therefore $(7,8) \in E$. There
				are $6$ vertices left with degree $1$, and $6$
				edges that must come from $7$ and $8$. If any
				of these degree $1$ verticies paired up then we
				would have the problem that $7$ and $8$ need to
				pair up with $3$ vertices each (for a total of
				$6$ edges), but there are only $4$ verticies
				that will accept $4$ edges
				total.\\
				Therefore each of these degree $1$ vertices
				must share an edge with either $7$ or $8$.
				There are $\binom{6}{3}$ ways to choose this.
			\item
				We know that the number of odd degree verticies
				must be even. For all the possible graphs with $8$
				vertices of degree $1$, we know that none of
				them are connected since a connected graph must
				have a spanning tree which would mean the
				number of edges $\geq$ number of vertices $-1$,
				but number of edges = $\frac{1}{2}\sum $
				degrees $= 4 <$ number of vertices $-1$. Part
				$a$ counts the number with $6$ verticies of
				degree $1$, however we must account relabeling
				verticies so we multiply by $\binom{8}{2}$. And the
				graphs with number of $4$ degree vertices $\geq
				4$ are not trees, the reason being is that
				number of edges = $\frac{1}{2}\sum $ degrees
				$\geq \frac{1}{2}(4 + 4 + 4  + 4 + 1 + 1 + 1 +
				1) = 10 >$ the number of vertices, and trees
				must have the property that number of verticies
				$=$ number of edges $+ 1$.\\ So the total
				number of trees is $$\binom{6}{3}\binom{8}{2}$$

		\end{enumerate}
\end{ques}

\begin{ques}
	\textbf{4.}
		We can count the number of spanning trees of the $K_{n-1}$
		subgraph, and then attatch the $n$th vertex with an edge to one
		of the verticies of this spanning tree. The resulting graph
		would be a spanning tree of $K_n$ with $n$ as a leaf. There are
		$n-1$ ways to attach $n$ since there are $n-1$ verticies with
		which $n$ can attach. These methods of coming up with such
		spanning
		trees account for all spanning trees of $K_n$ with $n$ as a
		leaf since for any spanning tree of $K_n$ with
		$n$ as a leaf, we can remove $n$ to have a spanning tree of
		$K_{n-1}$, and so the method of adding $n$ back to this
		spanning tree of $K_{n-1}$ along with the edge that was removed
		will yield the original spanning tree. We also know that the
		images of these mapings dont intersect since the vertex $n$ is
		attatched is unique to each mapping.\\
		From Cayleys's formula we have $(n-1)^{n-3}$ spanning trees of
		$K_{n-1}$, and there are $n-1$ of these mappings, there are
		$$(n-1)(n-1)^{n-3} = (n-1)^{n-2}$$
		total spanning trees with $n$ as a leaf.
\end{ques}

\begin{ques}
	\textbf{5.}
		We can consider the connected component of $1$. If the
		connected component has $k$ vertices, there are
		$\binom{n-1}{k-1}$ ways to choose these other verticies. Of these
		vertices, by Cayley's formula there are $k^{k-2}$ possible
		trees of this connected component. The other connected
		component has the rest of verticies, so there are
		$(n-k)^{n-k-2}$ possible trees of this connected component.
		Therefore for a forrest with two connected components with the
		connected component of $1$ having size $k$ we have,
		$k^{k-2}\binom{n-1}{k-1}k^{k-2}(n-k)^{n-k-2}$ possible forrests\\
		To get all possible forrests with two connected components, we
		can sum $k$ from $1$ to $n-1$:
		$$\sum_{k=1}^{n-1}\binom{n-1}{k-1}k^{k-2}(n-k)^{n-k-2}$$
\end{ques}

\begin{ques}
	\textbf{6.}
		Every pair $\{i,j\} \subset V$ appears as an edge in the same
		number of trees. The reason for this is because for any pair
		$i,j$ there was nothing special about $i$ or $j$ and we can
		simply relabel the verticies to have any verticy pair take the
		places of $i,j$ and recount the trees in the same way. We can
		let the number of times any of these pairs of verticies show up
		as an edge in a tree be $k$. We know that the total number of
		trees is $n^{n-2}$. Each of these trees has $n-1$ edges, and so
		there are $(n-1)n^{n-2}$ total edges of these trees. We know
		that the total number of edges of these trees must equal the
		total number of times each pair shows up as an edge in a tree
		multiplied by the total number of pairs. We know the number of
		pairs is $\binom{n}{2}$ and total number of times a pair shows
		up is $k$. So we have 
		$$\binom{n}{2}k = (n-1)n^{n-2}$$
		$\frac{n(n-1)}{2}k = (n-1)n^{n-2}$, so $k = 2n^{n-3}$. And $k$
		is the number of times $(1,2)$ shows up as an edge in a tree
\end{ques}
\end{document}
