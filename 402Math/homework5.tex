%29.09.16
\documentclass[12pt]{article}
\usepackage{amsmath, amssymb, amsthm, epsfig}

\newenvironment{ques}{\vspace{2 ex}}{\vspace{2 ex}}
\renewcommand{\theenumi}{\alph{enumi}}

\theoremstyle{definition}

\newenvironment{Proof}{\noindent {\sc Proof.}}{$\Box$ \vspace{2 ex}}
\newtheorem{Wp}{Writing Problem}
\newtheorem{Ep}{Extra Credit Problem}

\oddsidemargin-1mm
\evensidemargin-0mm
\textwidth6.5in
\topmargin-15mm
%\headsep25pt
\textheight8.75in
\footskip27pt

\pagestyle{empty}
\begin{document}

\noindent \textit{\textbf{Math 402, Fall 2016}} \hspace{1.3cm}
\textit{\textbf{HOMEWORK $\#$5}} \hspace{1.3cm} \textit{\textbf{Peter
Gylys-Colwell}} 

\vspace{1cm}

\begin{ques}
	\textbf{9.1} 
		\begin{enumerate}
			\item
				This is not an equivalence relation:\\ 
				Consider $1, 2$, we have $2 - 1 \geq 0$ but $1
				- 2 \not \geq 0$
			\item
				This is an equivalence relation: for any $a \in
				\mathbb{Z}$, if $|a| = |b|$ then $b = a$ or $b
				= -a$ so $|b| = |a|$, and if for some $c \in
				\mathbb{Z}$ and $c\ R\ b$ then $c = b$ or $c = -b$
				so $c = a$ or $c = -a$ so $c\ R\ a$
			\item
				This is not an equivalence relation:\\
				Consider $1, 0, -1$. We have $1\ R\ 0$ and $-1\
				R\ 0$ but we dont have $1\ R\ -1$
			\item
				This is not an equivalence relation:\\
				Consider $1, 0, -1$. We have $1\ R\ 0$ and $-1\
				R\ 0$ but we dont have $1\ R\ -1$
		\end{enumerate}
\end{ques}

\begin{ques}
	\textbf{9.3}
		All points that belong to the same equivalence class have the property
		$$y - x = a$$
		for some $a \in \mathbb{R}$ therefore the equivalence classes
		are lines of the form
		$$y = a + x$$
\end{ques}

\begin{ques}
	\textbf{9.5}
		\begin{enumerate}
			\item
				We have $\langle J\rangle  = \{I, J, -I, -J\}$, and so we have
				$$\langle J\rangle I = \langle J\rangle$$
				$$\langle J\rangle K = \{K, L, -K, -L\}$$
			\item
				We have $\langle -I\rangle  = \{I, -I\}$, and we have
				$$\langle -I\rangle  J = \{J, -J\}$$
				$$\langle -I\rangle  K = \{K, -K\}$$
				$$\langle -I\rangle  L = \{L, -L\}$$
				$$\langle -I\rangle  I = \langle -I\rangle$$
		\end{enumerate}

\end{ques}

\begin{ques}
	\textbf{9.6}
		We have
		$$HI = H$$
		$$Hf = \{f, fg\}$$
		$$Hf^2 = \{f^2, g\}$$
		$$Hf^3 = \{f^3, f^3g\}$$
		As for the left cosets we have
		$$IH = H$$
		$$fH = \{f, f^3g\}$$
		$$f^2H = \{f^2, g\}$$
		$$f^3H = \{f^3, fg\}$$
\end{ques}

\begin{ques}
	\textbf{9.10}
		We have $H = \{I, \{1\}, \{1, 2\}, \{2\}\}$, and so
		$$HI = H$$
		$$H\{1, 2, 3, 4\} = \{\{1,2,3,4\}, \{2,3,4\},\{3,4\},\{1,3,4\}\}$$
		$$H\{1,2,3\} = \{\{1,2,3\},\{2,3\},\{3\},\{1,3\}\}$$
		$$H\{1,2,4\} = \{\{1,2,4\},\{2,4\},\{4\},\{1,4\}\}$$
\end{ques}

\begin{ques}
	\textbf{9.11}
		For any sets $A, B, C$ we have 
		$$A\ R\ A$$
		since we can create a bijective map from each element of $A$
		back to itself. We also have
		$$A\ R\ B \Rightarrow B\ R\ A$$
		Since every bijective function from $A$ to $B$ has a bijective
		inverse function from $B$ to $A$. Finally
		$$A\ R\ B \text{ and } B\ R\ C \Rightarrow A\ R\ C$$
		Since we the composition of bijective functions is bijective
		and so if we compose the bijective function from $A$ to $B$ and
		the bijective function from $B$ to $C$ we get a bijective
		function from $A$ to $C$
\end{ques}

\begin{ques}
	\textbf{9.14}
		$R$ is not an equivalence relation for any non-abelian group $G$. Consider
		$a,b,e \in G$ such that $e$ is the identity and $a$ and $b$ do
		not commute. We have
		$$a\ R\ e, b\ R\ e$$
		but it is not the case that 
		$$a\ R\ b$$
\end{ques}

\begin{ques}
	\textbf{9.16} 
		\begin{enumerate}
			\item
				For any $a, b, c$ in $G$ we have
				$$a^{-1}a = e \in H$$
				so $a\ R\ a$, we have 
				$$a^{-1}b \in H \Rightarrow (a^{-1}b)^{-1} =
				b^{-1}a \in H$$
				so $a\ R\ b \Rightarrow b\ R\ a$. Finally we have
				$$a^{-1}b, b^{-1}c \in H \Rightarrow
				a^{-1}bb^{-1}c = a^{-1}c \in H$$
				so $a\ R\ b$ and $b\ R\ c \Rightarrow a\ R\ c$
			\item
				let $\bar a$ denote the equivalence class of $a$, we have 
				$$\bar a = aH$$
				Since
				$$x \in \bar a \text{ iff } a^{-1}x \in H
				\text{ iff } x \in aH$$
			\item 
				$$aH = bH \Leftrightarrow a^{-1}aH = H =
				a^{-1}bH \Leftrightarrow a^{-1}b \in H$$

		\end{enumerate}
\end{ques}

\begin{ques} 
	\textbf{9.18} 
		$$Hx = Ky \Leftrightarrow H = Kyx^{-1}$$
		In order for $e \in H$, from the above equality we know
		$(yx^{-1})^{-1} \in K$ so $yx^{-1} \in K$ so 
		$$Kyx^{-1} = K = H$$
\end{ques}

\begin{ques}
	\textbf{9.19} 
		Consider $f_3 = x + 1$, $f_3 \in H$ so $f_2 \circ f_3 \in f_2H$.\\
		$$f_2 \circ f_3 = 2(x + 1) = 2x + 2$$
		however there is no $f \in H$ such that 
		$$f_1 \circ f = 2x + 2$$
		The proof for this is that we know all $f \in H$ have the form $x + n$, so 
		$f_1 \circ f$ has the form $2(x + n) + 1 = 2x + 2n + 1$, $2n +
		1$ must be an odd number, and so there is no $n \in \mathbb{Z}$
		such that $2n + 1 = 2$\\
		This proves that $f_2H \neq f_1H$ since there is an element in
		$f_2H$ not in $f_1H$\\
		To show $Hf_2 = f_1H \cup f_2H$, we wll show every element of
		$Hf_2$ is in $f_1H \cup f_2H$ and then every element of $f_1H \cup
		f_2H$ is in $Hf_2$\\
		For any $f_3 \in H$ we have $f_3 = x + n$ for some $n \in
		\mathbb{Z}$ so we have 
		$$f_3 \circ f_2 = 2x + n$$
		if $n$ is even
		$$= 2x + 2k = 2(x + k) \in f_2H$$
		and if $n$ is odd
		$$= 2x + 2k + 1 = 2(x + k) + 1 \in f_1H$$
		For some $k \in \mathbb{Z}$\\
		therefore no matter what
		$$f_3 \circ f_2 \in f_1H \cup f_2H$$
		And 
		$$f_1 \circ f_3 = 2x + 2n + 1 = (2x) + (2n + 1) \in Hf_2$$	
		and
		$$f_2 \circ f_3 = 2x + 2n = (2x) + (2n) \in Hf_2$$
		so
		$$Hf_2 = f_1H \cup f_2H$$
\end{ques}
\end{document}
