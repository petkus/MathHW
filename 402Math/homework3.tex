%29.09.16
\documentclass[12pt]{article}
\usepackage{amsmath, amssymb, amsthm, epsfig}

\newenvironment{ques}{\vspace{2 ex}}{\vspace{2 ex}}
\renewcommand{\theenumi}{\alph{enumi}}

\theoremstyle{definition}

\newenvironment{Proof}{\noindent {\sc Proof.}}{$\Box$ \vspace{2 ex}}
\newtheorem{Wp}{Writing Problem}
\newtheorem{Ep}{Extra Credit Problem}

\oddsidemargin-1mm
\evensidemargin-0mm
\textwidth6.5in
\topmargin-15mm
%\headsep25pt
\textheight8.75in
\footskip27pt

\pagestyle{empty}
\begin{document}

\noindent \textit{\textbf{Math 402, Fall 2016}} \hspace{1.3cm} \textit{\textbf{HOMEWORK $\#$3}} \hspace{1.3cm} \textit{\textbf{Peter Gylys-Colwell}} 

\vspace{1cm}

\begin{ques}
\textbf{2.1} 
	\begin{enumerate}
		\item $\mathbb{R}^+$ is not a group since none of the elements
			have inverses besides $\{0\}$. The inverse of $a \in
			\mathbb{R}^+$ is $-a \notin \mathbb{R}^+$\\
		\item This is a group, $0$ is the identity, the inverse of $a
			\in 3\mathbb{Z}$ is $-a$ and $3$ divides $-a$ since 3
			divides $a$ so $-a \in 3\mathbb{Z}$, and addition is
			assosiative. Therefore $3\mathbb{Z}$ under addition
			satisfies all the requirements of a group
		\item This is not a group since there is no identity. $a * b
			\geq 0$ for all $a,b \in \mathbb{R}$, therefore if $a <
			0$, $a * b \neq a$ for all $b \in \mathbb{R}$
		\item This is a group since $1$ is the identity, $-1$ and $1$
			are their own inverses and multiplication is
			assosiative.
		\item This is a group since for any $a, b \in \mathbb{Q}$ such
			that $\sqrt{a}, \sqrt{b} \in \mathbb{Q}$, $\sqrt{ab}
			\in \mathbb{Q}$. The inverse of $a$ would be $\frac1a$
			which by the way the square root works, $\sqrt \frac1a
			\in \mathbb{Q}$. The identity would be $1$ and
			multiplication is associative. Therefore all the
			conditions are met for this set under multiplication to
			be a group
		\item This is not a group since the operation is not
			associative. Subtraction is not associative.
		\item This is a group, we have $(0,1)$ be the identity, for a
			given $(a,b)$ in the set, $(-a, \frac1b)$ is its
			inverse, and addition and multiplication are
			associative and so $*$ is associative too
		\item This is a group, $0$ is the identity. The operation is
			accociative: 
			$$a * (b * c) = a * (b + c - bc) = a + b + c - a(b + c - bc) $$
			$$= a + b + c - ab - ac + abc = (a + b - ab) + c -
			c(a+b - ab) = (a * b) * c$$ 
			And the inverse of any $a \in \mathbb{R} - \{1\}$ is
			$\frac{a}{a-1}$ since 
			$$a * \frac{a}{a-1} = \frac1{a-1}(a(a-1) + a - a^2) =
			0$$ which is well defined since $a-1 \neq 0$. Therefore
			all the conditions of being a group are satisfied
		\item This is a group, $1$ is the identity, addition is
			associative, and the inverse of a given $a \in
			\mathbb{Z}$ is $-a$: $a - 1 - (a - 1) = 0$
	\end{enumerate}

\end{ques}

\begin{ques}
	\textbf{2.3}
	Under intersection $P(X)$ is a group, but not under union. As
		established in an earlier homework assignments, for a given $A
		\in P(X)$, the operation 
		$$F(B) = A \cup B$$
		is not injective and therefore does not have an inverse, and
		therefore the element $A$ cannot have an inverse under the
		union operation.\\
		For intersection, we have proven in a previous homework that
		the intersection operation is assosiative. We have the empty
		set be the identity since
		$$\emptyset \cap A = A \cap \emptyset = \emptyset$$
		and for the inverse of a given element $A$ we have 
		$$A^{-1} = X - A$$
		since 
		$$A \cap (X - A) = (X - A) \cap A = \emptyset$$

\end{ques}

\begin{ques}
	\textbf{2.5}
		No, $c$ has no inverse, there is no element $c^{-1}$ in the
		table such that $c^{-1} c a = a$
\end{ques}

\begin{ques}
	\textbf{2.7}
	$$\begin{array}{c|cc}
			& a & b \\
			\hline
			a & a & b \\
			b & b & a
		\end{array}$$
	$a$ is the identity, and $b$ is it's own inverse
\end{ques}

\begin{ques}
	\textbf{2.8}
		Yes it forms a group. The identity function is $f(x) = 1$ since 
		$$1g(x) = g(x)1 = g(x)$$
		for all functions $g(x)$. For any function $f(x)$ where $f(x)
		\neq 0 \ \forall x \in \mathbb{R}$, we define its inverse to be
		$\frac1{f(x)}$. Since $f(x) \neq 0$, the inverse is well
		defined. And multiplication is associative, so the operation of
		multiplying these functions is associative. Therefore all the
		criteria for being a group are met.
\end{ques}

\begin{ques}
	\textbf{2.10}
		Matrix multiplication is associative, the Identity matrix would be:
		$$\left[ \begin{matrix}
			1 & 0 \\
			0 & 1 
		\end{matrix} \right ]$$
		which as established in linear algebra, changes nothing under
		matrix multiplication. Since the determinanant of these
		matricies is non-zero, they all have inverses:
		$$\left[ \begin{matrix}
			a & b \\
			-b & a 
		\end{matrix} \right ]^{-1}
		=
		\frac{1}{a^2 + b^2}
		\left[ \begin{matrix}
			a & -b \\
			b & a 
		\end{matrix} \right ]$$
		which are also of the form $\left[ \begin{matrix}
			a & b \\
			-b & a 
		\end{matrix} \right ]$. Therefore all the requirements to be a
		group are met
\end{ques}

\begin{ques}
	\textbf{3.3}
	Example:
		$$A = \left[ \begin{matrix}
			0 & 1\\
			1 & 0
		\end{matrix} \right ],
		B = \left[ \begin{matrix}
			1 & 1\\
			1 & 1
		\end{matrix} \right ],
		C = \left[ \begin{matrix}
			1 & 0\\
			0 & 1
		\end{matrix} \right ]$$
		Some straight forward calculations yield $AB = BC$ even though $A \neq C$
\end{ques}

\begin{ques}
	\textbf{3.4}
		Because $G$ is a group, $\exists x^{-1} \in G$. Therefore if we apply
		$x^{-1}$ to both sides of the equation we have:
		$$x \cdot g = x$$
		$$x^{-1} \cdot x g = x^{-1}\cdot x$$
		$$e \cdot g = e$$
		and by the definition of the identity, this can only be the case if $g = e$

\end{ques}

\begin{ques}
	\textbf{3.7}
		If an element occured more than once in a row or column of a
		table, that is equivalent to saying there exists $a \in G$ such
		that
		$$\exists \ b, c \in G: b \neq c, \text{ and either } ab = ac, \text{ or } ba = ca$$
		However we know $a^{-1} \in G$, therefore
		$$a^{-1} ab = a^{-1} a c \text{ or } baa^{-1} = caa^{-1}$$
		and so 
		$$b = c$$
		which is a contradicion. Therefore the maping of elements in
		$G$ to the elements in the rows and columns must be injective.
		And because the number of elements in each row is the same as
		the number of elements in $G$, we know that the mapping is
		surjective. Therefore every element occurs precisely once in
		each row and column.
\end{ques}

\begin{ques}
	\textbf{3.11}
		For a given $x,y \in G$, with $x \neq y$ we have for the element $xy$:
		$$(xy)^{-1} = xy$$
		however 
		$$(xy)(yx) = x(yy)x = e$$
		as well, therefore they are both the inverse of $xy$, which is unique. So
		$$(xy)^{-1} = xy = yx$$
\end{ques}

\begin{ques}
	\textbf{3.14}
		First we will show there is an Identity element. We know some
		element $a \in G$ there is an element $e \in G$ such that 
		$$ae = a$$
		from this, we have $\forall y \in G$
		$$(ae)y = ay = a(ey)$$
		And therefore $ey = y$ for all $y$, Therefore for $\forall x \in G$:
		$$x(ey) = xy = (xe)y$$
		and so $x = xe$. Therefore $e$ satisfies all the properties of
		the identity element for $G$.\\
		From there we have for any $a \in G$:
		$$\exists a^{-1} \in G: a a^{-1} = e$$
		And so we have
		$$a(a^{-1}a) = a = (a^{-1}a)a$$
		and so $a^{-1}a = e$. Therefore $a^{-1}$ satisfies the
		properties to be the inverse of $a$.\\
		Theres an identity, all elements have an inverse, and $G$ is
		closed under an assosicative operation, therefore $G$ is a
		group.
\end{ques}

\begin{ques}
	\textbf{3.15}
		We will let the cardinality of $G$ be $n$. First we will show
		that there must be an Identity element.\\
		If we take some $a \in G$ we look at the set 
		$$\{a, a^2, a^3, a^4, \dots a^n, a^{n+1}\}$$
		Since there are $n$ possible values in $G$ but there are $n+1$
		elements in this set, by the pigeonhole principle, two of these
		values must be the same: $a^i = a^j$ for some $i < j$.\\
		Therefore we have $a^i = a^i a^{j-i}$. We will relabel
		$a^{j-i}$ as $e$. We now have for all $b \in G$
		$$(a^ie)b = a^{i}(eb) = a^{i}b$$
		and so $eb = b$. Similarly for any $c \in G$
		$$c(eb) = (ce)b = cb$$
		so $ec = c$. Therefore $e$ satisfies all the properties to be
		the identity of $G$.\\
		From the following property for any $x, a, b \in G$
		$$a \neq b \Rightarrow ax \neq bx$$
		We know the mapping of $f(a) = a \cdot x$ is injective, and
		therefore since it maps from $G$ to $G$, it is surjective.
		Therefore we know there is some $x^{-1} \in G$ such that
		$x^{-1} x = e$. Also that
		$$x(x^{-1}x) = (xx^{-1})x = x$$
		and so $xx^{-1} = e$. So $x^{-1}$ satisfies all the properties
		of the inverse of $x$. Therefore all elements in $G$ have an inverse.\\
		All the properties required to be a group are satisfied and
		so $G$ is a group
\end{ques}

\begin{ques}
	\textbf{3.16}
		Consider $G = \mathbb{N}$ under addition.\\
		The conditions are met since for $a,b,c \in \mathbb{N}$:
		$$a + b = c + b \Leftrightarrow a = c$$
		and 
		$$a + b \in \mathbb{N} \Leftrightarrow a,b \in \mathbb{N}$$
		However there is no identity or inverses over $G$.
\end{ques}


\end{document}
