%29.09.16
\documentclass[12pt]{article}
\usepackage{amsmath, amssymb, amsthm, epsfig}

\newenvironment{ques}{\vspace{2 ex}}{\vspace{2 ex}}
\renewcommand{\theenumi}{\alph{enumi}}

\theoremstyle{definition}

\newenvironment{Proof}{\noindent {\sc Proof.}}{$\Box$ \vspace{2 ex}}
\newtheorem{Wp}{Writing Problem}
\newtheorem{Ep}{Extra Credit Problem}

\oddsidemargin-1mm
\evensidemargin-0mm
\textwidth6.5in
\topmargin-15mm
%\headsep25pt
\textheight8.75in
\footskip27pt

\pagestyle{empty}
\begin{document}

\noindent \textit{\textbf{Math 402, Fall 2016}} \hspace{1.3cm}
\textit{\textbf{HOMEWORK $\#$4}} \hspace{1.3cm} \textit{\textbf{Peter
Gylys-Colwell}} 

\vspace{1cm}

\begin{ques}
	\textbf{4.9} 
		If the positive rationals over multiplication were cyclic,
		there would be some generator $x \in \mathbb{Q}$ such that for any $y \in
		\mathbb{Q}$, $x^n = y$ where $n \in \mathbb{Z}$. This is not
		possible, consider the number
		$$y = \frac{x + 1}{2} \in \mathbb{Q}^+$$
		we know $y \in (1,x)$ but for any $n \in \mathbb{Z}$, $x^n
		\notin (1,x)$. And so $(\mathbb{Q}^+, \cdot)$ cannot be cyclic.
\end{ques}

\begin{ques}
	\textbf{4.10}
		\begin{enumerate}
			\item  We can write any integer in the form $7n + r$,
				where $r$ is the number modulo $7$. And so for
				any two numbers we have:
				$$(7k_1 + k_2)(7g_1 + g_2) = 7(k_1 + g_1) + g_2
				+ k_2 =  (7g_1 + g_2)(7k_1 + k_2)$$
				and so the operation is commutative. Similarly
				since multiplication is associative, we know
				multiplication modulo a number will also be
				associative.\\
				We also know $1$ is still
				an identity element\\
				Looking at each element, we have 
				$$2 \odot 4 = 1,\ \ 3 \odot 5 = 1,\ \ 6 \odot 6 = 1$$
				and so every element has an inverse. Therefore
				the collection under $\odot$ is a group

			\item The group is cyclic, consider $3$:
				$$3^2 = 2, 3^3 = 6, 3^4 = 4, 3^5 = 5, 3^6 = 1$$
				All elements are accounted for.
		\end{enumerate}
		
\end{ques}

\begin{ques}
	\textbf{4.14}
		Consider an infite set $X$, with $P(X)$ being its powerset. \\
		An example of an infinite group where all elements have finite order is
		$$(P(X), \triangle)$$
		As we have proven earlier, powersets under symmetric difference are groups.\\
		The order of any $A \in P(X)$ is $2$ since $A \triangle A
		\triangle A = A$, but $P(X)$ is infinite.
\end{ques}

\begin{ques}
	\textbf{4.24}
		$o(y) = 2$ implies $y = y^{-1}$.\\
		We have 
		$$yxy = x^2$$
		$$yxyyxy = x^2 x^2$$
		$$y(x^2)y = x^4$$
		$$yyxyy = x^4$$
		And so 
		$$x = x^4$$
		so the order of $x$ must be $3$
\end{ques}

\begin{ques}
	\textbf{4.32}
		Let $z = xy$.\\
		Observe that since $G$ is abelian, $z^k = x^ky^k$. Therefore
		$$z^k = e \Leftrightarrow x^ky^k = e$$ $x^ky^k = e$ implies one
		of two cases. The first case is that $x = y^{-1}$ which means
		$m = n$ so their greatest common multiple is $m = n$, so we
		would be done since $x,y$ have order $m$. The other case is
		$x^k = e, y^k = e$ which is if and only if $k$ is a multiple of
		both $n$ and $m$, and so the least common multiple of $m,n$
		would be the order of $z$.
\end{ques}

\begin{ques}
	\textbf{4.33}
		\begin{enumerate}
			\item We have for any $x = (a_x, b_x, c_x)$ in the group described:
				$$x^3 = (a_x + a_x + a_x, b_x + b_x + b_x, c_x
				+ (c_x +c_x + a_xb_x) + (a_x + a_x)b_x) \mod 3$$
				$$= (3a_x, 3b_x, 3c_x + 3a_xb_x) \mod 3$$
				Since all these terms are multiples of $3$, we have:
				$$x^3 = (0,0,0) = e$$
				Therefore for any $x,y$ in the group, we have
				$$(xy)^3 = e = e \dot e = x^3y^3$$
				And similarly 
				$$(xy)^4 = (xy)^3xy = xy = xx^3yy^3 = x^4y^4$$
				However the group is not abelian, consider
				$$(1, 0, 0)(1, 1, 0) = (2, 1, 0) \neq (2,1,1) =
				(1,1,0)(1,0,0)$$

			\item We have the following logic for any $x,y$ in the
				group described:
				$$(xy)^{n+2} = x^{n+2}y^{n+2}$$
				$$(xy)^{n+1}xy = x(x^{n+1}y^{n+1})y$$
				Since $(xy)^{n+1} = x^{n+1}y^{n+1}$
				$$x^{n+1}y^{n+1}xy = x(xy)^{n+1}y$$
				Applying $x^{-1}$ on the left and $y^{-1}$ on
				the right for both sides:
				$$x^ny^{n+1}x = (xy)^{n+1}$$
				$$x^ny^nyx = (xy)^nxy$$
				And since $x^ny^n = (xy)^n$, they have the same
				inverse. Applying this inverse to the left for
				both sides yeilds the property that $G$ must be
				commutative:
				$$yx = xy$$
		\end{enumerate}
\end{ques}

\begin{ques}
	\textbf{5.6}
		\begin{enumerate}
			\item If $m | n$, let $x$ be the genorator of $G$. We know
				$$x^{n/m}$$
				is an element of $G$ since $n/m \in
				\mathbb{Z}$, and the element has order $m$.\\
				For the other direction, if there is some
				element $y$ with order $m$, \\
				We know for some $k$ we have $x^k = y$ (where
				$x$ is the genorator of $G$) and so if we consider the 
				cyclic subgroup $H = <y>$, we know 
				$$|H| = o(y) = \frac{n}{(k,n)} = m$$
				(where $(k,n)$ denotes the gcd of $k,n$) and so
				$m(k,n) = n$, which means $m$ divides $n$.

			\item As established in part $a$, there is an element of
				order $m \in G$ iff $m$ divides $|G|$. So the possible orders
				are
				$$\{1, 2, 4, 5, 8, 10, 20, 40\}$$
		\end{enumerate}
\end{ques}

\begin{ques}
	\textbf{5.7} If $(m,n) = 1$ then $x^m$ has order $n$ since $(x^m)^k =
		e$ if and only if $n$ divides $k$, and so $x^m$ is a genorator
		of a subgroup $H$ that is the same size as $G$, so $H = G$.\\
		Conversly if $x^m$ has order $n$, since both $n$ and $m$
		divides $(m,n)n$, $(m,n)n$ must be the order of $x^m$, and so
		$(m,n)n = n$ so $(m,n) =1$
\end{ques}

\begin{ques} 
	\textbf{5.17} Counterexample:\\
		Consider $G = \mathbb{Z}$ mod $9$ under addition. So
		$$G = \{0, 1, 2, 3, 4, 5, 6, 7, 8\}$$
		and let
		$$H = \{0, 1, 2, 7, 8\}$$
		We have that $H$ is closed under inverses since $1 + 8 = 0$ and
		$2 + 7 = 0$. However $H$ is not closed under addition so cannot
		be a group: $1 + 2 \notin H$
\end{ques}

\begin{ques}
	\textbf{5.18} 
		\begin{enumerate}
			\item If $H,J$ are proper subgroups of $G$ such that $H
				\cup J = G$, then we will reach the following
				contradiction:\\
				We know that $J / H \neq \emptyset$ and $H/J
				\neq \emptyset$ since that would mean $H \cup J
				= H$ or $H \cup J = J$ which means $H$ or $J$
				are not proper subgroups\\
				Therefore we know there is some elements $h \in
				H: h \notin J$, and $j \in J: j \notin J$\\
				Now we have $hj \in G$ so $hj \in H \cup J$,
				however if $hj \in H$ then $h^{-1}hj \in H$ so
				$j \in H$ which is not possible, and similarly
				if $hj \in J$ then $hjj^{-1} \in J$ so $h \in
				J$ which is not possible.\\
				Therefore we have a contradiction since $hj$
				must be in $H \cup J$ but it cannot be in
				either $H$ or $J$.

			\item The Klien's four group is an example:
				$$\{e, a, b, c\}$$
				such that $a^2 = b^2 = c^2 = e$
				Therefore the union of the proper subgroups:
				$$\{e, a\}, \{e, b\}, \{e, c\}$$
				is equal to the original group.
		\end{enumerate}
\end{ques}

\begin{ques}
	\textbf{8.7} Consider the elements $(1,2)$ and $(2,3)$ which are well
		defined in $S_n$ for all $n \geq 3$. We have
		$$(1,2) \circ (2,3) = (1,2,3) \neq (2,3) \circ (1,2) = (1,3,2)$$
		And so $S_n$ is not commutative for $n \geq 3$
\end{ques}

\begin{ques}
	\textbf{8.10} 
		\begin{enumerate} 
			\item Since these cycles are disjoint, we know they
				commute. Therefore we know that we can reorder
				$f^k$ as so:
				$$f^k = f_1^kf_2^k\dots f_m^k$$
				And we know each of these terms can be the
				identity if and only if $k$ is a multiple of
				the orders of each of the cycles. Therefore,
				the least common multiple of the orders of each
				of the cycles must be the order of $f$
				
			\item This permutation written in cycle notation is
				$$(1, 6, 4, 9)(2, 7, 11)(3, 5, 8)(10, 12)$$
				Since we know the order of this permutation is
				the lcm of the order of each of these cycles,
				we only need to calculate each of the cycles
				orders.\\
				We know that the order of a cycle is the length
				of a cycle, so the orders of each of the cycles
				above are $\{4, 3, 3, 2\}$ and the lcm of these
				4 numbers is $12$ so the order of the original
				permutation is $12$
		\end{enumerate}
\end{ques}

\begin{ques}
	\textbf{8.11}
		\begin{enumerate} 
			\item Let 
				$$x = (1, 2, 3, 4, 5)$$
				$$y = (1, 6, 7, 8, 9)$$
				We have that
				$$xy = (1, 2, 3, 4, 5, 6, 7, 8, 9)$$
				Which means that $o(x) = o(y) = 5$ and $o(xy) = 9$
			\item Each element must have order equal to the
				lcm of the disjoint cycles that make up the
				element. We know that the order of the cycle is
				equal to the size of the cycle. Therefore we
				are looking for the maximum lcm of a collection
				of sizes that add up to $\leq 9$. We will call
				this collection $S$\\
				We notice that each of the numbers in $S$ must
				be relatively prime to each other, since
				otherwise we could factor out a factor from two
				of these numbers to get a smaller sum, but the
				same lcm.\\
				Potential options from this reasoning yields that $S$ can be
				$$\{2,3\}, \{4,3\}, \{2,5\},\{3,5\}, \{4,5\}, \{2, 7\}$$
				And from calculating the lcm of each we find
				that $S = \{4,3\}$ with the lcm $= 20$. And so
				the maximum order of an element in $S_9$ is
				$20$.

		\end{enumerate}
\end{ques}


\end{document}
