%29.09.16
\documentclass[12pt]{article}
\usepackage{amsmath, amssymb, amsthm, epsfig}

\newenvironment{ques}{\vspace{2 ex}}{\vspace{2 ex}}
\renewcommand{\theenumi}{\alph{enumi}}

\theoremstyle{definition}

\newenvironment{Proof}{\noindent {\sc Proof.}}{$\Box$ \vspace{2 ex}}
\newtheorem{Wp}{Writing Problem}
\newtheorem{Ep}{Extra Credit Problem}

\oddsidemargin-1mm
\evensidemargin-0mm
\textwidth6.5in
\topmargin-15mm
%\headsep25pt
\textheight8.75in
\footskip27pt

\pagestyle{empty}
\begin{document}

\noindent \textit{\textbf{Math 402, Fall 2016}} \hspace{1.3cm}
\textit{\textbf{HOMEWORK $\#$7}} \hspace{1.3cm} \textit{\textbf{Peter
Gylys-Colwell}} 

\vspace{1cm}

\begin{ques}
	\textbf{12.17}
		Since every element has order $2$ and commutes with every other
		element, any injective mapping from $K$ to $K$ is an
		automorphim as long as the identity maps to itself.\\
		To count the number of injections, we have $a$ can map to $3$
		elements, then $b$ can map to the remaining $2$ elements, and
		$c$ must map to whats left. Therefore there are $3\cdot 2 = 6$ automorphisms
\end{ques}

\begin{ques}
	\textbf{12.20}
		\begin{enumerate}
			\item
				For any elements $x,y \in G$ we have
				$$\varphi (xy) = (xy)^n$$
				Since $G$ is abelian
				$$= x^ny^n = \varphi(x)\varphi(y)$$
				We can show $\varphi$ is a injective which
				would imply it is bijective since it is a
				mapping from $G$ to $G$. If $x^n = y^n$,
				we have 
				$$y^nx^{-n} = e$$
				since $G$ is abelian we have
				$$(x^{-1}y)^n = e$$
				since $n$ is relatively prime to $|G|$, we know
				that $n$ cannot be a multiple of the order of
				$x^{-1}y$ since the order must divide $|G|$ by
				lagranges thm, unless $x^{-1}y = e$. This
				implies $x = y$ since inverses are unique
			\item
				Since  $\varphi$ is surjective, for any $x \in
				G$ there is a $y \in G$ such that 
				$$\varphi(y) = y^n = x$$
		\end{enumerate}
\end{ques}

\begin{ques}
	\textbf{12.23}
		\begin{enumerate}
			\item
				We know that the mapping $\varphi(h) = ghg^{-1}$ is an
				automorphim of $G$ for any $g \in G$. Applying these
				mappings to $H$ we have
				$$\varphi(H) = gHg^{-1}$$
				And so if for all these mappings we have
				$\varphi(H) \subseteq H$ then we have
				$$gHg^{-1} \subseteq H$$
				Which is means $H$ is normal
			\item
				Consider $G =$ the Klien's 4group. Let $H =
				\{e, a\}$ and let $\varphi$ be the automorphim
				with  $\varphi(e) = e, \varphi(a) = b, \varphi(b) = c,
				\varphi(c) = a$. We have
				$$\varphi(H) = \{e, b\} \not \subseteq H$$
		\end{enumerate}
\end{ques}

\begin{ques}
	\textbf{12.31}
		No, consider $G=$ the Klien's 4group. $\psi(a) = b, \psi(b) =
		c,$ and $\psi(c) = a$. If we let $H = \{e, a\}$ we have
		$\varphi(a) = \psi(a) = b$, but $\varphi(b) = b$ so $\varphi$
		is not injective so not an automorphim.
\end{ques}

\begin{ques}
	\textbf{12.34}
		Letting $H$ bet the set of inner automorphisms of $G$, we have for any
		$A(x) = axa^{-1}, B(x) = bxb^{-1}, C(x) = cxc^{-1} \in H$. 
		$$A \circ B = abxb^{-1}a^{-1} \in H$$
		since $ab \in G$. Since $H$ is closed under the group operation
		and since Aut(G) is finite, that is sufficient to show $H$ is a
		subgroup. To check if normal we have for any $\varphi \in $Aut(G):
		$$\varphi \circ H \circ \varphi^{-1} = \{A(x) = \varphi(a
		\varphi^{-1}(x) a^{-1}): a \in G\}$$
		$$= \{A(x) = \varphi(a)x\varphi(a^{-1}): a\in G\}$$
		and since $\varphi$ is an automorphim on $G$,
		$$\{A(x) = bx b^{-1}: b\in G\} = H$$
		So $H$ is normal
\end{ques}

\begin{ques}
	\textbf{13.2}
		$H$ consists of only an identity element and an element of
		order $2$. Lets call this element $a$ and the identity $e$.
		If there existed a homomorphism $\varphi : Q_8 \to H$ then we know
		$\varphi(I) = e$ since $(\varphi(I))^2 = \varphi(I)$. If there is some
		element $q \in Q_8$ such that $\varphi(q) = a$. A property of
		$Q_8$ is that for any element $q \in Q_8$, there is an element
		$j$ such that $j^2 = q$. Therefore we have
		$$(\varphi(j))^2 = \varphi(q) = a$$
		But there is no element $k \in H$ such that $k^2 = a$ so there
		is nothing that $\varphi(j)$ can map to.
\end{ques}

\begin{ques}
	\textbf{13.6} 
		We have 
		$$\{0(3\mathbb{Z}/12\mathbb{Z}), 1(3\mathbb{Z}/12\mathbb{Z}),
		2(3\mathbb{Z}/12\mathbb{Z})\}$$
\end{ques}

\begin{ques} 
	\textbf{13.8} 
		We can define a homomorphism $\varphi: G \to (\mathbb{Z}, +)$
		such that for a given $\frac a b \in G$ with $a$ and $b$ in
		their most reduced state (relatively prime to each other) we
		have $\varphi(\frac a b) = m(a) - m(b)$ where $m(x)$ is the number of
		times $2$ divides $x$ (note that since $a$ and $b$ are
		relatively prime, $m(a)$ and/or $m(b)$ is zero). $\varphi$ is a
		homomorphism since for any $\frac{a}{b},
		\frac{c}{d}$, 
		$$\varphi(\frac a b) + \varphi(\frac c d) = m(a) - m(d) + m(c)
		- m(b)=  m(ac) - m(bd) = \varphi(\frac{ac}{bd})$$
		The kernal of $\varphi$ would be $\frac{a}{b} \in G$ such that
		$m(a) = m(b) = 0$ which is precisely $H$. Lastly $\varphi$ is
		surjective since for any $z \in \mathbb{Z}$ we have
		$\varphi(\frac{2^z}{1}) = z$ if $z \geq 0$ and
		$\varphi(\frac{1}{2^z}) = z$ if $z < 0$. Therefore by the
		Fundamental Theorem we have our desired result
		
\end{ques}

\begin{ques} 
	\textbf{13.9} 
		Let $\varphi : G \to \{\mathbb{R} - \{0\},\cdot \} \times
		\{\mathbb{R} - \{0\},\cdot \}$ with 
		$$\varphi \left(\begin{matrix} a & b\\ 0 & c \end{matrix}\right) = (a,c)$$
		$\varphi$ is a homomorphism since 
		$$\varphi \left(\begin{matrix} a & b\\ 0 & c
		\end{matrix}\right)\varphi \left(\begin{matrix} i & j\\
		0 & k \end{matrix}\right) = (ai, ck) =\varphi\left(
		\left(\begin{matrix} a & b\\ 0 & c \end{matrix}\right)
		\left(\begin{matrix} i & j\\ 0 & k
		\end{matrix}\right)\right) = \varphi\left(\begin{matrix} ai & aj + bk\\
		0 & ck \end{matrix}\right)$$
		We know $(1,1)$ is the identity of $\{\mathbb{R} - \{0\},\cdot \} \times
		\{\mathbb{R} - \{0\},\cdot \}$, and $(a,c)$ is precisely when
		the input matrix is in $H$ so ker($\varphi$) $= H$. It is clear
		$\varphi$ is surjective since we can choose $a, c$ to be
		anything in the matrix. Therefore
		by the Fundamental Theorem we have our desired result.
		$$G/H = \{\mathbb{R} - \{0\},\cdot \} \times \{\mathbb{R} -
		\{0\},\cdot \}$$
		
\end{ques}
\end{document}
