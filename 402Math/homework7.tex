%29.09.16
\documentclass[12pt]{article}
\usepackage{amsmath, amssymb, amsthm, epsfig}

\newenvironment{ques}{\vspace{2 ex}}{\vspace{2 ex}}
\renewcommand{\theenumi}{\alph{enumi}}

\theoremstyle{definition}

\newenvironment{Proof}{\noindent {\sc Proof.}}{$\Box$ \vspace{2 ex}}
\newtheorem{Wp}{Writing Problem}
\newtheorem{Ep}{Extra Credit Problem}

\oddsidemargin-1mm
\evensidemargin-0mm
\textwidth6.5in
\topmargin-15mm
%\headsep25pt
\textheight8.75in
\footskip27pt

\pagestyle{empty}
\begin{document}

\noindent \textit{\textbf{Math 402, Fall 2016}} \hspace{1.3cm}
\textit{\textbf{HOMEWORK $\#$7}} \hspace{1.3cm} \textit{\textbf{Peter
Gylys-Colwell}} 

\vspace{1cm}

\begin{ques}
	\textbf{10.2}
		\begin{enumerate}
			\item
				We know that $42$ must divide $32|H|$ since
				$|H|$ is the order of $32$ and adding $32$ $H$
				is the same as multiplying $32$ by $H$,
				therefore
				$$|H| = \frac{LCM(32,48)}{32} = \frac{32 * 3 *
				7}{32} = 21$$
				And by Lagrange's thm we have 
				$$|H||G:H| = |G| = 48$$
				so $|G:H| = 2$
			\item
				Using the same logic above we have
				$$|H| = \frac{LCM(24, 54)}{24} = \frac{24 * 3 *
				3}{24} = 9$$
				and so
				$$|G:H| = \frac{|G|}{|H|} = \frac{54}{9} = 6$$
			\item
				Using the same logic:
				$$|H| = \frac{LCM(100,112)}{100} = \frac{2 * 2
				* 7 * 100}{100} = 28$$
				So 
				$$|G:H| = \frac{|G|}{|H|} = \frac{112}{28} = 4$$
		\end{enumerate}

\end{ques}

\begin{ques}
	\textbf{10.5}
		Given any element $a \in G$, by Lagrange's thm we have $|\langle
		a \rangle|$ divides $|G| = 8$. Since $G$ is not cyclic we know
		$\langle a \rangle \neq G$ so $|\langle a \rangle| \neq 8$.
		Therefore the only options for $|\langle a \rangle|$ are $1, 2,
		4$. Since all these numbers divide $4$, we know that $a^4 = e$

\end{ques}

\begin{ques}
	\textbf{10.6}
		We know that the intersection of two subgroups is a group.
		Therefore if we let $A = H \cap K$, we have that $A$ is a
		subgroup of both $H$ and $K$ so by Lagrange's thm we have that
		$|A|$ divides both $|H| = 12$ and $|K| = 5$. The only number
		that divides both $12$ annd $5$ is $1$, so $|A| = 1$ so $A =
		\{e\}$

\end{ques}

\begin{ques}
	\textbf{10.14}
		\begin{enumerate}
			\item
				This is because we know for the element $x$ of order $6$,
				$|\langle x \rangle| = 6 = |G|$ and a subgroup
				of $G$ that is the same size of $G$ is
				equivalent to $G$. Therefore $G = \langle x
				\rangle$ is cyclic
			\item
				By Lagrange's thm for any element $a \in G$,
				$|\langle a \rangle|$ divides $|G| = 6$
				$|\langle a \rangle|$ cannot equal $6$ since
				$G$ is not cyclic, so $a$ must have either
				order 1, 2, or 3. We know that only $e$ has
				order $1$. We cannot have all the elements have
				order $2$, otherwise we would have two elements
				$a$, $b$, then we would have $ab$, each having
				order $2$ so they are their own inverses. So we
				have $\{e, a, b, ab\}$ which is closed with size $4$,
				so we must introduce another element $c$, which
				would bring us to $\{e, a, b, c, ab, ac, bc,
				abc\}$ which is too large. Therefore there is
				some element $a$ of order $3$
			\item
				If we take some element $b \in G: b \notin
				\langle a \rangle$, we already know $e, a, a^2
				\in G$. We know since $b \notin
				\langle a \rangle$ that $ab, a^2b \notin
				\langle a \rangle$ since $b$ is not equal to any
				power of $a$. Looking at $ab$, we can deduce
				$ab \neq a^2b$, since applying $b^{-1}$ on the
				right yields $a \neq a^2$ which is true. Therefore we have
				$$\{e, a, a^2, b, ab, a^2b\} \subseteq G$$
				Are all unique elements
			\item 
				We cannot have $b^2$ be a seperate element in
				the above set since $|G| = 6$, if $b^2 = a$
				then $b = a^2$ which is not true, if $b^2 a^2$
				then $b = a$ which is not true, if $b^2 = ab$
				then applying $b^{-1}$ on the right yields $b =
				a$ which is not true, and finally if $b^2 =
				a^2b$ then $b = a^2$ which is not true.
				Therefore $b^2 = e$.
			\item
				We have $(ba)^{-1} = a^{-1}b^{-1} = a^2b$ but
				since we concluded $a^2b$ has order 2
				$(a^2b)^{-1} = a^2b = ba$. Similarly
				$(ba^2)^{-1} = ab = (ab)^{-1}$
		\end{enumerate}
\end{ques}

\begin{ques}
	\textbf{10.15}
		By Lagrange's thm we have $|G| = |G:H||H|$ and $|H| = |H:K||K|$
		so $|K| = \frac{|H|}{|H:K|}$. We also have
		$$|G:K|= \frac{|G|}{|K|}$$
		Substituting the equalities for $|G|$ and $|K|$ yields
		$$|G:K| = \frac{|H:K||G:H||H|}{|H|} = |G:H||H:K|$$
\end{ques}

\begin{ques}
	\textbf{10.16}
		Because $|G|$ is odd we know the order of none of the elements
		except for the identity can be $2$.\\
		If the order of some element $a \in G$ was $2$ then $|\langle a
		\rangle| = 2$ but by Lagrange's thm $|\langle a \rangle|$
		should divide $|G|$ which cannot happen since $|G|$ is odd.\\
		Therefore for all elements $a \in G$, we know $a^{-1} \neq a$\\
		Therefore if we take the product of all the elements in $G$, we
		know that for every element in that product, it's inverse is
		present in that product as well. Since $G$ is abelian we can
		rearrange the product so that each element and it's inverse
		present in the product cancel out, to be left with $e$
\end{ques}

\begin{ques}
	\textbf{12.1}
		\begin{enumerate}
			\item
				This is an epimorphism. We have its a homomorphism since
				$$|xy| = |x||y|$$
				for every $x \in \mathbb{R}^+$ we have $x =
				|y|$ where $y = x$ with $y \in \mathbb{R} -
				\{0\}$. However both $y$ and $-y$ map to $x$ so
				it is not one-to-one
			\item
				This is an isomorphism. We know the function
				$f(x) = \sqrt{x}$ is one-to-one and onto on the
				positive real line.\\
				It is a homomorphism since
				$$\sqrt{xy} = \sqrt{x}\sqrt{y}$$
			\item
				This is a epimorphism. We know it is not injective
				since $\varphi((x - 1)) = \varphi((x-1)(x-2)) =
				0$. We do know it is surjective since given any
				$r \in \mathbb{R}$ we have $\varphi( rx) = r$.
				It is homomorphic since $\varphi(P_1(x) +
				P_2(x)) = P_1(1) + P_2(1) = \varphi(P_1(1)) +
				\varphi(P_2(1))$
			\item
				This is also an epimorphism. We know it is not
				injective since $\varphi(x + 1) = \varphi(x +
				2) = 1$. We do know it is surjective since
				every polinomial's antiderivative is a
				polinomial. Finally we know it is a homomorphism since
				$$\varphi(P_1(x) + P_2(x)) = P_1'(x) + P_2'(x)
				= \varphi(P_1) + \varphi(P_2)$$
			\item
				This is the same as applying the element $A \in
				G$ on the left hand side, we know $G$ is
				commutative (proved in previous hw that
				symetric difference is commutative), which
				means for any $BC \in G$ we have $\varphi(BC) =
				ABC \neq ABAC = BC = \varphi(B)\varphi(C)$.
				Therefore $\varphi$ is not a homomorphism
		\end{enumerate}
\end{ques}

\begin{ques}
	\textbf{12.9} 
		There is no isomorphic subgroup of $Q_8$ to $V$. The reason is
		because $V$ has $4$ elements of order $2$ but $Q_8$ only has
		one: $-I$. We know that isomorphisms preserve orders so no
		such isomorphism can exist
\end{ques}

\begin{ques} 
	\textbf{12.12} 
		For any group $G$ of order $8$, by Lagrange's thm we know
		elements can only have order $1,2,4,8$ (because the size of
		their cyclic subgroup equals the order of its element).\\
		The first case is if an element $a$ has order $8$. In this case
		$G = \langle a \rangle$ is a cyclic group.\\
		If an element $a$ has order $4$ then for an element $b \notin
		\langle a \rangle$ we have two cases:\\ 
		The first case is that $\langle b \rangle$ and $\langle a
		\rangle$ do not intersect, in which case we have the cosets $e
		\langle b \rangle, a \langle b \rangle, a^2 \langle b \rangle,
		a^3 \langle b \rangle$. Therefore $|\langle b \rangle| = 2$
		since we need the sum of the sizes of the cosets to be $= G$.
		So $b$ has order $2$. So we have $G = \{e, a, a^2, a^3, b, ab,
		a^2b, a^3b\}$. This is isomorphic to $D_8$\\
		For the case where $\langle b \rangle$ and $\langle a \rangle$
		do intersect, we can narrow it down to show only $b^2 = a^2$ is
		possible. We know that $a \neq b$, and since $(b^{3})^{-1} = b$
		and $(a^3)^{-1} = a$ if $a^3 = b^3$ that would imply $a = b$,
		so $a^3 \neq b^3$.\\
		Looking at other terms we have $(ab)(ba) = a(b^2)a = a^4 = e$.
		We know that $ab$ can have either order $4$ or $2$, if order
		$2$, we have $ab = ba$ so $a,b$ commute. Which would mean we have
		$$G = \{e, a, b, a^2, a^3, b^3, ab, a^3b = ab^3\}$$
		As for the other case when $ab$ has order $4$, we have 
		$$G = \{e, a, b, a^2, a^3, b^3, ab, (ab)^2 = (ba)^2,  ba\}$$
		This is isomorphic to the quaternians (where $a = J$, $b = K$,
		$ab = L$, and $a^2 = -I$)\\		
		That is all the cases where an element has
		order $4$. The rest of the cases are where all elements have
		order $2$.\\ We can add on two elements $a, b$ so we have $\{e,
		a, b, ab\}$ and since $ab$ has order $2$, $ab = (ab)^{-1} =
		b^{-1}a^{-1} = ba$. The set is closed so we add another element
		$c$, which gives us
		$$\{e, a, b, c, ab, ac, bc, abc\}$$
		Using the same logic as before we know $ac = ca$ and $bc = cb$.
		So all elements commute. There is no other possible changes to
		made to the elements of order $2$, so we are done. In total we
		have counted $5$ possible subgroups, each with different
		properties, which means they are not isomorphic.

\end{ques}

\begin{ques} 
	\textbf{12.13} 
		\begin{enumerate}
			\item
				Given any $x,y \in G$ we have since $H$ is abelian
				$$\varphi(yx) = \varphi (y)\varphi(x) = \varphi
				(x)\varphi(y) = \varphi(xy)$$
				Since $\varphi$ is one to one, we know there
				exists an inverse mapping $\varphi^{-1}$ from
				the image of $\varphi$ back to $G$. Applying
				the inverse shows that $G$ is abelian:
				$$\varphi^{-1}(\varphi(xy)) = \varphi^{-1}(\varphi(yx))$$
				$$xy = yx$$
				So $G$ is abelian
			\item
				Given any $x, y \in H$, since $\varphi$ is onto
				we know there is some $x',y' \in G$ such that
				$\varphi(x') = x, \varphi(y') = y$. Therefore
				since $G$ is abelian we have
				$$xy = \varphi(x)\varphi(y) = \varphi(x'y') =
				\varphi(y')\varphi(x') =
				\varphi(y')\varphi(x') = yx$$
				So $H$ is abelian since $xy = yx$

			\item
				As shown in problem $a$ we know $\varphi$ being
				an isomorphism and $H$ abelian $\Rightarrow$
				$G$ abelian. And in problem $b$ we showed the
				other way, $H$ abelian $\Leftarrow$ $G$
				abelian


		\end{enumerate}
\end{ques}
\end{document}
