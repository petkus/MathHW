%29.09.16
\documentclass[12pt]{article}
\usepackage{amsmath, amssymb, amsthm, epsfig}

\newenvironment{ques}{\vspace{2 ex}}{\vspace{2 ex}}
\renewcommand{\theenumi}{\alph{enumi}}

\theoremstyle{definition}

\newenvironment{Proof}{\noindent {\sc Proof.}}{$\Box$ \vspace{2 ex}}
\newtheorem{Wp}{Writing Problem}
\newtheorem{Ep}{Extra Credit Problem}

\oddsidemargin-1mm
\evensidemargin-0mm
\textwidth6.5in
\topmargin-15mm
%\headsep25pt
\textheight8.75in
\footskip27pt

\pagestyle{empty}
\begin{document}

\noindent \textit{\textbf{Math 402, Fall 2016}} \hspace{1.3cm}
\textit{\textbf{HOMEWORK $\#$5}} \hspace{1.3cm} \textit{\textbf{Peter
Gylys-Colwell}} 

\vspace{1cm}

\begin{ques}
	\textbf{10.11} 
\end{ques}

\begin{ques}
	\textbf{11.1}
		We know that the determinant is multiplicative, We also know
		for any $A \in GL(2,\mathbb{R})$, where $|A|$ signifies the
		determinant, we have $|A| = \frac{1}{|A^{-1}|}$.\\
		Therefore for any $B \in SL(2, \mathbb{R})$:
		$$|ABA^{-1}| = |A|1\frac{1}{|A|} = 1$$
		so
		$$ABA^{-1} \in S(2,\mathbb{R}$$
		Therefore
		$$SL(2,\mathbb{R}) \triangleleft GL(2, \mathbb{R})$$
\end{ques}

\begin{ques}
	\textbf{11.2}
		$H$ is not a normal group of $GL(2,\mathbb{R})$, consider
		$$A =\left (\begin{array}{cc}
			0 & 1\\
			1 & 0
		\end{array}\right) = A^{-1} \in GL(2,\mathbb{R}) \text{ and } 
		B = \left (\begin{array}{cc}
			2 & 1\\
			0 & 1 
		\end{array}\right) \in H$$
		We have
		$$ABA^{-1} = \left(\begin{array}{cc}
			1 & 0\\
			1 & 2
		\end{array}\right) \notin H$$
\end{ques}

\begin{ques}
	\textbf{11.3}
		We know that $H = \{e, a\}$ where $e$ is the identity and $a = a^{-1}$.\\
		It is known that $e$ commutes with every element in $G$ so $e \in Z(G)$. Since
		$H$ is normal we have
		for any $g \in G$
		$$gag^{-1} \in \{e, a\}$$
		If $gag^{-1} = e$ then applying $g$ on the right yields $ga = g$
		so $a = e$ which is not true.  Therefore $gag^{-1} = a$, and so
		applying $g$ on the right yields $ga = ag$ and so $a \in Z(G)$,
		so $H \subseteq Z(G)$
\end{ques}

\begin{ques}
	\textbf{11.11}
		
		
\end{ques}

\begin{ques}
	\textbf{11.16} NOT DONNNNNNNNNNNNNNNNNNNNNNNNNNNNNNNNNNNNNNNNNNNNNNNNNN
		Given any element $a \in \mathbb{Q}$ and $z \in \mathbb{Z}$ we
		have since addition is commutative for $aza^{-1} \in a\mathbb{Z}a^{-1}$
		$$aza^{-1} = aa^{-1}z = z \in \mathbb{Z}$$
		so we know $\mathbb{Z}$ is a normal subgroup of $\mathbb{Q}$.
		Therefore $\mathbb{Q} / \mathbb{Z}$ is a group.\\
		We know $\mathbb{Q} / \mathbb{Z}$ is infinite, if it were not
		we could list the group:
		$$\{a_1\mathbb{Z}, a_2\mathbb{Z}, a_3\mathbb{Z}, \dots a_n\mathbb{Z}\}$$
		and come up with an element $a_{n+1}\mathbb{Z} \in \mathbb{Z}$
		
\end{ques}

\begin{ques}
	\textbf{11.17}
		Since $G$ is abelian, for any $a, b \in G$ we have
		$$aHbH = \{ah_1bh_2: h_1, h_2 \in H\} = \{bh_2ah_1: h_1, h_2 \in H\} = bHaH$$
		so $G/H$ is abelian
\end{ques}

\begin{ques}
	\textbf{11.18} 
		We know every element of $G$ has the form $x^n$ where $x$ is
		the generator of $G$, therefore every element of $G/H$ also has
		the form $Hx^n$\\
		We have 
		$$Hx^{k}Hx^j = Hx^{k+j}$$
		Therefore $Hx$ is the generator of $G/H$ since every term in
		$G/H$ is of the form $(Hx)^n = Hx^n$ which means $G/H$ is
		cyclic.
\end{ques}

\begin{ques} 
	\textbf{11.26} 
		We will show there is a bijection between the elements of
		$gHg^{-1}$ and $H$ which would imply $|gHg^{-1}| = |H|$.\\
		We will define this bijection as $f : H \to gHg^{-1}$, $f(h) =
		ghg^{-1}$. We have that if
		$$f(h_1) = f(h_2)$$
		then 
		$$gh_1g^{-1} = gh_2g^{-1}$$
		$$g^{-1}g h_1 g^{-1} g = g^{-1}gh_2g^{-1}g$$
		$$h_1 = h_2$$
		so $f$ is one-to-one. We also know $f$ is onto since $\forall
		ghg^{-1} \in gHg^{-1}$, $h \in H$ so $f(h) = ghg^{-1}$.\\
		Therefore $f$ is a bijection so $|gHg^{-1}| = |H|$.
\end{ques}

\end{document}
