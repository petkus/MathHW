%29.09.16
\documentclass[12pt]{article}
\usepackage{amsmath, amssymb, amsthm, epsfig}

\newenvironment{ques}{\vspace{2 ex}}{\vspace{2 ex}}
\renewcommand{\theenumi}{\alph{enumi}}

\theoremstyle{definition}

\newenvironment{Proof}{\noindent {\sc Proof.}}{$\Box$ \vspace{2 ex}}
\newtheorem{Wp}{Writing Problem}
\newtheorem{Ep}{Extra Credit Problem}

\oddsidemargin-1mm
\evensidemargin-0mm
\textwidth6.5in
\topmargin-15mm
%\headsep25pt
\textheight8.75in
\footskip27pt

\pagestyle{empty}
\begin{document}

\noindent \textit{\textbf{Math 402, Fall 2016}} \hspace{1.3cm}
\textit{\textbf{HOMEWORK $\#$6}} \hspace{1.3cm} \textit{\textbf{Peter
Gylys-Colwell}} 

\vspace{1cm}

\begin{ques}
	\textbf{10.11}
		\begin{enumerate}
			\item
				Given any two left cosets $aH, bH$, we can
				construct a bijection $f : aH \to bH$ as
				$f(ah) = bh$. Which would imply the sets have
				the same size \\
				$f$ is one-to-one since if $f(ah_1) = f(ah_2)$
				then $bh_1 = bh_2$, applying $b^{-1}$ on the
				left yields $h_1 = h_2$ so $ah_1 = ah_2$.\\
				$f$ is onto since for any $bh$, we can choose the
				same $h$ for $ah$ and we have $f(ah) = bh$\\
				Therefore $f$ is a bijection
			\item
				Given any right coset and left coset $Ha, bH$,
				we can construct a bijection $f : Ha \to bH$
				where $f(ha) = bh$. Which would imply the sets
				have the same size.\\
				$f$ is one-to-one since if $f(ha_1) = f(ah_2)$,
				then $bh_1 = bh_2$, applying $b^{-1}$ on the
				left yields $h_1 = h_2$.\\
				$f$ is onto since for any $bh$, we can choose
				the same $h$ for $ha$, and we would have $f(ha)
				= bh$.\\
				Therefore $f$ is a bijection
		\end{enumerate}

\end{ques}

\begin{ques}
	\textbf{11.1}
		We know that the determinant is multiplicative, We also know
		for any $A \in GL(2,\mathbb{R})$, where $|A|$ signifies the
		determinant, we have $|A| = \frac{1}{|A^{-1}|}$.\\
		Therefore for any $B \in SL(2, \mathbb{R})$:
		$$|ABA^{-1}| = |A|1\frac{1}{|A|} = 1$$
		so
		$$ABA^{-1} \in S(2,\mathbb{R}$$
		Therefore
		$$SL(2,\mathbb{R}) \triangleleft GL(2, \mathbb{R})$$
\end{ques}

\begin{ques}
	\textbf{11.2}
		$H$ is not a normal group of $GL(2,\mathbb{R})$, consider
		$$A =\left (\begin{array}{cc}
			0 & 1\\
			1 & 0
		\end{array}\right) = A^{-1} \in GL(2,\mathbb{R}) \text{ and } 
		B = \left (\begin{array}{cc}
			2 & 1\\
			0 & 1 
		\end{array}\right) \in H$$
		We have
		$$ABA^{-1} = \left(\begin{array}{cc}
			1 & 0\\
			1 & 2
		\end{array}\right) \notin H$$
\end{ques}

\begin{ques}
	\textbf{11.3}
		We know that $H = \{e, a\}$ where $e$ is the identity and $a = a^{-1}$.\\
		It is known that $e$ commutes with every element in $G$ so $e \in Z(G)$. Since
		$H$ is normal we have
		for any $g \in G$
		$$gag^{-1} \in \{e, a\}$$
		If $gag^{-1} = e$ then applying $g$ on the right yields $ga = g$
		so $a = e$ which is not true.  Therefore $gag^{-1} = a$, and so
		applying $g$ on the right yields $ga = ag$ and so $a \in Z(G)$,
		so $H \subseteq Z(G)$
\end{ques}

\begin{ques}
	\textbf{11.11}
		We can go through each subgroup of $D_4$ and check which is normal\\
		List of all subgroups:\\
		$$\{I\}, \{I, R^2\}, \{I, F\}, \{I, FR\}, \{I, FR^2\}, \{I, FR^3\}$$
		$$\{I, R, R^2, R^3\}, \{I, R^2, F, FR^2\}, \{I, R^2, FR, FR^3\}, D_4$$
		We know that every subgroup $H$ of size $4$ must be normal since
		there are only two possible cosets (since $D_4$ is size $8$),
		$H, xH$ where $x \notin H$, Therefore $xH = \{a \in G: a \notin H\} = Hx$.\\
		As for the groups of size $2$, none of them are normal: $F\{I,
		R^2\} \neq \{I, R^2\}F$, $R\{I, F\} \neq \{I, F\}R$, $R\{I,
		FR^2\} \neq \{I, FR^2\}R$, $R\{I, FR^3\} \neq \{I, FR^3\}R$.\\
		Since $D_4$ and $\{I\}$ are included, total there are $6$ normal subgroups
		
		
\end{ques}

\begin{ques}
	\textbf{11.16}
		Given any element $a \in \mathbb{Q}$ and $z \in \mathbb{Z}$ we
		have since addition is commutative for $aza^{-1} \in a\mathbb{Z}a^{-1}$
		$$aza^{-1} = aa^{-1}z = z \in \mathbb{Z}$$
		so we know $\mathbb{Z}$ is a normal subgroup of $\mathbb{Q}$.
		Therefore $\mathbb{Q} / \mathbb{Z}$ is a group.\\
		Next we can show every element
		$$\frac{a}{b}\mathbb{Z} \in \mathbb{Q}/\mathbb{Z}$$
		has finite order. We can show this by seeing that
		$$(\frac{a}{b}\mathbb{Z})^b = (\frac{a}{b} + \frac{a}{b} +
		\dots \frac{a}{b} )\mathbb{Z} = a\mathbb{Z} = e\mathbb{Z}$$ 
		Since $a \in \mathbb{Z}$.\\
		Finally we can show $\mathbb{Q} / \mathbb{Z}$ is infinite, if it were not
		we could list the group:
		$$\{a_1\mathbb{Z}, a_2\mathbb{Z}, a_3\mathbb{Z}, \dots a_n\mathbb{Z}\}$$
		Where $a_1, \dots , a_n$ are all $\in (0, 1)$. We can always
		choose our $a_i$ in such a way since there is always
		an element $\in (0,1)$ in each coset because we can add or
		subtract the integer part of any element in the coset to be
		only left with the fractional portion of that number\\
		Since there are infinite rational numbers in $(0,1)$, we can
		choose $a_{n+1} \in (0,1)$ such that $a_{n+1} \notin \{a_1,
		\dots a_n\}$. We know there is no integer $z$ such
		that $a_{n+1} + z = a_i$ for any $i: 0 <
		i \leq n$ since $|a_{n+1} - a_i| < 1$ so $a_{n+1} \notin
		a_i\mathbb{Z}$ therefore 
		$$a_{n+1}\mathbb{Z} \notin \{a_1\mathbb{Z}, a_2\mathbb{Z},
		a_3\mathbb{Z}, \dots a_n\mathbb{Z}\}$$
		which is a contradiction
		
\end{ques}

\begin{ques}
	\textbf{11.17}
		Since $G$ is abelian, for any $a, b \in G$ we have
		$$aHbH = \{ah_1bh_2: h_1, h_2 \in H\} = \{bh_2ah_1: h_1, h_2 \in H\} = bHaH$$
		so $G/H$ is abelian
\end{ques}

\begin{ques}
	\textbf{11.18} 
		We know every element of $G$ has the form $x^n$ where $x$ is
		the generator of $G$, therefore every element of $G/H$ also has
		the form $Hx^n$\\
		We have since $H$ is normal 
		$$Hx^{k}Hx^j = Hx^{k+j}$$
		Therefore $Hx$ is the generator of $G/H$ since every term in
		$G/H$ is of the form $(Hx)^n = Hx^n$ which means $G/H$ is
		cyclic.
\end{ques}

\begin{ques} 
	\textbf{11.26} 
		We will show there is a bijection between the elements of
		$gHg^{-1}$ and $H$ which would imply $|gHg^{-1}| = |H|$.\\
		We will define this bijection as $f : H \to gHg^{-1}$, $f(h) =
		ghg^{-1}$. We have that if
		$$f(h_1) = f(h_2)$$
		then 
		$$gh_1g^{-1} = gh_2g^{-1}$$
		$$g^{-1}g h_1 g^{-1} g = g^{-1}gh_2g^{-1}g$$
		$$h_1 = h_2$$
		so $f$ is one-to-one. We also know $f$ is onto since $\forall
		ghg^{-1} \in gHg^{-1}$, $h \in H$ so $f(h) = ghg^{-1}$.\\
		Therefore $f$ is a bijection so $|gHg^{-1}| = |H|$.
\end{ques}

\end{document}
