%29.09.16
\documentclass[12pt]{article}
\usepackage{amsmath, amssymb, amsthm, epsfig}

\newenvironment{ques}{\vspace{2 ex}}{\vspace{2 ex}}
\renewcommand{\theenumi}{\alph{enumi}}

\theoremstyle{definition}

\newenvironment{Proof}{\noindent {\sc Proof.}}{$\Box$ \vspace{2 ex}}
\newtheorem{Wp}{Writing Problem}
\newtheorem{Ep}{Extra Credit Problem}

\oddsidemargin-1mm
\evensidemargin-0mm
\textwidth6.5in
\topmargin-15mm
%\headsep25pt
\textheight8.75in
\footskip27pt

\pagestyle{empty}
\begin{document}

\noindent \textit{\textbf{Math 402, Fall 2016}} \hspace{1.3cm}
\textit{\textbf{HOMEWORK $\#$7}} \hspace{1.3cm} \textit{\textbf{Peter
Gylys-Colwell}} 

\vspace{1cm}

\begin{ques}
	\textbf{6.3}
		No, if there were some generator $(a, b) \in \mathbb{Z} \times
		\mathbb{Z}$ we have that $(a, b)^n = (na, nb)$ but there is no
		possible power $n \in \mathbb{Z}$ such that $(a, b-1) = (na,
		nb) = (a, b)^n$ since $a = na$ implies that $n = 1$ but then
		$nb \neq b - 1$.
\end{ques}

\begin{ques}
	\textbf{6.5}
		For any element $(a, b) \in A \times B$ we have $(a, b) \circ
		(a^{-1}, b^{-1}) = (e_a, e_b)$, and $(a^{-1}, b^{-1}) \in A
		\times B$ since, $A, B$ are groups. Therefore every element has
		an inverse. We already know the operations are assosiative
		since crossing two associative operations is an associative
		operation, and
		finally we know $A \times B$ is closed under these operations
		since we just apply the operations component wise and $A, B$
		are closed under their respective operation. Therefore $A
		\times B$ is a subgroup of $G \times H$
\end{ques}

\begin{ques}
	\textbf{6.10}
		We have
		$$\{(0,0)\}, \langle (1, 0) \rangle, \langle (1, 1) \rangle, \langle (0,
		1) \rangle, \langle (0, 2) \rangle, \langle (1, 2) \rangle$$
		For a total of $6$ subgroups
\end{ques}

\begin{ques}
	\textbf{6.12}
		(i). If a generator of $G, H$ is $g, h$ respectivly, then a
		generator of $G \times H$ is $(g,h)$ for any element $(a, b)
		\in G \times H$ since $G, H$ are cyclic we know there is some
		$n, m$ such that $(a, b) = (g^n, b^m)$,
\end{ques}

\begin{ques}
	\textbf{13.10}
\end{ques}

\begin{ques}
	\textbf{13.11}
\end{ques}

\begin{ques}
	\textbf{13.16} 
\end{ques}

\begin{ques} 
	\textbf{13.20} 
\end{ques}

\begin{ques} 
	\textbf{13.25} 
\end{ques}
\end{document}
