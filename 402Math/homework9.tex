%29.09.16
\documentclass[12pt]{article}
\usepackage{amsmath, amssymb, amsthm, epsfig}

\newenvironment{ques}{\vspace{2 ex}}{\vspace{2 ex}}
\renewcommand{\theenumi}{\alph{enumi}}

\theoremstyle{definition}

\newenvironment{Proof}{\noindent {\sc Proof.}}{$\Box$ \vspace{2 ex}}
\newtheorem{Wp}{Writing Problem}
\newtheorem{Ep}{Extra Credit Problem}

\oddsidemargin-1mm
\evensidemargin-0mm
\textwidth6.5in
\topmargin-15mm
%\headsep25pt
\textheight8.75in
\footskip27pt

\pagestyle{empty}
\begin{document}

\noindent \textit{\textbf{Math 402, Fall 2016}} \hspace{1.3cm}
\textit{\textbf{HOMEWORK $\#$9}} \hspace{1.3cm} \textit{\textbf{Peter
Gylys-Colwell}} 

\vspace{1cm}

\begin{ques}
	\textbf{6.3}
		No, if there were some generator $(a, b) \in \mathbb{Z} \times
		\mathbb{Z}$ we have that $(a, b)^n = (na, nb)$ but there is no
		possible power $n \in \mathbb{Z}$ such that $(a, b-1) = (na,
		nb) = (a, b)^n$ since $a = na$ implies that $n = 1$ but then
		$nb \neq b - 1$.
\end{ques}

\begin{ques}
	\textbf{6.5}
		For any element $(a, b) \in A \times B$ we have $(a, b) \circ
		(a^{-1}, b^{-1}) = (e_a, e_b)$, and $(a^{-1}, b^{-1}) \in A
		\times B$ since, $A, B$ are groups. Therefore every element has
		an inverse. We already know the operations are assosiative
		since crossing two associative operations is an associative
		operation, and
		finally we know $A \times B$ is closed under these operations
		since we just apply the operations component wise and $A, B$
		are closed under their respective operation. Therefore $A
		\times B$ is a subgroup of $G \times H$
\end{ques}

\begin{ques}
	\textbf{6.10}
		We have
		$$\{(0,0)\}, \langle (1, 0) \rangle, \langle (1, 1) \rangle, \langle (0,
		1) \rangle, \langle (0, 2) \rangle, \langle (1, 2) \rangle$$
		For a total of $6$ subgroups
\end{ques}

\begin{ques}
	\textbf{6.12}
		(i). If $(a, b)$ is a generator of $G \times H$, then for any
		$g \in G$ and $h \in H$ we have for some $n \in \mathbb Z$,
		since $G \times H$ is a cyclic for $(g,h) \in G \times H$ we have
		$(a, b)^n = (a^n, b^n) = (g,h) \Leftrightarrow a^n = g, b^n =
		h$ and so $a, b$ are generators of $G, H$ respectively\\
		(ii). For any subgroup $A \times B$ of $G \times H$ we know
		that for any $(a, b) \in A \times B$, $(a, b)^{-1} = (a^{-1},
		b^{-1}) \in A \times B$, we know the group operations
		must be closed and assosiative as well. Therefore $A$ and $B$
		satisify all the conditions to be subgroups of $G, H$
		respectively since the inverse of every element in $A, B$ is
		contained in $A, B$ respectively and the sets are closed under
		their respective group operation.
\end{ques}

\begin{ques}
	\textbf{13.10}
		\begin{enumerate}
			\item
				If $G$ is abelian then $G \times G$ is abelian.
				We know that any subgroup of an abelian group
				is normal, and so this would imply $D$ is
				normal. Conversly if $G$ was not abelian, we
				can take elements $a, b \in G$ that dont
				commute, we have for $(b, b) \in D$
				$$(a, b) (b, b) (a, b)^{-1} = (aba^{-1},
				bbb^{-1}) = (aba^{-1}, b)$$
				Since $ab \neq ba$ we know $(ab)a^{-1} \neq
				baa^{-1} = b$ which means
				$$(a, b) (b, b) (a, b)^{-1} = (aba^{-1}, b)
				\notin D$$
			\item 
				Let $\varphi: G \times G$ be defined as
				$\varphi(a, b) = ab^{-1}$, we have 
				$$\varphi(a,b)\varphi(c,d) = ab^{-1}cd^{-1} =
				ac(bd)^{-1} = \varphi(ac,bd)$$
				So $\varphi$ is a homomorphism. $D$ is
				precisely the kernel of $\varphi$ since
				$\varphi(a,b) = e \Leftrightarrow ab^{-1} = e
				\Leftrightarrow a = b$. Therefore by the
				fundamental theorem we have 
				$$(G\times G)/D \cong G$$
		\end{enumerate}
\end{ques}

\begin{ques}
	\textbf{13.11}
		\begin{enumerate}
			\item
				We can define a homomorphism $\varphi: G \to
				G/H \times G/K$ with $\varphi(g) = (gH, gK)$.
				To show it is a homomorphism we have for $a,b
				\in G$:
				$$\varphi(a)\varphi(b) = (aH, aK)(bH, bK) =
				(abH, abK) = \varphi(ab)$$
				We know that ker($\varphi$) $= H \cap K$ since
				$\varphi(g) = (H, K) \Leftrightarrow g \in K$
				and $g \in H$. Therefore by the Fundamental Theorem we have
				$$G/(H \cap K) \cong \varphi(G)$$
				We know $\varphi(G)$ must be a subgroup of $G/H
				\times G/K$ since the image of a homomorphism
				is a group. And so we are done
			\item 
				If $G = HK$ we can show the $\varphi$ from part
				$a$ is surjective which would imply
				$\varphi(G)= G/H \times G/K$. For any $(aH,bK) \in G/H
				\times G/K$. Since $G = HK$, $a = h_ak_a,
				b=h_bk_b$ where $h_a,h_b \in H, k_a, k_b \in K$. 
				Now since $H,K$ are normal:
				$$(h_ak_aH, h_bk_bK) = (h_aHk_a, h_bK) = (k_aH, h_bK)$$
				and so we have
				$$\varphi(k_ah_b) = (k_aH, h_bK)$$
				And so $\varphi$ is surjective.
		\end{enumerate}
\end{ques}

\begin{ques}
	\textbf{13.16} 
		They are isomorphic. We have
		$$\frac{G \times H}{A \times B} = \{(a, b)(G, H): (a, b) \in A
		\times B\} = \{(aG, bH): a \in A, b \in B\} = G/A \times H/B$$
\end{ques}

\begin{ques} 
	\textbf{13.20} 
		We can commpose $\varphi$ with the canonical homomorphism $\rho
		: K \to K/J$. The composition of homomorphisms is a
		homomorphism so $\varphi \circ \rho$ is a homomorphism.
		Now we can use the Fundamental Theorem, letting $f = \varphi
		\circ \rho$ we have $f: G \to K/J$ is a homomorphism and is
		surjective since $\rho$ is surjective and $\varphi$ is
		surjective.
		$$G/\ker(f) \cong K/J$$
		And $\ker(f)$ is some normal subgroup $H$ of $G$.
\end{ques}

\begin{ques} 
	\textbf{13.25} 
		\begin{enumerate}
			\item
				We have for $D_3$, the symmetry group of the
				triangle which is not abelian, we established
				in class $H = \{e, FR\}$ is a normal subgroup
				and $H$ is abelian since there is only two
				elements and one of them is $e$.  We also know
				$D_3/H$ is abelian since $D_3/H = \{eH, RH,
				R^2H\}$ and all the $R$s commute. So $D_3$ is
				metablelian. 
			\item
				Let $H$ be the subgroup of $G$ that is
				abelian along with $G/H$ being abelian. We know
				$\varphi(H)$ is an abelian subgroup of $K$ that
				is normal since the image of an abelian group
				is an abelian group for any homomorphism and
				from thm 13.3 we get normality. We will call
				$\varphi(H)$ $J$ for convienence. We have
				that $K/J$ is abelian since for any
				$aJ, bJ \in K/J$ since $\varphi$ is
				surjective there is some $a_g, b_g \in G:
				\varphi(a_g) = a, \varphi(b_g) = b$ and so we
				have 
				$$aJbJ = abJ = \varphi(a_g)\varphi(b_g)J =
				\varphi(a_gb_gH)$$
				and since $G/H$ is abelian
				$$= \varphi(b_ga_gH) = baJ = bJaJ$$
				And so $K$ is metablelian

		\end{enumerate}
\end{ques}
\end{document}
