\documentclass[12pt]{article}
\usepackage{amsmath, amssymb, amsthm, epsfig}

\newenvironment{definition}{\vspace{2 ex}{\noindent{\bf Definition}}}
        {\vspace{2 ex}}

\newenvironment{ques}[1]{\textbf{Exersise #1}\vspace{1 mm}\\ }{\bigskip}

\renewcommand{\theenumi}{\alph{enumi}}

\theoremstyle{definition}

\newenvironment{Proof}{\noindent {\sc Proof.}}{$\Box$ \vspace{2 ex}}
\newtheorem{Wp}{Writing Problem}
\newtheorem{Ep}{Extra Credit Problem}

\oddsidemargin-1mm
\evensidemargin-0mm
\textwidth6.5in
\topmargin-15mm
\textheight8.75in
\footskip27pt


\renewcommand{\l}{\left }
\renewcommand{\r}{\right }

\newcommand{\R}{\mathbb R}
\newcommand{\Q}{\mathbb Q}
\newcommand{\Z}{\mathbb Z}
\newcommand{\C}{\mathbb C}

\newcommand{\s}{\sin}
\renewcommand{\c}{\cos}

\renewcommand{\t}{\theta}
\renewcommand{\a}{\alpha}

\newcommand{\norm}[1]{\left\lVert#1\right\rVert}

\newcommand{\T}{\mathfrak{T}}

\pagestyle{empty}
\begin{document}

\noindent \textit{\textbf{Math 441, Fall 2017}} \hspace{1.3cm}
\textit{\textbf{HOMEWORK $\#$2}} \hspace{1.3cm} \textit{\textbf{Peter
Gylys-Colwell}} 

\vspace{1cm}

\begin{ques}{3.7}
	Given $x_0 \in \mathbb R^n$ and any $\epsilon > 0$, let $\delta =
	\epsilon$. For any $x \in B(x_0, \delta)$ we have $\lVert x
	- x_0 \rVert < \delta = \epsilon$. We also have $| f(x) - f(x_0)
	|= | \lVert x \rVert - \lVert x_0 \rVert |$. \\
	We know that we have $| \lVert x \rVert - \lVert x_0 \rVert | = \lVert
	x \rVert - \lVert x_0 \rVert$ or $-(\lVert x \rVert - \lVert x_0
	\rVert) = \lVert x_0 \rVert - \lVert x \rVert$. By the triangle
	inequality we know both $\lVert x \rVert - \lVert x_0 \rVert \leq \lVert x -
	x_0 \rVert$ and $\lVert x_0 \rVert - \lVert x \rVert \leq \lVert x_0 -
	x \rVert = \lVert x - x_0 \rVert$. And so $| f(x) - f(x_0)| = | \lVert
	x \rVert - \lVert x_0 \rVert | \leq \lVert x - x_0 \rVert < \epsilon$\\
	Thus $f$ is continuous
\end{ques}

\begin{ques}{3.9}
	\begin{enumerate}
		\item
			If there exists some $N$ such that $x_j = x_k$ for all
			$j,k > N$ then $\delta(x_j, x_k) = 0 < \epsilon$ for
			all $\epsilon > 0$, and so by definition $x_n$
			converges. Conversly if $x_n$ converges, let $\epsilon
			= 1/2$. We have that for some $N$, $\delta(x_j,x_k) <
			1/2$ for all $j,k > N$. Since $\delta(x_j,x_k) >
			\epsilon$ if and only if $x_j \neq x_k$, we have that
			$x_j = x_k$ for all $j,k > N$
		\item
			For any $x_0 \in X$ and any $\epsilon > 0$, let $\delta
			= 1/2$. We have that $\delta(x,x_0) < \delta$ if and
			only if $x = x_0$, by definition of the discrete
			metric. Therefore $B_(x_0,\delta) = \{ x_0\}$ and as
			one of the properties of the metric, we have $d(f(x_0),
			f(x_0)) = 0 < \epsilon$. Therefore by definition $f$ is continuous
	\end{enumerate}
\end{ques}

\begin{ques}{3.11}
	For a given $\epsilon > 0$, $f$ continuous means that 
	there exists a $\delta > 0 $ such that $d(x,x_i) < \delta$ implies $\rho
	(f(x) , f(x_i)) < \epsilon$. $x_n \to x$ means there is a $N > 0$ such
	that for $k > N$, $d(x_k,x) < \delta$. Therefore for $k > N$ we have $\rho
	(f(x) , f(x_k)) < \epsilon$. Thus $f(x_n) \to f(x)$
\end{ques}

\begin{ques}{3.14}
	For any $\theta_0 \in [0,2\pi)$, given $\epsilon > 0$ let $\delta =
	\min(\epsilon, 1)$.
	For any $\theta \in [0,2\pi)$ with $|\theta - \theta_0| < \delta$ we
	have 
	$$\lVert f(\theta) - f(\theta_0) \rVert = \sqrt{(\cos(\theta) -
	\cos(\theta_0))^2 + (\sin(\theta) - \sin(\theta_0))^2}$$
	$$= \sqrt{\cos^2(\theta) - 2\cos(\theta_0)\cos(\theta) +
	\cos^2(\theta_0) +\sin^2(\theta) - 2\sin(\theta_0)\sin(\theta) +
	\sin^2(\theta_0)}$$
	$$= \sqrt{2(1 - (\cos(\theta_0)\cos(\theta) + \sin(\theta_0)\sin(\theta)))}$$
	Using the sum formula ($\cos(a - b) = \cos a\cos b + \sin a \sin b$) we have:
	$$= \sqrt{2(1 - \cos(\theta - \theta_0))}$$
	We can use the Taylor series of $\c \theta$ to get a bound 
	$$= \sqrt{2\l(1 - \l(1 - \frac {(\t - \t_0)^2}{2} + \frac{(\t -
	\t_0)^4}{4!} - \dots \r )\r ) }$$
	Since we know $|\t - \t_0| < 1$ (since $\delta \leq 1$) we know that
	$(\t - \t_0)^2 \geq \l(\frac{(\t - \t_0)^4}{12} - \dots \r )$ since a
	convergent alternating series terms are always greater than the sum of
	the terms following it. Therefore we have
	$$ = \sqrt{\l((\t - \t_0)^2 - \frac{(\t -
	\t_0)^4}{12} + \dots \r )} \leq \sqrt{(\t - \t_0)^2} = |\t - \t_0| <
	\delta \leq \epsilon$$
	And so $F$ is continous.\\
	$F$ is surjective since for any $p = (x,y) \in S^1$, we let $\t$ be the
	angle that $p$ makes with the origin and the point $(1,0)$ and from the
	geometric definition of $\s$ and $\c$ we have that $\s \t = y$, $\c \t
	= x$ and so $F(\t) = p$. $F$ is injective since if $F(\t) = F(\a)$,
	then we have $\s \t = \s \a$, since $\t, \a \in [0, 2\pi)$ we know that
	$\t = \pi - \a$ or $\t = \a$, then we have that $\c \t = \c \a$ which
	means that $\t = 2\pi - \a$ or $\t = \a$. We cannot have $\pi - \a = \t
	= 2\pi - \a$ since then $0 = \pi$ therefore $\t = \a$.\\
	Therefore $F$ is bijective and so has an inverse mapping $F^{-1}$.\\
	However we have that $F^{-1}(1,0) = 0$, but for $\epsilon = \pi$ and
	any $\delta > 0$, letting $a < \min \{1/2, \delta\}$ we have that for
	$$ p = \l(-\sqrt{1 - \frac{a^2}{4}} , \frac a 2 \r) \in S^1$$
	with $\norm{p - 0} < \delta$ and $F^{-1}(p) = \t$ where
	$$\t = \sin^{-1}\l(-\sqrt{1 - \frac{a^2}{4}}\r), \t = \c^{-1} \l( \frac a 2 \r) $$
	We have that $\t > \pi$ since $\l(-\sqrt{1 - \frac{a^2}{4}}\r) < 0$ and
	$\s^{-1} \a > \pi$ for any $\a \in (0, -1)$.\\
	Therefore $|F^{-1}(p) - F^{-1}(0)| > \epsilon$. So $F^{-1}$ is not
	continous
	
\end{ques}

\begin{ques}{3.17}
	\begin{enumerate}
		\item
			By definition of open for metric spaces, we have that
			for any $a \in \emptyset$, for any $\epsilon > 0$,
			$B(a, \epsilon)$ is itself the empty set since $a$ does
			not exist so $B(a, \epsilon ) \subseteq \emptyset$.
			Thus the empty set is open
		\item
			For any $a \in X$ and $\epsilon > 0$ we have that $B(a,
			\epsilon) = \{x \in X: \delta(x,a) < \epsilon\}$ thus
			$B(a,\epsilon) \subseteq X$ and so $X$ is open
		\item
			For any $a \in B(x, \epsilon)$, let $ \epsilon' =
			\epsilon - \delta(x, a)$. Thus we have for any $y \in
			B(a, \epsilon')$ we have $\delta(y,a) < \epsilon' =
			\epsilon - \delta(x,a)$. Thus from the triangle
			inequality we have:
			$$\delta(y,x) \leq \delta(x,a) + \delta(a,y) < \epsilon$$
			Thus $y \in B(x, \epsilon)$, so $B(a, \epsilon')
			\subseteq B(x, \epsilon)$. Thus $B(x, \epsilon)$ is open
		\item
			For any $x \in U_1 \cap \dots \cap U_k$, since each
			$U_i$ is open there exists for each $U_i$ $\epsilon_i >
			0$ where $B(x, \epsilon_i) \subseteq U_i$. Let
			$\epsilon = \min \{\epsilon_1, \epsilon_2, \dots
			\epsilon_k \}$. We have that $B(x, \epsilon) \subseteq
			B(x, \epsilon_i)$ for all $i$. This is because we have
			for any $a \in B(x, \epsilon)$ we have that
			$\delta(a,x) < \epsilon \leq \epsilon_i$ and thus $a
			\in B(x, \epsilon_i)$. Therefore $B(x, \epsilon)
			\subseteq U_i$ for all $i$, so $B(x, \epsilon)
			\subseteq U_1 \cap U_2 \cap \dots U_k$. Thus $U_1 \cap
			\dots U_k$ is open
	\end{enumerate}
\end{ques}

\begin{ques}{\S 13, 3}
	In example 4 we have $X - X = \emptyset$ which is countable so $X \in
	\T_c$, and $X - \emptyset = X$ so $\emptyset \in \T_c$.\\
	For any collection of sets $A \subseteq \T_c$ we have from Demorgans laws:
	$$X - \l(\bigcup_{U \in A}U \r) = \bigcap_{U \in A} (X - U)$$
	Intersections of countable sets are countable, therefore $\l(\bigcup_{U
	\in A}U \r) \in \T_c$.\\
	For a finite collection $A \subset \T_c$ we have from Demorgans laws:
	$$X - \l(\bigcap_{U \in A}U \r) = \bigcup_{U \in A} (X - U)$$
	Finite unions of countable sets are countable. Therefore $\bigcup_{U
	\in A} (X - U) \in \T_c$. Thus all the axioms of a topology are
	satisfied, so $\T_c$ is a topology.\\
	However we have $\T_\infty$ is not necessarily a topology:\\
	Let $X = \Z$. Let $U = \{x \in \Z: x < 0\}$ and $V = \{x \in \Z: x >
	0\}$. We have that $X - U = \{x \in \Z: x \geq 0\}$ is an infinite set
	and $X-V = \{x \in Z: x \leq 0\}$ is an infinite set, so $U,V \in
	\T_\infty$. However we have
	$$X - \l(V \cup U\r) = \{0\}$$
	Is not infinite. Thus $U \cup V \notin \T_\infty$. So $\T_\infty$ does
	not satisfy the axioms of a topology.
	
\end{ques}

\begin{ques}{\S 13, 4}
	\begin{enumerate}
		\item
			We have that $X, \emptyset \in \T_\alpha$ for all $\alpha$,
			so $X, \emptyset \in \bigcap \T_\alpha$\\
			We have that for any collection of sets $A \subseteq
			\bigcap \T_\alpha$, we have that for  each
			$\T_\alpha$,  $A \subseteq \T_\alpha$, and thus since
			$\T_\alpha$ is a topolgy
			$$\bigcup_{U \in A} U \in \T_\alpha$$
			so $\bigcup_{U \in A} U \in \bigcap\T_\alpha$.\\
			For a finite collection $A \subseteq \bigcap
			\T_\alpha$, we again have that for  each
			$\T_\alpha$,  $A \subseteq \T_\alpha$, and thus since
			$\T_\alpha$ is a topolgy, we have that 
			$$\bigcap_{U \in A} U \in \T_\alpha$$
			So $\bigcap_{U \in A} U \in \bigcap\T_\alpha$.\\
			$\bigcup \T_\alpha$ is not necessarily a topology:\\
			Consider $X = \{a, b, c\}$. We have the topolgies
			$$\T_1 = \{ \emptyset, X, \{a\}, \{a, c\}\}, \T_2 = \{
			\emptyset, X, \{b\}, \{b, c\}\}$$
			$$\T_1 \cup \T_2 = \{ \emptyset, X, \{a\}, \{b\}, \{a,
			c\}, \{b, c\}\}$$
			However $\{a, c\} \cap \{b, c\} = \{c\} \notin \T_1
			\cup \T_2$. So $\T_1 \cup \T_2$ is not a topology
		\item
			The largest topology contained in all $\T_\alpha$ is 
			$$\bigcap \T_\alpha$$
			This follows directly from the conditions. $\bigcap
			\T_\alpha$ is contained in all $\T_\alpha$ and any
			topology that is contained in all $\T_\alpha$ must be
			contained in their intersection.\\
			The smallest topology that contains each $\T_\alpha$ is
			the Topology $\T$ that consists of all the sets of
			$\bigcup \T_\a $ as well as the sets obtained by taking
			any sequence of unions and finite intersections of
			sets in $\bigcup \T_\a $. From this construction it is clear that $\T$
			contains all $\T_\a$. It is also clear $\T$ is a
			topology as well since taking any union or finite
			intersection of sets of $\T$ will again produce a set obtained
			by taking a sequence of unions and finite
			intersections of sets in $\bigcup \T_\a$\\
			The reasoning that $\T$ is the smallest of the desired
			Topologies, and is unique is as follows:\\
			If there exists a smaller topology $G \subset \T$ where
			$\T_\a \subset G$ for each $\T_\a$ then there must be
			some $A \in \T$ with $A \notin G$. We know $A$ has the
			form
			$$A_1 \cup A_2 \dots \cap A_i \cap \dots \cup A_k
			\dots$$
			Where each $A_i \in \T_\a$ for some $\T_a$. However,
			this means that $G$ is not a topology since we would
			have that $A$ is union of sets in $G$ that is not in
			$G$. This also implies that if there is a topology $F$
			that contains each $\T_\a$ then $F \cap \T = \T$ since
			for any $A \in \T$ with $A \notin F$ we have the same
			problem. Thus $\T$ must be unique, since for any $F
			\neq \T$ we have $ \T \subset F$ so $\T$ is smaller
			than $F$.

			
		\item
			The smallest topology containing $\T_1, \T_2$ would be
			$$\T_1 \cup \T_2 = \{\emptyset, X, \{a\}, \{b\}, \{a, b\} \{b, c\}\}$$
			The largest topology containing $\T_1, \T_2$ would be
			$$\T_1 \cap \T_2 = \{\emptyset, X, \{a\}\}$$
	\end{enumerate}
\end{ques}
\end{document}
