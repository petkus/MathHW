%29.09.16
\documentclass[12pt]{article}
\usepackage{amsmath, amssymb, amsthm, epsfig}

\newenvironment{definition}{\vspace{2 ex}{\noindent{\bf Definition}}}
        {\vspace{2 ex}}

%\newenvironment{ques}{\vspace{2 ex}}
	\newenvironment{ques}[1]{\textbf{Exersise #1}\vspace{1 mm}\\ }{\bigskip}

\renewcommand{\theenumi}{\alph{enumi}}

\theoremstyle{definition}

\newenvironment{Proof}{\noindent {\sc Proof.}}{$\Box$ \vspace{2 ex}}
\newtheorem{Wp}{Writing Problem}
\newtheorem{Ep}{Extra Credit Problem}

\oddsidemargin-1mm
\evensidemargin-0mm
\textwidth6.5in
\topmargin-15mm
%\headsep25pt
\textheight8.75in
\footskip27pt

\pagestyle{empty}
\begin{document}

\noindent \textit{\textbf{Math 441, Fall 2017}} \hspace{1.3cm}
\textit{\textbf{HOMEWORK $\#$2}} \hspace{1.3cm} \textit{\textbf{Peter
Gylys-Colwell}} 

\vspace{1cm}

\begin{ques}{3.7}
	Given $x_0 \in \mathbb R^n$ and any $\epsilon > 0$, let $\delta =
	\epsilon$. For any $x \in B(x_0, \delta)$ we have $\lVert x
	- x_0 \rVert < \delta = \epsilon$. We also have $| f(x) - f(x_0)
	|= | \lVert x \rVert - \lVert x_0 \rVert |$. \\
	We know that we have $| \lVert x \rVert - \lVert x_0 \rVert | = \lVert
	x \rVert - \lVert x_0 \rVert$ or $-(\lVert x \rVert - \lVert x_0
	\rVert) = \lVert x_0 \rVert - \lVert x \rVert$. By the triangle
	inequality we know both $\lVert x \rVert - \lVert x_0 \rVert \leq \lVert x -
	x_0 \rVert$ and $\lVert x_0 \rVert - \lVert x \rVert \leq \lVert x_0 -
	x \rVert = \lVert x - x_0 \rVert$. And so $| f(x) - f(x_0)| = | \lVert
	x \rVert - \lVert x_0 \rVert | \leq \lVert x - x_0 \rVert < \epsilon$\\
	Thus $f$ is continuous
\end{ques}

\begin{ques}{3.9}
	\begin{enumerate}
		\item
			If there exists some $N$ such that $x_j = x_k$ for all
			$j,k > N$ then $\delta(x_j, x_k) = 0 < \epsilon$ for
			all $\epsilon > 0$, and so by definition $x_n$
			converges. Conversly if $x_n$ converges, let $\epsilon
			= 1/2$. We have that for some $N$, $\delta(x_j,x_k) <
			1/2$ for all $j,k > N$. Since $\delta(x_j,x_k) >
			\epsilon$ if and only if $x_j \neq x_k$, we have that
			$x_j = x_k$ for all $j,k > N$
		\item
			For any $x_0 \in X$ and any $\epsilon > 0$, let $\delta
			= 1/2$. We have that $\delta(x,x_0) < \delta$ if and
			only if $x = x_0$, by definition of the discrete
			metric. Therefore $B_(x_0,\delta) = \{ x_0\}$ and as
			one of the properties of the metric, we have $d(f(x_0),
			f(x_0)) = 0 < \epsilon$. Therefore by definition $f$ is continuous
	\end{enumerate}
\end{ques}

\begin{ques}{3.11}
	For a given $\epsilon > 0$, $f$ continuous means for that given $\epsilon > 0$
	there exists a $\delta > 0 $ such that $d(x,x_i) < \delta$ implies $\rho
	(f(x) , f(x_i)) < \epsilon$. $x_n \to x$ means there is a $N > 0$ such
	that for $k > N$, $d(x_k,x) < \delta$ and therefore for $k > N$, $\rho
	(f(x) , f(x_k)) < \epsilon$. Thus $f(x_n) \to f(x)$
\end{ques}

\begin{ques}{3.14}
	For any $\theta_0 \in [0,2\pi)$, given $\epsilon > 0$ let $\delta = $.
	For any $\theta \in [0,2\pi)$ with $|\theta - \theta_0| < \delta$ we
	have 
	$$\lVert f(\theta) - f(\theta_0) \rVert = \sqrt{(\cos(\theta) -
	\cos(\theta_0))^2 + (\sin(\theta) - \sin(\theta_0)^2}$$
	$$= \sqrt{\cos^2(\theta) - 2\cos(\theta_0)\cos(\theta) +
	\cos^2(\theta_0) +\sin^2(\theta) - 2\sin(\theta_0)\sin(\theta) +
	\sin^2(\theta_0)}$$
	$$= \sqrt{2(1 - (\cos(\theta_0)\cos(\theta) + \sin(\theta_0)\sin(\theta)))}$$
	Using the sum formula ($\cos(a - b) = \cos a\cos b + \sin a \sin b$) we have:
	$$= \sqrt{2(1 - \cos(\theta - \theta_0))}$$
	A common property of $\sin$ is that $|\sin x| < |x|$ since $|x|$ is the arc
	length while $\sin$ is the vertical length of point on the unit circle.
	Therefore $\sin^2 (\theta - \theta_0) = 1 - \cos^2 (\theta - \theta_0)
	< (\theta - \theta_0)^2 < \delta^2$. And so we have
	$$\lVert f(\theta) - f(\theta_0) \rVert < $$ 
	
\end{ques}

\begin{ques}{3.17}
\end{ques}

\begin{ques}{\S 13, 3}
\end{ques}

\begin{ques}{\S 13, 4}
\end{ques}
\end{document}
