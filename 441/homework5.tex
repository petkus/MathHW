\documentclass[12pt]{article}
\usepackage{amsmath, amssymb, amsthm, epsfig}

\newenvironment{definition}{\vspace{2 ex}{\noindent{\bf Definition}}}
        {\vspace{2 ex}}

\newenvironment{ques}[1]{\textbf{Exersise #1}\vspace{1 mm}\\ }{\bigskip}

\renewcommand{\theenumi}{\alph{enumi}}

\theoremstyle{definition}

\newenvironment{Proof}{\noindent {\sc Proof.}}{$\Box$ \vspace{2 ex}}
\newtheorem{Wp}{Writing Problem}
\newtheorem{Ep}{Extra Credit Problem}

\oddsidemargin-1mm
\evensidemargin-0mm
\textwidth6.5in
\topmargin-15mm
\textheight8.75in
\footskip27pt


\renewcommand{\l}{\left }
\renewcommand{\r}{\right }

\newcommand{\R}{\mathbb R}
\newcommand{\Q}{\mathbb Q}
\newcommand{\Z}{\mathbb Z}
\newcommand{\C}{\mathbb C}

\newcommand{\s}{\sin}
\renewcommand{\c}{\cos}

\renewcommand{\t}{\theta}
\renewcommand{\a}{\alpha}

\newcommand{\norm}[1]{\left\lVert#1\right\rVert}

\newcommand{\T}{\mathcal{T}}

\pagestyle{empty}
\begin{document}

\noindent \textit{\textbf{Math 441, Fall 2017}} \hspace{1.3cm}
\textit{\textbf{HOMEWORK $\#$5}} \hspace{1.3cm} \textit{\textbf{Peter
Gylys-Colwell}} 

\vspace{1cm}

\begin{ques}{\S 19, 3}
	% (finite products only)
	For finite products (which we are considering) the box and product
	topologies are the same.\\
	For any $y, x \in \prod_{\a \in [n]} X_\a$ with $y \neq x$, we can consider the
	components of $x$ and $y$:
	$$x = (x_1, x_2, \dots x_n), y = (y_1, y_2, \dots y_n)$$
	where $x_i, y_i \in X_i$. Since $y \neq x$ there exists some $k$ such that
	$x_k \neq y_k$. Since $X_k$ is Hausdorf there exists open sets $U_k, V_k
	\subseteq X_k$ such that $U_k \cap V_k = \emptyset$ and $x_k \in U_k, y_k
	\in V_k$. We can the open sets 
	$$U = U_k \times \prod_{\a \in [n], \a \neq k} X_\a, V = V_k \times \prod_{\a
	\in [n], \a \neq k} X_\a$$
	From the definition of box and product topologies it is clear that these
	are open sets. We also know that $x \in U, y \in V$ since their $k$th
	compenent is contained in the $k$th component of the open set and any other
	$i$th component is contained in $X_i$. Thus if we have that $U \cap V =
	\emptyset$ then the product topologies is Hausdorf.\\
	We have that $U \cap V = \emptyset$ since if $s \in U$ and $s \in V$ then
	we have that the $k$th component of $s$ is contained in $U_k$ and $V_k$
	which is not possible since $U_k$ and $V_k$ are dsijoint.
\end{ques}

\begin{ques}{\S 20, 3}
	(a) \ We know that $d$ is continous if the inverse image of every basis elt
	is open. A basis of $\R$ is the set of open intervals of the form $I =
	(a,b): a < b$.  Considering any $I$, for any elt $x \in d^{-1}(I)$ we have
	that if $b \leq 0$ then $d^{-1}(I) = \emptyset$ is open since $d$ maps to
	nonegative numbers. Otherwise, for any $(x,y) \in d^{-1}(I)$, we can define
	the following radius:
	$$r_{x,y} = \min \{ d(x,y) - a, b - d(x,y)\}$$
	We can consider the open set in $X \times X$ of the product of balls of
	radius $r$ centered around $x$ and $y$:  $U_{x,y} = B_{r/2}(x) \times B_{r/2}(y)$.
	We have that $U_{x,y} \subseteq d^{-1}(I)$ the reason is because for any
	$(s,t) \in U_{x,y}$ we have from the triangle inequality that $d(s,t) \leq
	d(s,x) + d(x,y) + d(y,t)$ and $d(s,x) + d(y,t) \leq r$ so 
	$$d(s,t) < r + d(x,y) \leq b$$
	And we also have $d(x,y) \leq d(x,s) + d(s,t) + d(t,y)$ so
	$$d(s,t) \geq d(x,y) - d(x,s) - d(y,t) > d(x,y) - r \geq a$$
	Thus $d(s,t) \in I$ so $(s,t)
	\in d^{-1}(I)$. Therefore we have that $U_{x,y} \subseteq d^{-1}(I)$. We
	have that $d^{-1}(I)$ is the union of these open sets for each arbitrary
	$(x,y) \in d^{-1}(I)$ and thus open:
	$$d^{-1}(I) = \bigcup_{(x,y) \in d^{-1}(I)} U_{x,y}$$
	This is clear since for each $p = (x,y) \in d^{-1}(I)$ there is a $U_{x,y}$
	which contains $p$ and each of these $U_{x,y}$ is contained in $d^{-1}(I)$
	so the union is contained in $d^{-1}(I)$. Therefore $d$ is continous\\
	\\
\end{ques}

\begin{ques}{\S 21, 1}
	The question is asking if $d|A \times A$ generates the same topology as the
	subspace topology.\\
	We can show that the basis for each topology are contained within the
	other, and thus the topologies contain each other so are equal. The basis of
	the subspace topology is the intersections of the
	basis elements of $X$ with $A$. Therefore the basis consists of sets of the
	form $B_r(x) \cap A$. For the metric topology we have the basis $B'_r(x) =
	\{p \in A: d(p,x) < r\}$. We have that $B'_r(x) = B_r(x) \cap A$ so it is
	clear that the metric topology is contained in the subspace topology. For
	any $B_r(x)$, if $x \in A$ then $B_r(x) \cap A = B'_r(x)$. If $x \notin A$
	then for any $y \in B_r(x) \cap A$ we have that $B'_{r - d(y,x)}(y)
	\subseteq B_r(x)$ therefore we have
	$$B_r(x) \cap A = \bigcup_{y \in B_r(x) \cap A} B'_{r - d(y,x)}(y)$$
	So $B_r(x) \cap A$ is contained in the topology of of the metric space
	since it is a union of open sets.
\end{ques}

\begin{ques}{\S 22, 2}
	(a) \ We know that $p$ is surjective since for any $y \in Y$ we have
	$p(f(y)) = y$. Thus all we have to check is the converse of continuity. For any
	$U \subseteq Y$ we have the following
	$$f^{-1}(p^{-1}(U)) = U$$
	The reason for this is because for any $u \in U$, $p(f(u)) = u$ so $f(u)
	\in p^{-1}(u)$ so $u \in f^{-1}(p^{-1}(u))$ so $U \subseteq
	f^{-1}(p^{-1}(U))$ and conversly if for some $x \in f^{-1}(p^{-1}(U))$ we
	have $f(x) \in p^{-1}(U)$ then $p(f(x)) \in U \Rightarrow x \in U$ so
	$f^{-1}(p^{-1}(U)) \subseteq U$. Therefore since $f$ is continous, if
	$p^{-1}(U)$ is open, then $f^{-1}(p^{-1}(U))$ is open so $U$ must be open.
	Therefore $p$ is a quotient map.\\
	\\
	(b) \ We know that $r$ is surjective and continous, thus we only have to
	check the converse of continuity.\\
	For any set $S \subseteq A$, if $r^{-1}(S)$ is open, since
	$r(S) = S$ and $r(A - S) \cap S = \emptyset$, we have $r^{-1}(S) \cap A =
	S$ so
	$$r^{-1}(S) = (r^{-1}(S) \cap A) \bigcup (r^{-1}(S) \cap (X - A) ) = S
	\cup R$$
	Where $R = r^{-1}(S) \cap (X - A)$ so $R \cap A = \emptyset$. Therefore we
	have that $f^{-1}(S) = R \cup S$ is open and $f^{-1}(S) \cap A = S$ is open
	in $A$ thus we have the converse of continuity: $f^{-1}(S)$ open
	$\Rightarrow S$ open 
\end{ques}

\begin{ques}{\S 22, 3}
	$q$ is a quotent map as follows. For any basis element $I = (a,b)$ of $\R$
	we have that 
	$$q^{-1}(I) = (I \times R) \cap A$$
	$I \times R$ is open in $\R \times \R$ is open so $q^{-1}(I)$ is open in
	$A$, thus $q$ is continous. We have that the saturated open sets of $q$ are of
	the form $U = (V \times \R) \cap A$ where $V$ is an open set of $\R$. This is
	because any open set $S \subseteq \R \times \R$ can be written as the union
	of a product of open sets:
	$$S = \bigcup M_\a \times V_\a$$
	So we have 
	$$\pi_1^{-1}(\pi_1(S)) = \pi^{-1}_1\l (\bigcup M_a \r ) = \l
	(\bigcup M_a \r) \times \R = V \times \R$$
	For some open set $V$ in $\R$. Thus the open sets which are the preimage of
	their image for $\pi_1$ are as the described form. Saturated sets of the
	restriction of $\pi_1$ to $A$ would therefore be saturated sets intersected
	with $A$. We have that for any open saturated set $U = V \times \R$, $q(U)
	= V$ which is open in $\R$. Thus $q$ maps opens saturated sets to open
	saturated sets, and thus is a quotent map.\\
	\\
	$q$ is not open since the open set $B_1(0,2) \cap A$ gets mapped to $[0,1)$
	which is not open in $\R$. $q$ is not closed since it maps the closed set 
	$$C = \{(x,y) \in A: x \neq 0, y = \frac 1 x\}$$
	To the open set $(0,\infty)$.
\end{ques}

\begin{ques}{\S 22, 4}
	(a) \ We have from the definition of quotient space there exists the quotient map
	$p:X \to X^*$ with $p(x,y) = [x,y]$ where $[x,y]$
	is the equivalence class of $(x,y)$. We also have that the map $g:X \to \R$
	with $g(x,y) = x + y^2$ is constant on $p^{-1}([x,y])$ since $(x,y) \sim (x',y')
	\Leftrightarrow g(x,y) = g(x',y')$. Therefore from Corollary 22.3 if $g$ is
	a quotient map, then $X^*$ is homeomorphic to $\R$.\\
	From Lemma 21.4 we know that $g$ is continous. For
	checking the converse of continuity, we have that $g$ is an open map:\\
	For any basis element of $X$: $I_1 \times I_2 = (a,b) \times (c,d)$ we have that 
	$$g(I_1 \times I_2) = (a + \inf_{y \in I_2}\{ y^2\}, b + \sup_{y \in I_2}\{y^2\})$$
	Which is a open set. Therefore $g$ is a quotient map. \\
	\\
	(b) \ We have from the definition of quotient space there exists the quotient map
	$p:X \to X^*$ with $p(x,y) = [x,y]$ where $[x,y]$
	is the equivalence class of $(x,y)$. We also have that the map $g:X \to
	\R^+ \cup \{0\}$ with $g(x,y) = x^2 + y^2$ is constant on $p^{-1}([x,y])$
	since $(x,y) \sim (x',y') \Leftrightarrow g(x,y) = g(x',y')$. Therefore
	from Corollary 22.3 if $g$ is a quotient map, then $X^*$ is homeomorphic to
	$\R^+ \cup \{0\}R$.\\
	From Lemma 21.4 we know that $g$ is continous. For
	checking the converse of continuity, we have that $g$ is an open map:\\
	For any basis element of $X$: $I_1 \times I_2 = (a,b) \times (c,d)$ we have
	two possible cases, if $0 \notin I_1$ or $\notin I_2$ then 
	$$g(I_1 \times I_2) = (\inf_{x \in (a,b)} \{x^2\} + \inf_{y \in (c,d)}
	\{y^2\}, b + \sup_{y \in I_2}\{y^2\})$$
	Otherwise if $0 \in I_1$ and $0 \in I_2$ then 
	$$g(I_1 \times I_2) = [0, \sup\{a^2, b^2\} + \sup\{c^2, d^2\})$$
	Which is a open set in $R^+ \cup \{0\}$. Therefore $g$ is a quotient map. \\
	
\end{ques}
\end{document}
