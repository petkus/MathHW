\documentclass[12pt]{article}
\usepackage{amsmath, amssymb, amsthm, epsfig}

\newenvironment{definition}{\vspace{2 ex}{\noindent{\bf Definition}}}
        {\vspace{2 ex}}

\newenvironment{ques}[1]{\textbf{Exercise #1}\vspace{1 mm}\\ }{\bigskip}

\renewcommand{\theenumi}{\alph{enumi}}

\theoremstyle{definition}

\newenvironment{Proof}{\noindent {\sc Proof.}}{$\Box$ \vspace{2 ex}}
\newtheorem{Wp}{Writing Problem}
\newtheorem{Ep}{Extra Credit Problem}

\oddsidemargin-1mm
\evensidemargin-0mm
\textwidth6.5in
\topmargin-15mm
\textheight8.75in
\footskip27pt


\renewcommand{\l}{\left }
\renewcommand{\r}{\right }

\newcommand{\R}{\mathbb R}
\newcommand{\Q}{\mathbb Q}
\newcommand{\Z}{\mathbb Z}
\newcommand{\C}{\mathbb C}
\newcommand{\N}{\mathbb N}

\renewcommand{\b}{\setminus}

\newcommand{\id}{\text{id}}

\newcommand{\s}{\sin}
\renewcommand{\c}{\cos}

\renewcommand{\t}{\theta}
\renewcommand{\a}{\alpha}

\newcommand{\norm}[1]{\left\lVert#1\right\rVert}

\newcommand{\T}{\mathcal{T}}

\pagestyle{empty}
\begin{document}

\noindent \textit{\textbf{Math 441, Fall 2017}} \hspace{1.3cm}
\textit{\textbf{HOMEWORK $\#$9}} \hspace{1.3cm} \textit{\textbf{Peter
Gylys-Colwell}} 

\vspace{1cm}


% Do this later with path homotopy instead 
\begin{ques}{1}
	Checking equivalence relation axioms for any paths $f, g, h : [a,b] \to Y$
	from $x$ to $y$ in $Y$:\\
	Reflexivity:\\
	For $f$, we define $H : [a,b] \times I \to Y$ as $H(s,t) = f(s)$. $H$ is
	continous since it is equal to the composition of the continous maps
	$\id_{[a,b]} : {[a,b]} \times I \to {[a,b]}$ and $f: {[a,b]} \to Y$. We
	have that $H$ is a path homotopy from $f$ to $f$ since $H(s,0) = f(s) =
	H(s,1)$, $H(a,t) = f(a) = x, H(b,t) = f(b) = y$ and thus $f \sim f$\\
	Symmetry:\\
	If $f \sim g$ then there exists $H: {[a,b]} \times I \to Y$ where $H(s,0) =
	f(s)$ and $H(s,1) = g(s)$, $H(a,t) = x, H(b,t) = y$. We can define $H':
	{[a,b]} \times I \to Y$ where $H' = H \circ (\id_{[a,b]}, 1 -\id_I)$. $H'$
	is continous since $(\id_{[a,b]}, 1 - \id_I)$ is component-wise continous
	and so continous and thus $H'$ is the composition of continous functions.
	We have that $H'$ is a path homotopy from $g$ to $f$ since $H'(s,0) = H(s,1
	- 0) = g(s)$, $H'(s,1) = H(s,1-1) = f(s)$ and $H'(a,t) = H(a,1-t) = x,
	G'(b,t) = H(b,1-t) = y$. Thus $g \sim f$\\
	Transitivity:\\
	If there exists a path homotopy $H$ from $f$ to $g$ and path homotopy $G$
	from $g$ to $h$ ($f \sim g, g \sim h$) we can define the homotopy
	$F:{[a,b]} \times I \to Y$ using the pasting lemma as follows.\\ Consider
	$H':{[a,b]} \times [0,1/2] \to Y$ as $H' = H \circ (\id_{[a,b]}, \frac 1
	2\id_I)$ and $G': {[a,b]} \times [1/2, 1] \to Y$ as $G' = G \circ
	(\id_{[a,b]}, \frac 1 2 + \frac 1 2 \id_I)$. Both these mappings are
	continous since they are the composition of continous mappings, and their
	domains intersect on $S = {[a,b]} \times \{1/2\}$. We have that $H'(S) =
	G'(S)$ since $H'(s,1/2) = H(s,1) = g(s) = G(s,0) = G'(s,1/2)$. Thus we
	define $F: {[a,b]} \times I \to Y$ using the pasting lemma. $F$ is a
	path homotopy from $f$ to $h$ since $F(s,0) = H'(s,0) = H(s,0) = f(s)$ and
	$F(s,1) = G'(s,1) = G(s,1) = h(s)$. Also $F(a,t) = H(a,1/2 t) = x$ or $=
	G(a,1/2 + 1/2 t) = x$ and $F(b,t) = F(b,t) = H(b,1/2 t) = y$ or $= G(b,1/2
	+ 1/2 t) = y$. Thus $f \sim h$
\end{ques}

\begin{ques}{2}
	If there exists $\t : S^1 \to \R$ such that $p \circ \t = \id_{S^1}$, from
	Exercise $\S 24, 2$ we know there exists $t \in S^1$ such that $\t(t) =
	\t(-t)$. Then we have $p(\t(t)) = p(\t(-t))$ which is a contradiction since
	that implies $t = -t$.
\end{ques}

\begin{ques}{3}
	We have the map $f: [0,2\pi] \to S^1$ with $f(\theta) = (\cos(\t),
	\s(\t))$. We know that $f$ is continous, and $[0, 2\pi] $ is simply
	connected, however $S^1$ is not simply connected. Conversly we have the
	constant map $g: S^1 \to \{0\}$ where $g(s) = 0$. $g$ is continous and
	$\{0\}$ is simply connected, while $S^1$ is not.
\end{ques}

\begin{ques}{4}
	We can assume $S_1$ is the circle with center $\{(0,0)\}$ not in the set
	since shifting and scaling $\R^2$ are homeomorphisms, thus $A$ is
	homeomorphic to such a set. We have the inclusion mapping $i:S^1 \to A$
	which is the the continous identity mapping of $S^1 \subset A$. There exists the
	retraction $\rho: A \to S^1$ with $\rho(x) = \frac x {|x|}$. $\rho$ satisfies
	$\rho \circ i = \id$ since every $x \in S^1$ has norm $1$ so $\frac x {|x|}
	= x$. We have that any loop $f:[a,b] \to S^1$ based at $x \in S^1$ that is
	not null homotopic invokes a loop based at $i(x)$ that is not null
	homotopic in $A$ and thus $A$ is multiply connected since $S^1$ multiply
	connected implies there exists non null homotopic loops in $A$. We get this
	loop in $A$ as $f' = i \circ f$. We have that $f'$ cannot be null homotopic
	since if there existed a homotopy $H': [a,b] \times I \to A$ from $f'$ to
	the constant loop, then we would have the homotopy $H: [a,b] \times I \to
	S^1$ from $f$ to the constant loop defined as $H(s,t) = \rho(H'(s,t))$
	which would be a contradiction. Checking $H$ is the described homotopy:\\
	$H(a,t) = \rho(H'(a,t)) = x = H(b,t) = \rho(H'(b,t)) = H(b,t)$\\
	$H(s,0) = \rho(H'(s,0)) = \rho(i(f(s))) = f(s)$\\
	$H(s,1) = \rho(H'(s,1)) = \rho(i(x)) = x$\\
	And thus we are done
\end{ques}

\begin{ques}{7}
	We already know the direct product of connected spaces is connected, thus
	$X_1 \times \dots \times X_n$ is connected.\\
	For any loop $f: I \to X_1 \times \dots \times X_n$ based at $x = (x_1,
	\dots x_n)$, we will show $f$ is null homotopic and thus $X_1 \times \dots
	\times X_n$ is simply connected. If we consider the compotnents of $f:$
	$f_1, f_2, \dots f_n$, these are loops in $X_1, X_2, \dots X_n$ based at
	$x_1, x_2, \dots x_n$ respectively. Thus since $X_1, X_2, \dots X_n$ are
	simply connected there exists path homotopies $H_1, H_2, \dots H_n : [a,b]
	\times I \to X_1, X_2, \dots X_n$ from $f_1,f_2 \dots f_n$ to the constant
	loops. We have that $H = (H_1, H_2, \dots H_n): $ is a path homotopy from
	$f$ to the constant loop and thus $f$ is null homotopic. Checking $H$ is a
	path homotopy:\\
	$H(a,t) = (H_1(a,t), H_2(a,t), \dots H_n(a,t)) = x = (H_1(b,t), H_2(b,t),
	\dots H_n(b,t)) = H(b,t)$,\\
	$H(s, 0) = (H_1(s,0), H_2(s,0), \dots H_n(s,0)) = (f_1(s),f_2(s), \dots
	f_n(s)) = f(s)$\\
	$H(s, 1) = (H_1(s,1), H_2(s,1), \dots H_n(s,1)) = (x_1,x_2, \dots
	x_n) = x$\\

\end{ques}

\begin{ques}{8}
	We can define the continous map $f:S^{n-1} \times \R^+ \to R^n \b \{0\}$
	with $f(x,r) = rx$. Notice this mapping is the same as the polar coordinate
	representation of $R^n$. $f$ has the continous inverse $f^{-1}(x) =
	(\frac{x}{|x|}, |x|)$ and thus is a homeomorphism. Thus since $S^{n-1}$ and
	$\R^+$ are simply connnected for $n \geq 3$ we know that $\R^n$ is simply
	connected.
\end{ques}

\begin{ques}{9}
	If we have some homeomorphism $f:\R^2 \to \R^n$ for $n \geq 3$, we can
	consider the restriction $f':\R^2\b \{0\} \to \R^n\b \{f(0)\}$. This
	restriction must be a homeomorphism, however this is not possible since
	$\R^2 \b \{0\}$ is not simply connected yet $\R^n \b \{f(0)\} \cong \R^n \b
	\{0\}$ is simply connected as proven in Exercise 8. 
\end{ques}
\end{document}
