\documentclass[12pt]{article}
\usepackage{amsmath, amssymb, amsthm, epsfig}

\newenvironment{definition}{\vspace{2 ex}{\noindent{\bf Definition}}}
        {\vspace{2 ex}}

\newenvironment{ques}[1]{\textbf{Exersise #1}\vspace{1 mm}\\ }{\bigskip}

\renewcommand{\theenumi}{\alph{enumi}}

\theoremstyle{definition}

\newenvironment{Proof}{\noindent {\sc Proof.}}{$\Box$ \vspace{2 ex}}
\newtheorem{Wp}{Writing Problem}
\newtheorem{Ep}{Extra Credit Problem}

\oddsidemargin-1mm
\evensidemargin-0mm
\textwidth6.5in
\topmargin-15mm
\textheight8.75in
\footskip27pt


\renewcommand{\l}{\left }
\renewcommand{\r}{\right }

\newcommand{\R}{\mathbb R}
\newcommand{\Q}{\mathbb Q}
\newcommand{\Z}{\mathbb Z}
\newcommand{\C}{\mathbb C}

\newcommand{\s}{\sin}
\renewcommand{\c}{\cos}

\renewcommand{\t}{\theta}
\renewcommand{\a}{\alpha}

\newcommand{\norm}[1]{\left\lVert#1\right\rVert}

\newcommand{\T}{\mathcal{T}}

\pagestyle{empty}
\begin{document}

\noindent \textit{\textbf{Math 441, Fall 2017}} \hspace{1.3cm}
\textit{\textbf{HOMEWORK $\#$4}} \hspace{1.3cm} \textit{\textbf{Peter
Gylys-Colwell}} 

\vspace{1cm}

\begin{ques}{\S 17, 11}
	If we have the Hausdorff spaces $X, Y$, for each pair $(x,y), (x',y') \in X \times
	Y$, since $X, Y$ are Hausdorff there are neighborhoods $U_1, U_2$ of $x$
	and $x'$ respectivly and $V_1, V_2$ of $y, y'$ respectivly that are
	disjoint. Thus we have $(x, y) \in U_1 \times V_1$ and $(x', y') \in U_2
	\times V_2$ (by definition of the product topology these sets are
	neighborhoods) where $U_1 \times V_1$ and $U_2 \times V_2$ are disjoint.
	Thus $X \times Y$ is Hausdorff.
\end{ques}

\begin{ques}{\S 17, 12}
	If $X$ is Hausdorff and we have a subspace $A \subseteq X$, then for any
	pair of points $x, y \in A$, since $X$ is Hausdorff we have neighborhoods
	$U, V$ of $x, y$ respectivly in $X$. Thus we have that $U \cap A$, $V \cap
	A$ are neighborhoods of $x, y$ in $A$. Since $U$ is disjoint from $V$, we
	know that $U \cap A$ is disjoint from $V \cap A$, and thus $A$ is Hausdorff
\end{ques}

\begin{ques}{\S 17, 13}
	
\end{ques}

\begin{ques}{\S 18, 4}
	
\end{ques}

\begin{ques}{\S 18, 7}
	(a) \ 
	
\end{ques}
\end{document}
