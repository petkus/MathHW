\documentclass[12pt]{article}
\usepackage{amsmath, amssymb, amsthm, epsfig}

\newenvironment{definition}{\vspace{2 ex}{\noindent{\bf Definition}}}
        {\vspace{2 ex}}

\newenvironment{ques}[1]{\textbf{Exersise #1}\vspace{1 mm}\\ }{\bigskip}

\renewcommand{\theenumi}{\alph{enumi}}

\theoremstyle{definition}

\newenvironment{Proof}{\noindent {\sc Proof.}}{$\Box$ \vspace{2 ex}}
\newtheorem{Wp}{Writing Problem}
\newtheorem{Ep}{Extra Credit Problem}

\oddsidemargin-1mm
\evensidemargin-0mm
\textwidth6.5in
\topmargin-15mm
\textheight8.75in
\footskip27pt


\renewcommand{\l}{\left }
\renewcommand{\r}{\right }

\newcommand{\R}{\mathbb R}
\newcommand{\Q}{\mathbb Q}
\newcommand{\Z}{\mathbb Z}
\newcommand{\C}{\mathbb C}

\newcommand{\s}{\sin}
\renewcommand{\c}{\cos}

\renewcommand{\t}{\theta}
\renewcommand{\a}{\alpha}

\newcommand{\norm}[1]{\left\lVert#1\right\rVert}

\newcommand{\T}{\mathcal{T}}

\pagestyle{empty}
\begin{document}

\noindent \textit{\textbf{Math 441, Fall 2017}} \hspace{1.3cm}
\textit{\textbf{HOMEWORK $\#$4}} \hspace{1.3cm} \textit{\textbf{Peter
Gylys-Colwell}} 

\vspace{1cm}

\begin{ques}{\S 17, 11}
	If we have the Hausdorff spaces $X, Y$, for each pair $(x,y), (x',y') \in X \times
	Y$, since $X, Y$ are Hausdorff there are neighborhoods $U_1, U_2$ of $x$
	and $x'$ respectivly and $V_1, V_2$ of $y, y'$ respectivly that are
	disjoint. Thus we have $(x, y) \in U_1 \times V_1$ and $(x', y') \in U_2
	\times V_2$ (by definition of the product topology these sets are
	neighborhoods) where $U_1 \times V_1$ and $U_2 \times V_2$ are disjoint.
	Thus $X \times Y$ is Hausdorff.
\end{ques}

\begin{ques}{\S 17, 12}
	If $X$ is Hausdorff and we have a subspace $A \subseteq X$, then for any
	pair of points $x, y \in A$, since $X$ is Hausdorff we have neighborhoods
	$U, V$ of $x, y$ respectivly in $X$. Thus we have that $U \cap A$, $V \cap
	A$ are neighborhoods of $x, y$ in $A$. Since $U$ is disjoint from $V$, we
	know that $U \cap A$ is disjoint from $V \cap A$, and thus $A$ is Hausdorff
\end{ques}

\begin{ques}{\S 17, 13}
	If $X$ is Hausdorff then for any $(x,y) \in X \times X$ where $x \neq y$ we
	have that there exists open sets $U, V \subset X$ such that $x \in U, y \in
	V$ and $U \cap V = \emptyset$. Since $U \cap V = \emptyset$ we know that
	$(U \times V)\cap \Delta = \emptyset$ since $U$ and $V$ do not have any
	points that are the same. By definition we know $U \times V$ is open in $X
	\times X$ as well. Thus we can take the following arbitrary union where $U_x$ and
	$V_y$ denote the open sets described above for each $(x,y)$
	$$W = \bigcup_{\{(x,y) \in X \times X: x \neq y\}} U_x \times U_y$$
	$W$ is an open set since it is the union of open sets. We also have that $W
	= (X \times X) - \Delta$. It is clear $W \subseteq (X \times X) - \Delta$
	since $W$ is a union of open sets that do not intersect with $\Delta$. We
	also have that $(X \times X) - \Delta \subseteq W$ since for each $(x, y)
	\in X \times X$ there is a $U_x \times V_y$ with $(x,y) \in U_x \times V_y
	\subseteq W$. Thus $\Delta$ is the complement of the open set $W$, so
	$\Delta$ is closed\\
	\\
	Conversly we know that the basis that generates the topology of $X \times
	X$ is all sets of the form $U \times V$ where $U$ and $V$ are open sets of
	$X$. Therefore letting $W = (X \times X) - \Delta$, for any $(x,y) \in W$,
	since $W$ is open we know there exists a basis element $U \times V$
	contained in $W$ which contains $(x,y)$. Thus since $W \cap \Delta =
	\emptyset$ we have that $U \times V \cap  \Delta = \emptyset$ and $U \cap V
	= \emptyset$. Thus $X$ is Hausdorff since for any $x,y$ we have found open
	sets $U, V$such that $x \in U, y \in V$ and $U \cap V = \emptyset$ 

\end{ques}

\begin{ques}{\S 18, 4}
	The case for $g$ is equivalent to the case for $f$ if we just relabel
	the sets $X$ and $Y$, thus this only needs to be proven for $f$.\\
	For any open set $M \subseteq X \times Y$, since the topology of
	$X \times Y$ is generated by the product of open sets of $X$ and $Y$ we
	know that $M = \bigcup \ U_i \times V_i$ where $U_i,V_i$ are open sets of $X, Y$
	respectivly. Thus we have that
	$$f^{-1}(M) = f^{-1}\l(\bigcup U_i \times V_i\r) = \bigcup f^{-1}(U_i \times V_i)$$
	We have that $f^{-1}(U_i \times V_i) = U_i$ if $y_0 \in V_i$ and
	$f^{-1}(U_i \times V_i) = \emptyset$ if $y_0 \notin V_i$. Thus we have
	$$f^{-1}(M) = \bigcup U_i \text{ or } = \emptyset$$
	Thus $f^{-1}(M)$ is open and so $f$ is continous. \\
	It is clear that $f$ is injective since $f(a) = f(b) \Rightarrow
	(a,y_0) = (b, y_0) \Rightarrow a= b$\\
	Finally we must check $f^{-1} : f(X) \to X$ is continous. We have
	$(f^{-1})^{-1} = f$. For open set $U \subseteq X$ we have that $f(U) = U
	\times y_0$. We have that $f(X) = X \times y_0$ and so 
	$$U \times y_0 = (U \times Y) \cap (X \times y_0) = (U \times Y) \cap f(X)$$
	$U \times Y$ is an open set, and thus $(U \times Y) \cap f(X)$ is an open
	set in $f(X)$, so $f^{-1}$ is continous on $f(X)$. Thus $f$ is a homeomorphism
\end{ques}

\begin{ques}{\S 18, 7}
	(a) \ It suffices to show that the inverse image of any basis element is
	open then $f$ is continous. For $\R$ the basis elements can be the open
	intervals $(a,b)$. For any open interval $(a,b)$ we have that
	$$f^{-1}(a,b) = \{x \in \R: a < f(x) < b\}$$
	If $f^{-1}(a,b) = \emptyset$ then $f^{-1}(a,b)$ is open, otherwise, for
	each $p \in f^{-1}(a,b)$, since $\lim_{x \to p^+} f(x) = f(p)$, letting
	$\epsilon = \min\{|f(p)-a|/2, |f(p) - b|/2\} > 0$ we  know there
	exists a $\delta > 0$ such that for all $x \in [p, p + \delta)$, we have
	that $|f(x) - f(p)| < \epsilon$ and thus 
	$$-\epsilon + f(p) < f(x) < f(p) + \epsilon$$
	From how we have chosen $\epsilon$ we have that $a \leq -\epsilon + f(p) <
	\epsilon + f(p) \leq b$. Thus $f(x) \in (a,b)$.\\
	Thus we have that $[p, p+ \delta) \subseteq f^{-1}(a,b)$.\\
	For each point $p \in f^{-1}(a,b)$ we can construct the interval $I_p = [p, p +
	\delta)$ described above. We have that 
	$$f^{-1}(a,b) = \bigcup_{p \in f^{-1}(a,b)} I_p$$
	As we have shown, $I_p \subseteq f^{-1}(a,b)$ for each $p$ so 
	$\bigcup_{p \in f^{-1}(a,b)} I_p \subseteq f^{-1}(a,b)$. We also have for
	every $p \in f^{-1}(a,b)$, $p \in I_p$ so the other direction of
	containment is also true.\\
	In $R_l$ the $I_p$s are open sets, and thus $f^{-1}(a,b)$ is an open set.
	Therefore $f$ is continous.
	
\end{ques}
\end{document}
