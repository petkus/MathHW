%29.09.16
\documentclass[12pt]{article}
\usepackage{amsmath, amssymb, amsthm, epsfig}

\newenvironment{definition}{\vspace{2 ex}{\noindent{\bf Definition}}}
        {\vspace{2 ex}}

%\newenvironment{ques}{\vspace{2 ex}}
	\newenvironment{ques}[1]{\textbf{Exersise #1}\vspace{1 mm}\\ }{\bigskip}

\renewcommand{\theenumi}{\alph{enumi}}

\theoremstyle{definition}

\newenvironment{Proof}{\noindent {\sc Proof.}}{$\Box$ \vspace{2 ex}}
\newtheorem{Wp}{Writing Problem}
\newtheorem{Ep}{Extra Credit Problem}

\oddsidemargin-1mm
\evensidemargin-0mm
\textwidth6.5in
\topmargin-15mm
%\headsep25pt
\textheight8.75in
\footskip27pt

\pagestyle{empty}
\begin{document}

\noindent \textit{\textbf{Math 441, Fall 2017}} \hspace{1.3cm}
\textit{\textbf{HOMEWORK $\#$1}} \hspace{1.3cm} \textit{\textbf{Peter
Gylys-Colwell}} 

\vspace{1cm}

\begin{ques}{\S 1, 9}
	For any $s \in S = A - (B \cup C)$ we have $s \in A$ and $s \notin B$,
	as well as $s \in A$ and $s\notin B$. Therefore $s \in R = (A - B) \cap
	(A - C)$ and so $S \subseteq R$. For any $r \in R$ we have $r \in A -
	B$ as well as $r \in A-C$ so $r$ must be in $A$. Also since $r \in
	A-B$, $r \notin B$ and similarly since $r \in A-C$, $r\notin C$.
	Therefore $R \subseteq S$, and so $R = S$\\ 
	\\
	For the other law we have
	for any $s \in S = A- (B\cap C)$ we have $s \in A$ and $s$ is not in
	both $B$ and $C$. Therefore $s$ must not be in either $B$ or $C$ so $s
	\in A-B$ or $s \in A-C$ whcih means $s \in R = (A-B)\cup (A-C)$.
	Therefore $S \subseteq R$. We also have for any $r \in R$, $r$ is in $A
	- B$ or $A -C$ which means $r \in A$ and $r$ is not in both $B$ and $C$
	which means $r \in S$. Therefore $R \subseteq S$ and so $R = S$

\end{ques}

\begin{ques}{\S 2, 1}
	\begin{enumerate}
		\item
			For any $a \in A_0$, by definition we have $f(a) \in
			f(A_0)$ and therefore 
			$$a \in f^{-1}(f(A_0))$$
			which means $A_0 \subseteq f^{-1}(f(A_0))$. If $f$ is
			injective then if there exists $b \notin A_0$ with $b
			\in A_0 - f^{-1}(f(A_0))$ then $f(b) \in f(A_0)$ which
			means there exists $a \in A_0$ such that $f(b) = f(a)$
			which contradicts injectivity. Therefore $A_0 -
			f^{-1}(f(A_0)) = \emptyset$ and so $A_0 =
			f^{-1}(f(A_0))$

		\item
			For any $b \in B_0$ we have by definition $f(f^{-1}(b))
			\subseteq B_0$ and so $f(f^{-1}(B_0)) \subseteq B_0$.
			If $f$ is surjective then for any $b \in B_0$ there is
			a $a \in A$ such that $f(a) = b$ and therefore $a \in
			f^{-1}(b)$ and so $b \in f(f^{-1}(b)) \subseteq
			f(f^{-1}(B_0))$ and therefore $B_0 \subseteq
			f(f^{-1}(B_0))$. This means that $B_0 = f(f^{-1}(B_0))$

	\end{enumerate}
\end{ques}

\begin{ques}{\S 2, 2}
	\begin{enumerate}
		\item
			Given any $b \in B_0$, since $B_0 \subseteq B_1$ we
			know $b \in B_1$. By the definition of $f^{-1}(B_1)$ we
			have that $f^{-1}(b) \subseteq f^{-1}(B_1)$ since $b
			\in B_1$. And since $f^{-1}(B_0)$ is a unioun of
			these preimages which are contained in  $B_1$, we know
			$f^{-1}(B_0) \subseteq f^{-1}(B_1)$

		\item
			Given any $a \in A$ with $a \in f^{-1}(B_0 \cup B_1)$
			or equivalently $f(a) \in B_0 \cup B_1$ we know that
			$f(a)$ must be in either $B_0$ or $B_1$ and so $a$ is
			in either $f^{-1}(B_0)$ or $f^{-1}(B_1)$. Therefore $a
			\in f^{-1}(B_0) \cup f^{-1}(B_1)$ and so $f^{-1}(B_0
			\cup B_1) \subseteq f^{-1}(B_0) \cup f^{-1}(B_1)$.
			Conversly if $f(a)$ is in $B_0$ or in $B_1$ then $f(a)
			\in B_0 \cup B_1$ and so $f^{-1}(B_0) \cup f^{-1}(B_1)
			\subseteq f^{-1}(B_0 \cup B_1)$. Therefore we
			have equality

		\item
			Given any $a \in A$ with $a \in f^{-1}(B_0 \cap B_1)$
			or equivalently $f(a) \in B_0 \cap B_1$ we know that
			$f(a)$ must be in both $B_0$ and $B_1$ and so $a$ is
			in $f^{-1}(B_0)$ and $f^{-1}(B_1)$. Therefore $a
			\in f^{-1}(B_0) \cap f^{-1}(B_1)$ and so $f^{-1}(B_0
			\cap B_1) \subseteq f^{-1}(B_0) \cap f^{-1}(B_1)$.
			Conversly if $f(a)$ is in $B_0$ and in $B_1$ then $f(a)
			\in B_0 \cap B_1$ and so $f^{-1}(B_0) \cap f^{-1}(B_1)
			\subseteq f^{-1}(B_0 \cap B_1)$. Therefore we
			have equality

		\item
			Given any $a \in A$ with $a \in f^{-1}(B_0 - B_1)$
			or equivalently $f(a) \in B_0 - B_1$ we know that
			$f(a)$ must be in $B_0$ and not $B_1$ and so $a$ is
			in $f^{-1}(B_0)$ and $f^{-1}(B_1)$. Therefore $a
			\in f^{-1}(B_0) - f^{-1}(B_1)$ and so $f^{-1}(B_0
			- B_1) \subseteq f^{-1}(B_0) - f^{-1}(B_1)$.
			Conversly if $f(a)$ is in $B_0$ and not in $B_1$ then $f(a)
			\in B_0 - B_1$ and so $f^{-1}(B_0) - f^{-1}(B_1)
			\subseteq f^{-1}(B_0 - B_1)$. Therefore we
			have equality

		\item
			Given any $b \in f(A_0)$ we know there exists some $a
			\in A$ with $f(a) = b$, since $a \in A_0 \subseteq A_1$
			we have that $a \in A_1$ and so $f(a) \in f(A_1)$.
			Therefore $f(A_0) \subseteq f(A_1)$

		\item
			Given any $b \in f(A_0 \cup A_1)$ we know there exists
			$a \in A_0 \cup A_1$ with $f(a) = b$ and so $a$ is
			either in $A_0$ or $A_1$ so $f(a) \in f(A_0) \cup
			f(A_1)$ and so $f(A_0 \cup A_1) \subseteq f(A_1) \cup
			f(A_0)$.  Conversly for any $f(a) \in f(A_0) \cup
			f(A_1)$ we know $f(a)$ is in either $f(A_0)$ or
			$f(A_1)$ and so $a \in A_0$ or $a \in A_1$ therefore $a
			\in A_0 \cup A_1$ and therefore $f(a) \in f(A_0 \cup
			A_1)$. Therefore we have equality
		\item
			Given any $b \in f(A_0 \cap A_1)$ we know there exists
			$a \in A_0 \cap A_1$ with $f(a) = b$ and therefore
			since $a$ is in $A_0$ and $A_1$, $f(a) \in f(A_0)$ and
			$f(a) \in f(A_1)$ so $f(a) f(A_0) \cap f(A_1)$.
			Therefore $f(A_0 \cap A_1) \subseteq f(A_0) \cap
			f(A_1)$. If $f$ is injective, for any $b \in f(A_0)
			\cap f(A_1)$ we know $b$ is in both $f(A_0)$ and in
			$f(A_1)$. Therefore there exists elements $a_0 \in A_0,
			a_1 \in A_1$ such that $f(a_0) = b \in f(A_0)$ and
			$f(a_1) = b \in f(A_1)$. Since $f$ is injective however
			$a_0 = a_1$ and we know $a = a_0 = a_1$ is in both
			$A_0$ and $A_1$. Therefore $f(a) \in f(A_0 \cap A_1)$,
			and thus $f(A_0) \cap f(A_1) \subseteq f(A_0 \cap A_1)$
			and we have equality of the sets

		\item
			 For any $b \in f(A_0) - f(A_1)$ we know $b$ is in
			 $f(A_0)$ and not $f(A_1)$. Since $b \in f(A_0)$ we
			 know there exists $a \in A_0$ such that $f(a) = b$.
			 Since $f(a) \notin f(A_1)$, we know $a \notin A_1$ and
			 so $a \in A_0 - A_1$, thus $f(a) \in f(A_0 - A_1)$ and
			 thus $f(A_0) - f(A_1) \subseteq f(A_0 - A_1)$. If $f$
			 is injective, given any $b \in f(A_0 - A_1)$ we know
			 there exists $a \in A_0 - A_1$ with $f(a) = b$ and
			 therefore since $a$ is in $A_0$ and not $A_1$, $f(a)
			 \in f(A_0)$. We also have $f(a) \notin f(A_1)$ since
			 if there exists $a_1 \in A_1$ with $f(a_1) = f(a)$
			 then $a = a_1$ since $f$ is injective, but this is not
			 possible since $a \notin A_1$ while $a_1 \in A_1$.
			 Thus $f(a) \in f(A_0) - f(A_1)$.  Therefore $f(A_0 -
			 A_1) \subseteq f(A_0) - f(A_1)$. Thus we have equality
			 of the sets

	\end{enumerate}
\end{ques}

\begin{ques}{\S 2, 4}
	\begin{enumerate}
		\item
			For any $a \in (g \circ f)^{-1}(C_0)$ there must exist a $c \in C_0$ such that $g \circ f (c) = a$

		\item

		\item

		\item

	\end{enumerate}
\end{ques}

\begin{ques}{\S 2, 5}
	\begin{enumerate}
		\item

		\item

		\item

		\item

	\end{enumerate}
\end{ques}

\begin{ques}{\S 3, 4}


\end{ques}

\end{document}
