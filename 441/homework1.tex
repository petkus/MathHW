%29.09.16
\documentclass[12pt]{article}
\usepackage{amsmath, amssymb, amsthm, epsfig}

\newenvironment{definition}{\vspace{2 ex}{\noindent{\bf Definition}}}
        {\vspace{2 ex}}

%\newenvironment{ques}{\vspace{2 ex}}
	\newenvironment{ques}[1]{\textbf{Exersise #1}\vspace{1 mm}\\ }{\bigskip}

\renewcommand{\theenumi}{\alph{enumi}}

\theoremstyle{definition}

\newenvironment{Proof}{\noindent {\sc Proof.}}{$\Box$ \vspace{2 ex}}
\newtheorem{Wp}{Writing Problem}
\newtheorem{Ep}{Extra Credit Problem}

\oddsidemargin-1mm
\evensidemargin-0mm
\textwidth6.5in
\topmargin-15mm
%\headsep25pt
\textheight8.75in
\footskip27pt

\pagestyle{empty}
\begin{document}

\noindent \textit{\textbf{Math 441, Fall 2017}} \hspace{1.3cm}
\textit{\textbf{HOMEWORK $\#$1}} \hspace{1.3cm} \textit{\textbf{Peter
Gylys-Colwell}} 

\vspace{1cm}

\begin{ques}{\S 1, 9}
	For any $s \in S = A - (B \cup C)$ we have $s \in A$ and $s \notin B$,
	as well as $s \in A$ and $s\notin B$. Therefore $s \in R = (A - B) \cap
	(A - C)$ and so $S \subseteq R$. For any $r \in R$ we have $r \in A -
	B$ as well as $r \in A-C$ so $r$ must be in $A$. Also since $r \in
	A-B$, $r \notin B$ and similarly since $r \in A-C$, $r\notin C$.
	Therefore $R \subseteq S$, and so $R = S$\\ 
	\\
	For the other law we have
	for any $s \in S = A- (B\cap C)$ we have $s \in A$ and $s$ is not in
	both $B$ and $C$. Therefore $s$ must not be in either $B$ or $C$ so $s
	\in A-B$ or $s \in A-C$ whcih means $s \in R = (A-B)\cup (A-C)$.
	Therefore $S \subseteq R$. We also have for any $r \in R$, $r$ is in $A
	- B$ or $A -C$ which means $r \in A$ and $r$ is not in both $B$ and $C$
	which means $r \in S$. Therefore $R \subseteq S$ and so $R = S$

\end{ques}

\begin{ques}{\S 2, 1}
	\begin{enumerate}
		\item
			For any $a \in A_0$, by definition we have $f(a) \in f(A_0)$ and therefore 
			$$a \in f^{-1}(f(A_0))$$
			which means $A_0 \subseteq f^{-1}(f(A_0))$. If $f$ is
			injective then if there exists $b \notin A_0$ with $b
			\in A_0 - f^{-1}(f(A_0))$ then $f(b) \in f(A_0)$ which
			means there exists $a \in A_0$ such that $f(b) = f(a)$
			which contradicts injectivity. Therefore $A_0 -
			f^{-1}(f(A_0)) = \emptyset$ and so $A_0 =
			f^{-1}(f(A_0))$

		\item
			For any $b \in B_0$ we have by definition $f(f^{-1}(b))
			\subseteq B_0$ and so $f(f^{-1}(B_0)) \subseteq B_0$.
			If $f$ is surjective then for any $b \in B_0$ there is
			a $a \in A$ such that $f(a) = b$ and therefore $a \in
			f^{-1}(b)$ and so $b \in f(f^{-1}(b)) \subseteq
			f(f^{-1}(B_0))$ and therefore $B_0 \subseteq
			f(f^{-1}(B_0))$. This means that $B_0 = f(f^{-1}(B_0))$

	\end{enumerate}
\end{ques}

\begin{ques}{\S 2, 2}
	\begin{enumerate}
		\item

		\item

		\item

		\item

	\end{enumerate}
\end{ques}

\begin{ques}{\S 2, 4}
	\begin{enumerate}
		\item

		\item

		\item

		\item

	\end{enumerate}
\end{ques}

\begin{ques}{\S 2, 5}
	\begin{enumerate}
		\item

		\item

		\item

		\item

	\end{enumerate}
\end{ques}

\begin{ques}{\S 3, 4}
\end{ques}

\end{document}
