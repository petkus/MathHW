\documentclass[12pt]{article}
\usepackage{amsmath, amssymb, amsthm, epsfig}
\setlength\parindent{0pt}

\newenvironment{definition}{\vspace{2 ex}{\noindent{\bf Definition}}}
        {\vspace{2 ex}}

\newenvironment{ques}[1]{\textbf{Exersise #1}\vspace{1 mm}\\ }{\bigskip}

\renewcommand{\theenumi}{\alph{enumi}}

\theoremstyle{definition}

\newenvironment{Proof}{\noindent {\sc Proof.}}{$\Box$ \vspace{2 ex}}
\newtheorem{Wp}{Writing Problem}
\newtheorem{Ep}{Extra Credit Problem}

\oddsidemargin-1mm
\evensidemargin-0mm
\textwidth6.5in
\topmargin-15mm
\textheight8.75in
\footskip27pt


\renewcommand{\l}{\left }
\renewcommand{\r}{\right }

\newcommand{\R}{\mathbb R}
\newcommand{\Q}{\mathbb Q}
\newcommand{\Z}{\mathbb Z}
\newcommand{\C}{\mathbb C}
\newcommand{\N}{\mathbb N}
\newcommand{\F}{\mathbb F}
\renewcommand{\H}{\mathbb H}
\newcommand{\ndiv}{\hspace{-4pt}\not|\hspace{2pt}}

\newcommand{\tensor}{\otimes}

\newcommand{\s}{\sin}
\renewcommand{\c}{\cos}

\renewcommand{\t}{\theta}
\renewcommand{\a}{\alpha}

\newcommand{\norm}[1]{\left\lVert#1\right\rVert}

\newcommand{\T}{\mathcal{T}}

\newcommand{\Tor}{\text{Tor}}
\newcommand{\Ann}{\text{Ann}}
\newcommand{\End}{\text{End}}

\newcommand{\id}{\text{id}}
\pagestyle{empty}
\begin{document}

\noindent \textit{\textbf{Math 505, WINTER 2018}} \hspace{1.3cm}
\textit{\textbf{HOMEWORK $\#$1}} \hspace{1.3cm} \textit{\textbf{Peter
Gylys-Colwell}} 

\vspace{1cm}

\begin{ques}{2.1}
	We have
	$$x^5 + x^2 - x - 1 = (x - 1)(x + 1)(x^3 + x + 1)$$
	The roots are 
	$$\pm 1, \frac{\a}{3^{2/3}} - \frac{1}{3^{1/3}\a},
	\zeta_3^2\frac{1}{3^{1/3}\a}-\zeta_3\frac{\a}{3^{2/3}},
	\zeta_3\frac{1}{3^{1/3}\a}-\zeta_3^2\frac{\a}{3^{2/3}}$$ 
	where
	$\zeta_3 = \frac 1 2 + \frac {\sqrt 3} {2}$ is the third root of unity and
	$$\a = \sqrt[3]{\frac{\sqrt{93} - 9}{2}}$$
	(I found these values using wolfram alpha for the roots of $x^3 + x + 1$,
	then plugged in the values to verify). From this we get the splitting field
	is $$\Q(\a, \sqrt[3] 3, \zeta_3)$$ Since $\sqrt[3] 3 = 2(\zeta_3 - 1/2) \in
	\Q(\zeta_3)$, $$= \Q(\a, \zeta_3)$$ we have $$[\Q(\a,\zeta_3) : \Q] =
	[\Q(\a,\zeta_3) : \Q(\a)][\Q(\a):\Q]$$ $[\Q(\a): \Q] = 3$ since the degree
	of the irreducible polynomial is 3.  $\zeta_3 \notin \Q(\a)$ since $\Q(\a)
	\subset \R$ while $\zeta_3 \notin \R$, so $[\Q(\zeta_3,\a) : \Q(\a)] \geq
	2$. Over $\Q$ the irreducible polynomial of $\zeta_3$ is $x^2 + x + 1$,
	thus we have $2 = [\Q(\zeta_3) : \Q] \geq [\Q(\zeta_3,\a) : \Q(\a)]$, So
	$[\Q(\zeta_3,\a):\Q(\a)] = 2$.\\ Thus
	$$[\Q(\a,\zeta_3):\Q] = 6$$
\end{ques}

\begin{ques}{2.2}
	For any element $a \in D$, since $D$ is finite dimensional over $k$ (lets
	say of degree $n$) the $n +1$ vectors $1, a, a^2 \dots a^n$ are linearly
	dependent and thus there exists $k_0, k_1, \dots k_n \in k$ where
	$$k_0 + k_1a + \dots k_na^n = 0$$
	Thus $a$ is algebraic over $k$. Since $k$ is algebraically closed, this
	means $a \in k$. Thus $D \subseteq k$ so $D = k$
\end{ques}

\begin{ques}{2.3}
\end{ques}

\begin{ques}{2.4}
	($\Leftarrow$)\ If every irreducible polynomial in $k[x]$ that has root in
	$K$ splits over $K$
	\\
	($\Rightarrow$)\ If $K$ is the splitting field for $f$ and $g$ is any
	irreducible polynomial over $k$ with roots $\a \in K$ and $\beta$, we have
	that $k(\a) \cong k(\beta)$. The isomorphism $\sigma: k(\a) \to k(\beta)$
	fixes $k$, and thus $f = \sigma(f)$. From the theorem proved in
	lecture, letting $K'$ be the splitting field of $f$ over $k(\beta)$, we
	know that this induces an isomorphism of field extentions $$K|k(\alpha)
	\cong K'|k(\beta)$$ Where $K \cong K'$. Since both $K$ and $K'$ are the
	splitting field of $f$ over $k$, they must be equal.  Thus $k(\beta)
	\subset K$ so $\beta \in K$
	
\end{ques}

\begin{ques}{2.5}
	We can consider the group structure of multiplication over the units of
	$\F_p$. By Lagrange's Theorem, for any unit $\a \in \F_p$, $\a^p = \a$,
	thus $\a$ is a root of $x^p - x$. Thus all $p$ elements of $\F_p$
	(including $0$ since $0^p - 0 = 0$) are roots of $x^p - x$. Since $x^p - x$
	can have at most $p$ roots (since polynomials have at most their degree
	number of roots), there can be no multiple roots since it has $p$ distinct
	roots.
\end{ques}

\begin{ques}{2.6}
	($\Rightarrow$) suppose $\a$ is a root of multiplicity $\geq 2$. We have
	that $\a^n = 1$, by Lagranges theorem applied to the group of units with
	multiplication in the prime field of $k$ we know that either $\a = 1$ or $n|p$.\\
	If $\a \neq 1$ and $n|p$ then we have $x-\a$ divides $x^n - 1$ which yields
	$$x^n - 1 = (x - \a)(x^{n-1} + \a x^{n-2} + \a^2x^{n-3} + \dots \a^{n-1})$$
	$\a$ must be a root of the second polynomial, which means
	$$\a^{n-1} + \a^{n-1} + \dots \a^{n-1} = n\a^{n-1} = 0$$
	Since $\a^n = 1, \a^{n-1} = \a^{-1} \neq 0$. Thus it must be the case that
	$n = 0 \Rightarrow p|n$\\
	\\
	If $\a = 1$ then $x - 1$ divides $x^n - 1$ yielding
	$$x^n - 1 = (x - 1)(x^{n-1} + x^{n-2} + x^{n-3} + \dots 1)$$
	In order for $1$ of multiplicity $\geq 2$ it must be a root of 
	$$x^{n-1} + x^{n-2} + x^{n-3} + \dots 1$$
	plugging in $1$ yields a sum of $n$ $1$s. In order for that sum to be zero,
	it must be the case that $p|n$
	\\
	($\Leftarrow$) \ If $p | n$, again we have
	$$x^n - 1 = (x - 1)(x^{n-1} + x^{n-2} + \dots + x + 1)$$
	$1$ is a root of multiplicity $\geq 2$ since $1$ is a root of $x - 1$ and
	a root of $x^{n-1} + \dots + x + 1$. Again this is because plugging in $1$ we get
	a sum of $n$ $1$s and since the characterstic divides $n$, that sum is $0$
\end{ques}
\end{document}
