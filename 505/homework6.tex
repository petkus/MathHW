\documentclass[12pt]{article}
\usepackage{amsmath, amssymb, amsthm, epsfig}
\setlength\parindent{0pt}

\newenvironment{definition}{\vspace{2 ex}{\noindent{\bf Definition}}}
        {\vspace{2 ex}}

\newenvironment{ques}[1]{\textbf{Exersise #1}\vspace{1 mm}\\ }{\bigskip}

\renewcommand{\theenumi}{\alph{enumi}}

\theoremstyle{definition}

\newenvironment{Proof}{\noindent {\sc Proof.}}{$\Box$ \vspace{2 ex}}
\newtheorem{Wp}{Writing Problem}
\newtheorem{Ep}{Extra Credit Problem}

\oddsidemargin-1mm
\evensidemargin-0mm
\textwidth6.5in
\topmargin-15mm
\textheight8.75in
\footskip27pt


\renewcommand{\l}{\left }
\renewcommand{\r}{\right }

\newcommand{\R}{\mathbb R}
\newcommand{\Q}{\mathbb Q}
\newcommand{\Z}{\mathbb Z}
\newcommand{\C}{\mathbb C}
\newcommand{\N}{\mathbb N}
\newcommand{\F}{\mathbb F}
\renewcommand{\H}{\mathbb H}
\newcommand{\ndiv}{\hspace{-4pt}\not|\hspace{2pt}}

\newcommand{\tensor}{\otimes}

\renewcommand{\t}{\theta}
\renewcommand{\a}{\alpha}
\renewcommand{\b}{\beta}

\newcommand{\norm}[1]{\left\lVert#1\right\rVert}

\newcommand{\T}{\mathcal{T}}

\newcommand{\Tor}{\text{Tor}}
\newcommand{\Ann}{\text{Ann}}
\newcommand{\End}{\text{End}}
\newcommand{\Aut}{\text{Aut}}
\newcommand{\Gal}{\text{Gal}}
\newcommand{\orb}{\text{orb}}
\newcommand{\im}{\text{im}}
\newcommand{\Tr}{\text{Tr}}
\newcommand{\s}{\sigma}

\newcommand{\id}{\text{id}}
\pagestyle{empty}
\begin{document}

\noindent \textit{\textbf{Math 505, WINTER 2018}} \hspace{1.3cm}
\textit{\textbf{HOMEWORK $\#$6}} \hspace{1.3cm} \textit{\textbf{Peter
Gylys-Colwell}} 

\vspace{1cm}

\begin{ques}{6.1}
	For $n = p_1^{a_1} \dots p_r^{a_r}$ being the prime factorization of $n$ we
	have the equality
	$$\prod_{i=1}^r \Q(\zeta_{p_i^{a_i}}) = \Q(\zeta_n)$$
	We have '$\subseteq$' by the fact that $\Q(\zeta_n)$ contains each
	$\Q(\zeta_{p_i^{a_i}})$ (since $\zeta_n^{n / p_i^{a_i}}$ is a primitive
	$p_i^{a_i}$ root of unity, $\zeta_n | \zeta_{p_i^{a_i}}$) and thus must
	contain the composite. We have '$\supseteq$' by the fact that the product
	$$\zeta = \zeta_{p_1^{a_1}}\zeta_{p_2^{a_2}} \dots \zeta_{p_r^{a_r}}$$
	is a primitive $n$th root of unity. 
	% The reason for this is because clearly
	% $\zeta^n = 1$. And for $a<n$
	% $$\zeta^a = \zeta_{p_1^{a_1}}^a \zeta_{p_2^{a_2}}^a \dots \zeta_{p_r^{a_r}}^a$$
	\\
	We can use a simple inductive argument on $r$ to show 
	$$\cap_{i = 1}^r
	\Q(\zeta_{p_i^{a_i}}) = \Q$$
	as well as
	$$\Gal(\Q(\zeta_n)/\Q) \cong \times_{i=1}^{r} \Gal(\Q(\zeta_{p_i^{a_i}})/\Q)$$
	For $r = 1$ the statements are trivial since $\zeta_n = \zeta_{p_1^{a_1}}$\\
	\\
	For the inductive step, using the identity for Galois Extensions $K_1/\Q, K_2/\Q$
	$$[K_1K_2 : \Q] = \frac{[K_1:\Q][K_2:\Q]}{[K_1 \cap K_2 : \Q]}$$
	Where $K_1 = \Q(\zeta_{p_1^{a_1}}), K_2 = \Q(\zeta_{p_2^{a_2}})$, $K_1K_2 =
	\Q(\zeta_n)$. Since 
	$$[K_1K_2 : \Q] = \varphi(n) = \varphi(p_1^{a_1})\varphi(p_2^{a_2}) =
	[K_1:\Q][K_2:\Q]$$
	we have $[K_1 \cap K_2 : \Q] = 1 \Rightarrow K_1 \cap
	K_2 = \Q$. we let $K_1 = \prod_{i=1}^r
	\Q(\zeta_{p_i^{a_i}})$ and $K_2 = \q(\zeta_{p_{r+1}^{a_{r+1}}})$ and once
	again we have
	$$[K_1K_2 : \Q] = \varphi(n) = \varphi(p_1^{a_1}) \dots
	\varphi(p_r^{a_r})\varphi(p_{r+1}^{a_{r+1}}) =
	[K_1:\Q][K_2:\Q]$$
	$[K_1 \cap K_2 : \Q] = 1 \Rightarrow K_1 \cap
	K_2 = \Q$. \\
	We can again use a simple inductive argument and the problem 3 statement
	from last week for $k = K_1 \cap K_2$:
	$$\Gal(K_1K_2/k) \cong \Gal(K_1/k) \times \Gal(K_2/k)$$
	to prove
	$$\Gal(\Q(\zeta_n)/\Q) \cong \times_{i=1}^{r} \Gal(\Q(\zeta_{p_i^{a_i}})/\Q)$$
	The base case $r = 1$ is trivially true. For $K_1 = \prod_{i=1}^r
	\Q(\zeta_{p_i^{a_i}})$ and
	% page 598
\end{ques}

\begin{ques}{6.2}
	% Cor 28 pg 600
\end{ques}

\begin{ques}{6.3}
\end{ques}

\begin{ques}{6.4}
	There is a five cycle in $A_5$
	No complex roots
\end{ques}

\begin{ques}{6.5}
	% zeta_29 has group Z/4 times Z/7
\end{ques}

\begin{ques}{6.6}
\end{ques}
\end{document}
