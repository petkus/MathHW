\documentclass[12pt]{article}
\usepackage{amsmath, amssymb, amsthm, epsfig}
\setlength\parindent{0pt}

\newenvironment{definition}{\vspace{2 ex}{\noindent{\bf Definition}}}
        {\vspace{2 ex}}

\newenvironment{ques}[1]{\textbf{Exersise #1}\vspace{1 mm}\\ }{\bigskip}

\renewcommand{\theenumi}{\alph{enumi}}

\theoremstyle{definition}

\newenvironment{Proof}{\noindent {\sc Proof.}}{$\Box$ \vspace{2 ex}}
\newtheorem{Wp}{Writing Problem}
\newtheorem{Ep}{Extra Credit Problem}

\oddsidemargin-1mm
\evensidemargin-0mm
\textwidth6.5in
\topmargin-15mm
\textheight8.75in
\footskip27pt


\renewcommand{\l}{\left }
\renewcommand{\r}{\right }

\newcommand{\R}{\mathbb R}
\newcommand{\Q}{\mathbb Q}
\newcommand{\Z}{\mathbb Z}
\newcommand{\C}{\mathbb C}
\newcommand{\N}{\mathbb N}
\newcommand{\F}{\mathbb F}
\renewcommand{\H}{\mathbb H}
\newcommand{\ndiv}{\hspace{-4pt}\not|\hspace{2pt}}

\newcommand{\tensor}{\otimes}

\newcommand{\s}{\sin}
\renewcommand{\c}{\cos}

\renewcommand{\t}{\theta}
\renewcommand{\a}{\alpha}
\renewcommand{\b}{\beta}

\newcommand{\norm}[1]{\left\lVert#1\right\rVert}

\newcommand{\T}{\mathcal{T}}

\newcommand{\Tor}{\text{Tor}}
\newcommand{\Ann}{\text{Ann}}
\newcommand{\End}{\text{End}}
\newcommand{\Aut}{\text{Aut}}
\newcommand{\Gal}{\text{Gal}}
\newcommand{\orb}{\text{orb}}

\newcommand{\id}{\text{id}}
\pagestyle{empty}
\begin{document}

\noindent \textit{\textbf{Math 505, WINTER 2018}} \hspace{1.3cm}
\textit{\textbf{HOMEWORK $\#$4}} \hspace{1.3cm} \textit{\textbf{Peter
Gylys-Colwell}} 

\vspace{1cm}

\begin{ques}{4.1}
	We have the root $\a = \sqrt[4]{-2} = \zeta_8\sqrt[4]{2}$ and every other root is
	of the form $\zeta_4^n\a$ where $\zeta_4 = i, \zeta_8 = \frac 1 {\sqrt 2} +
	i \frac 1 {\sqrt 2}$ are roots of unity.\\
\end{ques}

\begin{ques}{4.2}
	
\end{ques}

\begin{ques}{4.3}
	
\end{ques}

\begin{ques}{4.4}
	We have that $K = \Q(\zeta_n) \cap \Q(\zeta_m)$ is an extentsion of $\Q$.
	From the correspondences of Galois theory, we know that $\Aut(K/\Q)$ must be the
	a subgroup of $\Gal(\Q(\zeta_m)/\Q)$ as well as $\Gal(\Q(\zeta_n)/\Q)$.
	However we have
	$$\Gal(\Q(\zeta_m)/\Q) \cong C_m, \Gal(\Q(\zeta_n)/\Q) \cong C_n$$
	Every subgroup of $C_m$ is of the form $C_d$ where $d|m$ thus the only
	possible subgroup of $C_m, C_n$ is $C_1$
\end{ques}

\begin{ques}{4.5}
	We can use strong induction, first establishing a base case:\\
	For $n = 1$ we have
	$$\Phi_1(-x) = -x - 1 = -\Phi_2(x)$$
	for $n = 3$:
	$$\Phi_3(-x) = x^2 - x + 1 = \Phi_6(x)$$
	For the inductive step we use the well established identity:
	$$x^n - 1 = \prod_{d | n} \Phi_d(x)$$
	We can reorder the product for $2n$ since each divisor of $n$ must be odd:
	$$x^{2n} - 1 = \prod_{d|2n}\Phi_d = \prod_{d|n}\Phi_d(x) \Phi_{2d}(x) $$
	We also have the factorization $x^{2n} - 1 = (x^n - 1)(x^n + 1)$. Since $n$
	is odd, $x^n + 1 = -((-x)^n - 1)$:
	$$= -(x^n - 1)((-x)^n - 1) = -\prod_{d | n} \Phi_d(x)\prod_{d | n}
	\Phi_d(-x) $$
	So we have
	$$\prod_{d|n}\Phi_d(x) \Phi_{2d}(x) =-\prod_{d | n} \Phi_d(x)\prod_{d | n}
	\Phi_d(-x) $$
	From our inductive hypothesis, for each $d < n, d \neq 1$ we have
	$\Phi_d(-x) = \Phi_{2d}(x)$, thus we can divide on both sides
	$$\Phi_{2n}(x)\Phi_1(x)\prod_{d|n} \Phi_d(x) \prod_{d|n, 1 < d <
	n}\Phi_{d}(-x) =-\prod_{d | n} \Phi_d(x)\prod_{d | n} \Phi_d(-x) $$
	$$\Phi_{2n}(x)\Phi_1(x) = -\Phi_2(-x)\Phi_n(-x)$$
	Since $\Phi_1(x) = -\Phi_2(-x)$ we get our equality
	$$\Phi_{2n}(x) = \Phi_n(-x)$$
\end{ques}

\begin{ques}{4.6}
\end{ques}
\end{document}
