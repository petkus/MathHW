\documentclass[12pt]{article}
\usepackage{amsmath, amssymb, amsthm, epsfig}
\setlength\parindent{0pt}

\newenvironment{definition}{\vspace{2 ex}{\noindent{\bf Definition}}}
        {\vspace{2 ex}}

\newenvironment{ques}[1]{\textbf{Exersise #1}\vspace{1 mm}\\ }{\bigskip}

\renewcommand{\theenumi}{\alph{enumi}}

\theoremstyle{definition}

\newenvironment{Proof}{\noindent {\sc Proof.}}{$\Box$ \vspace{2 ex}}
\newtheorem{Wp}{Writing Problem}
\newtheorem{Ep}{Extra Credit Problem}

\oddsidemargin-1mm
\evensidemargin-0mm
\textwidth6.5in
\topmargin-15mm
\textheight8.75in
\footskip27pt


\renewcommand{\l}{\left }
\renewcommand{\r}{\right }

\newcommand{\R}{\mathbb R}
\newcommand{\Q}{\mathbb Q}
\newcommand{\Z}{\mathbb Z}
\newcommand{\C}{\mathbb C}
\newcommand{\N}{\mathbb N}
\newcommand{\F}{\mathbb F}
\renewcommand{\H}{\mathbb H}
\newcommand{\ndiv}{\hspace{-4pt}\not|\hspace{2pt}}

\newcommand{\tensor}{\otimes}

\newcommand{\s}{\sin}
\renewcommand{\c}{\cos}

\renewcommand{\t}{\theta}
\renewcommand{\a}{\alpha}
\renewcommand{\b}{\beta}

\newcommand{\norm}[1]{\left\lVert#1\right\rVert}

\newcommand{\T}{\mathcal{T}}

\newcommand{\Tor}{\text{Tor}}
\newcommand{\Ann}{\text{Ann}}
\newcommand{\End}{\text{End}}
\newcommand{\Aut}{\text{Aut}}
\newcommand{\Gal}{\text{Gal}}
\newcommand{\orb}{\text{orb}}

\newcommand{\id}{\text{id}}
\pagestyle{empty}
\begin{document}

\noindent \textit{\textbf{Math 505, WINTER 2018}} \hspace{1.3cm}
\textit{\textbf{HOMEWORK $\#$4}} \hspace{1.3cm} \textit{\textbf{Peter
Gylys-Colwell}} 

\vspace{1cm}

\begin{ques}{4.1}
	We have the root $\a = \sqrt[4]{-2} = \zeta_8\sqrt[4]{2}$ and every other root is
	of the form $\zeta_4^n\a$ where $\zeta_4 = i, \zeta_8 = \frac 1 {\sqrt 2} +
	i \frac 1 {\sqrt 2}$ are roots of unity. Thus the splitting field is
	$$\Q(i,\zeta_8, \sqrt[4]{2}) = \Q(i, \sqrt[4]{2})$$
	We get the equality since $\zeta_8 = \frac 1 {\sqrt 2} + i \frac 1 {\sqrt 2} \in
	\Q(i,\sqrt[4]{2})$ \\
	We have that the minimal polynomial of $\sqrt[4]{2}$ (over both $\Q$ and $\Q(i)$) is 
	$$x^4 - 2 = (x - \sqrt[4]{2})(x - i\sqrt[4]{2})(x + \sqrt[4]{2})(x +
	i\sqrt[4]{2})$$
	The reason this is minimal is because it has no roots in $\Q$ and combining
	any two of the linear factors does not yield a polynomial in $\Q[x]$ since
	the constant term will be of the form $\pm\sqrt 2$ or $\pm i\sqrt 2$.\\
	Letting $K$ be the splitting filed we have that any $\varphi \in \Gal(K/\Q)$
	is fully determined by $\varphi(\sqrt[4]{2})$ and $\varphi(i)$. When fixing
	$i$ we have that the group of automorphisms is cyclic.
	$$\Gal(K/ \Q(i)) = C_4$$
	Where we have the generator $g$ Defined by $g(\sqrt[4]{2}) =
	i\sqrt[4]{2}$. Notice that
	$$g(\sqrt[4]{2}) = i\sqrt[4]{2}, g^2(\sqrt[4]{2}) = -\sqrt[4]{2},
	g^3(\sqrt[4]{2}) = -i\sqrt[4]{2}, g^4(\sqrt[4]{2}) = \sqrt[4]{2}$$
	The orbit of $g$ contains all possible automorphisms. Over $\Q$ we can
	permute the roots of $x^2 + 1$. We have the automorphism $f$ sending $i \to
	-i$ while fixing $\sqrt[4]{2}$. Notice we have
	$$f \circ g = g^3 \circ f, f^2 = \id$$
	The first equality comes from evaluating at the roots $f \circ
	g(\sqrt[4]{2}) = f(i\sqrt[4]{2}) = -i\sqrt[4]{2}$, $f \circ g(i) = f(i) =
	-i$. Since $\Gal(K/\Q)$ is the group generated by $f$ and $g$ we can conclude
	$$\Gal(K/\Q) \cong D_8$$
	where $D_8$ is the order $8$ dihedral group 
\end{ques}

\begin{ques}{4.2}
	% We know that $x^3 + x + t$ is irreducible (since it has no roots and its
	% degree $3$) and thus seperable (since irreducible polynomials over a field
	% of characteristic $0$ is seperable).\\
	% Leting $\a_0, a_1, a_2$ be the roots of $x^3 + x + t$. The splitting field is
	% $$K = \C(t)(a_1,a_2,a_3)$$
\end{ques}

\begin{ques}{4.3}
	If there was an infinite number of roots of unity in $K$ then for each $N >
	0$ there must be a $n$ such that $n > N$, and $\zeta_n \in K$ (since there
	are only be finitely many roots of unity of degree less than $N$). Thus we have
	$$[K:\Q] \geq [\zeta_n:\Q] = \varphi(n)$$
	The Euler Phi function is bounded below (this is a commonly known lower bound)
	$$\varphi(n) \geq \frac{\sqrt n }{\sqrt 2}$$
	Thus we would contradict finiteness of $[K:\Q]$ since it is bounded fromm
	below by a number that for large choice of $N$ becomes arbitrarily large.
	$$[K:\Q] \geq \varphi(n) \geq \frac{\sqrt{n}}{\sqrt 2}
	\geq\frac{\sqrt{N}}{\sqrt 2}$$
\end{ques}

\begin{ques}{4.4}
	We have that $K = \Q(\zeta_n) \cap \Q(\zeta_m)$ is an extentsion of $\Q$.
	We have the chain of extentsions 
	$$\Q \subset K \subset \Q(\zeta_m) \subset
	\Q(\zeta_{m n})$$
	$$\Q \subset K \subset \Q(\zeta_n) \subset
	\Q(\zeta_{m n})$$
	This yields
	$$[\Q(\zeta_{n m}): \Q(\zeta_n)][\Q(\zeta_n) : K] = [\Q(\zeta_{n
	m}): K] = [\Q(\zeta_{n m}): \Q(\zeta_m)][\Q(\zeta_m) : K]$$
	Thus if we show it is the case that $[\Q(\zeta_{n m}): \Q(\zeta_n)] =
	\varphi(m)$ and $[\Q(\zeta_{n m}): \Q(\zeta_m)] = \varphi(n)$ then it
	must be the case that $[\Q(\zeta_m) : K] = \varphi(m)$. From the following
	fact this means that $K = \Q$:
	$$[\Q(\zeta_m) : K][K : \Q] = [\Q(\zeta_m):\Q] = \varphi(m) \Rightarrow [K:\Q] = 1$$
	Therefore all we must show is $[\Q(\zeta_{n m}): \Q(\zeta_n)] =
	\varphi(m)$ (since labeling is arbitrary this will imply $[\Q(\zeta_{n
	m}): \Q(\zeta_m)] = \varphi(n)$).\\
	We have that
	$$\varphi(mn) = [\Q(\zeta_{mn}): \Q] = [\Q(\zeta_{mn}):
	\Q(\zeta_m)][\Q(\zeta_m):\Q] = [\Q(\zeta_{mn}):
	\Q(\zeta_m)]\varphi(m)$$
	Since $m$ and $n$ are relatively prime, we have that $\varphi(mn) =
	\varphi(m)\varphi(n)$ and thus dividing by $\varphi(m)$on both sides yields
	the desired result
	$$[\Q(\zeta_{mn}): \Q(\zeta_m)] = \varphi(n)$$
\end{ques}

\begin{ques}{4.5}
	We can use strong induction, first establishing a base case:\\
	For $n = 1$ we have
	$$\Phi_1(-x) = -x - 1 = -\Phi_2(x)$$
	for $n = 3$:
	$$\Phi_3(-x) = x^2 - x + 1 = \Phi_6(x)$$
	For the inductive step we use the well established identity:
	$$x^n - 1 = \prod_{d | n} \Phi_d(x)$$
	We can reorder the product for $2n$ since each divisor of $n$ must be odd:
	$$x^{2n} - 1 = \prod_{d|2n}\Phi_d = \prod_{d|n}\Phi_d(x) \Phi_{2d}(x) $$
	We also have the factorization $x^{2n} - 1 = (x^n - 1)(x^n + 1)$. Since $n$
	is odd, $x^n + 1 = -((-x)^n - 1)$:
	$$= -(x^n - 1)((-x)^n - 1) = -\prod_{d | n} \Phi_d(x)\prod_{d | n}
	\Phi_d(-x) $$
	So we have
	$$\prod_{d|n}\Phi_d(x) \Phi_{2d}(x) =-\prod_{d | n} \Phi_d(x)\prod_{d | n}
	\Phi_d(-x) $$
	From our inductive hypothesis, for each $d < n, d \neq 1$ we have
	$\Phi_d(-x) = \Phi_{2d}(x)$, thus we can divide on both sides
	$$\Phi_{2n}(x)\Phi_1(x)\prod_{d|n} \Phi_d(x) \prod_{d|n, 1 < d <
	n}\Phi_{d}(-x) =-\prod_{d | n} \Phi_d(x)\prod_{d | n} \Phi_d(-x) $$
	$$\Phi_{2n}(x)\Phi_1(x) = -\Phi_2(-x)\Phi_n(-x)$$
	Since $\Phi_1(x) = -\Phi_2(-x)$ we get our equality
	$$\Phi_{2n}(x) = \Phi_n(-x)$$
\end{ques}

\begin{ques}{4.6}
	(a) \ We have
	$$n! = |\Gal(K/\Q)| = [K:\Q]$$
	If $f$ were reducible, then $f$ can be factored as such $f(x) = g(x)h(x)$
	where $\deg(h),\deg(g) \geq 1$.
	Letting $H$ be the splitting field of $h$ over $\Q$ we have 
	$$[K:\Q] = [K:H][H:\Q]$$
	We know that $[H:\Q] \leq \deg(h)!$ and $[K:H] \leq \deg(g)!$. Since
	$n = \deg(h) + \deg(g)$ and $\deg(h),\deg(g) \geq 1$, it is the case
	$$n! > \deg(h)! \deg(g)! \geq [K:H][H:\Q] = n!$$
	Which is a contradiction. Thus $f$ cannot be be factored\\
	\\
	(b)\ Any automorphism $\varphi \in \Aut(\Q(\a)/\Q)$ is fully determined by
	$\varphi(\a)$. If it was the case that $\varphi$ was not the identity, then
	it must be the case $\varphi(\a) = \beta$ where $\beta \neq \a$ is a root of $f$.
	Thus $\beta \in \Q(\a)$. Then it would be the case that letting $h(x) =
	(x-\a)(x-\b) \in \Q(\a)[x]$, that $h(x)|f(x) \Rightarrow f(x) = h(x)g(x)$.
	From this we have the following
	$$n! = [K:\Q] = [K:\Q(\a)][\Q(\a):\Q]$$
	We have that $K/\Q(\a)$ is the splitting field of $g$ and so $[K:\Q(\a)]
	\leq \deg(g)!$ and $[\Q(\a):\Q] = \deg(f) = n$. Since $\deg(g) = \deg(f) -
	2 = n-2$ we are led to the contradiction
	$$n! \leq n \cdot (n-2)!$$
	Thus the only possible automorphism is the identity\\
	\\
	(c)\ If $\a^n = a\in \Q$ then the minimal polynomial of $\a$ would have to be 
	$$x^n - a$$
	So $f(x) = x^n - a$ since $f$ is the minimal polynomial of $\a$.\\
	All other roots of $f$ would be of the form $\a\zeta_n^k$ for some $k$, and
	this will not yield a Galois group isomorphic to $S_n$. The Galois group
	would be the direct product of cycic groups generated by the two automorphisms
	that send $\a \to \a\zeta_n$ and $\zeta_n \to \zeta_n^p$ (where $p$ has
	order $\varphi(n)$ in $(\Z/n)^*$)
\end{ques}
\end{document}
