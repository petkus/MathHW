\documentclass[12pt]{article}
\usepackage{amsmath, amssymb, amsthm, epsfig}
\newenvironment{definition}{\vspace{2 ex}{\noindent{\bf Definition}}}
        {\vspace{2 ex}}

\newenvironment{ques}[1]{\textbf{Exersise #1}\vspace{1 mm}\\ }{\bigskip}

\renewcommand{\theenumi}{\alph{enumi}}

\theoremstyle{definition}

\newenvironment{Proof}{\noindent {\sc Proof.}}{$\Box$ \vspace{2 ex}}
\newtheorem{Wp}{Writing Problem}
\newtheorem{Ep}{Extra Credit Problem}

\oddsidemargin-1mm
\evensidemargin-0mm
\textwidth6.5in
\topmargin-15mm
\textheight8.75in
\footskip27pt


\renewcommand{\l}{\left }
\renewcommand{\r}{\right }

\newcommand{\R}{\mathbb R}
\newcommand{\Q}{\mathbb Q}
\newcommand{\Z}{\mathbb Z}
\newcommand{\C}{\mathbb C}
\newcommand{\N}{\mathbb N}
\newcommand{\F}{\mathbb F}
\renewcommand{\H}{\mathbb H}
\newcommand{\ndiv}{\hspace{-4pt}\not|\hspace{2pt}}

\newcommand{\tensor}{\otimes}

\newcommand{\s}{\sin}
\renewcommand{\c}{\cos}

\renewcommand{\t}{\theta}
\renewcommand{\a}{\alpha}

\newcommand{\norm}[1]{\left\lVert#1\right\rVert}

\newcommand{\T}{\mathcal{T}}

\newcommand{\Tor}{\text{Tor}}
\newcommand{\Ann}{\text{Ann}}
\newcommand{\End}{\text{End}}

\newcommand{\id}{\text{id}}
\pagestyle{empty}
\begin{document}

\noindent \textit{\textbf{Math 505, WINTER 2018}} \hspace{1.3cm}
\textit{\textbf{HOMEWORK $\#$1}} \hspace{1.3cm} \textit{\textbf{Peter
Gylys-Colwell}} 

\vspace{1cm}

\begin{ques}{1.1}
	We have the  isomorphism
	$$\phi : \R[x]/(x^2 + x + 1) \to \R[\zeta_3]$$
	$$x \to \zeta_3$$
	Where $\zeta_3 = 1/2 + \frac{\sqrt 3 }{ 2}i$ is the third root of unity.
	This is an isomorphism since $x^2 + x + 1$ is the minimal polynomial of
	$\zeta_3$ over $\R$.\\
	We have that $\R[\zeta_3] \cong \C$ since by definition $\C = \R[i]$,
	$\zeta_3 \in \C$ so $\R[\zeta_3] \subseteq \C$ and $i = (\zeta_3 -
	1/2)\frac 2 {\sqrt 3}$ so $\C \subseteq \R[\zeta_3]$
\end{ques}

\begin{ques}{1.2}
	Let $\alpha = \sqrt 2 + \sqrt 3$. It is clear $\alpha \in \Q(\sqrt 2, \sqrt
	3)$ so $\Q(\alpha) \subseteq \Q(\sqrt 2, \sqrt 3)$. We have $$\frac{\a ^3 -
	9\a}{2} = \frac{11\sqrt 2 + 9\sqrt 3 - 9(\sqrt 2 + \sqrt 3)}{2} = \sqrt 2$$
	$$\sqrt 3 = \a - \frac{\a ^3 - 9\a}{2}$$ So $\sqrt 2, \sqrt 3 \in \Q(\a)
	\Rightarrow \Q(\sqrt 2, \sqrt 3) \subseteq \Q(\a)$, thus $\Q(\sqrt 2, \sqrt
	3) = \Q(\a)$ so $\a$ is a primitive element
\end{ques}

\begin{ques}{1.3}
	We have the factorization
	$$x^5 + x^2 - x - 1 = (x + 1)(x - 1)(x^2 + x + 1)$$
	Where $x^2 + x + 1$ is irriducible since the roots are $\pm \zeta_3 \notin
	\Q$. Thus either $\a = \pm 1$ which yields a degree 1 extension $\Q[\a] =
	\Q$, or $\a = \pm \zeta_3$ which yields a degree 2 extension since 2 is the
	degree of the minimal polynomial of $\a$: $x^2 + x + 1$
\end{ques}

\begin{ques}{1.4}
	For any choice of $a$, we have that $\a$ is a root of the following
	$$m_0(x) = (x - \a)(x + \a)(x -\frac 1 \a)(x + \frac 1 \a) = (x^2 - \a^2)(x^2
	- \frac 1 {\a^2}) = x^4 - (4a + 2)x^2 + 1 \in \Q$$
	Thus the minimal polynomial must divide $m_0$ (while having $\a$ as a root),
	which yields the possiblities besides $m_0$
	$$m_1(x) = (x - \a)(x \pm \frac 1 \a) = x^2 - (\a \pm \frac 1 \a)x \pm 1$$
	$$m_2(x) = (x - \a)(x + \a) = x^2 - \a^2$$
	$$m_3(x) = x - \a$$
	The minimal polynomial of $\a$ is the smallest degree polynomial of the
	ones listed above with coefficents in $\Q$. Each polynomial is possible. \\
	If $\a \in \Q$ then $m_\a = m_3$. Such is the case when $a = 0$.\\ 
	If $\a^2
	\in \Q$ and conditions were not met above, then $m_\a = m_2$ this is the
	case iff $\sqrt{a^2 + a} \in \Q$ since $\a^2 = 2a + 2\sqrt{a^2 + a} + 1$.
	This is possible for example when $a = 1/3$\\ 
	For $\a - \frac 1 \a$, notice that $1/\a = \sqrt {a
	+ 1} - \sqrt a$ so $\a \pm 1/\a = \sqrt a$ or $\sqrt {a + 1}$. Thus if
	either $a$ or $a + 1$ are squares in $\Q$ and conditions were not met above
	then $m_\a = m_1$.\\ 
	And finally if none of the above were true then $m_\a
	= m_0$
\end{ques}

\begin{ques}{1.5}
	We know that $\a ^2 \in k(\a)$ so $k(\a^2) \subseteq k(\a)$\\
	Since $[k(\a): k]$ is odd, the minimal polynomial over $k,$ $m_\a,$ has odd
	degree ($2n-1$):
	$$m_\a(\a) = \a^{2n -1} + c_{2n-2}\a^{2n-2}+ \dots +c_2\a ^2 +c_1\a +
	c_0 = 0$$ 
	Multiplying by $\a$ on both sides in $K$ yields
	$$\a^{2n} + c_{2n-2}\a^{2n-1}+ \dots +c_2\a ^3 +c_1\a^2 +
	c_0\a = 0$$ 
	Subtracting all odd degree terms:
	$$\a^{2n} + \dots +c_1\a^2 = -c_{2n-2}\a^{2n-1} - \dots - c_2\a^3 - c_0\a$$ 
	Factoring out $\a$ and relabeling constants $k_i = -c_i$:
	$$\a^{2n} + \dots +c_1\a^2 = \a(k_{2n-2}\a^{2n-2} + \dots + k_2\a^2 + k_0)$$
	We have $\a$ in terms of a ratio of polynomials in $\a^2$:
	$$\a = \frac{\a^{2n} + \dots +c_1\a^2}{k_{2n-2}\a^{2n-2} + \dots +
	k_2\a^2 + k_0} \in k(\a^2)$$
	We know that this is well defined, ie $k_{2n-2}\a^{2n-2} + \dots + k_0 \neq 0$ is
	invertable, since it is a non-zero polynomial $f(\a) = k_{2n-2}\a^{2n-2} +
	\dots + k_2\a^2 + k_0$ of degree less than $m_\a$ and therefore cannot be
	zero otherwise we would contradict minimality of $m_\a$. $f$ is nonzero
	since $k_0 = -c_0$ is nonzero since if $c_0 = 0$ then 
	$$x|m_\a(x) = x^{2n -1} + c_{2n-2}x^{2n-2}+ \dots +c_2x ^2 +c_1x$$
	which contradicts $m_\a$ being irriducible.\\ Thus $k(\a) \subseteq
	k(\a^2)$ so $k(\a) = k(\a^2)$
\end{ques}

\begin{ques}{1.6}
	Since $A$ is a subring of $K$ we know $A$ is an integral domain. All we
	must show is that for any $\a \in A$, $\a^{-1} \in A$.\\
	We have that $k[\a] \subseteq A$ where $k[\a]$ is the smallest subring of
	$K$ to contain $k$ and $\a$. It turns out that $k[\a] = k(\a)$ and
	therefore $\a^{-1} \in k(\a) \subseteq A$, so $A$ is a field.\\
	The reason $k[\a] = k(\a)$ is since $\a \in K$ and $K$ algebraic over $k$,
	there is a minimal polynomial for $\a$, $m_\a(x) \in k[x]$. $k[\a]$ must
	contain all linear powers of $\a$ over $k$, with $m_\a(\a) = 0$. From this we
	have the isomorphism $k[\a] \cong k[x]/(m_\a(x))$ (this isomorphism is
	established more rigorously in Dummit and Foote's section of Field Theory)
	which is a field since $(m_\a(x))$ is maximal. Thus $k[\a]$ is a field so
	$k(\a) \subseteq k[\a]$.  Since $k(\a)$ is a ring we know $k[\a] \subseteq
	k(\a)$, so $k[\a] = k(\a)$
\end{ques}
\end{document}
