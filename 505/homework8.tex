\documentclass[12pt]{article}
\usepackage{amsmath, amssymb, amsthm, epsfig}
\setlength\parindent{0pt}

\newenvironment{definition}{\vspace{2 ex}{\noindent{\bf Definition}}}
        {\vspace{2 ex}}

\newenvironment{ques}[1]{\textbf{#1}\vspace{1 mm}\\ }{\bigskip}

\renewcommand{\theenumi}{\alph{enumi}}

\theoremstyle{definition}

\newenvironment{Proof}{\noindent {\sc Proof.}}{$\Box$ \vspace{2 ex}}
\newtheorem{Wp}{Writing Problem}
\newtheorem{Ep}{Extra Credit Problem}

\oddsidemargin-1mm
\evensidemargin-0mm
\textwidth6.5in
\topmargin-15mm
\textheight8.75in
\footskip27pt


\renewcommand{\l}{\left }
\renewcommand{\r}{\right }

\newcommand{\R}{\mathbb R}
\newcommand{\Q}{\mathbb Q}
\newcommand{\Z}{\mathbb Z}
\newcommand{\C}{\mathbb C}
\newcommand{\N}{\mathbb N}
\newcommand{\F}{\mathbb F}
\renewcommand{\H}{\mathbb H}
\newcommand{\ndiv}{\hspace{-4pt}\not|\hspace{2pt}}

\newcommand{\tensor}{\otimes}

\renewcommand{\t}{\theta}
\renewcommand{\a}{\alpha}
\renewcommand{\b}{\beta}
\newcommand{\vp}{\varphi}

\newcommand{\norm}[1]{\left\lVert#1\right\rVert}

\newcommand{\T}{\mathcal{T}}

\newcommand{\Tor}{\text{Tor}}
\newcommand{\Ann}{\text{Ann}}
\newcommand{\End}{\text{End}}
\newcommand{\Aut}{\text{Aut}}
\newcommand{\Gal}{\text{Gal}}
\newcommand{\orb}{\text{orb}}
\newcommand{\im}{\text{im}}
\newcommand{\Tr}{\text{Tr}}
\newcommand{\s}{\sigma}

\newcommand{\id}{\text{id}}
\pagestyle{empty}
\begin{document}

\noindent \textit{\textbf{Math 505, WINTER 2018}} \hspace{1.3cm}
\textit{\textbf{HOMEWORK $\#$8}} \hspace{1.3cm} \textit{\textbf{Peter
Gylys-Colwell}} 

\vspace{1cm}

\begin{ques}{Character Table of $A_4$}
	There are 4 Conjugacy classes of $A_4$:
	$$
	\begin{matrix}
	C_1 = \{1\}, &
	C_2 = \{g(123)g^{-1}\}, &
	C_3 = \{g(124)g^{-1}\}, &
	C_4 = \{g(12)(34)g^{-1}\} &
	\end{matrix}
	$$
	With sizes
	$$
	\begin{matrix}
	|C_1| = 1 &
	|C_2| = 4 &
	|C_3| = 4 &
	|C_4| = 3 &
	\end{matrix}
	$$
	$g \in A_4$. Thus there are 4 irreducible representations $\vp_1, \vp_2, \vp_3,
	\vp_4$ (there is always the trivial 1 dim representation which we will call
	$\vp_1$) \\
	We know that 
	$$12 = |A_4| = \dim_\C \C[A_4] = \dim^2 \vp_1 + \dim^2 \vp_2 + \dim^2 \vp_3
	+ \dim^2 \vp_3$$
	with $\dim^2 \vp_1 = 1$. The only way to have the above equality is with
	$\dim^2 \vp_2 = \dim^2 \vp_3 = 1, \dim^2 \vp_4 = 9$. Thus we have two
	nontrivial 1 dimensional irreducible representations and one 3 dimensional
	irreducible representation\\
	For the 1 dimensional case we know that the Character is equal to the
	representation which is a class function and thus fully defined by the
	image of Conjugacy class representatives
	\\
	We know that $(123), (124)$ has order $3$ so must map to a 3rd root of
	unity and $(12)(34)$ has order $2$ so must map to a 2nd root of unity.
	\\
	From these considerations we arrive at representations
	$$\begin{matrix}
	\vp_2((123)) = \zeta_3 & \vp_2(124) = \zeta_3^2 & \vp_2((12)(34)) = 1
	\end{matrix}$$
	$$
	\begin{matrix}
	\vp_3((123)) = \zeta^2_3 & \vp_3(124) = \zeta_3 & \vp_3((12)(34)) = 1
	\end{matrix}$$
	Using these representation along with the fact for any character $\chi_i(1)
	= \dim \vp_i$ we can arrive at the partially completed character table
	$$
	\begin{array}{c| c c c c}
	 & 1  & (123) & (124) & (12)(34) \\
	\hline
	\chi_1 & 1 & 1 & 1 & 1 \\
	\chi_2 & 1 & \zeta_3 & \zeta_3^2 & 1 \\
	\chi_3 & 1 & \zeta_3^2 & \zeta_3 & 1 \\
	\chi_4 & 3 & x_1 & x_2 & x_3 \\
	\end{array}
	$$
	We can solve for $x_1, x_2, x_3$ by using the orthogonality relations:
	$$1 + \zeta_3 + \zeta_3^2 + 3x_1 = 0 \Rightarrow x_1 = 0$$
	$$1 + \zeta_3 + \zeta_3^2 + 3x_2 = 0 \Rightarrow x_2 = 0$$
	$$1 + 1 + 1 + 3x_3 = 0 \Rightarrow x_3 = -1$$
	and thus complete our Character table
	$$
	\begin{array}{c| c c c c}
	 & 1  & (123) & (124) & (12)(34) \\
	\hline
	\chi_1 & 1 & 1 & 1 & 1 \\
	\chi_2 & 1 & \zeta_3 & \zeta_3^2 & 1 \\
	\chi_3 & 1 & \zeta_3^2 & \zeta_3 & 1 \\
	\chi_4 & 3 & 0 & 0 & -1 \\
	\end{array}
	$$
\end{ques}
\end{document}
