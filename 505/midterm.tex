\documentclass[12pt]{article}
\usepackage{amsmath, amssymb, amsthm, epsfig}
\setlength\parindent{0pt}

\newenvironment{definition}{\vspace{2 ex}{\noindent{\bf Definition}}}
        {\vspace{2 ex}}

\newenvironment{ques}[1]{\textbf{Problem #1}\vspace{1 mm}\\ }{\bigskip}

\renewcommand{\theenumi}{\alph{enumi}}

\theoremstyle{definition}

\newenvironment{Proof}{\noindent {\sc Proof.}}{$\Box$ \vspace{2 ex}}
\newtheorem{Wp}{Writing Problem}

\oddsidemargin-1mm
\evensidemargin-0mm
\textwidth6.5in
\topmargin-15mm
\textheight8.75in
\footskip27pt


\renewcommand{\l}{\left }
\renewcommand{\r}{\right }

\newcommand{\R}{\mathbb R}
\newcommand{\Q}{\mathbb Q}
\newcommand{\Z}{\mathbb Z}
\newcommand{\C}{\mathbb C}
\newcommand{\N}{\mathbb N}
\newcommand{\F}{\mathbb F}
\renewcommand{\H}{\mathbb H}
\newcommand{\ndiv}{\hspace{-4pt}\not|\hspace{2pt}}

\newcommand{\tensor}{\otimes}

\newcommand{\s}{\sin}
\renewcommand{\c}{\cos}

\renewcommand{\t}{\theta}
\renewcommand{\a}{\alpha}
\renewcommand{\b}{\beta}

\newcommand{\norm}[1]{\left\lVert#1\right\rVert}

\newcommand{\T}{\mathcal{T}}

\newcommand{\Tor}{\text{Tor}}
\newcommand{\Ann}{\text{Ann}}
\newcommand{\End}{\text{End}}
\newcommand{\Aut}{\text{Aut}}
\newcommand{\Gal}{\text{Gal}}
\newcommand{\orb}{\text{orb}}

\newcommand{\id}{\text{id}}
\pagestyle{empty}
\begin{document}

\noindent \textit{\textbf{Math 505, WINTER 2018}} \hspace{1.3cm}
\textit{\textbf{MIDTERM}} \hspace{1.3cm} \textit{\textbf{Peter
Gylys-Colwell}} 

\vspace{1cm}

%doneeeeeeeeee
\begin{ques}{1}
	For any $a \in K$ if $a > 0$ then since every positive element of $K$ is a
	square in $K$ $a$ has a square root. If $a < 0$ then $-a$ is positive with
	square root $\sqrt{-a}$. Thus $i\sqrt{-a}$ is the squareroot of $a$ since
	$(i\sqrt{-a})^2 = -1 \cdot -a = a$. Thus we know every element in $K$ has a
	square root\\
	For any $\a = x + iy \in K(i)$ we can write $\a$ in polar form (while the
	notation is analytical, all of the following arguments are algebraic)
	$$\a = re^{i\t} = r(\cos(\t) + i \sin(\theta))$$
	where $r = \sqrt{x^2 + y^2} \in K$ and $e^{i\t} = \cos(\t) + i \sin(\theta) =
	\frac{\a}{r}$.  We have that $\c(\t)$ and $\s(\t) \in K$ since applying the
	conjugate automorphism $i \to -i$
	$$\overline{e^{i\t}} = \cos(\t) - i\sin(\t)$$
	we get the following element is fixed under both automorphisms in
	$\Gal(K(i)/K)$ and thus in $K$
	$$\cos(\t) = \frac{e^{i\t} - \overline{e^{i\t}}}{2} \in K$$
	We have that
	$$\sqrt \a = \sqrt{r}(\cos(\t/2) + i\sin (\t/2)) \in K(i)$$
	Where $\sqrt r \in K$ and for appropriate choice of $n$ (although intuition
	comes from the half angle formula, this is defined purely algebraically)
	$$\cos(\t/2) =-1^n\sqrt{\frac{1 + \cos \t}{2}} \in K$$
	$$\sin (\t/2) = \sqrt{\frac{1 - \cos \t}{2}} \in K$$
	The only thing left to check is that we actually have $\l(\sqrt \a \r)^2 =
	\a$:
	$$\l(\sqrt \a \r)^2 = (\sqrt{r})^2(\cos(\t/2)^2 - \sin(\t/2)^2 +
	2i\cos(\t/2)\sin(\t/2))$$
	$$= (\sqrt{r})^2\l( \frac{1 + \cos \t}{2}- \frac{1 - \cos \t}{2}
	+(-1)^{n}2\sqrt{\frac{1 - \cos \t}{2}} \sqrt{\frac{1 + \cos \t}{2}}i \r)$$
	$$= r(\cos(\t) + (-1)^n\sqrt{1 - \cos^2(\t)}i) = r(\cos(\theta) + i\sin(\t)) = \a$$
	We know that $(-1)^n\sqrt{1 - \cos^2(\t)} = \sin(\t)$ since from how
	$\cos(\theta), \sin(\t)$ are defined
	$$\cos^2(\t) + \sin^2(\t) = \l(\frac{x}{\sqrt{x^2 + y^2}}\r)^2
	+ \l(\frac{y}{\sqrt{x^2 + y^2}}\r)^2 = \frac{x^2 + y^2}{x^2 + y^2} = 1$$
	Thus $\sin^2(\t) = 1 - \cos^{2}(\t) \Rightarrow \sin(\t) = \pm \sqrt{1 -
	\cos(\t)}$
\end{ques}

%done
\begin{ques}{2}
	For any finite extension $E$ of $K(i)$ generated by $\a_1, \a_2, \dots
	\a_n$ we have that $E$ is contained in the splitting field $L$ of the product
	of minimal polynomials $m_1, m_2, \dots m_n$ of the generators. When
	eliminating duplicates (where $m_i = m_j$) in the product, we get a
	seprable polynimial $p$. \\
	$p$ is seperable since if $\a$ was a double root,
	then it must be a root of either some $m_i \neq m_j$ which would contradict
	both $m_i, m_j$ being irreducible since one must divide the other. The
	other possibility is that $\a$ is a double root of $m_i$ but this cannot
	happen since $K$ is characteristic $0$ \\
	$K$ is characteristic $0$ since it has a total ordering so if it were the case
	$$0 = 0 + 1 + 1 + \dots 1$$
	then $0 < 0$ which would contradict our ordering
	\\
	We can now conclude $L$ is the splitting field of
	the seprable polynomial $p$ and is thus Galois. We have that $L$ is a
	finite extension since it is generated by the roots of $m_1, m_2, \dots
	m_n$ which is a finite set.\\
	\\
	If we have the 2-Sylow subgroup $H \subset G = \Gal(L/K)$, then we have
	$|H| = 2^n$ and $|G| = 2^nm$ where $m$ odd. Letting $F = L^H$ be the fixed
	field of $H$ we have
	$$[L:F][F:K] = [L:K] = |G| = 2^nm$$
	By the correspondence of Galois theory we have $|H| = [L:F] = 2^n$ and thus
	the degree of $F$ over $K$ is odd
	$$[F:K] = m$$
\end{ques}

%done
\begin{ques}{3}
	From the Primitive Element Theorem since $F$ is a seperable (seperable
	since $K$ is characteristic $0$) and finite extension of $K$ we know there
	exists $\a \in F$ such that 
	$$F = K(\a)$$
	We have that the degree of the minimal polynimial $d_\a$ satisfies
	$$d_\a = [K(\a) : K] = m$$
	Thus $d_\a$ is odd as we established $m$ to be odd in problem 2. Thus since
	every odd degree polynimial has a root in $K$, in order for $m_\a$ to be
	irriducible it must be linear. Hence 
	$$d_\a = [F : K] = 1$$
	Since $m = 1$ we have established $G$ is a 2-Group:
	$$|G| = 2^n$$
\end{ques}

%done
\begin{ques}{4}
	We have that $\Aut(L/K(i))$ is a subgroup of $\Gal(L/K)$ with the
	corresponding fixed field $K(i)$. As a consequence of the
	Fundamental Theorem of Galois Theory (which states that the size of a
	subgroup is equal to the index of the Galois Extention over the fixed
	field)
	$$[L:K(i)] = |\Aut(L/K(i))|$$
	And thus $L$ is a Galois extension of $K(i)$
\end{ques}

\begin{ques}{5}
	Letting $G_1 = \Gal(L/K(i))$, we have that $G_1$ is a subgroup of $G =
	\Gal(L/K)$. As we proved in problem 3, $|G| = 2^n$ and thus $|G_1| = 2^k$ for some
	$k \leq n$. If $G_1$ is nontrivial it must have a subgroup $H_1$ of size
	$2^{k-1}$ of index 2 follwing from the fact that every group of order $p^n$
	has a subgroup of order $p^{r}$ for all $r < n$. The reason for this is as follows:\\
	We can induct on $n$ where $|G| = p^n$, begining with the trivial base case
	$n = 1$ which has no subgroups except for the trivial group\\
	We have that $G$ has a nontrivial center $Z$ since from the class equations
	$$|G| = |Z| + \sum |G : C_G(g_i)| = p^n$$
	$p | [G:C_G(g_i)]$ so $p| |Z|$ and since $\id \in Z$, $|Z| \geq 1$ so $|Z|
	= p^k$ for some $k$\\
	From our classification of abelian groups we know abelian groups of size
	$p^k$ have subgroups of each order $p^i$ for any $i < k$. We can apply our
	inductive hypothesis to $G/Z$ since
	$$|G/Z| = p^{n - k}$$
	$G/Z$ has groups of all order $p^i$ for $i < (n-k)$ and thus from the
	correspondence theorem we get groups of all orders $p^{i + k}$ in $G$. Thus
	we get subgroups of $p^r$ for all $r < n$ either by having a subgroup of
	$Z$ when $r \leq k$ or from the correspondence of subgroups of the quotient
	group when $r > k$


\end{ques}

%done
\begin{ques}{6}
	Letting $F_1 = L^{H_1}/K(i)$ be the fixed field of $H_1$ from the Fundamental
	Theorem of Galois Theory
	$$[F_1 : K(i)] = |H_1| = 2$$
	This is a contradiction of our conclusion of problem 1 since
	letting $\a$ be a generator of $F_1/K(i)$ we have that the minimal
	polynimial of $\a$ must be quadratic but we have shown in problem 1 every
	quadratic polynimial splits and thus is reducible. Thus $G_1$ must be trivial. Thus
	$$1 = |G_1| = [L:K(i)] \Rightarrow L = K(i)$$
	Thus $K(i)$ is algebraically closed since we have shown any algebraic
	extension of $K(i)$ is degree 1
\end{ques}

%donnnnnnnnnnnnnnnnnnnne
\begin{ques}{7}
	Notice that for $\a \in \C \setminus \R$, $|\a| = 1 \Rightarrow \frac 1 \a
	= \overline \a$ where $\overline{a + bi} = a - bi$ denotes the conjugate.
	The conjugate is an isomorphism of $\C$ which fixes $\R$ and thus
	$\overline f = f$. We thus have that $\frac 1 \a$ is a root of $f$: 
	$$0 = \overline {f(\a)} = f(\overline \a) = f(\frac 1 \a)$$
	If we consider any other root of $f$ $\beta$, since $f$ is irriducible,
	there exists an isomorphism
	$$\varphi :k(\a) \to k(\beta)$$
	which fixes $k$ and maps $\a \to \b$. Thus $\varphi(1/\a) = \varphi(1/\b)$ and so
	$1/\b$ is a root of $f$
	$$0 = \varphi(f(1/\a)) = f(\varphi(1/\a)) = f(1/\b)$$
	Thus $f$ is reciprocal. \\
	$f$ is seperable since it is irriducible in a field
	of Characteristic $0$. Thus the $\deg(f)$ is equal to the number of roots.
	$f$ must be even degree since it has an even number of roots. There is
	an even number of roots since we can pair every root $\a$ with $1/\a$ and
	the only times $\a = \frac 1 \a$ is if $\a = \pm 1$ which would contradict
	$f$ being irreducible over $k$
\end{ques}

%donnnnnnnnnnnnnnnnne
\begin{ques}{8}
	$f$ is irreducible since if $f$ were reducible, then $f$ can be factored as
	such $f(x) = g(x)h(x)$ where $\deg(h),\deg(g) \geq 1$.  Letting $H$ be the
	splitting field of $h$ over $k$ we have 
	$$[K:k] = [K:H][H:k]$$
	We know that $[H:k] \leq \deg(h)!$ and $[K:H] \leq \deg(g)!$\\
	Since
	$n = \deg(h) + \deg(g)$ and $\deg(h),\deg(g) \geq 1$, it is the case
	$$n! > \deg(h)! \deg(g)! \geq [K:H][H:k] = n!$$
	Which is a contradiction. Thus $f$ cannot be be factored\\
	\\
	We have that $f$ is seperable following from the fact that we know the
	degree of a splitting field is bound from above by $\deg(f)!$
	$$n! = |\Aut(K/k)| \leq [K:k] \leq \deg(f)! = n!$$
	$$\Downarrow$$
	$$|\Aut(K/k)| = [K:k]$$
	Thus $K/k$ is Galois which implies that $f$ is seperable (Theorem 13 of
	Dummit and Foote section 14.2).\\
	\\
	If $\a \in K$ was a root of $f$ then any automorphism $\varphi \in
	\Aut(k(\a)/k)$ is fully determined by
	$\varphi(\a)$. If it was the case that $\varphi$ was not the identity, then
	it must be the case $\varphi(\a) = \beta$ where $\beta \neq \a$ is a root of $f$.
	Thus $k \in k(\a)$. Then it would be the case that letting $h(x) =
	(x-\a)(x-\b) \in k(\a)[x]$, that $h(x)|f(x) \Rightarrow f(x) = h(x)g(x)$.
	From this we have the following
	$$n! = [K:k] = [K:k(\a)][k(\a):k]$$
	We have that $K/k(\a)$ is the splitting field of $g$ and so $[K:k(\a)]
	\leq \deg(g)!$ and $[k(\a):k] = \deg(f) = n$. Since $\deg(g) = \deg(f) -
	2 = n-2$ we are led to the contradiction
	$$n! = n \cdot (n-2)!$$
	Thus the only possible automorphism is the identity\\
\end{ques}

\begin{ques}{9}
	Since $F = \overline k ^{\langle \sigma \rangle}$, we have that 
	$$\Aut(\overline k / F) = \langle \sigma \rangle$$
	Notice that for any finite extension $K \supset F$ (with $K \subset
	\overline k$) we have that any automorphism which fixes $F$
	$$\varphi:K/F \to K/F$$
	extends to an automorphism 
	$$\overline\varphi:\overline k/F \to \overline k/F$$
	since $\overline k$ was defined by taking splitting fields of polynomials,
	we can extend $\varphi$ to each splitting field to get our automorphism defined over
	all of $\overline k$\\
	From this fact it follows there is an embedding of groups
	$$\Aut(K/F) \subset \Aut(\overline k/F) = \langle \sigma \rangle$$
	Thus it must be the case that $K$ is cyclic over $F$
\end{ques}

%doneeeeeee
\begin{ques}{10}
	We know that every finite field is of the form $\F_{p^n}$. If $p = 2$ then
	every element is a square since the Frobenius endomorphism is bijective.\\
	If otherwise, we can consider the mapping
	$$s: \F_{p^n} \to \F_{p^n}$$
	$$\a \to \a^2$$
	We have that the polynimial $x^2 - s(\a)$ has at most two roots so only two
	elements can map to the same element in $s$. This fact along with knowing
	no nonzero elements are nilpotent give us a bound on the size of the image
	$$|s\l(\F_{p^n}\r)| \geq \frac{|\F^*_{p^n}|}{2} + |\{0\}| = \frac{p^n-1}{2}
	+ 1$$
	For any $\a \in \F_{p^n}$ we have that the set 
	$$\a - s(\F_{p^n})$$
	is also of size $\frac{p^n-1}{2} + 1$ since the addition mapping is
	one-to-one. Thus we have
	$$|s\l(\F_{p^n}\r)| + |\a - s(\F_{p^n})| \geq p^n + 1 > |\F_{p^n}|$$
	Thus from the pigeon hole principle the sets must intersect. So there
	exists $\b, \gamma \in \F_{p^n}$
	$$\a - \beta^2 \in s(\F^{p^n})$$
	$$\Downarrow$$
	$$\a = \beta^2 + \gamma^2$$
\end{ques}
\end{document}
