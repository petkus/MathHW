\documentclass[12pt]{article}
\usepackage{amsmath, amssymb, amsthm, epsfig}
\setlength\parindent{0pt}

\newenvironment{definition}{\vspace{2 ex}{\noindent{\bf Definition}}}
        {\vspace{2 ex}}

\newenvironment{ques}[1]{\textbf{Exersise #1}\vspace{1 mm}\\ }{\bigskip}

\renewcommand{\theenumi}{\alph{enumi}}

\theoremstyle{definition}

\newenvironment{Proof}{\noindent {\sc Proof.}}{$\Box$ \vspace{2 ex}}
\newtheorem{Wp}{Writing Problem}
\newtheorem{Ep}{Extra Credit Problem}

\oddsidemargin-1mm
\evensidemargin-0mm
\textwidth6.5in
\topmargin-15mm
\textheight8.75in
\footskip27pt


\renewcommand{\l}{\left }
\renewcommand{\r}{\right }

\newcommand{\R}{\mathbb R}
\newcommand{\Q}{\mathbb Q}
\newcommand{\Z}{\mathbb Z}
\newcommand{\C}{\mathbb C}
\newcommand{\N}{\mathbb N}
\newcommand{\F}{\mathbb F}
\renewcommand{\H}{\mathbb H}
\newcommand{\ndiv}{\hspace{-4pt}\not|\hspace{2pt}}

\newcommand{\tensor}{\otimes}

\renewcommand{\t}{\theta}
\renewcommand{\a}{\alpha}
\renewcommand{\b}{\beta}

\newcommand{\norm}[1]{\left\lVert#1\right\rVert}

\newcommand{\T}{\mathcal{T}}

\newcommand{\Tor}{\text{Tor}}
\newcommand{\Ann}{\text{Ann}}
\newcommand{\End}{\text{End}}
\newcommand{\Aut}{\text{Aut}}
\newcommand{\Gal}{\text{Gal}}
\newcommand{\orb}{\text{orb}}
\newcommand{\im}{\text{im}}
\newcommand{\Tr}{\text{Tr}}
\newcommand{\s}{\sigma}

\newcommand{\id}{\text{id}}
\pagestyle{empty}
\begin{document}

\noindent \textit{\textbf{Math 505, WINTER 2018}} \hspace{1.3cm}
\textit{\textbf{HOMEWORK $\#$7}} \hspace{1.3cm} \textit{\textbf{Peter
Gylys-Colwell}} 

\vspace{1cm}

\begin{ques}{7.1}
	Notice that $S_3$ is the group generated by $x = (123), y = (12)$ where $x^3
	= y^2 = (xy)^2$ and thus the answer in problem $6$ yields all the
	representations of $S_3$
\end{ques}

\begin{ques}{7.2}
	We have that $Q_8$ is generated by $i,j$ with $i^4 = j^4 = k^4 = 1$ and $ij = k$.
	For the 1 dimensional case $i, j$ must map to $\{1, \zeta_4, -1,
	-\zeta_4\}$. We have the representations
	$$i \to -1, j \to -1, k \to 1$$
	$$i \to -1, j \to 1, k \to -1$$
	$$i \to 1, j \to -1, k \to -1$$
	We cannot have $i \to \zeta_4$
	since then $i^2 = j^2 \to -1$ so $j \to
	\{\zeta_4, -\zeta_4 \}$ but then we have $ij = -ji$ map to different
	elements: $ij \to -1, -ji \to 1$. This covers all the possiblities of one
	degree representations \\
	For degree 2 we have the irreducible representation
	$$i \to
	\begin{bmatrix}
	\zeta_4 & 0\\
	0 & -\zeta_4
	\end{bmatrix}, 
	j \to 
	\begin{bmatrix}
	0 & 1\\
	-1 & 0
	\end{bmatrix},
	k \to 
	\begin{bmatrix}
	0 & \zeta_4\\
	\zeta_4 & 0
	\end{bmatrix}
	$$
	This is a complete list of representations since the sum
	of the degrees of each of these representations is equal to $8$ which
	matches with the degree of $\C[Q_8]$
\end{ques}

\begin{ques}{7.3}
	Letting $R,F$ be the generators of $D_n$ where $R^n = F^2 = RFRF = 1$ we
	have the irreducible faithful representation
	$$R \to 
	\begin{bmatrix}
	\zeta_n & 0\\
	0 & \zeta_n^{-1}
	\end{bmatrix}$$
	$$F \to 
	\begin{bmatrix}
	0 & 1\\
	1 & 0
	\end{bmatrix}$$
	This representation is irreducible since when viewed as a $\C[G]$ module we
	have that the module is isomorphic to the two by two matricies $M_2(\C)$
	(since the images of $R$ and $F$ generate $M_2(\C)$ as $\C$ algebras)
	and not a direct sum of matrix rings. 
\end{ques}

\begin{ques}{7.4}
	Fixing $g$, we have the representation
	$$T = \rho(g) \in GL_n(\C)$$
	is a linear transformation from $\C^n$ to $\C^n$. We can choose a basis so
	that $T$ is in Jordan canonacal form
	$$T = 
	\begin{bmatrix}
	\lambda_1 & 0 & 0 & \dots & 0\\
	0 & \lambda_2 & 0 &  & 0\\
	0 & 0 & \lambda_3 &  & 0\\
	\vdots &  &  & \ddots & \vdots\\
	0 & 0 & 0 & \dots & \lambda_n
	\end{bmatrix}
	$$
	We have that the minimal polynomial is seperable (and thus the matrix is
	fully diagonalizable) since $T^m = 1$ where $m$ is such that $g^m = 1$, and
	thus the minimal divides the seperable polynomial $x^m - 1$. Each
	$\lambda_i$ is a root of unity and thus $\Tr(\rho) = \Tr (T) = \sum_{i=1}^n
	 \lambda_i$ is a sum of roots of unity
\end{ques}

\begin{ques}{7.5}
	If we have a
	faithful irreducible representation
	$$\rho :\C[G] \to M_n(\C)$$
	We know that $\rho$ must map $Z(G)$ to the center of $M_n(\C)$ since
	$Z(G)$ is in the center of $k[G]$. The center of $M_n(\C)$ is the set of
	diagonal matricies $D_n(\C)$ which is isomorphic to $\C$ as rings. This
	induces the group homomorphism
	$$\rho \circ \pi: Z(G) \to \C^\times$$
	where $\pi : D_n(\C) \to \C$ is the ring isomorphism \\
	Since $G$ is finite each $g \in Z(G)$ has a power $m$ such that $g^m = 1$.
	Thus $g \to \zeta_{m}$ maps to some $m$th root of unity. Thus we have
	$$Z(G) \to \langle \zeta_{m_1}, \zeta_{m_2}, \dots \zeta_{m_n}\rangle
	\subset \C^\times$$
	It is the case that
	$$\langle \zeta_{m_1}, \zeta_{m_2}, \dots \zeta_{m_n}\rangle \subseteq \langle
	\zeta_{m}\rangle \cong \Z/(\varphi(m))$$
	where $m = \gcd(m_1, m_2, \dots m_n)$. Thus
	$$\Z(G) \subseteq \Z/(\varphi(m))$$
	is cyclic
\end{ques}

\begin{ques}{7.6}
	Aside from the trivial representation, there are two $1$ dimensional
	representations of $G$. Since $x^3 = y^2 = 1$ it must be the case that
	$$x \to \{1, \zeta_3, \zeta_3^2 \}, y \to \pm 1$$
	if $x \not \to 1$ then we would have $xy \to \{\zeta_6, \zeta_6^5,
	\zeta_3, \zeta_3^2 \}$ none of which square to $1$, thus $x \to 1$ and we
	have the nontrivial representation $y \to -1$\\
	For the two dimensional case we can choose a basis so that the
	representation of $x$ is in the Jordan Canonical form. If the form was
	$$x \to 
	\begin{bmatrix}
	\lambda & 1\\
	0 & \lambda
	\end{bmatrix}, \text{ then }
	\begin{bmatrix}
	\lambda & 1\\
	0 & \lambda
	\end{bmatrix}^3
	=
	\begin{bmatrix}
	\lambda^3 & 3\lambda^3\\
	0 & \lambda^3
	\end{bmatrix}
	=
	\begin{bmatrix}
	1 & 0\\
	0 & 1
	\end{bmatrix}
	$$
	would mean $3\lambda^3 = 0$ which is not possible. Thus
	$$x \to 
	\begin{bmatrix}
	\lambda_1 & 0\\
	0 & \lambda_2
	\end{bmatrix}
	$$
	Where $\lambda_1,\lambda_2$ have order $3$ in $\C$ and thus
	$\lambda_1,\lambda_2 \in \{1, \zeta_3, \zeta_3^2\}$. Thus we have the options
	$$x \to
	\begin{bmatrix}
	\zeta_3 & 0\\
	0 & \zeta_3
	\end{bmatrix},
	x \to 
	\begin{bmatrix}
	\zeta_3 & 0\\
	0 & \zeta_3^{-1}
	\end{bmatrix}$$
	Since we have $xyxy =
	1$, $y=y^{-1}$ it must be the case $xy \neq x^{-1}y = yx$, we have that the first
	option for $x$ is not possible since the image of $x$ is in the center of
	$M_2(\C)$ and so would commute with the image of $y$ which is not the case in $G$\\
	Thus the only possiblity is the second option, which leads to the image for $y$:
	$$y \to 
	\begin{bmatrix}
	0 & 1\\
	1 & 0
	\end{bmatrix}
	$$
	There can be no larger degree irreducible representations than $2$ since
	$\C[G]$ is a degree $6$ vector space over $\C$ and any $n \times n$ matrix
	ring is degree $\geq 9$ for $n \geq 3$
\end{ques}
\end{document}
