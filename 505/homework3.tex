\documentclass[12pt]{article}
\usepackage{amsmath, amssymb, amsthm, epsfig}
\setlength\parindent{0pt}

\newenvironment{definition}{\vspace{2 ex}{\noindent{\bf Definition}}}
        {\vspace{2 ex}}

\newenvironment{ques}[1]{\textbf{Exersise #1}\vspace{1 mm}\\ }{\bigskip}

\renewcommand{\theenumi}{\alph{enumi}}

\theoremstyle{definition}

\newenvironment{Proof}{\noindent {\sc Proof.}}{$\Box$ \vspace{2 ex}}
\newtheorem{Wp}{Writing Problem}
\newtheorem{Ep}{Extra Credit Problem}

\oddsidemargin-1mm
\evensidemargin-0mm
\textwidth6.5in
\topmargin-15mm
\textheight8.75in
\footskip27pt


\renewcommand{\l}{\left }
\renewcommand{\r}{\right }

\newcommand{\R}{\mathbb R}
\newcommand{\Q}{\mathbb Q}
\newcommand{\Z}{\mathbb Z}
\newcommand{\C}{\mathbb C}
\newcommand{\N}{\mathbb N}
\newcommand{\F}{\mathbb F}
\renewcommand{\H}{\mathbb H}
\newcommand{\ndiv}{\hspace{-4pt}\not|\hspace{2pt}}

\newcommand{\tensor}{\otimes}

\newcommand{\s}{\sin}
\renewcommand{\c}{\cos}

\renewcommand{\t}{\theta}
\renewcommand{\a}{\alpha}
\renewcommand{\b}{\beta}

\newcommand{\norm}[1]{\left\lVert#1\right\rVert}

\newcommand{\T}{\mathcal{T}}

\newcommand{\Tor}{\text{Tor}}
\newcommand{\Ann}{\text{Ann}}
\newcommand{\End}{\text{End}}
\newcommand{\Aut}{\text{Aut}}
\newcommand{\orb}{\text{orb}}

\newcommand{\id}{\text{id}}
\pagestyle{empty}
\begin{document}

\noindent \textit{\textbf{Math 505, WINTER 2018}} \hspace{1.3cm}
\textit{\textbf{HOMEWORK $\#$3}} \hspace{1.3cm} \textit{\textbf{Peter
Gylys-Colwell}} 

\vspace{1cm}

\begin{ques}{3.1}
	The algebraic closure of $\F_p$ is the infinite vectorspace over $\F_p$
	$$K = \bigcup_{n=1}^\infty \F_{p^n}$$
	We know that the algebraic closure must contain the splitting field of
	$x^{p^n} - x$ and thus each $\F_{p^n}$ is contained in the closure.
	Therefore the algebraic closure necessarily contains $K$. \\
	Conversly every algebraic extension of $\F_p(\alpha)/\F_p$ is a finite
	vectorspace over $\F_p$ and thus letting $n = [\F_p(\a):\F_p]$ it is the
	case that $\F(\a) \cong \F_{p^{n}}$ and thus is a subfield of $K$. Thus $K$
	contains all algebraic extensions of $\F_p$ which means $K$ contains the
	algebraic closure.
	% Conversly we know that the algebraic closure must be a vectorspace over
	% $\F_p$ with a countable basis.  We know the basis is countable since the
	% way the closure was contructed was by repeatedly adjoining the roots of the
	% countable set of irriducible polynomials. Thus the basis is a subset of the
	% countable union of countable roots. A countable union of countable sets is
	% countable, so the basis is countable. Any subfield of the closure with a
	% finite number of basis $n$ must be the splitting field of $x^{p^n} - x$ and
	% thus must be a subfield of $K$. Thus the algebraic closure must be
	% contained in $K$ since every algebraic extension of $\F_p$
\end{ques}

\begin{ques}{3.2}
	We have that $[\F_p(\sqrt \a):\F_p] = [\F_p(\sqrt \b):\F_p] = 2$. Therefore
	$|\F_p(\sqrt \a)| = |\F_p(\sqrt \b)| = p^2$. As we have established in
	lecture it is necessarily the case that they are the splitting field of
	$x^{p^2} - x$ over $\F_p$ and are thus isomorphic to $\F_{p^2}$.
\end{ques}

\begin{ques}{3.3}
	
\end{ques}

\begin{ques}{3.4}
	As we have established in lecture, every finite field is of the form
	$\F_{p^n}$ which is the splitting field for the seperable polynomial
	$x^{p^n}-x$ over $\F_p$. Thus since $x^{p^n}-x$ is seperable,
	$\F_{p^n}/\F_p$ is Galois $|\Aut(\F_{p^n}/\F_p)| = n$. Since the orbit of
	$F \in \Aut(\F_{p^n}/\F_p)$ is of size $n$, it must be the case
	$\Aut(\F_{p^n}/\F_p) = \text{orb}(F)$
\end{ques}

\begin{ques}{3.5}
\end{ques}

\begin{ques}{3.6}
\end{ques}
\end{document}
