\documentclass[12pt]{article}
\usepackage{amsmath, amssymb, amsthm, epsfig}
\setlength\parindent{0pt}

\newenvironment{definition}{\vspace{2 ex}{\noindent{\bf Definition}}}
        {\vspace{2 ex}}

\newenvironment{ques}[1]{\textbf{Exersise #1}\vspace{1 mm}\\ }{\bigskip}

\renewcommand{\theenumi}{\alph{enumi}}

\theoremstyle{definition}

\newenvironment{Proof}{\noindent {\sc Proof.}}{$\Box$ \vspace{2 ex}}
\newtheorem{Wp}{Writing Problem}
\newtheorem{Ep}{Extra Credit Problem}

\oddsidemargin-1mm
\evensidemargin-0mm
\textwidth6.5in
\topmargin-15mm
\textheight8.75in
\footskip27pt


\renewcommand{\l}{\left }
\renewcommand{\r}{\right }

\newcommand{\R}{\mathbb R}
\newcommand{\Q}{\mathbb Q}
\newcommand{\Z}{\mathbb Z}
\newcommand{\C}{\mathbb C}
\newcommand{\N}{\mathbb N}
\newcommand{\F}{\mathbb F}
\renewcommand{\H}{\mathbb H}
\newcommand{\ndiv}{\hspace{-4pt}\not|\hspace{2pt}}

\newcommand{\tensor}{\otimes}

\newcommand{\s}{\sin}
\renewcommand{\c}{\cos}

\renewcommand{\t}{\theta}
\renewcommand{\a}{\alpha}
\renewcommand{\b}{\beta}

\newcommand{\norm}[1]{\left\lVert#1\right\rVert}

\newcommand{\T}{\mathcal{T}}

\newcommand{\Tor}{\text{Tor}}
\newcommand{\Ann}{\text{Ann}}
\newcommand{\End}{\text{End}}
\newcommand{\Aut}{\text{Aut}}
\newcommand{\Gal}{\text{Gal}}
\newcommand{\orb}{\text{orb}}

\newcommand{\id}{\text{id}}
\pagestyle{empty}
\begin{document}

\noindent \textit{\textbf{Math 505, WINTER 2018}} \hspace{1.3cm}
\textit{\textbf{HOMEWORK $\#$3}} \hspace{1.3cm} \textit{\textbf{Peter
Gylys-Colwell}} 

\vspace{1cm}

\begin{ques}{3.1}
	The algebraic closure of $\F_p$ is the infinite vectorspace over $\F_p$
	$$K = \bigcup_{n=1}^\infty \F_{p^n}$$
	Where each $\F_{p^n}$ extends $\F_{p^{n-1}}$ as the splitting field of
	$x^{p^n} - x$: $\F_{p} \subset\F_{p^2} \subset \F_{p^3} \dots \subset
	\F_{p^n} \dots $.\\
	We know that the algebraic closure must contain the splitting field of
	$x^{p^n} - x$ and thus each $\F_{p^n}$ is contained in the closure.
	Therefore the algebraic closure necessarily contains $K$. \\
	Conversly every algebraic extension of $\F_p(\alpha)/\F_p$ is a finite
	vectorspace over $\F_p$ and thus letting $n = [\F_p(\a):\F_p]$ it is the
	case that $\F(\a) \cong \F_{p^{n}}$ and thus is a subfield of $K$. Thus $K$
	contains all algebraic extensions of $\F_p$ which means $K$ contains the
	algebraic closure.
\end{ques}

\begin{ques}{3.2}
	We have that $[\F_p(\sqrt \a):\F_p] = [\F_p(\sqrt \b):\F_p] = 2$. Therefore
	$|\F_p(\sqrt \a)| = |\F_p(\sqrt \b)| = p^2$. As we have established in
	lecture it is necessarily the case that they are the splitting field of
	$x^{p^2} - x$ over $\F_p$ and are thus isomorphic to $\F_{p^2}$, so
	isomorphic to each other.
\end{ques}

\begin{ques}{3.3}
	$\F_{p^n}$ is the splitting field of $x^{p^n} - x$, and for any $\a \in
	\F_{p^n}$, 
	$$\a^{p^n} - \a = F^n(\a) - \a = 0 \Rightarrow F^n(\a) = \a \Rightarrow F^n =
	\id_{\F_{p^n}}$$
	Thus ord$(F) |n$.\\
	If it is the case for $d \geq 1$, $F^d = \id_{\F_{p^n}}$, then for all $\a
	\in \F_{p^n}$, $\a^{p^d} = \a$ so $\a^{p^d} - \a = 0$. Since $x^{p^d} - x$
	has exactly $p^d$ roots, in order for every element of $\F_{p^n}$ to be a
	root, it would have to be the case 
	$$p^n = |\F_{p^n}| \leq p^d \Rightarrow n \leq d$$
	Thus $n$ must be the order of $F$
\end{ques}

\begin{ques}{3.4}
	As we have established in lecture, every finite field is of the form
	$\F_{p^n}$ which is the splitting field for the seperable polynomial
	$x^{p^n}-x$ over $\F_p$. Thus since $x^{p^n}-x$ is seperable,
	$\F_{p^n}/\F_p$ is Galois: $|\Aut(\F_{p^n}/\F_p)| = n$. Since the orbit of
	$F \in \Aut(\F_{p^n}/\F_p)$ is of size $n$, it must be the case
	$\Aut(\F_{p^n}/\F_p) = \text{orb}(F)$
\end{ques}

\begin{ques}{3.5}
	In the field $\F_{p^m}$ every element is a root of $x^{p^m} - x$. For $n|m$
	we will show that any element $\a$ in the splitting field of $x^{p^n} - x$
	is a root of $x^{p^m} - x$ thus showing that $\F_{p^n}$ is contianed in the
	splitting field of $x^{p^m} - x$ which is $\F_{p^m}$. Since $n|m$ we have
	that $p^m = p^np^np^n \dots p^n$ so 
	$$\a^{p^m} = \a^{p^np^np^n \dots p^n} = ((\a^{p^n})^{p^n} \dots )^{p^n}$$
	Since $\a^{p^n} = \a$, we get $\a^{p^m} = \a$ and thus is a root of
	$x^{p^m} - x$. Thus $\F_{p^n}$ embeds into $\F_{p^m}$. Since $\F_{p^m}$ is
	still the splitting field of the seperable polynomial $x^{p^m} - x$ over
	$\F_{p^n}$, it is Galois:
	$$[\F_{p^m} : \F_{p^n}] = |\Aut(\F_{p^m} / \F_{p^n})|$$
\end{ques}

\begin{ques}{3.6}
	We have that $F$ is not surjective. There is no $f(t) = \frac{p(t)}{q(t)}
	\in K$ where $F(f) = t$. The reason for this if
	$$t = F(f(t)) = \frac{(p(t))^p}{(q(t))^p}$$
	Then
	$$t(q(t))^p = (p(t))^p \in \F_{p^n}[x]$$
	This cannot be the case however since the degree of $(p(t))^p$ is divisible
	by $p$ while $t(q(t))^p$ is not (it is of the form $kp + 1$)
\end{ques}
\end{document}
