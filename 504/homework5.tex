\documentclass[12pt]{article}
\usepackage{amsmath, amssymb, amsthm, epsfig}

\newenvironment{definition}{\vspace{2 ex}{\noindent{\bf Definition}}}
        {\vspace{2 ex}}

\newenvironment{ques}[1]{\textbf{Exersise #1}\vspace{1 mm}\\ }{\bigskip}

\renewcommand{\theenumi}{\alph{enumi}}

\theoremstyle{definition}

\newenvironment{Proof}{\noindent {\sc Proof.}}{$\Box$ \vspace{2 ex}}
\newtheorem{Wp}{Writing Problem}
\newtheorem{Ep}{Extra Credit Problem}

\oddsidemargin-1mm
\evensidemargin-0mm
\textwidth6.5in
\topmargin-15mm
\textheight8.75in
\footskip27pt


\renewcommand{\l}{\left }
\renewcommand{\r}{\right }

\newcommand{\R}{\mathbb R}
\newcommand{\Q}{\mathbb Q}
\newcommand{\Z}{\mathbb Z}
\newcommand{\C}{\mathbb C}
\renewcommand{\H}{\mathbb H}

\newcommand{\s}{\sin}
\renewcommand{\c}{\cos}

\renewcommand{\t}{\theta}
\renewcommand{\a}{\alpha}

\newcommand{\norm}[1]{\left\lVert#1\right\rVert}

\newcommand{\T}{\mathcal{T}}

\pagestyle{empty}
\begin{document}

\noindent \textit{\textbf{Math 504, Fall 2017}} \hspace{1.3cm}
\textit{\textbf{HOMEWORK $\#$5}} \hspace{1.3cm} \textit{\textbf{Peter
Gylys-Colwell}} 

\vspace{1cm}

\begin{ques}{5.1}
	(1) \ We have that for any of the generators $e_q \in \R[Q_8]$ that $(e_1 +
	e_{-1})e = e_q + e_{-q} = e(e_1 + e_{-1})$ in other words, $a$ commutes
	with every generator in $\R[Q_8]$ and thus commutes with every element in
	$\R[Q_8]$ thus $a\R[Q_8] = \R[Q_8]a$.\\
	We have that $\H \cong \R[Q_8]/a\R[Q_8]$ since we can relabel the cosets of
	$e_1$ as $1$, $e_i$ as $i$, $e_j$ as $j$ and $e_k$ as $k$ to get $\H$
	\\
	(2) \ Let $a = e_{()} + e_{(12)}$ we have that if we apply $a$ to
	$e_{(13)}$ on the left we get 
	$$e_{(13)} + e_{(132)}$$
	However there is no possible elt $b \in \C[S_3]$ where $ba = e_{(13)}a$
	thus $a\C[S_3] \neq \C[S_3]a$. The reason no $b$ exists is because in order
	for $be_{()} + be_{(12)} = ba = e_{(13)} + e_{(132)}$
	$b$ must have either $e_{(13)}$ or
	$e_{(132)}$ as a term so that $be_{()} = e_{(13)}$ or $e_{(132)}$ however 
	$$e_{(13)}a = e_{(13)} + e_{(123)}$$
	and 
	$$e_{(132)}a = e_{(132)} + e_{(23)}$$
	Both of which produce terms that do not show up in $e_{(13)}a$
\end{ques}

\begin{ques}{5.2}
	We have that for any $a,b \in \Z$
	$$\begin{bmatrix} 0 & a \\ 0 & 0 \end{bmatrix}^2 = \begin{bmatrix} 0 & 0 \\
	b & 0 \end{bmatrix}^2 =	\begin{bmatrix} 0 & 0
	\\ 0 & 0 \end{bmatrix}$$
	We know that homomorphisms must map nilpotent elts to nilpotent elts, the
	only nilpotent elt in $\Z$ is $0$ thus
	$$\begin{bmatrix} 0 & a \\ 0 & 0 \end{bmatrix} \to 0, \begin{bmatrix} 0 & 0
	\\ b & 0 \end{bmatrix} \to 0$$
	We have that 
	$$\begin{bmatrix} 0 & 1 \\ 0 & 0 \end{bmatrix}\begin{bmatrix} 0 & 0 \\ 1 &
	0 \end{bmatrix} + \begin{bmatrix} 0 & 0 \\ 1 & 0
	\end{bmatrix}\begin{bmatrix} 0 & 1 \\ 0 & 0 \end{bmatrix}= \begin{bmatrix}
	1 & 0 \\ 0 & 1 \end{bmatrix} $$
	Thus we have that
	$$\begin{bmatrix} 1 & 0 \\ 0 & 1 \end{bmatrix} \to 0$$
	Therefore the only homomorphism is the zero homomorphism.
\end{ques}

\begin{ques}{5.3}
	(1) \ For any $a,b \in R$ we have $a + I, b + I \in R/I$, if
	$$(a + I)(b + I) = ab + I = 0 + I$$
	We have $ab + I = 0 \Leftrightarrow ab \in I$. If $R/I$ is an integral domain then
	either $a + I = 0$ or $b + I = 0 \Rightarrow a \in I$ or $b \in I$ so $I$
	is prime. Conversly if $I$ is prime then $ab + I = 0 \Rightarrow ab \in I
	\Rightarrow a\in I$ or $b \in I$ so $a + I =0$ or $b + I = 0$ thus $R/I$ is
	an integral domain\\
	\\
	(2) \ If $I$ is maximal then if there exists $a + I \in R/I$ where $a
	\notin I$ that is not invertable then we can define a new ideal $M = (a) +
	I$. Since $a$ is not invertable in $R/I$ we have that $1 \notin M$ since if
	$1 \in M$ then $1 = r_1a + r_2s$ where $s \in I$ and thus $r_1 + I$ would be
	the inverse of $a + I$ in $R/I$ which is a contradiction. But then we have
	that $M$ is a proper ideal which contains $I$ and is larger that $I$ since
	$a \in M, a \notin I$ and thus $I$'s maximality is contradicted. Thus every
	nonzero elt of $R/I$ has a multiplicative inverse.\\
	Conversly if $R/I$ is a field yet $I$ is not maximal then there exists an
	ideal $M \neq R, I$ with $I \subset M$ then choose $a \in M$ where $a
	\notin I$. We have that $a + I$ is invertable so there exists $b \in R$
	such that $ab + rs = 1$ where $r \in R, s \in I$. Notice that $a, s \in M$
	and thus $ab + rs \in M$ which means $1 \in M \Rightarrow M = R$ which
	contradicts $M$ proper. Thus $I$ is maximal.\\
	\\
	(3) \ By definition $R_p = S^{-1}R$ where $S = R - P$. From last weeks
	homework we have proven there is a bijective correspondence between prime
	ideals in $R$ not meeting $S$ and $R_P$. From how $S$ is definied prime
	ideals in $R$ not meeting $S$ are prime ideals contained in $P$. Thus we
	have our bijective correspondence.
\end{ques}

\begin{ques}{5.4}
	(1) \ We use one of the standard norms $N(a + ib) = a^2 + b^2$. We have to
	check that $N$ lets the euclidean algorithm work: $\forall a, b \neq 0 \in \C[i]$,
	$\exists d, r \in \C[i]$ such that $a = db + r$ where $N(r) < N(b)$ or $r = 0$. \\
	For any $\alpha, \beta \in \C[i]$, since $\C$ is a field we know $(a + bi)
	= \alpha/\beta \in \C$ where $a,b \in \R$. Thus we have
	$$\alpha = (a + bi) \beta$$
	We can choose integers $x, y$ such that $|a - x| \leq \frac 1 2$ and $|b -
	y| \leq \frac 1 2$. Thus 
	$$\alpha  = (x + iy)\beta + \l ( (a - x) + (b - y)i\r ) \beta$$
	Notice that $\alpha, (x + iy) \beta \in \C[i]$ and thus $((a - x) + (b -
	y)i)\beta \in \C[i]$ since $\C[i]$ is closed under addition.\\
	We have that either $(a - x) + (b - y)i = 0$ or
	$$N(((a - x) + (b - y)i)\beta) = ((a - x)^2 + (b -
	y)^2)N(\beta) \leq \l(\l (\frac 1 2 \r)^2 + \l(\frac 1 2 \r)^2\r)N(\beta)$$
	$$= \frac 1 2 N(\beta) < N(\beta)$$
	Thus letting $\gamma = a + iy$ and $\rho (a - x) + (b - y)i$ we have
	$$\alpha = \gamma \beta + \rho$$
	with $N(\rho) < N(\beta)$ or $\rho = 0$. Thus $\C[i]$ is a Euclidean domain.
	\\
	(2) \ We have
	$$6 = -i \cdot (1 + i)^2 \cdot 3$$
	$i$ is a unit, $1 + i$ is irreducible since $N(1 + i) = 2$ is prime and so
	if $ab = 1 + i$ then $N(ab) = N(a)N(b) = 2 \Rightarrow N(a)$ or $N(b) = 1
	\Rightarrow a$ or $b$ is a unit. $3$ is irreducible since $N(3) = 9$ so
	if $ab = 3$ then $N(ab) = N(a)N(b) = 9$, if $N(a)$ or $N(b) = 1$ then $a$
	or $b$ is a unit, otherwise if $N(a) = N(b) = 3$ then $\exists x, y \in
	\Z$ where $a = x + iy$ and $x^2 + y^2 = 3$, a simple check of possible
	numbers less than $3$ shows there is no solutions and thus this is not
	possible.
\end{ques}

\begin{ques}{5.5}
	(1) \ We have 
	$$9 = 3^2 = (2 + \sqrt{-5})(2 - \sqrt{-5})$$
	We now have to show $3, 2 + \sqrt{-5}$ and $2 - \sqrt{-5}$ are irreducible
	and thus this is two different factoraizations.\\
	We will use the traditional norm defined over the complex numbers $|x + iy|
	= x^2 + y^2$ which satisfy all the axioms of norms. First we have in
	$\Z[\sqrt{-5}]$ that $|x + y\sqrt{-5}| = 1 \Rightarrow x + y\sqrt{-5} = \pm
	1$. The reason is because $1 = |x + y\sqrt{-5}| = x^2 + 5y^2$, and since
	$x, y \in \Z$ we know $y = 0, x = \pm 1$. Second we know that there exists
	no $x + y\sqrt{-5}$ with $|x + y\sqrt{-5}| = 3$ since $x^2 + 5y^2 > 3$ if $y \neq
	0$ and other wise we have $x^2 + 5y^2 = x^2 + 0 \neq 3$ for all $x \in \Z$
	since $3$ is not a square in $\Z$\\
	Therefore we have that if $3 = \alpha \beta$ for $\alpha, \beta \in
	\Z[\sqrt{-5}]$ then 
	$$9 = |3| = |\alpha| |\beta|$$
	So since no norms can be 3, one of the norms must be $1$ and thus a unit.
	The argument is the same for $2 + \sqrt{-5}, 2 - \sqrt{-5}$:
	$$|2 + \sqrt{-5}| = |2 - \sqrt{-5}| = 9$$
	and thus any product equal to either of these terms must be the product of
	a unit and an elt of norm $9$.
	\\
	(2) \ Let $I = \langle 3, 2 + \sqrt{-5} \rangle$. If $I = \langle \lambda
	\rangle$ for some $\lambda \in \Z[\sqrt {-5}]$ then $3 = x\lambda$
	and $2 + \sqrt{-5} = y\lambda$ for some $x,y
	\in \Z[\sqrt{-5}]$.\\
	However we have already shown that $3$ and $2 + \sqrt{-5}$ are irreducible.
	Thus $\lambda$ is either a unit or $x, y$ is a unit. The only units are $1,
	-1$ so if lambda is not a unit then $3= \pm (2 + \sqrt{-5})$ which is
	obviously not the case.
	Hence $\lambda$ must be a unit so $\langle\lambda\rangle = \Z[\sqrt{-5}]
	\Rightarrow 1 \in I$. However this is not possible. If we have
	$$3\alpha + (2 + \sqrt{-5})\beta = 1$$
	Then multiplying by $2 - \sqrt{-5}$ yields
	$$3\alpha(2 - \sqrt{-5}) + 9 = 2 - \sqrt{-5}$$
	Which means $3$ divides $2 - \sqrt{-5}$, but $2 - \sqrt{-5}$ is irreducible
	and $3, -3$ are the only possible values of $3$ multiplied with a unit so
	this is impossible
\end{ques}

\begin{ques}{5.6}
	We have that the number of elements in the qotient is 8:
	$$R = \Z[x]/(2, x^3 + 1) \cong \Z_2[x]/(x^3 + 1) = \{x^2, x^2 + x, x^2 + x + 1, x^2 +
	1, x + 1, x, 1, 0\}$$
	Every ideal of $R$ is generated be a combination of these elements. However
	since $R$ is finite, all non-nilpotent elements of $R$ are units and thus
	generate all of $R$. We have that the only nilpotent elements are $(x + 1),
	(x^2 +x + 1), (x + 1)^2 = x^2 + 1,(x + 1)^3 = x^2 + x$ since $x^3 + 1 = (x
	+ 1)(x^2 - x + 1)$. 
	We know that these are the only nilpotents since $\Z_2[x]$ is a UFD so we know
	that any elt that is nilpotent must share
	an irreducible divisor with $(x^3 + 1)$ in $\Z_2[x]$ in order to divide a
	multiple of $(x^3 + 1)$ by a factor not divisible by $(x^3 + 1)$.\\
	Thus we have the proper Ideals
	$$(x + 1), (x^2 + x + 1), ((x + 1)^2), ((x + 1)^3), (x + 1, x^2 + x + 1),
	((x + 1)^2, x^2 + x + 1), ((x + 1)^3, x^2 + x + 1)$$

\end{ques}

\end{document}
