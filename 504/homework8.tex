\documentclass[12pt]{article}
\usepackage{amsmath, amssymb, amsthm, epsfig}
\newenvironment{definition}{\vspace{2 ex}{\noindent{\bf Definition}}}
        {\vspace{2 ex}}

\newenvironment{ques}[1]{\textbf{Exersise #1}\vspace{1 mm}\\ }{\bigskip}

\renewcommand{\theenumi}{\alph{enumi}}

\theoremstyle{definition}

\newenvironment{Proof}{\noindent {\sc Proof.}}{$\Box$ \vspace{2 ex}}
\newtheorem{Wp}{Writing Problem}
\newtheorem{Ep}{Extra Credit Problem}

\oddsidemargin-1mm
\evensidemargin-0mm
\textwidth6.5in
\topmargin-15mm
\textheight8.75in
\footskip27pt


\renewcommand{\l}{\left }
\renewcommand{\r}{\right }

\newcommand{\R}{\mathbb R}
\newcommand{\Q}{\mathbb Q}
\newcommand{\Z}{\mathbb Z}
\newcommand{\C}{\mathbb C}
\newcommand{\N}{\mathbb N}
\renewcommand{\H}{\mathbb H}

\newcommand{\s}{\sin}
\renewcommand{\c}{\cos}

\renewcommand{\t}{\theta}
\renewcommand{\a}{\alpha}

\newcommand{\norm}[1]{\left\lVert#1\right\rVert}

\newcommand{\T}{\mathcal{T}}

\newcommand{\Tor}{\text{Tor}}
\newcommand{\Ann}{\text{Ann}}

\newcommand{\id}{\text{id}}
\pagestyle{empty}
\begin{document}

\noindent \textit{\textbf{Math 504, Fall 2017}} \hspace{1.3cm}
\textit{\textbf{HOMEWORK $\#$8}} \hspace{1.3cm} \textit{\textbf{Peter
Gylys-Colwell}} 

\vspace{1cm}

\begin{ques}{8.1}
	Notice that the set of left proper ideals of $R$ form a partialy ordered
	set $P$ with inclusion as the ordering relation ($K \leq J \Leftrightarrow
	K \subseteq J)$. We know that $P$ is not empty since $I \in P$.\\
	If we show that every chain in $P$ has an upper bound in $P$ then by
	Zorn's Lemma $P$ has a maximal element (which is a maximal ideal).
	Considering any chain of proper ideals.
	$$I_1 \subseteq I_2 \subseteq I_3 \subseteq \dots $$
	we have
	$$U = \bigcup_{i = 1}^\infty I_i \in P$$
	$U$ is an ideal since for any $x,y \in U, r \in R$, there exists $I_n$ such that $x,
	y \in I_n$ then $x + y \in I_n \subseteq U, rx \in I_n \subseteq U$. $U$ is proper
	since $1 \notin I_i \forall i$ so $1 \notin U$. We have that 
	$$I_1 \subseteq I_2 \subseteq I_3 \subseteq \dots \subseteq U$$
	So every chain is bounded. Thus we are done.
\end{ques}

\begin{ques}{8.2}
	(a) \ For any $a \in D$, if we consider the set $1, a, a^2, a^3, \dots
	a^{n}$ where $n$ is the dimension of $D$ over $k$. We have $n+1$ elements
	and thus they are linearly dependent. So there exists a nonzero polinomial 
	$$f(a) = k_{n}a^{n} + k_{n - 1}a^{n - 1} + \dots k_1a + k_0 = 0$$
	(b) \ $D = k$ since for any $a \in D$ and $f \in k[x]$, $f(a) = 0$ we can
	factor $f$ completely since $k$ is completely
	$$f(a) = (a - a_n)(a - a_{n-1}) \dots (a - a_0)$$
	Where $a_n, a_{n-1}, \dots a_0 \in k$. Since $D$ is a domain we know $a =
	a_i$ for one of the $a_i$ and thus $a \in k$. Thus $D \subseteq k$. We
	already know $k \subseteq D$ since there is an embedding from $k$ to $D$.
\end{ques}

\begin{ques}{8.3}
	If $M$ is some $k[G]$ module with submodule $N \subset M$, we have the
	surjective homomorphism $\pi : M \to M/N$. Since $k$ is a subring of $k[G]$
	$M$ and $M/N$ are $k$ vectorspaces. We have a $k$ linear section $s :
	M/N \to M$ such that $\pi \circ s =$ id. The reason for this is because
	$M/N$ has a basis $B$ as a vectorspace so for each $b \in B$ there is some
	$m \in M$ with $\pi(m) = b$ then we define $s(b) = m$. $s$ is a fully
	defined $k$ linear map from where it sends its basis. We have that
	$$s'(x) = \frac{1}{|G|} \sum_{g \in G}e_gs(e_{g^{-1}}x)$$
	is a $k[G]$ module homomorphism. Checking the properties:\\
	$s'(0) = \frac{1}{|G|} \sum_{g \in G}e_gs(0) = 0$\\
	$s'(x + y) = s'(x) + s'(y)$, we can use the fact that $s(x + y) = s(x) + s(y)$
	$$s'(x + y) = \frac{1}{|G|} \sum_{g \in G}e_gs(e_{g^{-1}}(x+y)) =
	\frac{1}{|G|} \sum_{g \in G}e_gs(e_{g^{-1}}x)+ e_gs(e_{g^{-1}}y)) = s'(x) +
	s'(y)$$
	For $s'(rx) = rs'(x)$ for $r \in k[G]$ we have that $r = e_{g_1}k_1 +
	e_{g_2}k_2 + \dots e_{g_n}k_n$ so 
	$$s'(rx) = s'(e_{g_1}k_1x + e_{g_2}k_2x + \dots e_{g_n}k_nx) =
	k_1s'(e_{g_1}x) + k_2s'(e_{g_2}x) + \dots k_ns'(e_{g_n}x)$$
	We know $s'$ is k linear since 
	$$s'(kx) = \frac{1}{|G|} \sum_{g \in G}e_gs(e_{g^{-1}}kx) = \frac{1}{|G|}
	\sum_{g \in G}e_gks(e_{g^{-1}}x) = ks'(x)$$
	Thus we must only check that $s'(e_hx) = e_hs'(x)$. \\
	$$s'(e_hx) = \frac{1}{|G|} \sum_{g \in G}e_gs(e_{g^{-1}h}x) $$
	We can relabel $z = h^{-1}g$ and $z^{-1} = g^{-1}h$. Since $h^{-1}G = G$ we
	have the same sum
	$$= \frac{1}{|G|} \sum_{z \in G}e_{hz}s(e_{z^{-1}}x) = \frac{e_h}{|G|}
	\sum_{z \in G}e_{z}s(e_{z^{-1}}x) = e_hs'(x)$$
	Thus $s'$ is a $k[G]$ module homomorphism.\\
	We have that $\pi \circ s' = $id since
	$$\pi \circ s'(x) = \frac{1}{|G|} \sum_{g \in G}e_g\pi(s(e_{g^{-1}}x)) =
	\frac{1}{|G|} \sum_{g \in G}x = x$$
	Letting $Q = s'(M/N)$ we have the exact sequence
	$$0 \to N \to^{id}  M \to^{\pi} Q \to 0$$
	Since $\pi$ splits we know that $M = N \oplus Q$ and thus $M$ is
	semisimple. \\
	\textbf{Important result used in other problems}\\
	We can write the middle module of a short exact sequence as a
	direct sum if the sequence splits as follows:\\
	If we have
	$$0 \to N \to^{id}  M \to^{\pi} Q \to 0$$
	where there exists $s':Q \to M$ such that $\pi \circ s' = \id$ then we can show $M
	= N \oplus Q$ by showing every $m \in M$ can be written uniquely as a sum
	$m = n + q$ where $n \in N, q \in Q'$. Here we
	define $Q' = \text{im } s'$ so $Q' \cong Q$.
	We have that $s'(\pi(m)) = q$ and $n =  m - q$. We know $m - q \in N$ since in
	$M/N$ the coset of $q$ and $m$  are the same since $\pi(q) =
	\pi(s'(\pi(m))) = \pi(m)$ so $m-q = 0 \Rightarrow m-q \in N$. If we show $N
	\cap Q' = 0$ then we have uniqueness since if $q + n = q' + n' \Rightarrow q
	- q' + n - n' = 0 \Rightarrow q - q' \in N, n - n' \in Q' \Rightarrow q -
	q', n - n' \in N \cap Q' \Rightarrow q - q' = 0, n - n' = 0 \Rightarrow q =
	q', n = n'$.\\
	For any $p \in N \cap Q'$ we have that $s'$ is surjective to $Q'$ so
	there exists $q \in Q$ where $s'(q) = p$. We have $\pi(s'(q)) = q$. Since
	$p \in N$, $\pi(p) = 0$, thus $q = 0$. Since $s'$ is a homomorphism we know
	$s'$ maps $0$ to $0$, thus $p = 0$.
\end{ques}

\begin{ques}{8.4}
	Consider $R = \Z$ for some prime $p$ we have the sequence of $R$ modules
	$$0 \to \Z/(p) \to \Z/(p^2) \to (\Z/(p^2)) / (\Z/(p)) \cong \Z/(p) \to
	0$$
	We know $\Z/(p)$ is simple, yet $\Z/(p^2) \not \cong \Z/(p) \oplus \Z/(p)$
	since the generator must map to an element of order $p^2$, so
	$\Z/(p^2)$ is not simple
\end{ques}

\begin{ques}{8.5}
	(a) \\
	$(i \Rightarrow ii)$\\
	This follows directly from the definition. Letting $N = P$, $\pi = p$, $f =
	$id. By the definition of a projective there exists $s: P \to M$ with $p
	\circ s = $id.\\
	$(ii \Rightarrow iii)$\\
	Leting $M= R^P$, the free module with generating set $P$, we have the
	surjection $p:M \to P$ which is the identity mapping on the generators.
	Letting $Q = \ker \pi$ we have the exact sequence
	$$0 \to Q \to^{id}  R^P \to^{p} P \to 0$$
	Since $p$ splits, we know that $R^P = P \oplus Q$. (I showed this result in
	problem 8.3)
	\\
	$(iii \Rightarrow i)$\\
	For any R modules $M$, $N$, surjective homomorphism $\pi : M \to N$ and
	homomorphism $f: P \to N$ we can extend $f$ as $f':(P \oplus Q) \to N$ by
	setting $f' = (f,0)$. We have that $P \oplus Q$ is free so has some
	generators $g_1, g_2,
	\dots$ Since $\pi$ is surjective, there exists $m_1, m_2, \dots \in M$
	where $\pi(m_1) = f'(g_1), \pi(m_2) = f'(g_2) \dots $ Thus we can define a
	homomorphism using the universal propertiy of free modules
	$$g': P \oplus Q \to M \text{ where } g_1 \to m_1, g_2 \to m_2 \dots$$
	We have that $\pi(g'(g_i)) = f'(g_i)$ and since homomorphisms from free
	modules are entirely determined by the image of the generators, $\pi \circ
	g' = f'$. Thus if we restrict $g'$ to $g: P \to Q$ with $g(p)
	= g'(p)$ we get the mapping showing $P$ is projective since $f = f'$ on $P$.\\
	\\
	(b) \\
	$(i \Rightarrow ii)$\\
	This follows directly from the definition. To use the same notation in the
	assignments description of injective, letting $M = I$, $N = M$, $\pi = s$, $f =
	$ id it follows from the definition of injective there exists $p: M \to I$ with $p
	\circ s = $ id.\\
	$(ii \Rightarrow i)$\\
	For any $M$  and homomorphism $f:M \to I$ and injective homomorphism $\pi:M
	\to N$, we create a module $(N \times I)/Q$ where $Q$ is the image of the
	homomorphism $\phi: M  \to M \times N$, $\phi(m) = (\pi(m), -f(m))$.\\
	We have the natural projective map $s:N \times I \to (N \times I)/Q$.
	We also have the injective map $(0,\id):I \to N \times I$. It is the case
	that $s \circ (0,\id)$ is injective (I will show this later) and thus from
	(ii) there exists $p: (N \times I)/Q \to I$ where $p \circ s \circ (0,\id)
	= \id$. The $g: N \to I$ to show (i) is $g = p \circ s \circ (\id,0)$. We have that
	$$g \circ \pi  = p \circ s \circ (\id,0) \circ \pi$$
	In the module $(N \times I)/Q$ we have the equivalent cosets $(\pi(m),0) =
	(\pi(m),0) - (\pi(m), -f(m)) = (0,f(m))$ so $s \circ (\id,0) \circ \pi = s
	\circ
	(0,\id) \circ f$:
	$$ = p \circ s \circ (0,\id) \circ f = f$$
	Since from how $p$ was defined $p \circ s \circ (0,\id) = \id$. All
	that is left to show is that $s \circ (0, \id)$ is injective:\\
	We have that $S = \ker (s \circ (0,\id)) = 0$ since for any $i \in S$,
	$s(0,i) =0 \Rightarrow (0,i) \in Q$, $\phi$ is surjective to $Q$ so there
	exists $m \in M$ where
	$(\pi(m),-f(m)) = (0,i)$. However since $\pi$ is injective, the only
	possibility for $m$ is $0$ and since $-f(0) = 0$ we know that $i = 0$.

\end{ques}

\begin{ques}{8.6}
	(a) \ For any division ring $R$ and $R$ modules $M, P, N$. We know that
	division ring modules have a basis so let $B$ be the basis of $P$. If
	there exists surjective homomorphism $\pi:M \to N$ and homomorphism $f:P
	\to N$ we have that $f$ is fully determined by the image of $B$. Since
	$\pi$ is surjective for every $b \in B$ there is an $m_b \in M$ such that
	$\pi(m_b) = f(b)$. Thus we can use the universal property of free modules to
	define $g: P \to M$ where $g(b) = m_b$ for all $b \in B$. We have that for
	every $b \in B$, $\pi \circ g(b) = f(b)$ so $\pi \circ g = f$. So $P$
	is projective.\\
	For any injective homomorphism $\pi: M \to N$ and homomorphism $f : M \to
	P$ there exists a basis $B$ for $M$ where $\pi$ and $f$ are fully defined
	by the images of $B$. Since $\pi$ is injective, we know that $\pi(B)$ is a
	basis for $\pi(M)$. We have that $N/\pi(M)$ has a basis $E'$, and so $N$ has
	the basis $\pi(B) \cup E$ where $E$ is a set in $N$ whose cosets are $E'$.
	We can define $g:N \to P$ where $g(e) = 0$ for all $e \in E$ and $g(\pi(b))
	= f(b)$. Thus we have $g\circ\pi = f$ so $P$ is injective.
	\\
	\\
	(b) \ If $P$ is a free $R$ module it is clear that $P$ is projective from
	condition (iii).\\ 
	Conversely $P$ is finitely generated and thus we know
	$$P \cong R^r \oplus R/(a_1) \oplus \dots \oplus R/(a_n)$$
	with $a_1 | a_2 | \dots a_n$. \\
	With $P$ projective there exists $Q$ such that $P \oplus Q \cong R^B$ is a
	free $R$ module. This can only be the case if $a_1 =a_2 = \dots a_n = 0$
	which would mean $P$ is free. This is because any basis element of $P \oplus Q$
	which generates an element in $R/ (a_i)$ cannot be linearly independent
	since it is not torsion free.\\
	\\	
	(c) \
	We can use Baer's Criterion to show that $\Q$ is injective. Baer's
	criterion states that a module over a unit ring $R$ is injective if every
	module homomorphism from an ideal $I \subset R$ to $M$ can be extended to a
	homomorphism. We have that every module homomorphism $f:n\Z \to \Q$ extends
	to a homomorphism $f':\Z \to \Q$ by taking $y \in \Q$ such that $ny = f(n)$
	and we define $f'(x) = xy$.\\
	% $\Q$ is injective since for any injection $\pi: \Q \to N$, we have that $\Q
	% \subset N$. Notice that every $\Z$ module homomorphism is simply a group
	% homomorphism since the $\Z$ action reflects applying the group action. Thus we can
	% define $g:N \to \Q$ where for every element $n \in N$ if $\langle n \rangle
	% \cap \pi(\Q) = 0$ we have $g(n) = 0$, otherwise $zn = \pi(p/q)$, thus
	% $\pi(p/zq) = n$. We define $g(n) = p/(zq)$. We have that $g$ is a well
	% defined homomorphism as follows:\\
	% If $g(n) = p/q$ and $g(n) = c/d$, then since $\pi$ is injective and
	% $\pi(p/q) = \pi(c/d) = n$, $c/d = p/q$. Thus $g$ is well defined.\\
	% $g$ is a homomorphism since if $n = \pi(p/q), m = \pi(c/d)$ then $n + m =
	% \pi(p/q) + \pi(c/d) = \pi(p/q + c/d)$ so $g(n) + g(m) = g(n + m)$. If $n =
	% \pi(p/q), \langle m \rangle \cap \pi(\Q) = 0$, then $(n + m)^q$.
	% We have that $g \circ \pi = \id$ and thus (ii) is satisfied so $\Q$ is
	% injective.
	$\Q$ is not projective since if it were, then $\Q$ would be a submodule of
	some free $\Z$ module $F$. We would then have the projection map $\pi : F
	\to \Q$ and the inclusion map $i: \Q \to F$ where $\pi \circ i = \id$.\\
	We have that $i(1) = a_1b_1 + a_2b_2 + \dots a_nb_n$ where $b_i$s are basis
	elements of $F$ and $a_i \in \Z$. Choose $N \in \Z$ so that $N > |a_i|$ for
	all $a_i$. We have that
	$$i(1) = N\cdot i(1/N) = a_1b_1 + \dots a_nb_n$$
	Which means $N | a_i$ for all $i$ (since $i(1/N)$ is written as a unique
	sum of basis elements). This is a contradiction however since $N > |a_i|$
	so $a_i = 0$ which contradicts $1 = \pi(i(1)) \neq \pi(0) = 0$
\end{ques}

\begin{ques}{8.7}
	$(i \Rightarrow ii)$\\
	For $R$ modules $M, P$ and surjective
	homomorphism $\pi: P \to M$, since $P, M$ are semisimple we can write them
	as a sum of simple modules 
	$$P = \bigoplus P_i, M = \bigoplus M_j$$
	We can write $\pi$ as a direct sum of its components from each $P_i$. Since
	the set of homomorphisms from each simple module is a division ring, we
	know that either $\pi_i: P_i \to M$ is zero, or there exists $s_i: \pi_i(P_i)
	\to P$ such that $\pi_i \circ s_i = \id$. Thus since $\pi$ is surjective we
	can define over all $M$ $s:M \to P$ where $s(m) = \bigoplus s_i(m)$. We
	then have that $\pi \circ s = \id$ and thus (ii) is satisfied so $P$ is
	projective. \\
	\\
	$(ii \Rightarrow iii)$\\
	We have that for any $R$ module $I$ and $M$ we wish to show any injective
	$\pi :I \to M$ splits. We have the short exact sequence
	$$I \to^{\pi} M \to^{p} M/I$$
	Since $M/I$ is projective we know that $p$ splits. As I have shown in 8.3
	this means that $M = M/I \oplus I$. Thus since $\pi(I) = 0 \oplus I$ we can
	extend the inverse on the image
	$\pi^{-1}: \pi(I) \to I$ to $(0, \pi^{-1}): M//I \oplus I \to I$ with $\pi
	\circ (0, \pi^{-1}) = \id$.
	\\
	\\
	$(iii \Rightarrow i)$\\
	For any submodule $I$ of $R$ we have the inclusion mapping $i : I \to R$.
	Since $R$ is injective there exists $g:R \to I$ with $g \circ i = \id$.
	Thus we have that the short exact sequence 
	$$0 \to \ker g \to^i R \to^g I$$
	and $g$ splits ($g \circ i = \id$). Thus $R = (\ker g) \oplus I$ as we have
	shown in problem 8.3. Therefore $R$ is semisimple.

\end{ques}
\end{document}
