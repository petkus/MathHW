\documentclass[12pt]{article}
\usepackage{amsmath, amssymb, amsthm, epsfig}
\newenvironment{definition}{\vspace{2 ex}{\noindent{\bf Definition}}}
        {\vspace{2 ex}}

\newenvironment{ques}[1]{\textbf{Exersise #1}\vspace{1 mm}\\ }{\bigskip}

\renewcommand{\theenumi}{\alph{enumi}}

\theoremstyle{definition}

\newenvironment{Proof}{\noindent {\sc Proof.}}{$\Box$ \vspace{2 ex}}
\newtheorem{Wp}{Writing Problem}
\newtheorem{Ep}{Extra Credit Problem}

\oddsidemargin-1mm
\evensidemargin-0mm
\textwidth6.5in
\topmargin-15mm
\textheight8.75in
\footskip27pt


\renewcommand{\l}{\left }
\renewcommand{\r}{\right }

\newcommand{\R}{\mathbb R}
\newcommand{\Q}{\mathbb Q}
\newcommand{\Z}{\mathbb Z}
\newcommand{\C}{\mathbb C}
\newcommand{\N}{\mathbb N}
\renewcommand{\H}{\mathbb H}

\newcommand{\s}{\sin}
\renewcommand{\c}{\cos}

\renewcommand{\t}{\theta}
\renewcommand{\a}{\alpha}

\newcommand{\norm}[1]{\left\lVert#1\right\rVert}

\newcommand{\T}{\mathcal{T}}

\newcommand{\Tor}{\text{Tor}}
\newcommand{\Ann}{\text{Ann}}

\newcommand{\id}{\text{id}}
\pagestyle{empty}
\begin{document}

\noindent \textit{\textbf{Math 504, Fall 2017}} \hspace{1.3cm}
\textit{\textbf{HOMEWORK $\#$9}} \hspace{1.3cm} \textit{\textbf{Peter
Gylys-Colwell}} 

\vspace{1cm}

\begin{ques}{9.1}
	(a) \ For any ideal $I \subset R \times S$, we have the projection maps
	$\pi_R:R\times S \to R, \pi_S: \R \times S \to S$ where $\pi_R(r,s) = r$
	and $\pi_S(r,s) = s$. We have that $I$ is the intersection of the ideals $P
	= \pi_R(I) \times S$ and $Q = R \times \pi_S(I)$. It is clear that $R
	\times S = P + Q$ since $R \times \{0\} \subset Q$ and $\{0\} \times S
	\subset P$. Thus we can use the chinese remainder theorem.
	$$(R \times S)/I \cong (R \times S) / P \oplus (R \times S)/ Q$$
	We have that $(R \times S)/ P = (R \times S) / (\pi_R(I) \times S) \cong R
	/ \pi_R(I)$ and similarly $(R \times S)/Q \cong S/(\pi_S(I)$. Thus
	$$(R \times S)/I \cong R/\pi_R(I) \oplus S/\pi_S(I)$$
	Since $R, S$ are semi-simple, we know that the short exact sequences
	$$0 \to \pi_R(I) \to R \to R/(\pi_R(I)) \to 0$$
	$$0 \to \pi_S(I) \to S \to S/(\pi_S(I)) \to 0$$
	split, and we get
	$$R \cong R/(\pi_R(I)) \oplus \pi_R(I), S \cong  S/(\pi_S(I)) \oplus \pi_S(I)$$
	So
	$$R \times S \cong R/(\pi_R(I)) \oplus \pi_R(I) \times S/(\pi_S(I)) \oplus
	\pi_S(I)$$
	from our chinese remainder theorem identity:
	$$R \times S \cong (R \times S)/I \oplus (\pi_R(I) \oplus \pi_S(I))$$
	And $I =  (\pi_R(I) \oplus \pi_S(I))$ since every element in $I$ is the
	unique sum of elements in $R$ and $S$ that are also in $I$, thus $R \times
	S$ is semi-simple.
	% (b) \ 
\end{ques}

\begin{ques}{9.2}
	(a) \ We have the following isomorphism
	$$\C[\Z/n] \cong \C[x]/(x^n - 1]$$

	We have from the chinese remainder theorem 
	$$\C[x]/(x^n - 1) = \C[x]/\l(\prod_{m=0}^{n-1} (x - e^{\frac{2\pi im}{n}})
	\r) \cong \prod_{m=0}^{n-1} \C[x]/ (x - e^{\frac{2\pi im}{n}})$$
	Each $\C[x]/ (x - e^{\frac{2\pi im}{n}})$ is a free $\C$ module of rank 1
	and thus $\C[x]/ (x - e^{\frac{2\pi im}{n}}) \cong \C$. So we have 
	$$\C[\Z/n] \cong \prod_{m=0}^{n-1} \C$$
	% (b) \
\end{ques}

\begin{ques}{9.3}
	$(iii) \Rightarrow (ii) \Rightarrow (i)$:\\

	% ($(i) \Rightarrow (iii)$):
\end{ques}

\begin{ques}{9.4}
	(a) \ 
	% (b) \ 
	% (c) \ 
	% (d) \ 
	% (e) \ 
\end{ques}

\begin{ques}{9.5}
	(a) \ Letting $A(t) = \sum_{i \geq 0} h_it$ and $B(t) =  \sum_{i \geq 0}
	e_it$, we can using combinatorial reasoning to write these series in a new
	form.\\
	When we multiply out 
	$$\prod_{i = 1}^n (1 + x_it) = (1 + x_1t)(1 + x_2t) \dots (1 + x_nt) $$
	We get $B(t)$. The reasoning for this is because for each $k$, a $t^k$ only
	shows up in the product by choosing $k$ $x_it$ terms and multiplying by $1$
	for the other terms. Thus every $t^k$ term is of the form $x_{j_1}x_{j_2}
	\dots x_{j_k}$ where $1\leq j_1 < j_2 < \dots j_n \leq n$, and conversly
	all $x_{j_1}x_{j_2} \dots x_{j_k}$ show up uniquely as a coefficient of one
	of the $t^k$ terms by choosing $x_{j_1}x_{j_2} \dots x_{j_k}$ and
	multiplying out by $1$ for the other terms. Summing up all these terms we
	get the symmetric polynomials:
	$$\sum_{1 \leq j_1 < j_2 < \dots j_k \leq n} x_{j_1} \dots x_{j_k}t^k = e_kt^k$$
	Thus $B(t) = \prod_{i = 1}^n (1 + x_it)$ since each coefficient of $t^k$ is
	the same in both polynomials.\\
	For $A(t)$ we have the following product of the closed form of the
	geometric series
	$$\prod_{i = 1}^n \frac1{ 1 - x_it } = \prod_{i = 1}^n\l(1 + x_it +
	(x_it)^2 + (x_it)^3 \dots + (x_it)^k \dots \r)$$
	When we factor out this product we get $A(t)$. The reasoning for this is
	because for each $k$, a $t^k$ only shows up in the product if we choose
	$(x_{j_1}t)^{n_1}, (x_{j_2}t)^{n_2}, \dots (x_{j_l}t)^{n_l}$ so that $n_1 +
	n_2 \dots n_l = k$ and multiply by the $1$ term for every other
	term in the product. Thus every $t^k$ term is one of the terms in $h_k$. We
	have that every term of $h_k$ shows up uniquely as a coefficient of one of
	the $t^k$ since any monomial $x_{j_1}^{n_1}x_{j_2}^{n_2}\dots
	x_{j_l}^{n_l}$ of total degree $k$ shows up only by choosing the terms
	$(x_{j_1}t)^{n_1}(x_{j_2}t)^{n_2}\dots (x_{j_l}t)^{n_l}$ and $1$s in the
	other terms. Thus when we sum up all the $t^k$ terms we get $h_kt^k$.\\
	\\
	(b) \ From our product identities we have the equality
	$$A(t)B(-t) = \prod_{i = 1}^n (1 - x_it)\prod_{i = 1}^n \frac1{ 1 - x_it } = 1$$
	By factoring out $A(t)B(-t)$ we get the constant term $e_0h_0 = 1$, thus
	subtracting the constant term on both sides we get the sum of nonconstant
	terms is $0$. Thus for each $k \geq 1$ the coefficient of $t^k$ is zero. We
	have that every $t^k$ coefficient term is of the form
	$h_n(-1)^me_m$ where $m + n = k$.  Thus the sum of the coefficients of the
	$t^k$ terms is $h_k - h_{k-1}e_1 + h_{k-2}e_2 - \dots + (-1)^ke_k$. This
	coefficient must be zero, thus we have Newtons identity
	$$h_k - h_{k-1}e_1 + h_{k-2}e_2 - \dots + (-1)^ke_k = 0$$
	(c) \ From Newtons identity we can that $\Lambda_n =
	\Z[h_1, \dots h_n]$. We from lecture that $\Lambda_n = \Z[e_1, \dots
	e_n]$ thus if we show $\Z[h_1, \dots h_n] = \Z[e_1 , \dots e_n]$ we are done.
	By showing that $h_1, \dots h_n$ linearly spans $e_1 \dots e_n$ we are done
	(we already know $e_1, e_2 \dots e_n$ spans $h_1 \dots h_n$ since $e_1,
	\dots e_n$ generate all symmetric polynomials). Using induction we have the
	base case $h_0 = e_0$. From Newtons identity:
	$$(-1)^{k-1}(h_k - h_{k-1}e_1 + h_{k-2}e_2 - \dots e_{k-1}h_1) = e_k$$
	We have that each $e_k$ is a linear sum of $e_i$ and $h_j$ where $i < k$
	thus from our inductive hypothesis each $e_i$ is a linear sum of
	$h_j$s and thus $e_k$ is a linear sum of $h_j$s. 
\end{ques}

\end{document}
