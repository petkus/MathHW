\documentclass[12pt]{article}
\usepackage{amsmath, amssymb, amsthm, epsfig}
\newenvironment{definition}{\vspace{2 ex}{\noindent{\bf Definition}}}
        {\vspace{2 ex}}

\newenvironment{ques}[1]{\textbf{Exersise #1}\vspace{1 mm}\\ }{\bigskip}

\renewcommand{\theenumi}{\alph{enumi}}

\theoremstyle{definition}

\newenvironment{Proof}{\noindent {\sc Proof.}}{$\Box$ \vspace{2 ex}}
\newtheorem{Wp}{Writing Problem}
\newtheorem{Ep}{Extra Credit Problem}

\oddsidemargin-1mm
\evensidemargin-0mm
\textwidth6.5in
\topmargin-15mm
\textheight8.75in
\footskip27pt


\renewcommand{\l}{\left }
\renewcommand{\r}{\right }

\newcommand{\R}{\mathbb R}
\newcommand{\Q}{\mathbb Q}
\newcommand{\Z}{\mathbb Z}
\newcommand{\C}{\mathbb C}
\newcommand{\N}{\mathbb N}
\newcommand{\F}{\mathbb F}
\renewcommand{\H}{\mathbb H}
\newcommand{\ndiv}{\hspace{-4pt}\not|\hspace{2pt}}

\newcommand{\tensor}{\otimes}

\newcommand{\s}{\sin}
\renewcommand{\c}{\cos}

\renewcommand{\t}{\theta}
\renewcommand{\a}{\alpha}

\newcommand{\norm}[1]{\left\lVert#1\right\rVert}

\newcommand{\T}{\mathcal{T}}

\newcommand{\Tor}{\text{Tor}}
\newcommand{\Ann}{\text{Ann}}
\newcommand{\End}{\text{End}}

\newcommand{\id}{\text{id}}
\pagestyle{empty}
\begin{document}

\noindent \textit{\textbf{Math 504, Fall 2017}} \hspace{1.3cm}
\textit{\textbf{HOMEWORK $\#$9}} \hspace{1.3cm} \textit{\textbf{Peter
Gylys-Colwell}} 

\vspace{1cm}

\begin{ques}{9.1}
	(a) \ For any ideal $I \subset R \times S$, we have the projection maps
	$\pi_R:R\times S \to R, \pi_S: \R \times S \to S$ where $\pi_R(r,s) = r$
	and $\pi_S(r,s) = s$. We have that $I$ is the intersection of the ideals $P
	= \pi_R(I) \times S$ and $Q = R \times \pi_S(I)$. It is clear that $R
	\times S = P + Q$ since $R \times \{0\} \subset Q$ and $\{0\} \times S
	\subset P$. Thus we can use the chinese remainder theorem.
	$$(R \times S)/I \cong (R \times S) / P \oplus (R \times S)/ Q$$
	We have that $(R \times S)/ P = (R \times S) / (\pi_R(I) \times S) \cong R
	/ \pi_R(I)$ and similarly $(R \times S)/Q \cong S/(\pi_S(I)$. Thus
	$$(R \times S)/I \cong R/\pi_R(I) \oplus S/\pi_S(I)$$
	Since $R, S$ are semi-simple, we know that the short exact sequences
	$$0 \to \pi_R(I) \to R \to R/(\pi_R(I)) \to 0$$
	$$0 \to \pi_S(I) \to S \to S/(\pi_S(I)) \to 0$$
	split, and we get
	$$R \cong R/(\pi_R(I)) \oplus \pi_R(I), S \cong  S/(\pi_S(I)) \oplus \pi_S(I)$$
	So
	$$R \times S \cong R/(\pi_R(I)) \oplus \pi_R(I) \times S/(\pi_S(I)) \oplus
	\pi_S(I)$$
	from our chinese remainder theorem identity:
	$$R \times S \cong (R \times S)/I \oplus \pi_R(I) \oplus \pi_S(I)$$
	And $I =  \pi_R(I) \oplus \pi_S(I)$ thus $R \times
	S$ is semi-simple.\\
	\\
	(b) \ If $a$ is square free, since a PID is a UFD we can factor $a$ as a
	product of prime elements $a = p_1p_2 \dots p_n$. Thus from the chinese
	remainder theorem.
	$$R/(a) \cong R/(p_1) \oplus R/(p_2) \oplus \dots \oplus R/(p_n)$$
	each $R/(p_i)$ is a field (since the $p_i$s are irriducible elements) and
	thus simple. Thus $R/(a)$ is a direct sum of simple modules over $R$ and
	thus semi-simple.\\
	Conversly if $p^2 | a$ for some prime $p$ then the chinese remainder
	theorem results in
	$$R/(a) \cong R/(p^n) \oplus R/(b)$$
	where $p \ndiv \ b$ and $n \geq 2$. We have that the submodule generated by
	$(p,0)$ cannot split $R/(a)$. The reason for this is that our quotent
	becomes $(R/(a))/(p,0) \cong (R/(p^n) \oplus R/(b)/)(p,0) \cong R/(p) \oplus R/(b)$
	$$R/(a) \cong (R/(p) \oplus R/(b)) \oplus (p,0)R/(a)$$
	There would be no element in this new ring, which multiplied by $p^{n-1}$
	would be nonzero, but would be $0$ multiplied by $p^n$ as is the case for
	$(1,0) \in \R/(p^n) \oplus R/(b)$. Thus this split is not possible
\end{ques}

\begin{ques}{9.2}
	(a) \ We have the following isomorphism
	$$\C[\Z/n] \cong \C[x]/(x^n - 1]$$
	Where $e_0 \to 1, e_1 \to x, \dots e_{n-1} \to x^{n-1}$. This mapping we
	will call $\varphi$ is an homorphism as follows
	$$\varphi(e_ie_k) = \varphi(e_{i+k \pmod n}) = x^{i+k \pmod n} = x^ix^k =
	\varphi(e_i)\varphi(e_k)$$
	And $\varphi$ is bijective since there is a one to one and onto correspondence
	between generators as a $\C$ vectorspace.
	\\
	We have from the chinese remainder theorem 
	$$\C[x]/(x^n - 1) = \C[x]/\l(\prod_{m=0}^{n-1} (x - e^{\frac{2\pi im}{n}})
	\r) \cong \bigoplus_{m=0}^{n-1} \C[x]/ (x - e^{\frac{2\pi im}{n}})$$
	Each $\C[x]/ (x - e^{\frac{2\pi im}{n}})$ is a free $\C$ module of rank 1
	and thus $\C[x]/ (x - e^{\frac{2\pi im}{n}}) \cong \C$. So we have 
	$$\C[\Z/n] \cong \bigoplus_{m=0}^{n-1} \C$$
	With the explicit isomorphism as the composition of the described
	isomorphisms: $e_0 \to 1 \to 1$, and $$e_k \to x^k \to
	\bigoplus_{m=0}^{n-1} e^{\frac{2\pi i km}{n}}$$
	\\
	(b) \ We know that any division ring over $\C$ is $\C$ as shown in problem
	8.2. Thus we know $\C[S_3]$ is isomorphic to a direct sum of matricies over
	$\C$. Since $\C[S_3]$ is dimension $6$ over $\C$ the only possible
	dimension $6$ forms are $\C \oplus \C \dots \oplus \C$ and $M_2[\C] \oplus
	\C \oplus \C$. It cannot be $\C \oplus \C \oplus \dots \oplus \C$ however
	since the only elements of order $3$ are of the form 
	$$\bigoplus_{i \in I \subset [6]} e^{\frac{\pm2\pi}{3}} \bigoplus_{i \in
	I^c} 1$$
	So we would have $e_{123}$ maps to something of this form. but the only
	elements of order two are of the form
	$$\bigoplus_{i \in I \subset [6]} -1 \bigoplus_{i \in
	I^c} 1$$
	So $e_{12}, e_{23}$ would map to something of this form. But then
	$e_{12}e_{23} = e_{123}$ would not be compatable since multiplying elements
	of the order 2 form would not yield elements of the order 3 form. Thus 
	$$\C[S_3] \cong M_2[\C] \oplus \C \oplus \C$$
	For an explicit isomorphism we know 
	$$e_i \to 
	\begin{bmatrix}
		1 & 0 \\
		0 & 1
	\end{bmatrix}
	\oplus 1 \oplus 1$$
	We can consider the rational canonical form of matricies to figure out all
	matricies of order $2$. The minimal polynomial must divide $x^2 - 1$ which
	yields the possible matricies
	$$
	\begin{bmatrix}
		- 1 & 0 \\
		0 &  1
	\end{bmatrix},
	\begin{bmatrix}
		0 & 1 \\
		1 & 0
	\end{bmatrix}
	$$
	For matricies of order $3$, the minimal polynomial must divide $x^3 - 1
	= (x-1)(x-\zeta_3)(x+\zeta_3)$ (where $\zeta_3$ is the 3rd root of unity)
	and thus must be either $x^2 + x+1$ or $x^2 - (1\pm \zeta_3)x \pm \zeta_3$
	or $(x -1)$, $x\pm \zeta_3$. Which yields the possible matricies
	$$
	\begin{bmatrix}
		0 & -1 \\
		1 & -1
	\end{bmatrix},
	\begin{bmatrix}
		0 & \pm \zeta_3 \\
		1 & 1 \pm \zeta_3
	\end{bmatrix}
	\begin{bmatrix}
		\pm \zeta_3 & 0 \\
		0 & \pm \zeta_3
	\end{bmatrix}
	\begin{bmatrix}
		\pm 1 & 0 \\
		0 & \pm \zeta_3
	\end{bmatrix}
	$$
	After trying various conjugacy classes of the order two matricies, I found
	the following mapping to be compatable with multiplication:
	$$
	e_{12} \to \begin{bmatrix}
		0 & 1 \\
		1 & 0
	\end{bmatrix}
	e_{23} \to \begin{bmatrix}
		-1 & 0 \\
		-1 & 1
	\end{bmatrix}
	e_{13} \to \begin{bmatrix}
		1 & -1 \\
		0 & -1
	\end{bmatrix}
	e_{123} \to \begin{bmatrix}
		-1 & 1 \\
		-1 & 0
	\end{bmatrix}
	e_{13} \to \begin{bmatrix}
		0 & -1 \\
		1 & -1
	\end{bmatrix}
	$$
\end{ques}

\begin{ques}{9.3}
	$(iii) \Rightarrow (ii) \Rightarrow (i)$:\\
	From the identities we established leading up to the Artin Wedernburn
	theorem, we know that $R \cong \End_{R^{op}}(R^{op}) \cong M_{n}(D)$, where
	$M_n(D) \cong \End(M_1 \oplus M_2 \oplus \dots M_n)$. This comes from the
	identity $R^{op} \cong \oplus_{i=1}^n M_n$ where $M_n$ are simple modules.
	Since each $M_i$ is isomorphic to each other we have $R$ is the direct
	sum of only one matrix ring over a division ring. Thus $(iii) \Rightarrow
	(ii)$. In problem 7.6(1) we have shown $M_n(D)$ is simple. For Artinian
	condition, $M_n(D)$ as an $R$ module is isomorphic to $D^n \oplus D^n
	\oplus \dots \oplus D^n$ which are simple $M_n(D)$ modules. Thus, as we
	have established in lecture,
	$M_n(D)$ is Artinian since every every submodule is of the form $D^n \oplus
	D^n \oplus D^n \dots \oplus D^n$ and so a decsending chain must terminate
	on some number of $D^n$ as a direct sum.
	Thus $(ii) \Rightarrow (i)$.\\
	$(i) \Rightarrow (iii)$:\\
	$R$ contains a simple submodule $M$ as illustrated by the following
	process. If $R$ is simple, we are done, otherwise $R$ must contain some
	proper module $M_1$, either $M_1$ is simple or $M_1$ contains proper $M_2$.
	Continuing this logic yields a decsending chain of proper modules $R
	\supset M_1 \supset M_2 \dots$ since $R$ is Artinian there exists
	$M_n$ which terminates the chain, and thus must be simple.\\
	We know that Ann$_R(M)$ is a two sided ideal of $R$. Thus since $R$ is
	simple, we know Ann$_R(M) = 0$. We can now construct an isomorphism $\phi:R
	\to M^n$ for some $n$ and thus proving $R$ is semi-simple with every simple
	submodule of $R$ isomorphic to $M$. Since in lecture we established every
	simple $R$ module is a submodule of $R$, this implies $(iii)$. We
	construct this isomorphism as follows:\\
	Let $\phi_0:R \to M$ where $1 \to m_0$ for some $m_0 \in M$. Either $\ker
	\phi_0 = 0$ or $\exists r_0 \in \ker \phi_0$. Since Ann$_R(M)= 0$ there
	exists $m_1 \in M$ such that $r_1m_1 \neq 0$. We define the new homorphism
	$\phi_1:R \to M^2$ where $1 \to (m_0, m_1)$ thus $r_1 \notin \ker \phi_1$
	since $\phi_1(r_1) = (r_1m_0,r_1m_1) \neq (0,0)$, either $\ker \phi_2 = 0$ or
	there exists $r_2 \in \ker \phi_1$ and $m_2 \in M$ with $r_2m_2 \neq 0$.
	Thus we can define $\phi_2 : R \to M^3$ where $1 \to (m_0,m_1,m_2)$ and
	thus $r_2 \notin \ker \phi_2$. Continuing this process we get a decsending
	chain of submodules of $R$
	$$\ker \phi_0 \supset \ker \phi_1 \supset \ker \phi_2 \supset \dots $$
	Thus since $R$ is Artinian this chain terminates, which means there exists
	$\phi_k = \phi:R \to M^n$ with $\ker \phi = 0$. So $\phi$ is an imbedding
	into $M^n$. Thus $R$ is a submodule of $M^n$ and thus $R \cong M^k$ for
	some $k \leq n$.


\end{ques}

\begin{ques}{9.4}
	(a) We have the bilinear map 
	$$f:\Z/m \times \Z/n \to \Z/d$$
	$$(a,b) \to ab$$
	This map is bilinear since if $(a,b) = 0 \in \Z/m \times
	\Z/n$ then $ab = 0 \in \Z/d$  so $0 \to 0$, and $f(a,b) + f(c,b) = ab + cb
	= (a + c)b = f(a + c,b) $\\
	This induces the map
	$$\varphi : \Z/m \otimes \Z/n \to \Z/d$$
	$$a \otimes b \to ab$$
	This map is bijective and thus an isomorphism as follows:\\
	We know that evey $z \in \Z/d$ can
	be written as $ab$ since we let $a = 1$ and since $m \geq d$ we can let $b
	= z$. Thus this mapping is surjective.\\
	We have that there is only one generator. $\Z/m \tensor \Z/n$ is generated
	by $1_m \tensor 1_n = 1$ since both $\Z/m$ and $\Z/n$ have only the one
	generators. Since we are working with $\Z$ modules, this is equivalent to
	saying the group is cyclic. For any $a \tensor b$, from bilinearity
	$$a \tensor b = ab(1 + n\Z \tensor 1 + m\Z) = (ab
	+ n\Z + m\Z)(1 \tensor 1)= (ab + d\Z)(1 \tensor 1)$$
	Thus the characteristic is $d$ so the domain and codomain of $\varphi$
	have the same size. Since $\varphi$ is surjective, it is thus
	injective\\
	\\
	(b) \ We have that $\Z$ is a subring of $\Q$. Thus as established in
	Corollary 18 of 10.4 of Dummit and Foote, the tensor extends the scalers:
	$$\Q \tensor_\Z \Q \to \Q_\Q$$
	Where $\Q_\Q$ is a $\Q$ module. We know that $\Q_\Q \cong \Q$ as a group,
	thus we are done\\
	\\
	(c) \ For any $m \tensor r + I\in M \tensor R/I$ we have $m \tensor r + I = rm
	\tensor 1 + I$ thus every element can be written  in the form $rm \tensor 1 =
	m' \tensor 1$.
	$$\varphi: M \tensor R/I \to M/IM$$
	$$m \tensor 1\to m + IM$$
	This map is well defined since if $\varphi(m \tensor 1) = m + IM = m' +
	IM$, then $m = m' + in$ for some $i \in I$, $n \in M$. By bilinearity we
	have 
	$$m \tensor 1 = m' + in \tensor 1 = m' \tensor 1 + i(n \tensor 1) = m'
	\tensor 1 + n \tensor 0 = m' \tensor 1$$
	We have that $n \tensor 0 = n \tensor ( 0 + 0) = n \tensor 0 + n \tensor 0
	\Rightarrow n \tensor 0 = 0$ for that last equality. Thus $\varphi$ is well
	defined. $\varphi$ is surjective since for any $m + IM$ we have $\varphi
	(m \tensor 1) =m + IM$. $\varphi$ is injective since if $\varphi(m' \tensor 1) =
	\varphi(m \tensor 1)$ then as we have shown in proving $\varphi$ is well
	defined, $m +IM = m' + IM \Rightarrow m \tensor 1 = m' \tensor 1$. Thus
	$\varphi$ is an isomorphism\\
	\\
	(d) \ 
	We have the mapping 
	$$\varphi:\C \tensor_\R \C \to \C \times \C$$
	$$1 \tensor 1 \to (1,1), 1 \tensor i \to (1,i), i \tensor 1 \to (i,1), i
	\tensor i \to (i,i)$$
	This mapping is a ring isomorphism as follows.\\
	Since $\C \tensor_\R \C$ is an $\R$ vectorspace and thus a free module, we
	know the mapping is $\R$ linear and bijective since the map bijects basis.
	Thus all that needs to be checked is multiplication of the basis elements.
	From how the ring structure is defined over the tensor product (as
	established in prop 21 page 374 of Dummit and Foote), it is clear the
	multiplicative structure is preserved, since 
	$$\varphi((a\tensor b)(c \tensor d)) = \varphi(ab \tensor cd) = (ab,cd) =
	\varphi(a \tensor b)\varphi(c \tensor d)$$
	Where $a,b,c,d$ represents either $1$  or $i$\\
	\\
	(e) \ Using the hint we have that $\F_{p^n} \cong \F_p[x]/(f)$, thus
	$\F_{p^n}$ is a $\F_p$ free module. From Corollary 18 of Dummit and Foote
	we know as $\F_{p^n}$ modules
	$$\F_{p^n} \tensor_{\F_p} \F_{p^n} \cong \F_{p^n} \tensor_{\F_p}
	\F_{p}[x]/(f) \cong \F_{p^n}^n$$
	This applies as ring homorphisms since the multiplicative structure of
	$\F_{p^n}$ is preserved. ie for 
	$$(a \tensor (b_0 + b_1x + b_2x^2 \dots b_{n-1}x^{n-1}) \in \F_{p^n}
	\tensor_{\F_p} \F_{p}[x] / (f)$$
	We have
	$$(a \tensor (b_1 + b_2x + b_3x^2 \dots b_{n-1}x^{n-1}) = a(b_1(1 \tensor
	1) + b_2(1 \tensor x) + \dots b_n(1 \tensor x^{n-1})) \to a(b_0,b_1 \dots b_n)$$
\end{ques}

\begin{ques}{9.5}
	(a) \ Letting $A(t) = \sum_{i \geq 0} h_it$ and $B(t) =  \sum_{i \geq 0}
	e_it$, we can using combinatorial reasoning to write these series in a new
	form.\\
	When we multiply out 
	$$\prod_{i = 1}^n (1 + x_it) = (1 + x_1t)(1 + x_2t) \dots (1 + x_nt) $$
	We get $B(t)$. The reasoning for this is because for each $k$, a $t^k$ only
	shows up in the product by choosing $k$ $x_it$ terms and multiplying by $1$
	for the other terms. Thus every $t^k$ term is of the form $x_{j_1}x_{j_2}
	\dots x_{j_k}$ where $1\leq j_1 < j_2 < \dots j_n \leq n$, and conversly
	all $x_{j_1}x_{j_2} \dots x_{j_k}$ show up uniquely as a coefficient of one
	of the $t^k$ terms by choosing $x_{j_1}x_{j_2} \dots x_{j_k}$ and
	multiplying out by $1$ for the other terms. Summing up all these terms we
	get the symmetric polynomials:
	$$\sum_{1 \leq j_1 < j_2 < \dots j_k \leq n} x_{j_1} \dots x_{j_k}t^k = e_kt^k$$
	Thus $B(t) = \prod_{i = 1}^n (1 + x_it)$ since each coefficient of $t^k$ is
	the same in both polynomials.\\
	For $A(t)$ we have the following product of the closed form of the
	geometric series
	$$\prod_{i = 1}^n \frac1{ 1 - x_it } = \prod_{i = 1}^n\l(1 + x_it +
	(x_it)^2 + (x_it)^3 \dots + (x_it)^k \dots \r)$$
	When we factor out this product we get $A(t)$. The reasoning for this is
	because for each $k$, a $t^k$ only shows up in the product if we choose
	$(x_{j_1}t)^{n_1}, (x_{j_2}t)^{n_2}, \dots (x_{j_l}t)^{n_l}$ so that $n_1 +
	n_2 \dots n_l = k$ and multiply by the $1$ term for every other
	term in the product. Thus every $t^k$ term is one of the terms in $h_k$. We
	have that every term of $h_k$ shows up uniquely as a coefficient of one of
	the $t^k$ since any monomial $x_{j_1}^{n_1}x_{j_2}^{n_2}\dots
	x_{j_l}^{n_l}$ of total degree $k$ shows up only by choosing the terms
	$(x_{j_1}t)^{n_1}(x_{j_2}t)^{n_2}\dots (x_{j_l}t)^{n_l}$ and $1$s in the
	other terms. Thus when we sum up all the $t^k$ terms we get $h_kt^k$.\\
	\\
	(b) \ From our product identities we have the equality
	$$A(t)B(-t) = \prod_{i = 1}^n (1 - x_it)\prod_{i = 1}^n \frac1{ 1 - x_it } = 1$$
	By factoring out $A(t)B(-t)$ we get the constant term $e_0h_0 = 1$, thus
	subtracting the constant term on both sides we get the sum of nonconstant
	terms is $0$. Thus for each $k \geq 1$ the coefficient of $t^k$ is zero. We
	have that every $t^k$ coefficient term is of the form
	$h_n(-1)^me_m$ where $m + n = k$.  Thus the sum of the coefficients of the
	$t^k$ terms is $h_k - h_{k-1}e_1 + h_{k-2}e_2 - \dots + (-1)^ke_k$. This
	coefficient must be zero, thus we have Newtons identity
	$$h_k - h_{k-1}e_1 + h_{k-2}e_2 - \dots + (-1)^ke_k = 0$$
	(c) \ From Newtons identity we can that $\Lambda_n =
	\Z[h_1, \dots h_n]$. We from lecture that $\Lambda_n = \Z[e_1, \dots
	e_n]$ thus if we show $\Z[h_1, \dots h_n] = \Z[e_1 , \dots e_n]$ we are done.
	By showing that $h_1, \dots h_n$ linearly spans $e_1 \dots e_n$ we are done
	(we already know $e_1, e_2 \dots e_n$ spans $h_1 \dots h_n$ since $e_1,
	\dots e_n$ generate all symmetric polynomials). Using induction we have the
	base case $h_0 = e_0$. From Newtons identity:
	$$(-1)^{k-1}(h_k - h_{k-1}e_1 + h_{k-2}e_2 - \dots e_{k-1}h_1) = e_k$$
	We have that each $e_k$ is a linear sum of $e_i$ and $h_j$ where $i < k$
	thus from our inductive hypothesis each $e_i$ is a linear sum of
	$h_j$s and thus $e_k$ is a linear sum of $h_j$s. 
\end{ques}

\end{document}
