\documentclass[12pt]{article}
\usepackage{amsmath, amssymb, amsthm, epsfig}

\newenvironment{definition}{\vspace{2 ex}{\noindent{\bf Definition}}}
        {\vspace{2 ex}}

\newenvironment{ques}[1]{\textbf{Exersise #1}\vspace{1 mm}\\ }{\bigskip}

\renewcommand{\theenumi}{\alph{enumi}}

\theoremstyle{definition}

\newenvironment{Proof}{\noindent {\sc Proof.}}{$\Box$ \vspace{2 ex}}
\newtheorem{Wp}{Writing Problem}
\newtheorem{Ep}{Extra Credit Problem}

\oddsidemargin-1mm
\evensidemargin-0mm
\textwidth6.5in
\topmargin-15mm
\textheight8.75in
\footskip27pt


\renewcommand{\l}{\left }
\renewcommand{\r}{\right }

\newcommand{\R}{\mathbb R}
\newcommand{\Q}{\mathbb Q}
\newcommand{\Z}{\mathbb Z}
\newcommand{\C}{\mathbb C}
\newcommand{\N}{\mathbb N}
\renewcommand{\H}{\mathbb H}

\newcommand{\s}{\sin}
\renewcommand{\c}{\cos}

\renewcommand{\t}{\theta}
\renewcommand{\a}{\alpha}

\newcommand{\norm}[1]{\left\lVert#1\right\rVert}

\newcommand{\T}{\mathcal{T}}

\newcommand{\Tor}{\text{Tor}}
\newcommand{\Ann}{\text{Ann}}

\pagestyle{empty}
\begin{document}

\noindent \textit{\textbf{Math 504, Fall 2017}} \hspace{1.3cm}
\textit{\textbf{HOMEWORK $\#$6}} \hspace{1.3cm} \textit{\textbf{Peter
Gylys-Colwell}} 

\vspace{1cm}

\begin{ques}{7.1}
	(1) \ We have that the set $I = \{g \in k[x]: g(T)\}$ is an ideal of $k[x]$.
	Thus since $k[x]$ is a PID, we know it is generated by one element. This
	element is $m_T$ since if there was a different generator $g$ of $I$ which
	is not a unit multiple of $m_T$ then $g$ has degree less than degree of
	$m_T$ and $g(T) = 0$ which contradicts minimality of $m_T$. Thus we have
	that for any $f \in k[x]$, $f(T) = 0 \Leftrightarrow f \in I
	\Leftrightarrow$ $m_T$ divides $f$\\
	\\
	(2) \ We know that 
	$$V \cong k[x]/a_1(x) \oplus k[x]/a_2(x) \dots \oplus k[x]/a_{n-1}(x)
	\oplus k[x]/m_T(x)$$
	With $a_1|a_2| \dots a_{n-1} | m_T$. Thus in order for $f \in
	\Ann_{k[x]}(V)$, it would have to be the case that $a_1 | f, a_2 | f, \dots
	m_T | f$. Which is equivalent to $m_T | f$ since $a_1, a_2 \dots a_{n-1} |
	m_T$. Thus $\Ann_{k[x]}(V) = (m_T)$
\end{ques}

\begin{ques}{7.2}
	(1) \ $A$ is already in rational canonical form so $P$ is just the identity
	matrix. We have that det$(xI - A)$ is precisely $(x - 1)(x^2 - 3x + 2)$
	which is the characteristic polinomial.\\
	(2) \ We have that the characteristic polinomial splits completely as $(x -
	1)^2(x - 2)$, so the eigenvalues are $1, 1, 2$. Thus the jordan form is
	$$J = \begin{bmatrix}
	1 & 0 & 0\\
	0 & 1 & 0\\
	0 & 0 & 2\\
	\end{bmatrix}$$
	I solved the equations $Av = v, Aw = 2w$ to get eigenvectors. We
	get that the eigenvectors are $[0\ -1\ 1], [1\ 0\ 0], [0\ -2\ 1]$ for
	eigenvalues $2, 1, 1$ respectivly. Thus we know that $S^{-1}JS = A$ where
	$S$ is the matrix of eigenvectors. So $P^{-1} = S$, a straightforward
	inverse computation gives us $P$
	$$P^{-1} = \begin{bmatrix}
	0 & 1 & 0\\
	-1 & 0 & -2\\
	1 & 0 & 1\\
	\end{bmatrix} \Rightarrow
	P = \begin{bmatrix}
	0 & 1 & 2\\
	1 & 0 & 0\\
	0 & -1 & -1\\
	\end{bmatrix}
	$$
\end{ques}

\begin{ques}{7.3}
	(1) \ We have that the only possible reduced forms of the $k[x]$ modules are
	$$V \cong k[x]/(x) \oplus k[x]/(x(x^2+1)^2)$$
	$$V \cong k[x]/(x^2 + 1) \oplus k[x]/(x^2(x^2+1))$$
	$$V \cong k[x]/(x(x^2 + 1)) \oplus k[x]/(x(x^2+1))$$
	Factoring $x(x^2+1)^2 = x^5 + 2x^3 + x$, $x^2(x^2 + 1) = x^4 + x^2, x(x^2 +
	1) = x^3 + x$ we have the corresponding rational canonical forms
	$$\begin{bmatrix} 
	1 & 0 & 0 & 0 & 0 & 0\\ 
	0 & 0 & 0 & 0 & 0 & 0\\ 
	0 & 1 & 0 & 0 & 0 & -1\\ 
	0 & 0 & 1 & 0 & 0 & 0\\ 
	0 & 0 & 0 & 1 & 0 & -2\\ 
	0 & 0 & 0 & 0 & 1 & 0\\ 
	\end{bmatrix}
	\begin{bmatrix} 
	0 & -1 & 0 & 0 & 0 & 0\\ 
	1 & 0 & 0 & 0 & 0 & 0\\ 
	0 & 0 & 0 & 0 & 0 & 0\\ 
	0 & 0 & 1 & 0 & 0 & 0\\ 
	0 & 0 & 0 & 1 & 0 & -1\\ 
	0 & 0 & 0 & 0 & 1 & 0\\ 
	\end{bmatrix}
	\begin{bmatrix} 
	0 & 0 & 0 & 0 & 0 & 0\\ 
	1 & 0 & -1 & 0 & 0 & 0\\ 
	0 & 1 & 0 & 0 & 0 & 0\\ 
	0 & 0 & 0 & 0 & 0 & -1\\ 
	0 & 0 & 0 & 1 & 0 & 0\\ 
	0 & 0 & 0 & 0 & 1 & 0\\ 
	\end{bmatrix}$$
	\\
	(2), (3) \ $A$ has order $4$  means that $A$ has a minimal polinomial
	$M_A(x)$ which divides $x^4 - 1$. $x^4 - 1$ splits as $(x^2 + 1)(x+1)(x-1)$
	in $\Q$ and splits fully as $(x - 1)(x + 1)(x - i)(x + i)$ over $\C$. Thus
	the only possible degree $2$ $\Q[x]$ modules are
	$$\Q[x]/(x^2 - 1), \Q[x]/(x^2 + 1), \Q[x]/(x + 1)\oplus \Q[x]/(x + 1), \Q[x]/(x
	- 1)\oplus \Q[x]/(x - 1)$$
	Which leads to the matricies
	$$\begin{bmatrix} 
	0 & 1 \\ 
	1 & 0
	\end{bmatrix}
	\begin{bmatrix} 
	0 & -1 \\ 
	1 & 0
	\end{bmatrix}
	\begin{bmatrix} 
	0 & 1 \\ 
	-1 & 0
	\end{bmatrix}
	\begin{bmatrix} 
	0 & -1 \\ 
	-1 & 0
	\end{bmatrix}
	\begin{bmatrix} 
	-1 & 0 \\ 
	0 & -1
	\end{bmatrix}
	\begin{bmatrix} 
	-1 & 0 \\ 
	0 & 1
	\end{bmatrix}
	\begin{bmatrix} 
	1 & 0 \\ 
	0 & -1
	\end{bmatrix}
	$$
	A quick computation yields that the only elements of order $4$ are 
	$$
	\begin{bmatrix} 
	0 & -1 \\ 
	1 & 0
	\end{bmatrix}
	\begin{bmatrix} 
	0 & 1 \\ 
	-1 & 0
	\end{bmatrix}
	$$
	For the complex case, on top of the matricies in $\Q$ we also have the
	possible $\C[x]$ modules 
	$$\C[x]/(x^2 \pm (1 + i)x + i), \C[x]/(x^2 \pm (1 - i)x - i)$$
	$$\C[x]/(x \pm i)\oplus \C[x]/(x \pm i), \C[x]/(x \pm i)\oplus \C[x]/(x \pm 1)$$
	This yields the matricies
	$$\begin{bmatrix} 
	0 & -i \\ 
	1 & \pm(1 + i)
	\end{bmatrix}
	\begin{bmatrix} 
	0 & i \\ 
	-1 & \pm(1 - i)
	\end{bmatrix}
	\begin{bmatrix} 
	\pm i & 0 \\ 
	0 & \pm i
	\end{bmatrix}
	\begin{bmatrix} 
	\pm 1 & 0 \\ 
	0 & \pm i
	\end{bmatrix}
	\begin{bmatrix} 
	\pm i & 0 \\ 
	0 & \pm 1
	\end{bmatrix}
	$$
	A straightforward computation yields that every one of these matricies is
	of order $4$ (none have order $2$)
\end{ques}

\begin{ques}{7.4}
	$R$ is right Noetherian by the following reasoning. For any chain of ideals $0
	\subset I_1 \subset I_2 \subset \dots I_n \subset \dots R$, if the chain
	does not terminate then we can choose elements
	$$A_1, 	A_2, A_3, \dots 
	A_i = \begin{bmatrix} 
	a_i & b_i \\ 
	0 & c_i
	\end{bmatrix} \dots
	$$
	where $A_i \in I_i, A_{i+1} \in I_{i+1}$ and $A_{i+1} \notin I_i$. We have
	that $A_1R + A_2R + \dots A_iR \subset I_i$. We have $A_iR$ is of the form
	$$A_iR = \l\{\begin{bmatrix} 
	a_in & a_ip + b_iq \\ 
	0 & c_iq
	\end{bmatrix} | n \in \Z, p,q \in \Q \r\} = 
	\l \{ \begin{bmatrix} 
	a_in & p \\ 
	0 & q
	\end{bmatrix} | n \in \Z, p,q \in \Q\r\} 
	$$
	Thus we have that 
	$$A_1R + A_2R + \dots A_iR = 
	\l \{ \begin{bmatrix} 
	\text {gcd}(a_1, \dots a_i)n & p \\ 
	0 & q
	\end{bmatrix} | n \in \Z, p,q \in \Q\r\} 
	$$
	Since $A_{i+1}R \not \subseteq A_1R + A_2R + \dots A_iR$, we know that $\text
	{gcd}(a_1, \dots a_i) \not |a_{i+1}$ and thus we have that $\text
	{gcd}(a_1, \dots a_i) < \text {gcd}(a_1, \dots a_{i+1})$. So in a finite
	amount of iterations, there is an $n$ such that $\text {gcd}(a_1, \dots
	a_n) = 1$ which means
	$$A_1R + A_2R + \dots A_nR = R \Rightarrow I_n = R$$
	\\
	$R$ is not left Noetherian as illustrated in this chain
	$$R \begin{bmatrix} 
	1 & 1 \\ 
	0 & 0
	\end{bmatrix}
	\subset
	R \begin{bmatrix} 
	1 & \frac 1 2 \\ 
	0 & 0
	\end{bmatrix}
	\subset
	R \begin{bmatrix} 
	1 & \frac 1 4 \\ 
	0 & 0
	\end{bmatrix} 
	\subset
	\dots \subset
	R \begin{bmatrix} 
	1 & \frac 1 {2^n} \\ 
	0 & 0
	\end{bmatrix}  \subset \dots$$
	Elements of each Ideal are of the form
	$$\begin{bmatrix} 
	z & z \\ 
	0 & 0
	\end{bmatrix}
	,
	\begin{bmatrix} 
	z & \frac z 2 \\ 
	0 & 0
	\end{bmatrix}
	,
	\begin{bmatrix} 
	z & \frac z 4 \\ 
	0 & 0
	\end{bmatrix} 
	\dots 
	\begin{bmatrix} 
	1 & \frac z {2^n} \\ 
	0 & 0
	\end{bmatrix}   \dots$$
	For $z \in \Z$ and are thus each proper ideals.
\end{ques}

\begin{ques}{7.5}
	(1) \ Let $g_1 \dots g_n$ be a basis for $R$ over $k$. We can consider the
	number of these generators in an ideal $I$ which we will denote by $d(I)$.
	For any ascending chain
	$$0 \subset I_1 \subset I_2 \dots I_k \subset \dots \subset R$$
	Since $I_i \subset I_{i+1}$ we know that $d(I_i) \leq d(I_{i+1})$. Also
	notice that if $d(I_i) = d(I_{i+1})$ then $I_i = I_{i+1}$. This is because
	the set of generators in $I_i$ and $I_{i+1}$ must be the same since $I_i
	\subseteq I_{i+1}$. Thus for any $a \in I_{i + 1}$ we can write it as a
	linear combination of those generators in $I_{i+1}$ and thus $a$ is in
	$I_i$ as well. Thus we have a monotonic bounded sequence in the integers.
	$$0 \leq d(I_1) \leq d(I_2) \leq \dots d(I_k) \dots \leq n$$
	Thus it must be constant after some $N$. So $I_N = I_{N+1} = I_{N+2} \dots $ 
	the chain terminates. The same reasoning shows $R$ is artinian. For any
	descending chain
	$$R \supseteq I_1 \supseteq I_2 \dots I_k \supseteq  \dots \supset 0$$
	We have a monotonic bounded sequence
	$$n \geq d(I_1) \geq \dots d(I_k) \geq \dots 0$$
	And thus past some $N$ $d(I_k)$ is constant so $I_N = I_{N+1} = I_{N+2} \dots$
	the chain terminates.\\
	\\
	(2) \ $R/I$ is also a PID and thus Noetherian. We know PIDs are Noetherian
	since if we consider an infinite chain $0 \subseteq I_1 \subseteq I_2
	\subseteq \dots I_k \subseteq \dots R/I$ the union of all these ideals is an
	ideal (and principle) $J = \bigcup_{k \in \N} I_k = a(R/I)$. Thus $a$ must
	be in one of the $I_k$ and then the chain is constant past that ideal.\\
	We can do a similar argument for Artinian. If we have a descending chain
	$R/I \supseteq I_1/I \supseteq I_2/I \supseteq \dots I_k/I \supseteq \dots
	\supseteq I$ then the intersection of all these ideals is an ideal $J=
	\bigcap_{k\in\N} I_k = (a)$. We have that $I \subseteq J$ and therefore letting
	$I = (b) \neq 0$ we have that $a | b$ so $a \neq 0$. Thus since PIDs are
	UFDs, $a$ has a factorization 
	$$a = p_1^{n_1}p_2^{n_2} \dots p_k^{n_k}$$
	For each $I_i = (a_i)$ of our chain, we have that $a_i | a$ and $a_i |
	a_{i+1}$. Thus if we consider the number of prime factors of $a$ present in
	$a_i$ which we will denote as $d(a_i)$, we have a bounded monotonic
	sequence in $\N$
	$$d(a_1) \leq d(a_2) \leq \dots d(a_k) \leq \dots d(a)$$
	Thus it converges. So for some $N \in \N$, $d(a_N) = d(a_{N+1}) =
	d(a_{N+2}) \dots$ which means $a_N = a_{N+1} \dots  \Rightarrow I_N =
	I_{N+1} = I_{N+2} \dots $
\end{ques}

\begin{ques}{7.6}
	(1) \ If we consider any nonzero ideal $I$, then there is a unit $G \in I$.
	The reason for this is because if $B \in I$ with $B \neq 0$ then we can
	perform row operations (which works the same as in
	the vector space case) to multiply every column and row except for the row
	and column of some nonzero entry of $B$ by $0$. This new matrix $B'$ is
	still in $I$ since it is the product of row operation matricies with an
	element in $I$ and it has only one entry $b$ that is nonzero. Thus by
	performing column and row swaps we get matricies $B_1, B_2, B_3 \dots B_n \in I$
	where $B_i$ has the $(i,i)$ entry be the nonzero entry $b$ while every
	other entry is $0$. Thus $G = \sum_{i=1}^n B_i$ is the identity matrix
	multiplied by $b$. Since $D$ is a division ring $G$ is a unit since
	$b^{-1}$ multiplied by the identity matrix is the inverse of $G$. Thus $I = R$\\
	\\
	(2) \ It is clear that $I_k$ is closed under addition since we add
	component-wise so the columns that are not the $k$th column will stay zero.\\
	$I_k$ is a left ideal since for any $A \in R, B \in I_k$, 
	$$(AB)_{m,l} = \sum_{i = 0}^n A_{m,i}B_{i,l}$$
	Thus for every $l \neq k$ we get that $B_{i,l} = 0$ so $(AB)_{m,l} = 0$ so
	$AB$ has zero columns for every column that is not the $k$th column. Thus
	$AB \in I_k$. So $I_k$ is a left ideal.
\end{ques}

\end{document}
