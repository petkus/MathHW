\documentclass[12pt]{article}
\usepackage{amsmath, amssymb, amsthm, epsfig}

\newenvironment{definition}{\vspace{2 ex}{\noindent{\bf Definition}}}
        {\vspace{2 ex}}

\newenvironment{ques}[1]{\textbf{Exersise #1}\vspace{1 mm}\\ }{\bigskip}

\renewcommand{\theenumi}{\alph{enumi}}

\theoremstyle{definition}

\newenvironment{Proof}{\noindent {\sc Proof.}}{$\Box$ \vspace{2 ex}}
\newtheorem{Wp}{Writing Problem}
\newtheorem{Ep}{Extra Credit Problem}

\oddsidemargin-1mm
\evensidemargin-0mm
\textwidth6.5in
\topmargin-15mm
\textheight8.75in
\footskip27pt


\renewcommand{\l}{\left }
\renewcommand{\r}{\right }

\newcommand{\R}{\mathbb R}
\newcommand{\Q}{\mathbb Q}
\newcommand{\Z}{\mathbb Z}
\newcommand{\C}{\mathbb C}
\renewcommand{\H}{\mathbb H}

\newcommand{\s}{\sin}
\renewcommand{\c}{\cos}

\renewcommand{\t}{\theta}
\renewcommand{\a}{\alpha}

\newcommand{\norm}[1]{\left\lVert#1\right\rVert}

\newcommand{\T}{\mathcal{T}}

\newcommand{\Tor}{\text{Tor}}

\pagestyle{empty}
\begin{document}

\noindent \textit{\textbf{Math 504, Fall 2017}} \hspace{1.3cm}
\textit{\textbf{HOMEWORK $\#$6}} \hspace{1.3cm} \textit{\textbf{Peter
Gylys-Colwell}} 

\vspace{1cm}

\begin{ques}{6.1}
	(1) \ We know that degree $3$ polynomials are irreducible iff they have a
	root. We can use the rational root test to verify which polynomials have roots.\\
	A root of $x^3 + nx + 2$ must be $1, -1, 2$ or $-2$. We have that $1 + n + 2 =
	0 \Rightarrow n = -3$, $-1 - n + 2 = 0 \Rightarrow n = -1$, $2^3 + 2n + 2
	=0 \Rightarrow n = -5$ and $-2^3 - 2n + 2 = 0 \Rightarrow n = -3$. Thus
	$x^3 + nx + 2$ is irreducible iff $n\neq 1, -3, -5$\\
	\\
	(2) \ Over $\Z$, $x^8 - 1 = (x-1)(x + 1)(x^2 + 1)(x^4 + 1)$. We know that
	$x^2 + 1$ is irreducible since it has no roots, $x^4 + 1$ is irreducible
	since $(x-1)^4 + 1 = x^4 - 4x^3 + 6x^2 - 4x + 6$ is irreducible by
	eisenstein criteria. \\
	Over $\Z/2$, we have $x^8 - 1 = (x + 1)^8$\\
	Over $\Z/3$, $x^8 - 1 = (x-1)(x + 1)(x^2 + 1)(x^2 + x + 2)(x^2 + 2x + 2)$
	(a simple check of roots shows these deg 2 polynomials are irreducible)
\end{ques}

\begin{ques}{6.2}
	(1) \ If $n = m$ it is clear that $R^n \cong R^m$ since we can bijectively
	map the basis to each other. Since each element is a unique sum of the
	basis, our map will be bijective.\\
	If $n < m$, we can show that a surjective map $\varphi : R^n \to
	R^m$ is not possible. $\varphi$ is fully defined from where $\varphi$ sends
	the basis $g_1, g_2, \dots g_n$ to $R^m$. If $R^m$ has the basis $b_1,b_2, \dots
	b_m$, we have $\varphi(g_k) = \sum_i^m r_{i,k}b_i$ for $r_{i,k} \in R$, we can
	write this as an $n$ by $m$ matrix:
	$$A_{i,k} = [r_{i,k}]$$
	We can extend the domain of $\varphi$ from $R^n$ to $R^m$ by setting
	$\varphi(R^{m-n}) = 0$. Thus we would have $A$ extends to an $m \times m$
	matrix. However, since $\varphi$ is surjective, there exists a righthand
	inverse: $\rho : R^m \to R^n$ such that $\varphi \circ \rho = 1$. Thus our
	extension with $\rho: R^m \to R^m$, $\varphi: R^m \to R^m$ also satisfies
	$\varphi \circ \rho = 1$. If we consider the square matricies $B, A$
	for $\rho$ and $\varphi$ we have $$\text{det}(A) \text{det}(B) =
	\text{det}(\text{id}) = 1$$ However det$(A) = 0$ since it has a row of
	zeros and thus not a unit. Thus we have a contradiction.\\
	\\
	(2) \ Let $g_1, g_2$ be the generators of $R^2$ and $h$ the generator of
	$R$. We have that any $T \in$ End$_k(V)$ is of the form 
	$$T(k_1b_1 + k_2b_2 + \dots k_nb_n + \dots) =  k_1T(b_1) + k_2T(b_2) +
	\dots k_nT(b_n) + \dots $$
	Where the $b_i$s are the basis of $V$ and $k_i$s $\in k$.\\
	We can define the following isomorphism:\\
	$$\varphi(g_1) = T_1h, \varphi (g_2) = T_2 h$$
	Where $T_1(b_n) = b_n \forall n \in 2\Z$ and $T_1(b_n) = 0\  \forall n \in
	1 + 2\Z$, similarly $T_2(b_n) = b_n \forall n \in 1 + 2\Z$ and $T_2(b_n) =
	0\  \forall n \in 2\Z$. \\
	We have surjectivity since for any $Th \in R$ we have
	$$T = TT_1 + TT_2$$
	Thus $\varphi(Tg_1 + Tg_2) = Th$. We have injectivity since $TT_1 + TT_2 =
	FT_1 + FT_2 \Rightarrow T(b_1) = F(b_1), T(b_2) = F(b_2) \dots T(b_i) =
	F(b_i) \dots \Rightarrow T = F$

\end{ques}

\begin{ques}{6.3}
	(1) \ We can use the fact $M'' \cong M/M'$. If $M$ is noetherian it is
	clear that $M'$ is noetherian since it is isomorphic to a submodule of $M$.
	Submodules of noetherian modules are noetherian since any ascending chain
	of $M'$ is an ascending chain of $M$. Then we have that $M''$ is noetherian
	since any ascending chain in $M/M'$ has a corresponding ascending chain
	obtained from the cononical mapping $\pi : M/M' \to M$. We know that for
	any two submodules $N, N' \subset M/M'$ we have that $N \subset N'
	\Leftrightarrow \pi(N) \subset \pi(N')$, and thus a chain in $M'' = M/M'$
	terminates iff its image from $\pi$ terminates. \\
	Conversly, if $M''$ and $M'$ are noetherian then we have that any submodule
	of $M/M'$ is finitely generated and any submodule of $M'$ is finitely
	generated. We have that any submodule $S$ of $M$ is generated by the generators
	of $S \cap M'$ and the elements obtained by mapping generators of $S + M'
	\subset M/M'$ to one of their coset representatives.\\
	This is a generating set of $S$ since for any $s \in S$ we can write $s = m
	+ m'$ for $m \notin M', m' \in M'$ then we have that $m$ can be writen as a
	sum of generators obtained by coset representatives of generators of $S + M$
	with a difference of some elements in $S \cap M'$. Then we have the remaining
	elements are only in $S \cap M'$ and thus can be written as a sum of generators in
	$S \cap M'$.  Since this set of generators is finite (since $S \cap M'$ and
	$S + M$ are submodules of noetherian modules and thus finitely generated),
	$M$ is noetherian.\\
	\\
	(2) \ We can induct on the rank of the R-Modules. For rank $= 1$, we know
	that the only possible modules are ideals of $R$, which are noetherian.\\
	For a rank $n + 1$ R-Module $M$, we have that for a rank 1 sub module $R$,
	$M/R$ is a rank $n$ R-Module. Thus from our inductive hypothesis, since $R$
	and $M/R$ are noetherian, we know that $M$ is noetherian. 

\end{ques}

\begin{ques}{6.4}
	(1) \ If $rk(M) = n$, then we have a linear independent set $A = \{g_1,
	\dots g_n\} \subset M$. The submodule generated by $A$ is a free module
	$R^n$. If we consider the quotient $M/R^n$, every element $m$ not in $R^n$ (and
	thus not zero in $M/R^n$) cannot be written as a linear sum of elements in
	$A$. However we know that $\{m\} \cup A$ cannot be linearly independent
	since it would contradict $rk(M) = n$, therefore we have $r_mm = r_1g_1 +
	\dots r_ng_n$. Thus $r_mm = 0$ in $M/R^n$. So $M/R^n$ is torsion\\
	Conversly if $M/R^n$ is torsion then we know that $R^n$ is a submodule of
	$M$. Therefore there exists a set of $n$ independent elements in $M$ which
	are the generators of $R^n$. We have that there cannot exist any set of
	$n+1$ linearly independent elts of $M$ since they are all torsion in
	$M/R^n$, and thus when we multiply by appropriate $r \in R$ for each
	element we get $n+1$ elts in $R^n$. Since we know that any $n+1$ elements
	in $R^n$ are linearly dependent when $R$ is a PID we know we can find a
	non-zero linear combination of these elements to equal zero.\\
	\\
	(2) \ We have that sets of linearly independent elts $A = \{a_1, \dots
	a_n\} \subset M, B = \{b_1, \dots b_m\} \subset M'$ are linearly
	independent in $M \oplus M'$ as $A \times \{0\} \cup \{0\} \times B$. Thus
	$rk(M \oplus M') \geq rk(M) + rk(M')$. We have that $rk(M \oplus M') \leq
	rk(M) + rk(M')$ since if we have a set $G = \{(a_1,b_1), (a_2,b_2) \dots
	(a_{m+n+ 1}, b_{m + n + 1})\} \subset M \oplus M'$, since rank of $M < n +
	1$ there exists $r_i$s $\in R$ such that 
	$$ r_1a_1 + \dots r_na_n + r_{n+1}a_{m+n+1} = 0$$  
	Let us define $g =  r_1(a_1,b_1) + \dots r_n(a_n,b_n) +
	r_{n+1}(a_{m+n+1},b_{m+n+1})$. We have that $g = (0, b)$ for some $b \in M'$. Now
	in $M'$ we have that
	$$r_{n+1}b_{n+1} + \dots r_{n+m}b_{n+m} +
	r_{m+n+1}b = 0 $$
	Thus we can define $h = (r_{n+1}(a_{n+1},b_{n+1}) + \dots r_{n+m}(a_{n+m},
	b_{n+m}) + r_{m+n+1}g$ with $h = (a,0)$ for some $a \in M$. Thus we have
	that $g \cdot h = 0$ which is a linear combination of $(a_1,b_1) \dots
	(a_{n+m+1},b_{n+m+1})$\\
	\\
	(3) \ We have that any $(m,m') \in M \oplus M'$ we have that for any $r \in
	R$ we have that $r(m,m') = 0$ iff $rm = 0, rm' = 0$. Thus $(m,m') \in
	\Tor_R(M \oplus M')$ iff $m \in \Tor_R(M)$, $m' \in \Tor_R(M')$. So we have
	the cononical mapping $\pi: \Tor_R(M \oplus M') \equiv \Tor_R(M) \oplus
	\Tor_R(M')$ where $\pi(m,m') = (m,m')$ is an isomorphism.
\end{ques}

\begin{ques}{6.5}
	(1) \ Let the generators of $M$ be $g_1, \dots g_n$. Since $M$ is torsion
	there exists $r_1, \dots r_n \in R, r_i \neq 0$ where $r_1g_1 = r_2g_2 = \dots
	r_ng_n = 0$.  Thus for any $m \in M$ we have $(r_1r_2r_3 \dots r_n)m = 0$
	since $m$ is linear sum of elts of $g_1, \dots g_n$ and since $R$ is
	commutative we can rearrange so that $r_i$ multiplies with $g_i$ in each
	term. Thus $(r_1r_2 \dots r_n) \in \text{Ann}(M)$, and $(r_1r_2 \dots r_n)
	\neq 0$ since $R$ is an ID\\
	\\
	(2) \ Let $R = \Z$ and 
	$$S = \Z/2 \oplus \Z/4 \oplus \Z/8 \oplus \dots \Z/2^n \oplus \dots$$
	$M$ is defined as the set of finite tuples in $S$.  Thus we have that every
	element in $M$ is torsion since for any $m \in M$, there is an $N$ such
	that for $n > N$ the $n$th component of $m$ is zero. Thus $2^Nm = 0$.
	However ann$(R) = 0$ since for any $2^n \in R$ we have that the element $m$
	with $n+1$ component $1$ multiplies with $2^n$ to yield $2^n \neq 0$ in the
	$n+1$ th component. 
\end{ques}

\begin{ques}{6.6}
	If we consider the free module $R$, we know that every submodule is an
	ideal and thus from our assumption every ideal $I \subset R$ is free.
	Therefore for each $I$ there is a generating set $A$. $A$ can only have one
	element since the rank of $R$ is $1$ so every submodule has rank $1$. Thus
	$I = (a)R$ so $R$ is a PID
\end{ques}

\end{document}
