\documentclass[12pt]{article}
\usepackage{amsmath, amssymb, amsthm, epsfig}

\newenvironment{definition}{\vspace{2 ex}{\noindent{\bf Definition}}}
        {\vspace{2 ex}}

\newenvironment{ques}[1]{\textbf{Exersise #1}\vspace{1 mm}\\ }{\bigskip}

\renewcommand{\theenumi}{\alph{enumi}}

\theoremstyle{definition}

\newenvironment{Proof}{\noindent {\sc Proof.}}{$\Box$ \vspace{2 ex}}
\newtheorem{Wp}{Writing Problem}
\newtheorem{Ep}{Extra Credit Problem}

\oddsidemargin-1mm
\evensidemargin-0mm
\textwidth6.5in
\topmargin-15mm
\textheight8.75in
\footskip27pt


\renewcommand{\l}{\left }
\renewcommand{\r}{\right }

\newcommand{\R}{\mathbb R}
\newcommand{\Q}{\mathbb Q}
\newcommand{\Z}{\mathbb Z}
\newcommand{\C}{\mathbb C}
\renewcommand{\H}{\mathbb H}

\newcommand{\s}{\sin}
\renewcommand{\c}{\cos}

\renewcommand{\t}{\theta}
\renewcommand{\a}{\alpha}

\newcommand{\norm}[1]{\left\lVert#1\right\rVert}

\newcommand{\T}{\mathcal{T}}

\pagestyle{empty}
\begin{document}

\noindent \textit{\textbf{Math 504, Fall 2017}} \hspace{1.3cm}
\textit{\textbf{HOMEWORK $\#$6}} \hspace{1.3cm} \textit{\textbf{Peter
Gylys-Colwell}} 

\vspace{1cm}

\begin{ques}{6.1}
	(1) \ We know that degree $3$ polinomials are irreducible iff they have a
	root. We can use the rational root test to verify which polinomials have roots.\\
	A root of $x^3 + nx + 2$ must be $1, -1, 2$ or $-2$. We have that $1 + n + 2 =
	0 \Rightarrow n = -3$, $-1 - n + 2 = 0 \Rightarrow n = -1$, $2^3 + 2n + 2
	=0 \Rightarrow n = -5$ and $-2^3 - 2n + 2 = 0 \Rightarrow n = -3$. Thus
	$x^3 + nx + 2$ is irreducible iff $n\neq 1, -3, -5$\\
	\\
	(2) \ Over $\Z$, $x^8 - 1 = (x-1)(x + 1)(x^2 + 1)(x^4 + 1)$. We know that
	$x^2 + 1$ is irreducible since it has no roots, $x^4 + 1$ is irreducible since
\end{ques}

\begin{ques}{6.2}
	(1) \ If $n = m$ it is clear that $R^n \cong R^m$ since they are the same.
	If $n \neq m$ then we know that isomorphisms preserve the the rank of free
	modules (which is well defined when $R$ is commutative), and since they
	have different rank, there cannot exist an isomorphism between the two
	modules\\
	\\
	(2) \ 
\end{ques}

\begin{ques}{6.3}
\end{ques}

\begin{ques}{6.4}
\end{ques}

\begin{ques}{6.5}
\end{ques}

\begin{ques}{6.6}
\end{ques}

\end{document}
