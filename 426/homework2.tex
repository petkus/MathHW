\documentclass[12pt]{article}
\usepackage{amsmath, amssymb, amsthm, epsfig}
\setlength\parindent{0pt}

\newenvironment{definition}{\vspace{2 ex}{\noindent{\bf Definition}}}
        {\vspace{2 ex}}

\newenvironment{ques}[1]{\textbf{#1}\vspace{1 mm}\\ }{\bigskip}

\renewcommand{\theenumi}{\alph{enumi}}

\theoremstyle{definition}

\newenvironment{Proof}{\noindent {\sc Proof.}}{$\Box$ \vspace{2 ex}}
\newtheorem{Wp}{Writing Problem}
\newtheorem{Ep}{Extra Credit Problem}

\oddsidemargin-1mm
\evensidemargin-0mm
\textwidth6.5in
\topmargin-15mm
\textheight8.75in
\footskip27pt


\renewcommand{\l}{\left }
\renewcommand{\r}{\right }

\newcommand{\R}{\mathbb R}
\newcommand{\Q}{\mathbb Q}
\newcommand{\Z}{\mathbb Z}
\newcommand{\C}{\mathbb C}
\newcommand{\N}{\mathbb N}
\renewcommand{\i}{\text{int} \ }
\newcommand{\interior}[1]{%
  {\kern0pt#1}^{\mathrm{o}}%
}

\renewcommand{\sup}{\text{sup}}
\newcommand{\osc}{\text{osc}}
\newcommand{\diam}{\text{diam} \ }
\renewcommand{\-}{\backslash}

\newcommand{\s}{\sin}
\renewcommand{\c}{\cos}

\renewcommand{\t}{\theta}
\renewcommand{\a}{\alpha}

\newcommand{\norm}[1]{\left\lVert#1\right\rVert}

\newcommand{\T}{\mathfrak{T}}

\newcommand{\dist}{\text{dist}}

\pagestyle{empty}
\begin{document}

\noindent \textit{\textbf{Math 426, SPRING 2018}} \hspace{1.3cm}
\textit{\textbf{HOMEWORK $\#$2}} \hspace{1.3cm} \textit{\textbf{Peter
Gylys-Colwell}} 

\vspace{1cm}
\begin{ques}{2.4 18}
	(i) By definition of outer measure (where it is the infimum of measure of
	all open interval coverings) for any $\epsilon >0$ there exists a covering
	of $E$ by open intervals $I_k$
	$$E \subseteq \bigcup I_k = U$$
	with
	$$m^*(U) -m^*(E) < \epsilon$$
	Let $U_\epsilon$ denote such an open set. We have the $G_\delta$ set
	$$G = \bigcap_{n=1}^\infty U_{1/n}$$
	Notice that since $E \subseteq U_{1/n}$ for all $n$, $E \subseteq G$. Also
	notice $G\subseteq U_{1/n}$ for all $n$. Thus for every $n> 0$
	$$m^{*}(E) \le m^{*}(G) \le m^{*}(E) + \frac 1 n$$
	Thus $m^*(G) = m^*(E)$\\
	\\
	(ii)
	($\Rightarrow$): From Theorem 11 there exists $F_\sigma$ set $F$ where
	$$F \subseteq E$$
	$$m^*(E\backslash F) = 0$$
	Since $F$ is a measurable set from disjoint union additivity (which holds if only
	one of the disjoint sets is measurable) we have the desired result
	$$m^*(E) = m^*(F \cup (E \backslash F)) = m^*(F) + m^*(E \backslash F) = m^*(F)$$
	($\Leftarrow$): If there exists an $F_\sigma$ set $F \subseteq E$ where
	$m^*(F) = m^*(E)$ then since $F$ is measurable
	$$m^*(E) = m^*(F \cup (E \backslash F)) = m^*(F) + m^*(E \backslash F)$$
	and thus $m^*(E \backslash F) = 0$. We have that all sets of outer measure
	$0$ are measurable and thus $E$ is the union of measurable sets (and thus
	measurable)
	$$E = F \cup (E \backslash F)$$
\end{ques}

\begin{ques}{2.4 19}
	From problem 18 we know that there is a $G_\delta$ set $G =
	\cap_{k=1}^\infty U_k$ where 
	$$\lim_{N \to \infty} m^*\l(\bigcap_{k=1}^N U_k\r) = m^{*}(G) = m^*(E)$$
	Thus for any $\epsilon > 0$ and sufficiently large $N$ we have
	$$m^*\l(\bigcap_{k=1}^N U_k\r) < m^{*}(E) + \epsilon$$
	Letting $\mathcal O_N = \bigcap_{k=1}^N U_k$ be this open set we have
	$$m^*\l(\mathcal O_N \r) - m^{*}(E) < \epsilon$$
	From subadditivity 
	$$m^*(\mathcal O_N \backslash E) \geq m^*(\mathcal O_N) - m^*(E)$$
	If we had equality for every $\mathcal O_N$ however, for every $\epsilon > 0$
	we can choose a $\mathcal O_N$ as described above so that
	$$m^*(\mathcal O_N \backslash E) < \epsilon$$
	which from Theorem 11 establishes $E$ to be measurable. Thus equality
	cannot be the case for each $\mathcal O_N$
\end{ques}

\begin{ques}{2.4 20}
	($\Leftarrow$): This follows directly from the definition of measurable
	with the set in question $(a,b)$\\
	($\Rightarrow$): From the definition of outer measuer for any $\epsilon >
	0$ we can cover $E$ by a countable collection of open intervals $I_k$ such that
	$$\sum_{k=1}^\infty m^*(I_k) < m^*(E) + \epsilon$$
	From hypothesis
	$$\sum_{k=1}^\infty m^*(I_k) = \sum_{k=1}^\infty m^*(E \cap I_k) + m^*(I_k
	\backslash E) \geq m^*\l(\bigcup_{k=1}^\infty E \cap I_k \r) +
	m^*\l(\bigcup_{k=1}^\infty I_k \backslash E \r)$$
	Denoting $U = \bigcup_{k=1}^\infty I_k$ the open set, the above statement
	is equal to 
	$$m^*(U \cap E) + m^*(U \backslash E)$$
	since $E \subseteq U$, $U \cap E = E$ so tracing back from our first
	inequality yields
	$$m^*(E) + \epsilon > m^*(E) + m^*(U \backslash E)$$
	Thus $m^*(U \backslash E) < \epsilon$ which from Theorem 11 establishes $E$
	to be measurable
\end{ques}

\begin{ques}{2.5 25}
	Consider the example 
	$$B_1 = \R$$
	$$B_n = (n, \infty)$$
	We have that
	$$\bigcap_{n=1}^\infty B_n = \emptyset \Rightarrow
	m^*\l(\bigcap_{n=1}^\infty B_n\r) = 0$$
	while for each $B_n$, $m^*(B_n) = \infty$ so
	$$\lim_{n \to \infty} m^*(B_n) = \infty$$
\end{ques}

\begin{ques}{2.5 26}
	We have that 
	$$A \cap  \bigcup_{k=1}^\infty E_k = \bigcup_{k=1}^\infty A \cap E_k$$
	From subadditivity we have
	$$m^*\l(\bigcup_{k=1}^\infty A \cap E_k\r) \leq \sum_{k=1}^\infty m^*(A
	\cap E_k)$$
	We have that for all $N > 0$,
	$$m^*\l(\bigcup_{k=1}^\infty A \cap E_k\r) \geq m^*\l(\bigcup_{k=1}^N A
	\cap E_k \r) = \sum_{k=1}^N m^*(A \cap E_k)$$
	The second equality follows from Proposition 6. Since this is the case for
	all $N$, we can conclude
	$$m^*\l(\bigcup_{k=1}^\infty A \cap E_k\r) \geq \sum_{k=1}^\infty m^*(A
	\cap E_k)$$
	Thus since we have light inequalities both ways, we can conclude equality.\\
\end{ques}

\begin{ques}{2.5 28}
	For any countable collection of disjoint measurable sets $\{E_1, E_2, \dots\}$ 
	we can form an ascending chain of measurable sets by setting 
	$$A_k = \bigcup_{i=1}^k E_i$$
	We have that
	$$m^*\l(\bigcup_{s =1}^k E_s\r) = m^*\l(\bigcup_{s =1}^k A_s\r) = m^*(A_k)
	= \sum_{i=1}^k m^*(E_i)$$
	And thus from the continuity property
	$$m^*\l(\bigcup_{s =1}^\infty E_s\r) = m^*\l(\bigcup_{s =1}^\infty A_s\r) =
	\lim_{k \to \infty} m^*(A_k) = \sum_{i=1}^\infty m^*(E_i)$$
\end{ques}
\end{document}
