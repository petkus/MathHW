\documentclass[12pt]{article}
\usepackage{amsmath, amssymb, amsthm, epsfig}
\setlength\parindent{0pt}

\newenvironment{definition}{\vspace{2 ex}{\noindent{\bf Definition}}}
        {\vspace{2 ex}}

\newenvironment{ques}[1]{\textbf{#1}\vspace{1 mm}\\ }{\bigskip}

\renewcommand{\theenumi}{\alph{enumi}}

\theoremstyle{definition}

\newenvironment{Proof}{\noindent {\sc Proof.}}{$\Box$ \vspace{2 ex}}
\newtheorem{Wp}{Writing Problem}
\newtheorem{Ep}{Extra Credit Problem}

\oddsidemargin-1mm
\evensidemargin-0mm
\textwidth6.5in
\topmargin-15mm
\textheight8.75in
\footskip27pt


\renewcommand{\l}{\left }
\renewcommand{\r}{\right }

\newcommand{\R}{\mathbb R}
\newcommand{\Q}{\mathbb Q}
\newcommand{\Z}{\mathbb Z}
\newcommand{\C}{\mathbb C}
\newcommand{\N}{\mathbb N}
\renewcommand{\i}{\text{int} \ }
\newcommand{\interior}[1]{%
  {\kern0pt#1}^{\mathrm{o}}%
}

\renewcommand{\sup}{\text{sup}}
\newcommand{\osc}{\text{osc}}
\newcommand{\diam}{\text{diam} \ }
\renewcommand{\-}{\backslash}

\newcommand{\s}{\sin}
\renewcommand{\c}{\cos}

\renewcommand{\t}{\theta}
\renewcommand{\a}{\alpha}

\newcommand{\norm}[1]{\left\lVert#1\right\rVert}

\newcommand{\T}{\mathfrak{T}}

\newcommand{\dist}{\text{dist}}

\pagestyle{empty}
\begin{document}

\noindent \textit{\textbf{Math 426, SPRING 2018}} \hspace{1.3cm}
\textit{\textbf{HOMEWORK $\#$4}} \hspace{1.3cm} \textit{\textbf{Peter
Gylys-Colwell}} 

\vspace{1cm}
\begin{ques}{3.2 22}
	As the hint suggests, for any $\epsilon > 0$ define 
	$$E_n = \{x \in [a,b] | f(x)- f_n(x) < \epsilon \}$$
	Notice that $E_n$ is the preimage of $f_n$ of a ball around $f(x)$: $E_n =
	f_n^{-1}(B_\epsilon(f(x)))$ and thus is open. Next notice that if $m > n$
	then $E_n \subseteq E_m$. This is true since $f(x) \geq f_m(x) > f_n(x)$ so
	if $f(x) - f_n(x) > \epsilon$ then $f(x) - f_m(x) > \epsilon$\\
	Finally notice that the collection of $E_n$s cover $[a,b]$. This is true
	since for every $x \in [a,b]$ there is $N$ such that $f(x) - f_N(x) <
	\epsilon$ so $x \in E_N$. Thus since $[a,b]$ is compact we have a finite
	subcovering 
	$$E_{n_1} \dots E_{n_k}$$
	Since the largest indexed term $E_N$ in the subcovering contains all the other
	$E_{n_i}$ we have that $[a,b] \subseteq \bigcup E_{n_i} \subseteq E_N$.
	Thus for all $n > N$ we have the desired result $E_n \supseteq E_N
	\supseteq [a,b]$ and $f(x) - f_n(x) < \epsilon$ for all $n > N$
\end{ques}

\begin{ques}{3.2 23}
	Given measurable function $f(x)$, define $E_1 = f^{-1}([0,\infty]), E_2 =
	f^{-1}([-\infty, 0))$. We can now define the nonegative measurable
	functions $g_1(x) = \chi_{E_1}(x)f(x)$, $g_2(x) = -\chi_{E_2}(x)f(x)$.
	Since $E_1$ and $E_2$ are disjoint and cover all the domain of $f$ we get
	$$f(x) = g_1 - g_2$$
	From what was proven about nonegative measurable functions we have a
	sequence of simple functions with pointwise convergence $\phi_n \to g_1,
	\psi_n \to g_2$ and thus we have the sequence of sums of simple functions (which is
	again simple)
	$$\phi_n + \psi_n \to f$$
\end{ques}

\begin{ques}{3.3 25}
	We can express $\R \backslash F$ as the countable disjoint union of open
	intervals $I_n$.\\
	We will extend $f$ to all of $\R$ by using the following definition on each
	$I_n = (a_n,b_n)$, for $x \in I_n$ define 
	$$f(x) = \frac{f(b_n) - f(a_n)}{b_n-a_n}(x - a_n) + f(a_n)$$
	Notice this is the equation for a line with points $(a_n,f(a_n)), (b_n, f(b_n))$.\\
	If $I_n = (-\infty, p)$ or $= (p, \infty)$ define $f(x) = f(p)$ on $I_n$.\\
	We have that this extension is continuous on all of $\R$. It is clearly
	continuous on the interior of each $I_n$ and the interior of $F$ (since
	locally $f$ is a continous function on $F^i$ and $I_n$). On any boundary
	point $b \in F$ we have that $b$ is the boundary of some $I_n$. Thus either
	$I_n = (b, p)$ or $(p,b)$ either way, from our definition of the extension
	notice that $\lim_{x \to b^+}f(x) = f(b)$ and $\lim_{x \to b^-} f(x) =
	f(b)$ and thus $f$ is continous at $b$
\end{ques}

\begin{ques}{3.3 29}
	We can write $E$ as a countable disjoint union of measurable sets:
	$$E = \bigcup_{n \in \Z} E \cap (n, n+1]$$
	Calling $E_n$ each of these measurable sets, since $E_n$ has finite measure
	we can use Lusin's result in the finite measure case to say there is a
	$g_n$ defined on $F_n \subset E_n$ where $f = g_n$ on $F_n$ and $m(E_n
	\backslash F_n) \leq \epsilon / 2^n$ we can define the desired $F$ as
	$$F = \bigcup F_n$$
	We have that
	$$m(E \backslash F) = m(E \backslash \cup F_n) = \sum m(E_n \backslash F_n)
	\leq \sum \epsilon / 2^n = \epsilon$$
	with the desired $g$ defined as $g(x) = g_n(x)$ for $x \in E_n$. All that
	is left is to show $F$ is closed. For any limit point
	$p_n \to p$ of $F$,  $p \in (n,n+1]$ for some $n$, if $p \in (n,n+1)$ then
	it must be the case that for sufficiently large $n$, $p_n \in F_n$ so $p
	\in F_n$. If $p = n+1$ then the subsequence of $p_n$ contained in $F_n$
	converges to $p$ so is contained in $p_n$. 
\end{ques}

\begin{ques}{3.3 31}
	We can use Egoroff's Theorem to obtain for $k > 2$, measurable $E_k \subset
	E$ where $m(E \backslash E_k) < 1/k$ and $f_n$ converges uniformly on
	$E_k$. We will define 
	$$E_1 = E \backslash \bigcup_{k=2}^\infty E_k$$
	We have that 
	$$m(E_1) = m\l(\bigcap_{k=2}^\infty E \backslash E_k\r) = \lim_{k \to
	\infty} 1/k = 0$$
	and finally it is clear from how $E_1$ was constructed $E =
	\cup_{k=1}^\infty E_k$. 
\end{ques}

\begin{ques}{4.2 12}
	Let $F \subset E$ be the set where $f(x) = g(x)$ and $m(E \backslash F) =
	0$. Letting $F' = E \backslash F$, we have $f(x) = \chi_F g(x) + \chi_{F'}
	f(x)$. Integration yields
	$$\int_E f = \int_F g + \int_{F'} f$$
	Since $f$ is bounded there is some value $B$ where $|f(x)| < B$ for all $x
	\in F'$ and thus
	$$\l|\int_{F'} f \r| \leq \int_{F'} B = Bm(F') = 0$$
	Similarly since $g$ is bounded there is some $B'$ where $|g(x)| < B'$ and
	$$\int_E g = \int_F g + \int_{F'} g $$
	$\l|\int_{F'} g \r| \leq B'm(F') = 0$ and thus
	$$\int_E f = \int_F g = \int_E g$$
\end{ques}

\begin{ques}{4.2 16}
	Let $F = \{x \in E| f(x) \neq 0\}$ and define $F_n = \{x \in E| f(x) >
	1/n\}$. We have that $F = \cup_{n=1}^\infty F_n$ and thus
	$$m\l(F\r) = m\l(\bigcup_{n=1}^\infty F_n\r) = \sum_{n=1}^\infty m(F_n)$$
	Thus $m(F)=0$ iff $m(F_n)=0$ for all $n$. If some $m(F_n) > 0$ then we have
	$$0 = \int_E f = \int_{E \backslash F_n} f + \int_{F_n} f$$
	$\int_{E \backslash F_n} f \geq 0 $ since $f \geq 0$ and so
	$$\geq \int_{F_n} f \geq \int_{F_n} \frac 1 n = m(F_n)\frac 1 n > 0$$
	which is a contradiction, thus $m(F) = 0$
\end{ques}

\begin{ques}{4.3 24}
	(i) From the simple aproximation Theorem we can get an increasing sequence of
	simple functions to converge pointwise to $f$ on $E$: 
	$$\varphi_n \to f$$
	For each $\varphi_n$ we have an increasing sequence of simple functions
	$\psi_{kn} \to \varphi_n$
	with finite support defined as
	$$\psi_{kn} = \chi_{B_k}\varphi_n$$
	where $B_k$ is the ball of radius $k$ (thus the support can be at most $2k$)\\
	We have that the sequence of simple functions with finite support defined
	as $\psi_{nn}$ converges to $f$ pointwise. For any $x$ we have that there
	is a $N$ such that $f(x) - \varphi_n(x) < \epsilon$ for all $n > N$ and for
	$K$ sufficiently large, $x \in B_K$ so for $M = \max\{N,K\}$ we have that
	$f(x) - \psi_{m,m}(x) < \epsilon$ for all $m > M$\\
	\\
	(ii) We have that 
	$$\int_E f = \sup\l\{\int_E h | h \text{ bounded, measurable, of finite
	support and } 0 \leq h \leq f\r\}$$
	thus since we are taking a sup over a larger set
	$$\int_E f \geq \sup\l\{\int_E \varphi | \varphi \text{ simple, of finite
	support and } 0 \leq \varphi \leq f\r\}$$
	From $(i)$ we have a sequence $\varphi_n$ of simple and finite support
	functions that converge pointwise to $f$. From Fatous Lemma we have
	$$\int_E f \leq \lim \inf \int_E \varphi_n \leq \sup\l\{\int_E \varphi |
	\varphi \text{ simple, of finite support and } 0 \leq \varphi \leq f\r\}$$
	And thus we have equality
\end{ques}

\begin{ques}{4.3 26}
	consider the sequence of functions over $E = \R$
	$$f_n(x) = 1/n|x|$$
	This is a decreasing sequence of functions which converge pointwise to $0$, however
	$$\int_\R f_n = \infty$$
	for all $n$ so 
	$$\lim_{n \to \infty} \int_\R f_n = \infty \neq \int_\R 0$$
\end{ques}
\end{document}
