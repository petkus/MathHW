\documentclass[12pt]{article}
\usepackage{amsmath, amssymb, amsthm, epsfig}
\setlength\parindent{0pt}

\newenvironment{definition}{\vspace{2 ex}{\noindent{\bf Definition}}}
        {\vspace{2 ex}}

\newenvironment{ques}[1]{\textbf{#1}\vspace{1 mm}\\ }{\bigskip}

\renewcommand{\theenumi}{\alph{enumi}}

\theoremstyle{definition}

\newenvironment{Proof}{\noindent {\sc Proof.}}{$\Box$ \vspace{2 ex}}
\newtheorem{Wp}{Writing Problem}
\newtheorem{Ep}{Extra Credit Problem}

\oddsidemargin-1mm
\evensidemargin-0mm
\textwidth6.5in
\topmargin-15mm
\textheight8.75in
\footskip27pt


\renewcommand{\l}{\left }
\renewcommand{\r}{\right }

\newcommand{\R}{\mathbb R}
\newcommand{\Q}{\mathbb Q}
\newcommand{\Z}{\mathbb Z}
\newcommand{\C}{\mathbb C}
\newcommand{\N}{\mathbb N}
\renewcommand{\i}{\text{int} \ }
\newcommand{\interior}[1]{%
  {\kern0pt#1}^{\mathrm{o}}%
}

\renewcommand{\sup}{\text{sup}}
\newcommand{\osc}{\text{osc}}
\newcommand{\diam}{\text{diam} \ }
\renewcommand{\-}{\backslash}

\newcommand{\s}{\sin}
\renewcommand{\c}{\cos}

\renewcommand{\t}{\theta}
\renewcommand{\a}{\alpha}

\newcommand{\norm}[1]{\left\lVert#1\right\rVert}

\newcommand{\T}{\mathfrak{T}}

\newcommand{\dist}{\text{dist}}

\pagestyle{empty}
\begin{document}

\noindent \textit{\textbf{Math 426, SPRING 2018}} \hspace{1.3cm}
\textit{\textbf{HOMEWORK $\#$3}} \hspace{1.3cm} \textit{\textbf{Peter
Gylys-Colwell}} 

\vspace{1cm}
\begin{ques}{2.6 30}
	We know that any choice set is not measurable, we also know that every set
	which is countable or finite is measurable. Thus any choice set must be uncountable.
\end{ques}

\begin{ques}{2.7 34}
	Consider the function $\Psi^{-1}$ which is the inverse of the function
	$\Psi(x) = \varphi(x) + x$ where $\varphi$ is the Cantor-Lebesgue function.
	This inverse function is well defined since $\Psi$ is a bijection.
	$\Psi^{-1}$ is continuous since it is a strictly increasing surjection (it
	is strictly increasing since the inverse of a strictly increasing function
	is strictly increasing). By scaling the function $F(x) = \Psi(2x)$ we have
	a function defined on $[0,1]$ which maps a measurable set of positive
	measure onto the Cantor set
\end{ques}

\begin{ques}{2.7 37 OLD EDITION}
	It is not true, as illustrated by the function $\Psi^{-1}(x)$ with $E =
	[0,2]$ described in 34. We have that the pull back of $\Psi^{-1}$
	of a subset of the Cantor set is a nonmeasurable set (the subset of the
	Cantor set is measurable since the Cantor set is measure $0$)
\end{ques}

\begin{ques}{2.7 37}
	Let $B$ be a measure zero set. For any $\epsilon > 0$ we can cover $B$ with
	intervals $I_k$ such that $\sum_{k =1} ^\infty \ell(I_k) < \epsilon / c$.
	Notice that a implication of lipchitz is
	$$m^*(f(I_k)) \leq c \ell(I_k)$$
	The reason for this is if $I_k = (a,b)$, letting $x = \frac{a + b}2$ we
	have $I_k$ is a ball of radius $r = \ell(I_k)/2$ around $x$
	$$f(I_k) = f(B_r(x)) \subseteq B_{cr}(f(x))$$
	and $B_{cr}(f(x))$ is an interval of length $c\ell(I_k)$
	\\
	From this we have
	$$m^*(f(B)) \leq m^*\l(f\l( \bigcup_{k=1}^\infty I_k \r)\r)$$
	$$= m^*\l( \bigcup_{k=1}^\infty f\l( I_k \r)\r) \leq \sum_{k=1}^\infty
	m^*(f(I_k)) \leq \sum_{k=1}^\infty c \ell(I_k) < \epsilon$$
	And thus $f(B)$ is measure zero\\
	Let $F$ be a $F_\sigma$ set. We have that $F$ is a countable union of closed sets
	$$F = \bigcup_{i=1}^\infty C_i$$
	We have
	$$f(F) = \bigcup_{i=1}^\infty f(C_i)$$
	It is the case that every lipchitz function is closed and thus $f(C_i)$ is
	closed so $f(F)$ is a $F_\sigma$ set.\\
	To show that Lipshitz $\Rightarrow$ closed:\\
	for an limit point $y \in f(C)$ we have that $f(x_i) \in C$ converges to
	$y$. Thus $f(x_i)$ is Cauchy and by the lipchitz condition we get that
	$x_i$ is Cauchy and thus has a limit $x \in C$. Thus we have that $f(x) = y$ so
	$y \in f(C)$\\
	Thus since every measurable set $M = F \cup B$ is a union of a
	$F_\sigma$ set and a zero set we have that
	$$f(M) = f(F) \cup f(B)$$
	is also a union of a $F_\sigma$ set and a zero set and thus measurable

\end{ques}

\begin{ques}{2.7 42 OLD EDITION}
	Suppose for contradiction we have an enumeration of $X:$
	$$X = \{x_1, x_2, x_3, \dots \}$$
	We can construct an element in $X$ that is not enumerated in such a fashion:\\
	Choose any two elements $a_1, a_2 \in X$ and construct the disjoint closed balls
	$B_1, B_2$ around $a_1, a_2$ (whose radius would be $< |a_1-a_2|/2$). $x_1$
	cannot be in both sets so denote $C_1$ as the set $x_1$ is not in. Now
	choose $a_1,a_2$ in the interior of $C_1$ and again construct the disjoint
	closed balls $B_1, B_2$ of $a_1,a_2$ which are also contained in the
	interior of $C_1$. Again $x_2$ cannot be in both sets so denote $C_2$ as
	this closed ball.\\
	In general given a closed ball $C_n$ we choose two
	points $a_1,a_2 \in \i C_n$ and get disjoint balls $B_1, B_2$ centered
	around each point. We will denote $C_{n+1}$ as the ball which $x_{n+1}$ is
	not in\\
	From this we have the nested closed sets $C_1 \subset C_2 \subset C_3
	\subset \dots$ which cannot be closed when taking the intersection:
	$$\exists x \in \bigcap_{n=1}^\infty C_n$$
	$x$ was not in the enumeration since for every $n$, $x \in C_n, x_n \notin C_n$
\end{ques}

\begin{ques}{2.7 42}
	We know that the inverse image preserves $\sigma$-algebra operations:
	$$f^{-1}(A \cup B) = f^{-1}(A) \cup f^{-1}(B)$$
	$$f^{-1}(\cup_{i=1}^\infty A_i) = \cup_{i=1}^\infty f^{-1}(A_i)$$
	$$f^{-1}(A^c) = f^{-1}(A)^c$$
	Thus since every borel set $B$ is obtained by various $\sigma$-algebra
	operations of open sets $U_i$ and $f^{-1}$ preserves open sets, we have that
	$f^{-1}(B)$ is obtained by various $\sigma$-algebra operations of open sets
	$f^{-1}(U_i)$ and thus Borel
\end{ques}

\begin{ques}{3.1 2}
	This is not the case. Consider the example
	$$D = [0.1] \cap \Q , E = [0,1] - \Q$$
	$D, E$ are measurable since $m^*(D) = 0$ and $E = [0,1] - D$. Consider the function
	$$f(x) = 
	\begin{cases}
	1 & x \in D\\
	0 & x \in E
	\end{cases}$$
	$f$ is continuous on $D$ and on $E$ but not continuous on $D \cup E =
	[0,1]$. 
\end{ques}

\begin{ques}{3.1 4}
	No as illustrated by the following counterexample. We know that $\Psi^{-1}$
	described in problem 34 is not a measurable function.  It is however one-to-one
	and thus $(\Psi^{-1})^{-1}(c)$ is just a point and so measurable.
\end{ques}

\begin{ques}{3.1 8}
	(i)
	We know that the set of Borel sets are a subset of the set of Lebesgue
	sets. Thus the definition have direct implications: $E$ is Lebesgue if
	Borel and $\{x \in E|f(x) > c\}$ is Lebesgue if it is Borel\\
	\\
	(ii) We know that the inverse image preserves $\sigma$-algebra operations:
	$$f^{-1}(A \cup B) = f^{-1}(A) \cup f^{-1}(B)$$
	$$f^{-1}(A^c) = f^{-1}(A)^c$$
	We have that the Borel sets are generated by sets of the form $(c, \infty)$
	and thus $f^{-1}(B)$ is a set obtained by $\sigma$-algebra operations of
	sets of the form $f^{-1}(c, \infty)$ which are Borel sets since $f$ is
	Borel measurable. Thus $f^{-1}(B)$ is a Borel set\\
	\\
	(iii) We have that $(f \circ g)^{-1}(c, \infty) = g^{-1}(f^{-1}(c, \infty))
	= f^{-1}(B)$. Since $f$ is Borel measurable, $f^{-1}(c, \infty) = B$ is a
	Borel set. From (ii) $g^{-1}(B)$ is Borel and thus $f \circ g$ is Borel
	measurable\\
	\\
	(iv) We have that $(f \circ g)^{-1}(c, \infty) = g^{-1}(f^{-1}(c, \infty))
	= g^{-1}(B)$. Since $f$ is Borel measurable, $f^{-1}(c, \infty) = B$ is a
	Borel set. Notice that the argument for (ii) can be applied for Lebesgue
	measurable functions as well and so $g^{-1}(B)$ is Lebesgue. Thus $f
	\circ g$ is Lebesgue measurable
\end{ques}

\begin{ques}{3.1 10}
	No, consider $f$ is the identity function and $g$ is continuous but not
	measurable (for instance $\Psi^{-1}$ as described for problem 34 but
	extended to all of $\R$ by returning $2x$ on $\R \backslash [0,1]$).
\end{ques}
\end{document}
