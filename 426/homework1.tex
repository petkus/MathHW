\documentclass[12pt]{article}
\usepackage{amsmath, amssymb, amsthm, epsfig}
\setlength\parindent{0pt}

\newenvironment{definition}{\vspace{2 ex}{\noindent{\bf Definition}}}
        {\vspace{2 ex}}

\newenvironment{ques}[1]{\textbf{#1}\vspace{1 mm}\\ }{\bigskip}

\renewcommand{\theenumi}{\alph{enumi}}

\theoremstyle{definition}

\newenvironment{Proof}{\noindent {\sc Proof.}}{$\Box$ \vspace{2 ex}}
\newtheorem{Wp}{Writing Problem}
\newtheorem{Ep}{Extra Credit Problem}

\oddsidemargin-1mm
\evensidemargin-0mm
\textwidth6.5in
\topmargin-15mm
\textheight8.75in
\footskip27pt


\renewcommand{\l}{\left }
\renewcommand{\r}{\right }

\newcommand{\R}{\mathbb R}
\newcommand{\Q}{\mathbb Q}
\newcommand{\Z}{\mathbb Z}
\newcommand{\C}{\mathbb C}
\newcommand{\N}{\mathbb N}
\renewcommand{\i}{\text{int} \ }
\newcommand{\interior}[1]{%
  {\kern0pt#1}^{\mathrm{o}}%
}

\renewcommand{\sup}{\text{sup}}
\newcommand{\osc}{\text{osc}}
\newcommand{\diam}{\text{diam} \ }
\renewcommand{\-}{\backslash}

\newcommand{\s}{\sin}
\renewcommand{\c}{\cos}

\renewcommand{\t}{\theta}
\renewcommand{\a}{\alpha}

\newcommand{\norm}[1]{\left\lVert#1\right\rVert}

\newcommand{\T}{\mathfrak{T}}

\newcommand{\dist}{\text{dist}}

\pagestyle{empty}
\begin{document}

\noindent \textit{\textbf{Math 426, SPRING 2018}} \hspace{1.3cm}
\textit{\textbf{HOMEWORK $\#$1}} \hspace{1.3cm} \textit{\textbf{Peter
Gylys-Colwell}} 

\vspace{1cm}
\begin{ques}{1.4 36}
	We have that any $\sigma$-algebra which contains intervals of the form
	$[a,b)$ must contain all open intervals and conversley any $\sigma$-algebra
	which contains the open intervals must contain all intervals of the form
	$[a,b)$. Thus since the Borel sets are the smallest $\sigma$-algebra to
	contain the open intervals, it is also the smallest $\sigma$-algebra to
	contain intervals of the form $[a,b)$\\
	To show this containment:\\
	$(\subseteq):$ if $\mathcal A$ is $\sigma$-algebra containing the open
	intervals (and thus by closure under complements the closed intervals) then
	for any $[a,b)$ we have $(-\infty,a), [b,\infty) \in \mathcal A$ and thus
	$$[a,b) = ((-\infty,a) \cup [b,\infty))^C \in \mathcal A$$
	$(\supseteq):$ if $\mathcal A$ is $\sigma$-algebra containing the 
	intervals $[x,y)$ then for any $(a,b)$ we have the collection $[a + 1/n, b) \in
	\mathcal A$ and we have
	$$(a,b) = \bigcup_{n=1}^\infty [a + 1/n, b) \in \mathcal
	A$$
\end{ques}

\begin{ques}{2.1 3}
	We can split up the union into a union of disjoint sets:
	$$\bigcup_{k=1}^\infty E_k = \bigcup_{k=1}^\infty \l(E_k \backslash 
	\bigcup_{i=1}^{k-1}E_i\r)$$
	from countable additivity over disjoint sets we have
	$$m\l(\bigcup_{k=1}^\infty E_k\r) = m\l(\bigcup_{k=1}^\infty \l(E_k \backslash 
	\bigcup_{i=1}^{k-1}E_i\r)\r)$$
	$$= \sum_{k=1}^\infty m\l(E_k \backslash 
	\bigcup_{i=1}^{k-1}E_i\r)$$
	Since 
	$$m(E_k) = m \l(E_k \cap \l( 
	\bigcup_{i=1}^{k-1}E_i\r)\r) + m\l(E_k \backslash 
	\bigcup_{i=1}^{k-1}E_i\r)$$
	we get that
	$$m(E_k) \geq m\l(E_k \backslash \bigcup_{i=1}^{k-1}E_i\r)$$
	and thus
	$$m\l(\bigcup_{k=1}^\infty E_k\r) \leq \sum_{k=1}^\infty m(E_k)$$
\end{ques}

\begin{ques}{2.1 4}
	Countably Additive:\\
	For disjoint sets $E_1, E_2 \dots E_k \dots$, since the sets are disjoint
	the number of elements in the union is equal to the sum of the number of
	elements in each set. Thus
	$$c\l(\bigcup E_k\r) = \sum c(E_k)$$
	Translation Invariant:\\
	If a set $E$ is translated by $y$ we have that $E + y$ has the same number
	of elements and thus
	$$c(E) = c(E + y)$$
\end{ques}

\begin{ques}{2.2 6}
	We have the covering $(0,1)$ of $A$ where $\ell((0,1)) = 1$ and thus
	$m^*(A) \leq 1$.\\
	We also have from countable subadditivity
	$$1 \leq m^*([0,1] \- A) + m^*(A)$$
	It is the case $[0,1] \- A = \Q \cap [0,1]$ is a countable set and thus has
	outer measure zero thus
	$$1 \leq m^*(A)$$
	So $m^*(A) = 1$
\end{ques}

\begin{ques}{2.2 9}
	Since $B \subseteq A \cup B$ we have
	$$m^*(B) \leq m^*(A \cup B)$$
	by subadditivity
	$$m^*(A \cup B) \leq m^*(B) + m^*(A \- (A \cap B))$$
	Since $A \- (A \cap B) \subseteq A$
	$$0 \leq m^*(A\- (A \cap B)) \leq m^*(A) = 0$$
	$$m^*(A \cup B) \leq m^*(B)$$
	And thus $m^*(B) = m^*(A \cup B)$
\end{ques}

\begin{ques}{2.3 14}
	If we consider the sequence of balls $B_1, B_2, \dots$ where 
	$$B_n = \{x \in \R: |x| \leq n\}$$
	we have that
	$$E = \bigcup_{n=1}^\infty E \cap B_n$$
	from subadditivity
	$$m^*(E) \leq \sum_{n=1}^\infty m^*(E \cap B_n)$$
	since $m^*(E) > 0$ it must be the case $\sum_{n=1}^\infty m^*(E \cap B_n) >
	0$ which means one of the terms $m^*(E \cap B_N)$ in the sum must be $> 0$.
	$E \cap B_N$ is a bounded subset of $E$ with positive measure
\end{ques}

\begin{ques}{2.3 15}
	Consider the set of intervals
	$$\mathcal I = \{\dots \l[-2\epsilon , -\epsilon \r),
	\l[-\epsilon, 0\r), \l[0, \epsilon\r), \l[\epsilon, 2\epsilon\r)\dots \}$$
	Notice that $\mathcal I$ covers $\R$ and so we have
	$$E = \bigcup_{I_k \in \mathcal I} E \cap I_k$$
	Since $m(I_k) = \epsilon$ and $E \cap I_k \subseteq I_k$ we have the desired property
	$$m(E \cap I_k) \leq \epsilon$$
	Since each $I_k$ and $E$ is measurable from additivity
	$$m(E) = \sum_{I_k \in \mathcal I} m(E \cap I_k)$$
	thus since $m(E)$ is finite the sum converges which means for sufficiently
	large index, the sum of the rest of the terms is less than $\epsilon$. Let
	$E \cap I_N, E \cap I_{N+1}, E \cap I_{N+2}, \dots$ denote this tail of the
	sum. Letting $T =
	\bigcup_{n = N}^\infty E \cap I_n$ with 
	$$m(T) = \sum_{n=N}^\infty m(E \cap I_n) < \epsilon$$
	And thus we have $E$ is a finite union of measurable sets with measure less
	than $\epsilon$
	$$E = (E \cap I_1) \cup(E \cap I_2) \cup(E \cap I_3) \cup \dots (E \cap
	I_{N-1}) \cup T$$


\end{ques}

\end{document}
