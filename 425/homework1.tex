\documentclass[12pt]{article}
\usepackage{amsmath, amssymb, amsthm, epsfig}

\newenvironment{definition}{\vspace{2 ex}{\noindent{\bf Definition}}}
        {\vspace{2 ex}}

\newenvironment{ques}[1]{\textbf{Exercise #1}\vspace{1 mm}\\ }{\bigskip}

\renewcommand{\theenumi}{\alph{enumi}}

\theoremstyle{definition}

\newenvironment{Proof}{\noindent {\sc Proof.}}{$\Box$ \vspace{2 ex}}
\newtheorem{Wp}{Writing Problem}
\newtheorem{Ep}{Extra Credit Problem}

\oddsidemargin-1mm
\evensidemargin-0mm
\textwidth6.5in
\topmargin-15mm
\textheight8.75in
\footskip27pt


\renewcommand{\l}{\left }
\renewcommand{\r}{\right }

\newcommand{\R}{\mathbb R}
\newcommand{\Q}{\mathbb Q}
\newcommand{\Z}{\mathbb Z}
\newcommand{\C}{\mathbb C}
\newcommand{\N}{\mathbb N}
\renewcommand{\i}{\text{int} \ }
\newcommand{\interior}[1]{%
  {\kern0pt#1}^{\mathrm{o}}%
}

\newcommand{\osc}{\text{osc}}
\newcommand{\diam}{\text{diam} \ }

\newcommand{\s}{\sin}
\renewcommand{\c}{\cos}

\renewcommand{\t}{\theta}
\renewcommand{\a}{\alpha}

\newcommand{\norm}[1]{\left\lVert#1\right\rVert}

\newcommand{\T}{\mathfrak{T}}

\newcommand{\dist}{\text{dist}}

\pagestyle{empty}
\begin{document}

\noindent \textit{\textbf{Math 425, WINTER 2018}} \hspace{1.3cm}
\textit{\textbf{HOMEWORK $\#$1}} \hspace{1.3cm} \textit{\textbf{Peter
Gylys-Colwell}} 

\vspace{1cm}

\begin{ques}{19}
	(a) \ Consider any limit point $x$ of $D_k$. We will
	show $\osc(x) \geq 1/k \Rightarrow x \in D_k$ so $D_k$ is closed. (We will
	use the diameter definition of $\osc$, where $\osc(x) = \lim_{r \to 0}
	\diam f(B_r(x))$)\\
	For any $r > 0$, since $x$ is a limit point of $D_k$, there exists $y \in
	B_r(x) \cap D_k$. Since $B_r(x)$ is open there exists $B_s(y) \subset B_r(x)$.
	Since $y$ in $D_k$ we know that diam $f(B_s(y)) \geq 1/k$. Since $B_s(y)
	\subset B_r(x)$ we have 
	$$\diam f(B_r(x)) \geq \diam f(B_s(y)) \geq 1/k$$
	Thus for every $r > 0$ $\diam f(B_r(x)) \geq 1/k$ so we know 
	$$\osc(x) = \lim_{r \to 0} \diam f(B_r(x)) \geq 1/k$$
	\\
	(b) \ Since every discontinuity point has $\osc > 0$, the discontinuity set
	can be written as a countable union of $D_k$ where each $D_k$ is closed as
	proven in part a.
	$$D = \bigcup_{k=1}^\infty D_k$$
	\\
	(c) \ Since the continuity set is the complement of the discontinuity set,
	we have
	$$C = [a,b] \backslash \l(\bigcup_{k=1}^\infty D_k\r)$$
	From Demorgans law
	$$C = \bigcap_{k=1}^\infty\l( [a,b] \backslash D_k\r)$$
	Which is a countable intersection of open sets (since the complement of
	closed sets are open so $[a,b] \backslash D_k$ is open)
\end{ques}

\begin{ques}{27}
	(b)\ Consider the indicator function on the rationals $\chi_\Q: [0,1] \to
	\R$. We have that for any $n \in \N$, 
	$$x_k^* = \frac{2a + (2k -1)(b-a)}{2n} \in \Q$$ 
	Thus $\chi_\Q(x_k^*) = 1$ for all $k,n$ so our calc limit yields $1$, while
	$\chi_\Q$ is not Riemann integrable since it is continuous nowhere
\end{ques}

\begin{ques}{28}
	$(i \Rightarrow ii):$ This follows directly from the definition of a zero
	set. If $Z$ is a zero set, then for each $\epsilon > 0$ there is a
	countable couvering of $Z$ by open intervals  $(a_i,b_i)$ with total length
	$\sum b_i - a_i < \epsilon$. We can replace $(a_i, b_i)$ with $[a_i,b_i]$
	and since $(a_i,b_i) \subset [a_i,b_i]$ this is still a covering of $Z$,
	and the lengths are the same.\\
	$(ii \Rightarrow i):$ Given $\epsilon > 0$ from $ii$ there exists a countable
	covering $C_i = [a_i,b_i]$ with total length $\sum b_i - a_i < \epsilon / 2$.
	Since the covering is countable we can replace each $C_i = [a_i,b_i]$ with
	$U_i = (a_i - \frac \epsilon 2\frac 1 {4^i}, b_i +\frac \epsilon 2 \frac 1 {4^i})$. The $U_i$ make up a
	covering since each $C_i \subset U_i$ so $Z \subset \bigcup C_i \subset
	\bigcup U_i$. The total length is 
	$$\sum_{i=1}^\infty b_i +\frac \epsilon 2 \frac 1 {4^i} - (a_i -\frac
	\epsilon 2 \frac 1 {4^i}) = \sum b_i - a_i +\frac \epsilon 2
	\sum_{i=1}^\infty \frac 1 {2^i} = \frac \epsilon 2 + \sum b_i - a_i < \epsilon$$
\end{ques}

\begin{ques}{Additional Problem 1}
	For any $\epsilon > 0$ and $x \in [0,1] \backslash \Q$ we have that there
	exists $k \in \N$ such that $\frac 1 k < \epsilon$. We have that there are
	only finitely many $\frac p q \in \Q$ where $q < k$ (we can bound the
	number of these rational points from above by the sum of denominators less
	than $k$ which is finite). Thus there exists a minimum distance $d > 0$
	from these numbers and $x$ (this is because there are a finite number of
	distances, each of which are $ > 0 $ since $x \notin  \Q$ so not equal to
	any of these numbers). We have that $\forall y \in B_d(x)$, $f(y) <
	\epsilon$ and thus $f$ is continuous at $x$. The reason for this is as
	follows, if $y \notin \Q$ then $f(y) = 0$ and we are done. If $y \in \Q$
	then we have $|y - x| < d$ thus $y$ cannot have denominator $< k$ since
	otherwise we would contradict the minimality of $d$. Thus $f(y) \leq \frac
	1 k < \epsilon$ so $f(y) < \epsilon$.
\end{ques}

\begin{ques}{Additional Problem 2}
	If $x \in \partial S$, then for all $r > 0$, we have $B_r(x) \cap S, B_r(x)
	\cap S^c \neq \emptyset$. Thus there exists $y,z \in B_r(x)$ where
	$\chi_S(y) = 0$ and $\chi_S(z) = 1$. Thus $\diam \chi_S(B_r(x)) \geq 1$
	since $\chi_S$ cannot take any values besides $1$ and $0$, we have
	equality:  $\diam \chi_S(B_r(x)) = 1$ for all $r > 0$ 
	$$\osc(x) = \lim_{r \to 0} \diam \chi_S(B_r(x)) = 1$$
	If $x \notin \partial S$ then there exists $r > 0$ such that $B_r(x)
	\subset S$ or $S^c$. We thus have $\chi_S(B_r(x)) = \{1\}$ or $\{0\}$ which
	are sets with diameter $0$. Thus $\osc (x) = 0$
\end{ques}
\end{document}
