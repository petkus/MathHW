\documentclass[12pt]{article}
\usepackage{amsmath, amssymb, amsthm, epsfig}
\setlength\parindent{0pt}

\newenvironment{definition}{\vspace{2 ex}{\noindent{\bf Definition}}}
        {\vspace{2 ex}}

\newenvironment{ques}[1]{\textbf{Exercise #1}\vspace{1 mm}\\ }{\bigskip}

\renewcommand{\theenumi}{\alph{enumi}}

\theoremstyle{definition}

\newenvironment{Proof}{\noindent {\sc Proof.}}{$\Box$ \vspace{2 ex}}
\newtheorem{Wp}{Writing Problem}
\newtheorem{Ep}{Extra Credit Problem}

\oddsidemargin-1mm
\evensidemargin-0mm
\textwidth6.5in
\topmargin-15mm
\textheight8.75in
\footskip27pt


\renewcommand{\l}{\left }
\renewcommand{\r}{\right }

\newcommand{\R}{\mathbb R}
\newcommand{\Q}{\mathbb Q}
\newcommand{\Z}{\mathbb Z}
\newcommand{\C}{\mathbb C}
\newcommand{\N}{\mathbb N}
\renewcommand{\i}{\text{int} \ }
\newcommand{\interior}[1]{%
  {\kern0pt#1}^{\mathrm{o}}%
}

\newcommand{\osc}{\text{osc}}
\newcommand{\diam}{\text{diam} \ }

\newcommand{\s}{\sin}
\renewcommand{\c}{\cos}

\renewcommand{\t}{\theta}
\renewcommand{\a}{\alpha}

\newcommand{\norm}[1]{\left\lVert#1\right\rVert}

\newcommand{\T}{\mathfrak{T}}

\newcommand{\dist}{\text{dist}}

\pagestyle{empty}
\begin{document}

\noindent \textit{\textbf{Math 425, WINTER 2018}} \hspace{1.3cm}
\textit{\textbf{HOMEWORK $\#$2}} \hspace{1.3cm} \textit{\textbf{Peter
Gylys-Colwell}} 

\vspace{1cm}

\begin{ques}{51}
	If $f(x) < g(x)$ for all $x \in [a,b]$ then we have that $h(x) = g(x) -
	f(x)$ is a sum of riemann integrable functions, and thus riemann
	integrable. Leting $p$ be a point of continuity for $h$, we can choose a
	partition $P= \{x_0 = a, x_1, \dots x_n = b\}$ such that $p \in (x_1,x_2)$
	and $h(x_1,x_2) \subset B_{h(p)/2}(h(p))$. (in other words, $h(x) > h(p)/2$
	for all $x \in (x_1,x_2)$). We have the inequality
	$$\int_a^b h(x) \ dx \geq \sum_{i=1}^n m_i(x_i - x_{i-1}) > 0$$
	We have that $m_i \geq 0$ for all $i$, and $m_1 \geq h(p)/2 > 0$. Thus we
	have strict inequality.\\
	We know that integration preserves addition and subtraction:
	$$\int_a^b g(x) \ dx - \int_a^b f(x) \ dx = \int_a^b h(x) \ dx > 0$$
	Thus
	$$\int_a^b g(x) \ dx > \int_a^b f(x) \ dx$$
\end{ques}

\begin{ques}{53}
	We have that the discontinuity points $D(\max(f,g))$, $D(\min(f,g))$ are
	subsets of $D(f) \cup D(g)$. Since the finite union of zero sets is a zero
	set, this implies riemann integrablility.\\
	Showing the set inclusion above is equivalent to showing that if $f,g$ are
	continuous at $x$, then $\max(f,g), \min(f,g)$ are continuous at $x$. This
	was proven in homework for Math 424 using the epsilon delta definition of
	continuity
\end{ques}

\begin{ques}{62}
	$$2^ka_{2^k} = a_{2^k}+ a_{2^k}+ \dots + a_{2^k} \leq a_{2^{k-1}+1} +
	a_{2^{k-1}+2} + a_{2^{k-1}+3} + \dots a_{2^k} = \sum_{i=2^k+1}^{2^k} a_i$$
	Thus 
	$$\sum_{i=1}^{2^n} a_i = \sum_{k=1}^n\sum_{j=2^{k-1} + 1}^{2^k}a_j \geq
	\sum_{k = 1}^n 2^ka_k$$
	So by comparison $\sum_{k = 1}^n 2^ka_k$ converges if $\sum_{i=1}^{2^n}
	a_i$ converges which converges iff $\sum a_i$ converge. Conversly
	$$2^ka_{2^k} = a_{2^k}+ a_{2^k}+ \dots + a_{2^k} \geq a_{2^{k}} +
	a_{2^{k}+2} + a_{2^{k}+3} + \dots a_{2^{k + 1} - 1} =
	\sum_{i=2^k}^{2^{k+1}-1} a_i$$
	so now we have
	$$\sum_{i=1}^{2^n - 1} a_i = \sum_{k=1}^n\sum_{j=2^{k}}^{2^{k+1}-1}a_j \leq
	\sum_{k = 1}^n 2^ka_k$$
	Thus $\sum_{i=1}^{2^n - 1} a_i$ converge if $\sum_{k = 1}^n 2^ka_k$
	converge.
\end{ques}

\begin{ques}{Additional Problem 1}
	For any $x_j > x_{j-1} \in (a,b)$, from the mean value theorem there exists
	$x \in (x_j,x_{j-1})$ where 
	$$f'(x)(x_j - x_{j-1}) = f(x_j) - f(x_{j-1})$$
	Using the standard definition of $m_j, M_j$ established in lecture, we have
	$m_j \leq f'(x) \leq M_j$. Thus
	$$m_j(x_j - x_{j-1}) \leq f(x_j) - f(x_{j-1}) \leq M_j(x_j - x_{j-1})$$
	Thus for any partition of $(a,b)$ $P = \{x_0 = a, x_1, \dots x_{n-1}, x_n =
	b\}$ we have
	$$\underline{I}(P) = \sum_{i = 1}^n m_i(x_i - x_{i-1}) \leq \sum_{i=1}^n
	f(x_i) - f(x_{i-1}) = f(b) - f(a)$$
	$$\overline{I}(P) = \sum_{i = 1}^n M_i(x_i - x_{i-1}) \geq \sum_{i=1}^n
	f(x_i) - f(x_{i-1}) = f(b) - f(a)$$
	Thus since $\int_a^bf'(x) \ dx = \sup_P \{\underline{I}(P)\} = \inf_P
	\{\overline{I}(P)\}$. The first inequality yields $\int_a^b f'(x) \ dx \leq
	f(b) - f(a)$ and the second yields $\int_a^b f'(x) \ dx \geq f(b) - f(a)$.
	So $\int_a^b f'(x) \ dx = f(b) - f(a)$
\end{ques}

\begin{ques}{Additional Problem 2}
	(a) \ Since $k>0$, we can choose $x_0 > 0$ large enough so $x_0^k > y$. By
	the intermediate value theorem there exists $x \in (0, x_0)$ such that $x^k
	= y$ since $y \in (0^k, x_0^k)$. There is only one such $x$ since if there
	existed $x_1, x_2 > 0$ where $x_1^k = x_2^k = y$ then by the mean value
	theorem there exists $x \in (x_1,x_2)$ where 
	$$\frac d {dx} x^k = \frac{x_2^k - x_1^k}{x_2 - x_1} = 0$$
	$$kx^k = 0$$
	but this is the case iff $x = 0$ which is not true\\
	\\
	(b) \ Suppose for contradiction $\lim_{k \to \infty} y^{1/k} \neq 1$. This
	would mean there exists $r \neq 1$ such that for some $K$, $1 > r >
	y^{1/k}$ or $y^{1/k} > r > 1$ for all $k > K$\\
	Taking $k$ powers (since taking a $k \in \N$ power of positive numbers
	preserves inequalities)
	$$1 > r^k > y \text{ or } y > r^k > 1$$
	For all $k > K$. This is a contradiction since we know $\lim_{k \to \infty}
	r^k = 0$ or $\infty$ which cannot be the case for $y$
\end{ques}
\end{document}
