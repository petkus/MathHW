%29.09.16
\documentclass[12pt]{article}
\usepackage{amsmath, amssymb, amsthm, epsfig}

\newenvironment{definition}{\vspace{2 ex}{\noindent{\bf Definition}}}
        {\vspace{2 ex}}
\newenvironment{ques}{\vspace{2 ex}}{\vspace{2 ex}}

\renewcommand{\theenumi}{\alph{enumi}}

\theoremstyle{definition}

\newenvironment{Proof}{\noindent {\sc Proof.}}{$\Box$ \vspace{2 ex}}
\newtheorem{Wp}{Writing Problem}
\newtheorem{Ep}{Extra Credit Problem}

\oddsidemargin-1mm
\evensidemargin-0mm
\textwidth6.5in
\topmargin-15mm
%\headsep25pt
\textheight8.75in
\footskip27pt

\pagestyle{empty}
\begin{document}

\noindent \textit{\textbf{Math 403 B, Winter 2017}} \hspace{1.3cm}
\textit{\textbf{HOMEWORK $\#$2}} \hspace{1.3cm} \textit{\textbf{Peter
Gylys-Colwell}} 

\vspace{1cm}

\begin{ques}
	\textbf{16.24} Only do a,b,c
\end{ques}

\begin{ques}
	\textbf{17.1}
\end{ques}

\begin{ques}
	\textbf{17.20}
\end{ques}

\begin{ques}
	\textbf{A}
	\begin{enumerate}
		\item
		By definition of $1$, $1 \cdot 1 = 1$ so $1$ is an idempotent.
		For any $a \in R$
		\item

		\item


	\end{enumerate}
\end{ques}

\begin{ques}
	\textbf{B}
\end{ques}

\begin{ques}
	\textbf{C}
\end{ques}

\begin{ques}
	\textbf{D}
\end{ques}

\begin{ques}
	\textbf{E}
\end{ques}
\end{document}
