%29.09.16
\documentclass[12pt]{article}
\usepackage{amsmath, amssymb, amsthm, epsfig}

\newenvironment{definition}{\vspace{2 ex}{\noindent{\bf Definition}}}
        {\vspace{2 ex}}
\newenvironment{ques}{\vspace{2 ex}}{\vspace{2 ex}}

\renewcommand{\theenumi}{\alph{enumi}}

\theoremstyle{definition}

\newenvironment{Proof}{\noindent {\sc Proof.}}{$\Box$ \vspace{2 ex}}
\newtheorem{Wp}{Writing Problem}
\newtheorem{Ep}{Extra Credit Problem}

\oddsidemargin-1mm
\evensidemargin-0mm
\textwidth6.5in
\topmargin-15mm
%\headsep25pt
\textheight8.75in
\footskip27pt

\pagestyle{empty}
\begin{document}

\noindent \textit{\textbf{Math 403 B, Winter 2017}} \hspace{1.3cm}
\textit{\textbf{HOMEWORK $\#$2}} \hspace{1.3cm} \textit{\textbf{Peter
Gylys-Colwell}} 

\vspace{1cm}

\begin{ques}
	\textbf{16.24} Only do a,b,c
	\begin{enumerate}
		\item
		for any $a + bi, c + di, e + fi \in \mathbb Z [i]$, we have
		$$(a + bi) + (c + di) = (a + c) + (b + d)i = (b + di) + (a +
		bi) \in \mathbb Z [i]$$
		as well as
		$$(a + bi)(c + di) = ac - bd + (ad + bc)i = (c + di)(a + bi)
		\in \mathbb Z [i]$$
		Finally
		$$(a + bi + c + di)(e + fi) = (a + bi)e + (c + di)e + (a +
		bi)fi + (c + di)fi = (a + bi)(e + fi) + (c + di)(e + fi)$$
		And
		$$1(a + bi) = (a + bi)1 = a + bi$$
		Which means $\mathbb Z [i]$ satisfies all the properties to be
		a commutative ring with unity $1$.
		\item
		For $r = a + bi, s = c + di \in  \mathbb Z [i]$ we have
		$$N(rs) = N(ac - bd + (ad + bc)) = (ac - bd)^2 + (ad + bc)^2 =
		(ac)^2 - 2abcd + (bd)^2 + (ad)^2 + 2abcd + (bc)^2$$
		$$(ac)^2 + (bd)^2 + (ad)^2 + (bc)^2 = a^2(c^2 + d^2) + b^2(c^2
		+ d^2) = (a^2 + b^2)(c^2 + d^2) N(r)N(s)$$
		\item
		In order for $a$ to be a unit, there must be some $a^{-1} \in
		\mathbb Z [i]$ such that 
		$$aa^{-1}= 1$$
		Applying the norm to both sides we have
		$$N(aa^{-1}) = N(a)N(a^{-1}) = N(1) = 1$$
		However since the terms in $a$ and $a^{-1}$ are integers, the
		norms must be integers. Therefore in order for the product of
		their norms to be $1$, both norms must be $1$. Therefore $N(a)
		= 1$. Looking at the other direction, we can ceck every element
		with norm 1:\\
		$1, -1, i, -i$. Each of these terms have the respective inverse
		$1, -1, -i, i$. And so every element with norm one is a unit.
	\end{enumerate}
\end{ques}

\begin{ques}
	\textbf{17.1}
	\begin{enumerate}
		\item
		This is not a subring since it is not closed under multiplication:
		$$\left(\begin{matrix} 0 & 1 \\ 1 & 1
		\end{matrix}\right)\left(\begin{matrix} 0 & 1 \\ 1 & 1
		\end{matrix}\right) = \left(\begin{matrix} 1 & 1 \\ 1 & 2
		\end{matrix}\right) \notin S $$
		\item
		This is a subring since it is closed under multiplication and
		addition, and every element has an addative inverse:
		$$\left(\begin{matrix} a & 0 \\ b & c
		\end{matrix}\right)\left(\begin{matrix} d & 0 \\ e & f
		\end{matrix}\right) = \left(\begin{matrix} ad & 0 \\ be + cf & cf
		\end{matrix}\right) \in S $$
		$$\left(\begin{matrix} a & 0 \\ b & c
		\end{matrix}\right) + \left(\begin{matrix} d & 0 \\ e & f
		\end{matrix}\right) = \left(\begin{matrix} a + d & 0 \\ b + e & c + f
		\end{matrix}\right) \in S $$
		\item
		As established last quarter, $S$ is a group under
		multiplication and a group under division, and therefore is
		closed under multiplication, and so satisfies the requirements
		to be a subring.
		\item
		We have $S = M_2(\mathbb R)$, which has been established to be
		a ring. Therefore $S$ is a subring.
	\end{enumerate}
\end{ques}

\begin{ques}
	\textbf{17.20}
	If $aR = R$ then since $1 \in R$ there must be $1 \in aR$ which means
	there must be some $a^{-1}$ such that $aa^{-1} =1$ which means $a$ is a
	unit. For implication in the other direction, we have for any $x \in
	R$, assuming $a$ is a unit with multiplicative inverse $a^{-1}$, we
	have $a^{-1}x \in R$ and $a (a^{-1}x) = x \in aR$. Therefore every element of
	$R$ is an element of $aR$ and so $R \subseteq aR$, and since $R$ is
	closed under multiplication, for any $x \in R$, $ax \in R$, so $aR
	\subseteq R$ and so it follows 
	$$R = aR$$
\end{ques}

\begin{ques}
	\textbf{A}
	\begin{enumerate}
		\item
		We have
		$$a^2 = a \Rightarrow a^2 - a = 0 \Rightarrow a(a - 1) = 0$$
		Since $R$ is an integral domain, $a(a-1) = 0$ if and only if
		either $a$ or $a-1$ is zero, and since the additive inverse is
		unique, that means $a$ is either $1$ or $0$.
		\item
		The idempotents are $1$, $5$, and $6$.
		\item
		For any $(a,b) \in
		\mathbb Z \times \mathbb Z$ we have 
		$$(a, b)(a, b) = (a, b) \Rightarrow (a^2 - a, b^2 - b) = (0,0)
		\Rightarrow a^2 - a =0, b^2 - b = 0$$
		And since $\mathbb Z$ is an integral domain, from question Aa it means $a,b
		\in \{0, 1\}$ and so the idempotents are $(0,0), (1,1), (1,0),
		(0,1)$
	\end{enumerate}
\end{ques}

\begin{ques}
	\textbf{B}
	We can deduce the set of idempotents in $S$ is a subset of
	the idempotents in $R$ since $s \in S \Rightarrow s \in R$ and the
	conditions in either set is the same: $s^2 = s$.\\
	As shown in problem Aa, the only idempotents in $R$ are $1_R$ and $0_R$\\
	Subrings of an integral domain is an integral domain as well so $S$
	also has the property that the idempotents in $S$ are $1_S$ and $0_R$.
	Therefore we have.
	$$\{0_S, 1_S\} \subseteq \{0_R, 1_R\}$$
	From basic group theory we know the identity of a subgroup is equal to
	the identity of the containing group. Therefore $0_S = 0_R$ since $0$
	is the identity of the groups $R, S$ over additition. So we have $1_S
	\neq 0_S \Rightarrow 1_S \neq 0_R$. The only other element in $\{0_R,
	0_S\}$ that $1_S$ can be is $1_R$
\end{ques}

\begin{ques}
	\textbf{C}
	$U(R) = \{(1,1), (-1, 1), (1, -1), (-1,-1)\}$ since the only units in
	$\mathbb Z$ is $1$ and $-1$.
\end{ques}

\begin{ques}
	\textbf{D}
	True:\\
	Consider the subring
	$$S = 5\mathbb Z _{25} = \{0, 5, 10, 15, 20\}$$
	This is a subring since we have $x|5 \Leftrightarrow x \in S$ and for
	any $a, b \in S$, $ab | 5$ so $ab \in S$. We also have $a + b |5$ so $a
	+b \in S$. Thererfore $S$ is closed under addition and multiplication
	and is finite, so it is a subring. $S$ is isomorphic to $\mathbb Z _5$,
	let $\varphi: S \to \mathbb Z_5$ with $\varphi(5x) = x$. We have
	$$\varphi(5x)\varphi(5y) = xy = \varphi(5xy)$$
	and
	$$\varphi(5x) + \varphi(5y) = x + y = \varphi(5(x + y))$$
	So $\varphi$ is a homomorphism. We have $\varphi(0) = 0, \varphi(5) =
	1, \varphi(10) = 2, \varphi(15) = 3, \varphi(20) = 4$, and so $\varphi$
	is a bijections so an isomorphism.
\end{ques}

\begin{ques}
	\textbf{E}
	The four ideals are $R$, $R_r = \{(0,x): x \in \mathbb R\}, R_l =
	\{(x,0): x \in \mathbb R\}, \{(0,0)\}$. Since the idempotents of $R$
	are $(0,0), (1,0), (0,1), (1,1)$ and we know that the multiplicative
	identity of a subring must be one of these terms. If the identity is
	$(0,0)$ we get the subring $\{(0,0)\}$ since every term in $R$
	multiplies with $(0,0)$ to $(0,0)$. If the identity is $(1,1)$, then
	for any $(x,y) \in R$ $(1,1)(x,y) = (x,y)(1,1) = (x,y)$, and so the
	subring would have to be $R$ to satisfy the ideal property. If the
	identity is $(1,0)$ then for any $(x,y) \in R$ we have $(1,0)(x,y) =
	(x,y)(1,0) = (x,0)$ therefore the group would be $R_l$, and by symmetry
	for the identity being $(0,1)$, the group would be $R_r$.
\end{ques}
\end{document}
