%29.09.16
\documentclass[12pt]{article}
\usepackage{amsmath, amssymb, amsthm, epsfig}

\newenvironment{definition}{\vspace{2 ex}{\noindent{\bf Definition}}}
        {\vspace{2 ex}}
\newenvironment{ques}{\vspace{2 ex}}{\vspace{2 ex}}

\renewcommand{\theenumi}{\alph{enumi}}

\theoremstyle{definition}

\newenvironment{Proof}{\noindent {\sc Proof.}}{$\Box$ \vspace{2 ex}}
\newtheorem{Wp}{Writing Problem}
\newtheorem{Ep}{Extra Credit Problem}

\oddsidemargin-1mm
\evensidemargin-0mm
\textwidth6.5in
\topmargin-15mm
%\headsep25pt
\textheight8.75in
\footskip27pt

\pagestyle{empty}
\begin{document}

\noindent \textit{\textbf{Math 403 B, Winter 2017}} \hspace{1.3cm}
\textit{\textbf{HOMEWORK $\#$2}} \hspace{1.3cm} \textit{\textbf{Peter
Gylys-Colwell}} 

\vspace{1cm}

\begin{ques}
	\textbf{16.24} Only do a,b,c
\end{ques}

\begin{ques}
	\textbf{17.1}
\end{ques}

\begin{ques}
	\textbf{17.20}
\end{ques}

\begin{ques}
	\textbf{A}
	\begin{enumerate}
		\item
		We have
		$$a^2 = a \Rightarrow a^2 - a = 0 \Rightarrow a(a - 1) = 0$$
		Since $R$ is an integral domain, $a(a-1) = 0$ if and only if
		either $a$ or $a-1$ is zero, and since the additive inverse is
		unique, that means $a$ is either $1$ or $0$.
		\item
		The idempotents are $1$, $5$, and $6$.
		\item
		For any $(a,b) \in
		\mathbb Z \times \mathbb Z$ we have 
		$$(a, b)(a, b) = (a, b) \Rightarrow (a^2 - a, b^2 - b) = (0,0)
		\Rightarrow a^2 - a =0, b^2 - b = 0$$
		And since $\mathbb Z$ is an integral domain, that means $a,b
		\in \{0, 1\}$ and so the idempotents are $(0,0), (1,1), (1,0),
		(0,1)$
	\end{enumerate}
\end{ques}

\begin{ques}
	\textbf{B}
	We can deduce the set of idempotents in $S$ is a subset of
	the idempotents in $R$ since $s \in S \Rightarrow s \in R$ and the
	conditions in either set is the same: $s^2 = s$.\\
	As shown in problem Aa, the only idempotents in $R$ are $1_R$ and $0_R$\\
	Subrings of an integral domain is an integral domain as well so $S$
	also has the property that the idempotents in $S$ are $1_S$ and $0_R$.
	Therefore we have.
	$$\{0_S, 1_S\} \subseteq \{0_R, 1_R\}$$
	From basic group theory we know the identity of a subgroup is equal to
	the identity of the containing group. Therefore $0_S = 0_R$ since $0$
	is the identity of the groups $R, S$ over additition. So we have $1_S
	\neq 0_S \Rightarrow 1_S \neq 0_R$. The only other element in $\{0_R,
	0_S\}$ that $1_S$ can be is $1_R$
\end{ques}

\begin{ques}
	\textbf{C}
	
\end{ques}

\begin{ques}
	\textbf{D}
	True:\\
	Consider the subring
	$$S = 5\mathbb Z _{25} = \{0, 5, 10, 15, 20\}$$
	This is a subring since for any $a, b \in S$, $ab$
\end{ques}

\begin{ques}
	\textbf{E}
\end{ques}
\end{document}
