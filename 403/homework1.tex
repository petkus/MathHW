%29.09.16
\documentclass[12pt]{article}
\usepackage{amsmath, amssymb, amsthm, epsfig}

\newenvironment{definition}{\vspace{2 ex}{\noindent{\bf Definition}}}
        {\vspace{2 ex}}
\newenvironment{ques}{\vspace{2 ex}}{\vspace{2 ex}}

\renewcommand{\theenumi}{\alph{enumi}}

\theoremstyle{definition}

\newenvironment{Proof}{\noindent {\sc Proof.}}{$\Box$ \vspace{2 ex}}
\newtheorem{Wp}{Writing Problem}
\newtheorem{Ep}{Extra Credit Problem}

\oddsidemargin-1mm
\evensidemargin-0mm
\textwidth6.5in
\topmargin-15mm
%\headsep25pt
\textheight8.75in
\footskip27pt

\pagestyle{empty}
\begin{document}

\noindent \textit{\textbf{Math 403 B, Winter 2017}} \hspace{1.3cm}
\textit{\textbf{HOMEWORK $\#$1}} \hspace{1.3cm} \textit{\textbf{Peter
Gylys-Colwell}} 

\vspace{1cm}

\begin{ques}
	\textbf{16.1}
	This is a special case of thm 16.1 ii:\\
	We have 
	$$(-1)a + a = (-1 + 1)a = 0 \cdot a = 0$$
	and so subtracting $a$ on both sides yields 
	$$(-1)a = -a$$
\end{ques}

\begin{ques}
	\textbf{16.7}
	Since $F$ is a field we know there is $a^{-1} \in F$ such that
	$aa^{-1} = 1$. Therefore if we let $x = a^{-1}(-b)$ we satisfy the equation:
	$$a(a^{-1}(-b)) + b = (aa^{-1})(-b) + b = -b + b = 0$$
	We get that first equality since $\cdot$ is associative
\end{ques}

\begin{ques}
	\textbf{16.11}
	\begin{enumerate}
		\item
			The only unit is $(1,1)$ since for any $a,b \in
			\mathbb Z$, $ab = 1 \Leftrightarrow a=1, b=1$.
			The only zero-divisor is $(0,0)$ since for any
			$a,b \in \mathbb Z$, $ab = 0 \Leftrightarrow
			a=0$ and/or $b = 0$. Since the set of nilpotents
			elements is a subset of zero-divisors, it
			follows that the only nilpotent is also $(0,0)$.
		\item
			From previous knowledge of groups we know every
			element in $\mathbb Z_3$ has an inverse under
			the group operation of multiplication modulo
			$3$, therefore we know for any $(a,b) \in
			\mathbb Z_3 \oplus \mathbb Z_3$ there is a
			$(a^{-1}, b^{-1}) \in \mathbb Z_3 \oplus
			\mathbb Z_3$ such that $(a,b)(a^{-1},b^{-1}) =
			(1,1)$ and so every element in $\mathbb Z_3
			\oplus \mathbb Z_3$ is a unit. Since $3$ is
			prime there is no two numbers that can multiply
			together to be a multiple of $3$ unless one of
			the two numbers is already a multiple of $3$,
			only $(0,0)$ is a zero-divisor and
			from that it follows (since the set of
			nilpotents is a subset of zero-divisors) that
			$(0,0)$ is the only nilpotent
		\item
			The units are $(1,1), (1,5), (3,1), (3,5)$ with
			respective inverses $(1,1), (1,5), (3,1),
			(3,5)$. The zero-divisors are all the rest of
			the elements: $(0,2), (0,3), (0,4), (2,2),
			(2,3), (2,4)$. The nilpotents are $(0,0),
			(2,0)$.
	\end{enumerate}
\end{ques}

\begin{ques}
	\textbf{16.13}
	\begin{enumerate}
		\item
			If there were two multiplicative identities: $1
			\neq 1'$ we would have by definition of the
			multiplicative identity
			$$1 = 1\cdot1' = 1'$$
			and so $1 = 1'$
		\item
			If there were two multiplicative inverses, let
			$\beta$ and $\alpha$ be multiplicative
			inverses of $a$. We have
			$$\beta = \beta(a \alpha) = (\beta a)\alpha =
			\alpha$$
			And so $\beta = \alpha$
	\end{enumerate}
\end{ques}

\begin{ques}
	\textbf{A}
	From the definition we know that the center is abelian and from
	the definition of a division ring we know every element is a
	unit. Now all we need to show is that the center is closed
	under multiplication and addition. Given any $a, b \in $ the
	center of $R$ we have for any $x \in R$
	$$(a + b)x = ax + bx = xa + xb = x(a + b)$$ 
	and so $a + b$ is in the center. We also have
	$$(ab)x = axb = x(ab)$$
	and so $ab$ is in the center. Therefore the center is a field.
\end{ques}

\begin{ques}
	\textbf{B}
	$\mathbb Z \times \mathbb Z$ is not an integral domain.
	Consider any $a, b \in \mathbb Z / \{0\}$
	$$(a, 0) \cdot (0, b) = (0,0)$$
	and so $(a,0)$ and $(0,b)$ are non-zero zero-divisors.
\end{ques}

\begin{ques}
	\textbf{C}
	$\mathbb Z_{10}$ is not an integral domain. Consider
	$$2 \cdot 5 = 0$$
	and so $2$ and $5$ are non-zero zero-divisors. Observing that
	$S$ is the set of all even integers in $R$ we know that $S$ is
	closed under addition and multiplication since multiplying or
	adding to even numbers yields an even number. Addition is still
	commutative in $S$. Therefore $S$ is a subring of $R$. \\
	$S$ is an integral domain since for any $s \in S$ in order for
	$s \cdot a = 0$, $10$ must divide $sa$ and so $2 \cdot 5$ must
	divide $sa$. However since $s$ is even if it also has a factor
	of $5$ then it is a multiple of $10$ since it has a factor of
	both $5$ and $2$. If $s \neq 0$ mod $10$ then, $a$ must have a
	factor of $5$ and so if $a \in S$ then $a$ is $0$ since $a$
	would have a factor of $5$ and a factor of $2$. Therefore there
	is no non-zero term $a \in S$ such that $sa = 0$.\\
	$S$ is a field since $S$ is commutative (since $R$ is
	commutative) and each term is a unit with $6$ the
	multiplicative identity: $2 \cdot 8 = 6$, $4 \cdot 4 = 6$, $6
	\cdot 6 = 6$, $6 \cdot 2 = 2$, $6 \cdot 4 = 4$, and $6 \cdot 8
	= 8$.
\end{ques}

\begin{ques}
	\textbf{D}
	We in order for $\left[\begin{array}{cc} a & b \\ c & d
	\end{array}\right]$ to be in the center we have for any $w,x,y,z
	\in \mathbb R$:
	$$\left[\begin{array}{cc} a & b \\ c &
	d \end{array}\right]\left[\begin{array}{cc} w &
	x \\ y & z \end{array}\right] =
	\left[\begin{array}{cc} w & x \\ y & z
	\end{array}\right]\left[\begin{array}{cc}
	a & b \\ c & d \end{array}\right]$$
	$$= \left[\begin{array}{cc} aw + by & ax + bz
	\\ cw + dy & cx + dz \end{array}\right] =
	\left[\begin{array}{cc} wa + xc & wb + xd
	\\ ya + zc & yb + zd \end{array}\right] $$
	Equating the top left and bottom right corners gives us $by =
	cx$. The only way for those quantities be equal for any $x,y$ is if
	$b= c= 0$. From there, equating the top right and bottom left
	corners gives us $ax = xd$ and $dy = ya$. Dividing by $x$ for
	the first equation or $y$ for the second equation yields $a =
	d$. Therefore the center consisits of all matricies of the form
	$$\left[ \begin{array}{cc} a & 0 \\ 0 & a \end{array} \right]$$
	With $a \in \mathbb R$
\end{ques}

\begin{ques}
	\textbf{E}
	We can have 
\end{ques}
\end{document}
