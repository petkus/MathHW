%29.09.16
\documentclass[12pt]{article}
\usepackage{amsmath, amssymb, amsthm, epsfig}

\newenvironment{definition}{\vspace{2 ex}{\noindent{\bf Definition}}}
        {\vspace{2 ex}}
\newenvironment{ques}{\vspace{2 ex}}{\vspace{2 ex}}

\renewcommand{\theenumi}{\alph{enumi}}

\theoremstyle{definition}

\newenvironment{Proof}{\noindent {\sc Proof.}}{$\Box$ \vspace{2 ex}}
\newtheorem{Wp}{Writing Problem}
\newtheorem{Ep}{Extra Credit Problem}

\oddsidemargin-1mm
\evensidemargin-0mm
\textwidth6.5in
\topmargin-15mm
%\headsep25pt
\textheight8.75in
\footskip27pt

\pagestyle{empty}
\begin{document}

\noindent \textit{\textbf{Math 403 B, Winter 2017}} \hspace{1.3cm}
\textit{\textbf{HOMEWORK $\#$1}} \hspace{1.3cm} \textit{\textbf{Peter
Gylys-Colwell}} 

\vspace{1cm}

\begin{ques}
	\textbf{16.1}
		This is a special case of thm 16.1 ii:\\
		We have 
		$$(-1)a + a = (-1 + 1)a = 0 \cdot a = 0$$
		and so subtracting $a$ on both sides yields 
		$$(-1)a = -a$$
\end{ques}

\begin{ques}
	\textbf{16.7}
		Since $F$ is a field we know there is $a^{-1} \in F$ such that
		$aa^{-1} = 1$. Therefore if we let $x = a^{-1}(-b)$ we satisfy the equation:
		$$a(a^{-1}(-b)) + b = (aa^{-1})(-b) + b = -b + b = 0$$
		We get that first equality since $\cdot$ is associative
\end{ques}

\begin{ques}
	\textbf{16.11}
		\begin{enumerate}
			\item
				The only unit is $(1,1)$ since for any $a,b \in
				\mathbb Z$, $ab = 1 \Leftrightarrow a=1, b=1$.
				The only zero-divisor is $(0,0)$ since for any
				$a,b \in \mathbb Z$, $ab = 0 \Leftrightarrow
				a=0$ and/or $b = 0$. Since the set of nilpotents
				elements is a subset of zero-divisors, it
				follows that the only nilpotent is also $(0,0)$.
			\item
				From previous knowledge of groups we know every
				element in $\mathbb Z_3$ has an inverse under
				the group operation of multiplication modulo
				$3$, therefore we know for any $(a,b) \in
				\mathbb Z_3 \oplus \mathbb Z_3$ there is a
				$(a^{-1}, b^{-1}) \in \mathbb Z_3 \oplus
				\mathbb Z_3$ such that $(a,b)(a^{-1},b^{-1}) =
				(1,1)$ and so every element in $\mathbb Z_3
				\oplus \mathbb Z_3$ is a unit. Since $3$ is
				prime there is no two numbers that can multiply
				together to be a multiple of $3$ unless one of
				the two numbers is already a multiple of $3$,
				only $(0,0)$ is a zero-divisor and
				from that it follows (since the set of
				nilpotents is a subset of zero-divisors) that
				$(0,0)$ is the only nilpotent
			\item
				The units are $(1,1), (1,5), (3,1), (3,5)$ with
				respective inverses $(1,1), (1,5), (3,1),
				(3,5)$. The zero-divisors are all the rest of
				the elements: $(0,2), (0,3), (0,4), (2,2),
				(2,3), (2,4)$. The nilpotents are $(0,0),
				(2,0)$.
		\end{enumerate}
\end{ques}

\begin{ques}
	\textbf{16.13}
		\begin{enumerate}
			\item
				If there were two multiplicative identities: $1
				\neq 1'$ we would have by definition of the
				multiplicative identity
				$$1 = 1\cdot1' = 1'$$
				and so $1 = 1'$
			\item
				If there were two multiplicative inverses, let
				$\beta$ and $\alpha$ be multiplicative
				inverses of $a$. We have
				$$\beta = \beta(a \alpha) = (\beta a)\alpha =
				\alpha$$
				And so $\beta = \alpha$
		\end{enumerate}
\end{ques}

\begin{ques}
	\textbf{A}
		From the definition we know that the center is abelian and from
		the definition of a division ring we know every element is a
		unit. Now all we need to show is that the center is closed
		under multiplication and addition. Given any $a, b \in $ the
		center of $R$ we have for any $x \in R$
		$$(a + b)x = ax + bx = xa + xb = x(a + b)$$ 
		and so $a + b$ is in the center. and
		$$(ab)x = axb = x(ab)$$
		and so $ab$ is in the center. Therefore the center is a field.
\end{ques}

\begin{ques}
	\textbf{B}
		
\end{ques}

\begin{ques}
	\textbf{C}
		
\end{ques}

\begin{ques}
	\textbf{D}
\end{ques}

\begin{ques}
	\textbf{E}
\end{ques}
\end{document}
