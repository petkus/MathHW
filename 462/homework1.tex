%29.09.16
\documentclass[12pt]{article}
\usepackage{amsmath, amssymb, amsthm, epsfig}

\newenvironment{definition}{\vspace{2 ex}{\noindent{\bf Definition}}}
        {\vspace{2 ex}}
\newenvironment{ques}{\vspace{2 ex}}{\vspace{2 ex}}


\theoremstyle{definition}

\newenvironment{Proof}{\noindent {\sc Proof.}}{$\Box$ \vspace{2 ex}}
\newtheorem{Wp}{Writing Problem}
\newtheorem{Ep}{Extra Credit Problem}

\renewcommand{\theenumi}{\alph{enumi}}
\oddsidemargin-1mm
\evensidemargin-0mm
\textwidth6.5in
\topmargin-15mm
%\headsep25pt
\textheight8.75in
\footskip27pt

\pagestyle{empty}
\begin{document}

\noindent \textit{\textbf{Math 462, Winter 2017}} \hspace{1.3cm}
\textit{\textbf{HOMEWORK $\#$1}} \hspace{1.3cm} \textit{\textbf{Peter
Gylys-Colwell}} 

\vspace{1cm}
\begin{ques}
	\textbf{1}
	If there is a given bipartite graph $G = X \cup Y$ with the
	given degree sequence. With $X, Y$ the vertix sets that are not
	connected within eachother. We know that the sum of the degrees
	of verticies in $X$ must equal the sum of the degrees of
	verticies in $Y$. However this cannot be the case, since the sum of
	degrees of verticies that contains the degree $5$ will be $2$
	modulo $3$ since all the other degrees are $0$ mod $3$, and the
	sum from the other set will be $0$ mod $3$. Because the two
	sums are unequal mod $3$, they must be unequal in general. And
	so no bipartite graph with the given degree sequence can exist.
\end{ques}

\begin{ques}
	\textbf{2}
	We can induct on $d$:\\
	The base case is where $d$ = 1, in which case there is $1$ perfect
	matching (as proven in class.\\
	For the inductive step, given a bipartite graph $G = (X, Y)$ with all
	verticies having degree $d + 1$. We have proved in class that such a
	graph has a perfect matching $M$. If we consider the graph $H = G - M$
	where we remove all the edges in $M$ from $G$, by the definition of a
	perfect matching, every vertex in $G$ is connected to exactly $1$ edge
	in $M$, therefore removing $M$ would lower the degree of every vertex
	by $1$. Therefore $H$ is a bitartite graph with all verticies having
	degree $d$, which by the inductive hypothesis has $d$ distinct perfect
	matchings. Since $M$ is does not intersect with the set of edges in
	$H$, $M$ cannot intersect with any of the perfect matchings of $H$.
	Therefore these matchings of $H$ and $M$ make up $d + 1$ distinct
	perfect matchings of $G$
\end{ques}

\begin{ques}
	\textbf{3}
	For any nonempty subset $S \subseteq X$, since every vertex has degree
	$\geq 4$, we know that $|N(S)| \geq 4$ since all verticies in $X$ has
	$4$ neighbors. Therefore the Hall thm conditions is satisified for $|S|
	< 5$. For $|S| = 5$ we have $S = X$. Since all verticies in $Y$ have
	degree $\geq 1$ we know that $|N(X)| = |Y| > |X|$. Therefore the Hall
	thm is satisified for all subsets of $X$ which means there exists a
	perfect matching of $X$ to $Y$
\end{ques}

\begin{ques}
	\textbf{4}
	Letting $G = (M,W)$ be the bipartite graph where $M$ are the nodes of
	men, $W$ the nodes of women, and an edge represents whether the man and
	woman know each other. If we make a new graph $G'$ by adding two
	verticies to $W$ and edges between these new two verticies and every node
	in $M$, then the conditions in the problem imply that for every $2 \leq
	k \leq N$, every $k$ nodes in $X$ have at least $k$ neighbors in $G'$
	(we add two since every node in $X$ is connected to the two new added
	nodes in $W$). Every subset of size $1$ of $X$ in $G'$ has a neighbor
	set of size $\geq 1$ as well since every node in $X$ is connected to
	the added nodes. Therefore the conditions of Halls thm are satisified
	for $G'$ and so there exists a perfect matching $M$ from $X$ to $W \cup
	\{v_1, v_2\}$ where $v_1, v_2$ are the added verticies. When we remove
	$v_1, v_2$ and the edges connected to them from $M$, we also remove two
	verticies in $X$ from $M$, and so we are left with a matching $M'$ with
	$N-2$ verticies in $X$ and edges and verticies contained within $G$,
	which corresponds to the desired result: a matching of $N-2$ of the men
	to the women they know.
\end{ques}

\begin{ques}
	\textbf{5}
	\begin{enumerate}
		\item


		\item

	\end{enumerate}
\end{ques}

\begin{ques}
	\textbf{6}
\end{ques}
\end{document}
