%29.09.16
\documentclass[12pt]{article}
\usepackage{amsmath, amssymb, amsthm, epsfig}

\newenvironment{definition}{\vspace{2 ex}{\noindent{\bf Definition}}}
        {\vspace{2 ex}}
\newenvironment{ques}{\vspace{2 ex}}{\vspace{2 ex}}


\theoremstyle{definition}

\newenvironment{Proof}{\noindent {\sc Proof.}}{$\Box$ \vspace{2 ex}}
\newtheorem{Wp}{Writing Problem}
\newtheorem{Ep}{Extra Credit Problem}

\oddsidemargin-1mm
\evensidemargin-0mm
\textwidth6.5in
\topmargin-15mm
%\headsep25pt
\textheight8.75in
\footskip27pt

\pagestyle{empty}
\begin{document}

\noindent \textit{\textbf{Math 380a, Fall 2016}} \hspace{1.3cm}
\textit{\textbf{HOMEWORK $\#$5}} \hspace{1.3cm} \textit{\textbf{Peter
Gylys-Colwell}} 

\vspace{1cm}

\begin{ques}
	\textbf{1.} We will prove this by contradiction, consider 
		$$z = 1 + \frac12 + \frac13 + \dots \frac1n$$
		for some $n$ with $z$ being an integer.\\
		We will now define a value $M$ as the number with the prime
		factorization of $n!$, except that the power of
		$2$ in $M$'s prime factorization is equal to $i - 1$, where $i$ is
		the value such that
		$$2^i \leq n < 2^{i+1}$$
		The following property holds for $M$:
		$$k \in \{1, 2, 3, \dots n\} / \{2^i\} \Rightarrow k|M$$
		The reason is that the largest possible power of $2$ that
		divides a number $\leq n$ that is not $2^i$ would be $i - 1$,
		that is because the smallest possible number that has a factor
		of $2^i$ besides $2^i$ would be $2^{i + 1}$ which is larger
		than $n$. All the other prime factors for each number $\leq n$
		must be present in $M$ since $M$ has the same prime
		factorization as $n!$ besides the powers of $2$.\\
		Therefore we have by multiplying by $M$ on both sides of the
		original assumption:
		$$Mz = M + \frac M 2 + \frac M 3 + \dots \frac M {2^i} + \dots
		\frac M n$$
		However we reach a contradiction here, since the left hand
		side of the equality is an integer, and each term of the right
		hand side of the equality is an integer except for the term 
		$$\frac M {2^i}$$
		Which means the right hand side cannot be an integer. Therefore
		the equality above cannot hold

\end{ques}

\begin{ques}
	\textbf{2.} We can prove this by contradiction.\\
		Let $\frac{x}{y}$ be a root of $p$ such that $\frac{x}{y}$ is
		the most reduced form, so gcd$(x,y) = 1$.\\
		That means
		$$- a_0 = a_1\frac{x}{y} + a_2\frac{x^2}{y^2} +
		a_3\frac{x^3}{y^3} + \dots a_r\frac{x^r}{y^r}$$
		$$-y^ra_0 = a_1xy^{r-1} + a_2x^2y^{r-2} \dots a_rx^r$$
		Now we work through the following cases:\\
		If $y$ is even then since $y$ and $x$ are relatively prime, $2 \not| \ x$.\\
		Therefore we have 
		$$2|-y^ra_0$$ 
		However we have, $2|a_ix^iy^{r-i}$ for $i \neq r$, and since
		$2\not| \ x$ and $2 \not| \ a_r$, we have $2 \not| \ x^ra_r$. Which means
		$$a_1xy^{r-1} + a_2x^2y^{r-2} \dots a_rx^r \equiv 0 + 0 + 0
		\dots + 1 \pmod 2$$
		And so 
		$$2\not| \ a_1xy^{r-1} + a_2x^2y^{r-2} \dots a_rx^r$$ 
		Which means 
		$$-y^ra_0 \neq a_1xy^{r-1} + a_2x^2y^{r-2} \dots a_rx^r$$ 
		which is a contradiction.\\
		If $y$ is odd then we know $-y^ra_0$ must be odd since $a_0$ is also odd.\\
		However if $x$ is even, we know 
		$$a_1xy^{r-1} + a_2x^2y^{r-2} \dots a_rx^r $$
		Is even which would mean
		$$-y^ra_0 \neq a_1xy^{r-1} + a_2x^2y^{r-2} \dots a_rx^r$$
		which would be a contradiction. \\
		And if $x$ is odd, we have the following:
		Since $p(1)$ is odd, we know
		$$a_0 + a_1 + a_2 + \dots a_r \equiv 1 \mod 2$$
		And since $a_0, a_r$ are odd
		$$a_1 + a_2 + \dots a_{r-1} \equiv 1 \mod 2$$
		which means an odd amount of the terms above must be odd.\\
		Therefore since $x \equiv 1 \mod 2$ and $y \equiv 1 \mod 2$, we have 
		$$a_ix^iy^{r-i} \equiv a_i \mod 2$$
		And so 
		$$xy^{r-1}a_1 + x^2y^{r-2}a_2 + \dots x^{r-1}ya_{r-1} \equiv 1 \mod 2$$
		And we have $x^ra_r \equiv 1 \mod 2$
		$$xy^{r-1}a_1 + x^2y^{r-2}a_2 + \dots x^{r-1}ya_{r-1} + x^ra_r
		\equiv 0 \mod 2$$
		Which would lead to the same contradiction,
		$$xy^{r-1}a_1 + x^2y^{r-2}a_2 + \dots x^{r-1}ya_{r-1} + x^ra_r \neq -a_0y^r$$
		Therefore there can be no rational roots for $p$

\end{ques}

\begin{ques}
	\textbf{3.} To prove this, we choose a set of $n$ primes: 
		$$P = \{p_1^2, p_2^2, p_3^2, \dots p_n^2\}$$
		We know any two elements of $P$ are relatively prime since each
		element is a square of distinct primes\\
		And now consider the set
		$$S = \{0, -1, -2, \dots, -n + 2, -n + 1\}$$
		Then by the chinese remainder thm, we know there exits some $A$ such that
		$$A \equiv 0 \mod{p_1^2}, A + 1 \equiv 0 \mod{p_2^2},\ \ \dots
		\ \ A + n -1 \equiv 0 \mod{p_n^2}$$
		Which is the desired result
\end{ques}

\begin{ques}
	\textbf{4.} For any prime $p$ we know the number of times $p$ divides
		the numerator and denominator of $a_{m,n}$ is given by the following:\\
		number of times p divides the numerator: 
		$$\lfloor \frac{2m}p\rfloor + \lfloor \frac{2m}{p^2}\rfloor +
		\lfloor \frac{2m}{p^3}\rfloor \dots + \lfloor
		\frac{2m}{p^k}\rfloor + \lfloor \frac{2n}p\rfloor + \lfloor
		\frac{2n}{p^2}\rfloor + \lfloor \frac{2n}{p^3}\rfloor \dots +
		\lfloor \frac{2n}{p^k}\rfloor$$
		number of times p divides the denominator: 
		$$\lfloor \frac{m}p\rfloor + \lfloor \frac{m}{p^2}\rfloor +
		\lfloor \frac{m}{p^3}\rfloor \dots + \lfloor
		\frac{m}{p^k}\rfloor + \lfloor \frac{n}p\rfloor + \lfloor
		\frac{n}{p^2}\rfloor + \lfloor \frac{n}{p^3}\rfloor \dots +
		\lfloor \frac{m + n}{p}\rfloor + \lfloor \frac{m +
		n}{p^2}\rfloor + \lfloor \frac{m + n}{p^3}\rfloor \dots + \lfloor
		\frac{m + n}{p^k}\rfloor$$
		Comparing term for term these counts we will conclude
		$$\lfloor\frac{2m}{p^i}\rfloor + \lfloor\frac{2n}{p^i}\rfloor
		\geq \lfloor\frac m {p^i}\rfloor + \lfloor\frac{n}{p^i}\rfloor
		+ \lfloor\frac{m+n}{p^i}\rfloor$$
		For any $i \in \{1,2,\dots n\}$\\
		Which would imply that the count of times $p$ divides the
		numerator is $\geq$ the count of times $p$ divides the
		denominator. This would mean that the denominator divides the
		numerator since any prime that divides the denominator, divides
		the numerator at least the same number of times. And so we can conclude that
		$\frac{(2m)!(2n)!}{m!n!(m+n)!}$ is an integer.\\
		To prove
		$$\lfloor\frac{2m}{p^i}\rfloor + \lfloor\frac{2n}{p^i}\rfloor
		\geq \lfloor\frac m {p^i}\rfloor + \lfloor\frac{n}{p^i}\rfloor
		+ \lfloor\frac{m+n}{p^i}\rfloor$$
		We will go through the following cases, if
		$$\lfloor\frac{2m}{p^i}\rfloor + \lfloor\frac{2n}{p^i}\rfloor
		= \lfloor\frac{m}{p^i}\rfloor + \lfloor\frac{n}{p^i}\rfloor +
		\lfloor\frac{m}{p^i}\rfloor + \lfloor\frac{n}{p^i}\rfloor$$
		Then that would mean the rounded off term for $\frac m {p^i}$
		and $\frac n {p^i}$ must both be $< \frac 1 2$, and so
		$$\lfloor\frac m {p^i}\rfloor + \lfloor\frac{n}{p^i}\rfloor +
		\lfloor\frac{m+n}{p^i}\rfloor = \lfloor\frac{m}{p^i}\rfloor +
		\lfloor\frac{n}{p^i}\rfloor + \lfloor\frac{m}{p^i}\rfloor +
		\lfloor\frac{n}{p^i}\rfloor$$
		Similarly if
		$$\lfloor\frac{2m}{p^i}\rfloor + \lfloor\frac{2n}{p^i}\rfloor
		= \lfloor\frac{m}{p^i}\rfloor + \lfloor\frac{n}{p^i}\rfloor +
		\lfloor\frac{m}{p^i}\rfloor + \lfloor\frac{n}{p^i}\rfloor + 1$$
		Then the rounded off term for either $\frac m {p^i}$ or $\frac
		n {p^i}$ is $< \frac 1 2$ and the other rounded off term is
		$\geq \frac 1 2$, which would mean
		$$\lfloor\frac m {p^i}\rfloor + \lfloor\frac{n}{p^i}\rfloor +
		\lfloor\frac{m+n}{p^i}\rfloor \leq \lfloor\frac{m}{p^i}\rfloor +
		\lfloor\frac{n}{p^i}\rfloor + \lfloor\frac{m}{p^i}\rfloor +
		\lfloor\frac{n}{p^i}\rfloor + 1$$
		And lastly if
		$$\lfloor\frac{2m}{p^i}\rfloor + \lfloor\frac{2n}{p^i}\rfloor
		= \lfloor\frac{m}{p^i}\rfloor + \lfloor\frac{n}{p^i}\rfloor +
		\lfloor\frac{m}{p^i}\rfloor + \lfloor\frac{n}{p^i}\rfloor + 2$$
		Then the rounded off term for both $\frac m {p^i}$ and $\frac
		n {p^i}$ are $> \frac 1 2$, which would mean
		$$\lfloor\frac m {p^i}\rfloor + \lfloor\frac{n}{p^i}\rfloor +
		\lfloor\frac{m+n}{p^i}\rfloor = \lfloor\frac{m}{p^i}\rfloor +
		\lfloor\frac{n}{p^i}\rfloor + \lfloor\frac{m}{p^i}\rfloor +
		\lfloor\frac{n}{p^i}\rfloor + 1$$
		Therefore the inequality holds for each of the cases.
\end{ques}




\end{document}
