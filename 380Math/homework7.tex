%29.09.16
\documentclass[12pt]{article}
\usepackage{amsmath, amssymb, amsthm, epsfig}

\newenvironment{definition}{\vspace{2 ex}{\noindent{\bf Definition}}}
        {\vspace{2 ex}}
\newenvironment{ques}{\vspace{2 ex}}{\vspace{2 ex}}


\theoremstyle{definition}

\newenvironment{Proof}{\noindent {\sc Proof.}}{$\Box$ \vspace{2 ex}}
\newtheorem{Wp}{Writing Problem}
\newtheorem{Ep}{Extra Credit Problem}
\renewcommand{\theenumi}{\alph{enumi}}

\oddsidemargin-1mm
\evensidemargin-0mm
\textwidth6.5in
\topmargin-15mm
%\headsep25pt
\textheight8.75in
\footskip27pt

\pagestyle{empty}
\begin{document}

\noindent \textit{\textbf{Math 380a, Fall 2016}} \hspace{1.3cm}
\textit{\textbf{HOMEWORK $\#$7}} \hspace{1.3cm} \textit{\textbf{Peter
Gylys-Colwell}} 

\vspace{1cm}

\begin{ques}
	\textbf{1.} 
		If we foil out the product we have
		$$\prod_{i = 1}^n (1 + x^{2^i}) = 1 + x^2 + x^4 + x^6 \dots x^{2^n - 2}$$
		We know this since every number has a unique binary
		representation, and by multiplying by either a $2^{2^i}$ or by
		$1$ we are either choosing a $1$ or a $0$ for the $i$th digit of
		the binary representation of the power, since $i$ starts at $1$
		and not zero, we know only even powers show up\\
		The series we get is a geometric series whose limit is:
		$$= \frac{1 - (x^2)^{n+1}}{1 - x^2} \to \frac{1}{1 - x^2}$$
\end{ques}

\begin{ques}
	\textbf{2.} 
		We know that the limit must satisfy 
		$$L = \frac{e^L - 1}{2}$$
		and so $L$ is either $0$, or a number $w$, where $2 > w > 1$ such that 
		$$2w + 1 = e^w$$
		Analysing the function $f(x) = \frac{e^{x} - 1}{2} - x$ we
		have $f(x) > 0$ for $x < 0$, $f(x) < 0$ for $w > x > 0$ and
		$f(x) > 0$ for $x >w$\\
		Therefore we know the sequence must diverge for $\alpha = 2$
		since $x_0 = 2 > w$ and $x_{n+1} - x_n = f(x_n)$, $x_n$ would
		be an increasing unbounded sequence.\\
		For $\alpha = .5$, the sequence converges to $0$. We know that
		if $x_n \in (0,w)$ then $f(x_n) < 0$ so $x_n > x_{n+1}$. We
		also know that if $x_n > 0$ then $x_{n+1} > 0$ since $e^{x_n} >
		1$. Therefore $x_n$ is a decreasing sequence bounded from below
		by $0$
\end{ques}

\begin{ques}
	\textbf{3.} 
		The limit goes to infinity.\\
		We know that the limit must satisfy
		$$L = L + 10^{-10^L}$$
		so
		$$10^{-10^L} = 0$$
		so $L \to \infty$
\end{ques}

\begin{ques}
	\textbf{4.} 
		We can rewrite the sum as
		$$\frac{1}{n}\sum_{j=1}^{n}\frac{1}{1 + \frac{j^2}{n^2}}$$
		This is a riemann sum of the function
		$$\frac{1}{1 + x^2}$$
		over the interval $(0, 1)$.\\
		Therefore by the fundamental thm of calculus we have
		$$\lim_{n \to \infty} \frac{1}{n}\sum_{j=1}^{n}\frac{1}{1 +
		\frac{j^2}{n^2}} = \int_0^1 \frac{1}{1 + x^2}\, dx =
		\tan^{-1}(1) - \tan^{-1}(0) = \frac \pi 2$$
\end{ques}
\end{document}
