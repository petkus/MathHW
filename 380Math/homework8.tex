%29.09.16
\documentclass[12pt]{article}
\usepackage{amsmath, amssymb, amsthm, epsfig}

\newenvironment{definition}{\vspace{2 ex}{\noindent{\bf Definition}}}
        {\vspace{2 ex}}
\newenvironment{ques}{\vspace{2 ex}}{\vspace{2 ex}}


\theoremstyle{definition}

\newenvironment{Proof}{\noindent {\sc Proof.}}{$\Box$ \vspace{2 ex}}
\newtheorem{Wp}{Writing Problem}
\newtheorem{Ep}{Extra Credit Problem}
\renewcommand{\theenumi}{\alph{enumi}}

\oddsidemargin-1mm
\evensidemargin-0mm
\textwidth6.5in
\topmargin-15mm
%\headsep25pt
\textheight8.75in
\footskip27pt

\pagestyle{empty}
\begin{document}

\noindent \textit{\textbf{Math 380a, Fall 2016}} \hspace{1.3cm}
\textit{\textbf{HOMEWORK $\#$8}} \hspace{1.3cm} \textit{\textbf{Peter
Gylys-Colwell}} 

\vspace{1cm}

\begin{ques}
	\textbf{1.} 
		For the exponential generating function $f(x)$ of $a_n$, we
		have
		$$f(x) = \sum_{i = 0}^{\infty} \frac{a_i}{i!} x^i, f'(x) =
		\sum_{i = 0}^{\infty} \frac{a_{i+ 1}}{i!} x^i$$
		Since $a_{n+1} = a_{n} + na_{n-1}$ we have
		$$f'(x) = \sum_{i = 1}^{\infty} \frac{a_{i+ 1}}{i!} x^i + a_1 =
		\sum_{i = 1}^{\infty} \frac{a_{i}}{i!} x^i +\sum_{i =
		1}^{\infty} \frac{ia_{i - 1}}{i!} x^i + a_1$$
		$$ = \sum_{i = 0}^{\infty} \frac{a_{i}}{i!} x^i - a_0 +\sum_{i =
		1}^{\infty} \frac{a_{i - 1}}{(i - 1)!} x^i + a_1$$
		$$= f(x) - a_0 + xf(x) + a_1 = f(x) + xf(x)$$
		solving the differential equation we get
		$$\frac{f'(x)}{f(x)} = 1 + x$$
		$$\frac{d}{dx}\log(f(x)) = 1 + x$$
		$$f(x) = ce^{x + x^2}$$
		Since $f(0) = 1$, we have $c = 1$\\
		We can use the product of generating functions to get
		$$f(x) = e^{x}e^{2x} = \left(\sum_{k=0}^\infty \frac{x^k}{k!}
		\right)\left(\sum_{j=0}^\infty \frac{x^{2j}}{j!} \right) =
		\sum_{n=0}^\infty\sum_{\{k + 2j = n\}} \frac{x^n}{j!k!}$$
		And so we have
		$$a_n = n!\sum_{\{k + 2j = n\}} \frac{1}{j!k!}$$

\end{ques}

\begin{ques}
	\textbf{2.} 
		We have the generating functions 
		$$(x + 1)^{m+n} = \sum_{k = 0}^{m+n}\binom{m+n}{k}x^k$$
		And
		$$(x + 1)^m = \sum_{k = 0}^{m} \binom{m}{k}x^k, (x +
		1)^n = \sum_{k = 0}^{n} \binom{n}{k}x^k$$
		Since $(x+1)^m(x+1)^n = (x + 1)^{m+n}$, we have
		$$\left(\sum_{k = 0}^{m} \binom{m}{k}x^k\right)
		\left(\sum_{k = 0}^{n} \binom{n}{k}x^k\right) = \sum_{k
		= 0}^{m+n}\binom{m+n}{k}x^k$$
		When taking the product of two ordernary generating
		functions with sequences $a_n, b_n$ we know the product
		has generating sequence $c_n = \sum_{j = 0}^n
		a_jb_{n-j}$. Applying this to the product of generating
		functions on the left we have:
		$$\sum_{k=0}^{m+n}\sum_{j=0}^k
		\binom{m}{j}\binom{n}{k-j}x^k = \sum_{k =
		0}^{m+n}\binom{m+n}{k}x^k$$
		Since these are equal polinomials, the coefficents must
		be equal, so
		$$\sum_{j=0}^k \binom{m}{j}\binom{n}{k-j} =
		\binom{m+n}{k}$$

\end{ques}

\begin{ques}
	\textbf{3.} 
		\begin{enumerate}
			\item
				We have
				$$f(x) = \sum_{n = 0}^\infty \frac{a_n}{n!}x^n$$
				Differentiating $k$ times we have all the terms
				where $n < k$ dissapear and the rest have the
				form:
				$$f^{(k)}(x) = \sum_{n = k}^\infty \frac{a_n
				n(n-1)(n -2) \dots (n-k + 1)}{n!}x^{(n - k)}$$
				Therefore when we plug in $x = 0$, all the
				terms are zero except the term where $x$ has a
				zero power which is the term when $n = k$:
				$$f^{(k)}(0) = \frac{a_k k(k-1)(k-2) \dots 2
				\cdot 1}{k!} = \frac{a_kk!}{k!} = a_k$$
			\item
				Consider the generating function:
				$$f(x) = \sum_{n=0}^\infty x^n = \frac{1}{1 - x}$$
				If we take a product of $k$ of these functions we get
				$$f(x)f(x)\dots f(x) = (f(x))^k = \sum_{n=0}^\infty a_nx^n$$
				Where the $a_n$ counts all the ways to choose
				$x_1, x_2, \dots x_k$ exponents from the terms
				in each $f(x)$ in order for
				them to add up to $n$. Therefore $a_n$ is the
				sequence described in the problem.\\
				We can now calculate $a_n$ by taking the $n$th
				derivative and plugging in $0$, we have
				$$\frac{d}{dx} (f(x))^k = \frac{d}{dx}(1 -
				x)^{-k} = k(1-x)^{-k -1}$$
				And so we have
				$$\frac{d^n}{dx^n} (x -1)^{-k} = k(k+1)(k+2)
				\dots (k+n - 1)(1-x)^{-k-n}$$
				Plugging in zero we get
				$$a_n n! = k(k + 1) \dots (k+n-1)$$
				And so 
				$$a_n = \frac{k(k+1) \dots (k+n-1)}{n!} =
				\binom{n + k -1}{k-1}$$
		\end{enumerate}
\end{ques}

\begin{ques}
	\textbf{4.} 
		There is such a subset defined as follows:
		$S = \{n:$ the base $4$ representation of $n$ only contains
		$0$'s and $1$'s$\}$.\\
		We know that every positive integer has a unique base $4$
		representation. For a given $n > 0$ we can look at any digit
		place $d$. We will signify $d(n)$ to be the $d$th digit of $n$
		in base $4$\\
		We know that for any $x, y \in S$ none of the digits will carry
		in $x + 2y$ since $d(x) \leq 1, d(2y) \leq 2$ so $d(x + 2y) =
		d(x) + 2d(y)$\\
		If $d(n) = 0$, then there is only one possible way to have
		$d(n) = d(x) + d(2y)$ which is $d(x) = 0$ and $d(y) = 0$, if
		$d(n) = 1$, then the only possibility is  $d(x) = 1, d(y) = 0$,
		if $d(n) = 2$ then $d(x) = 0, d(y) = 1$, and finally if $d(n) =
		3$ then $d(x) = 1$ and $d(y) = 1$.\\
		Therefore for every digit of $n$, the digits of $x$ and $y$ are
		uniquely determined, and satisfy $d(n) = d(x + 2y)$, and so for
		any $n$, $x, y$ are uniquely determined and satisfy $n = x +
		2y$ So $S$
		has the desired property
		
\end{ques}
\end{document}
