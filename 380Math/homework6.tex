%29.09.16
\documentclass[12pt]{article}
\usepackage{amsmath, amssymb, amsthm, epsfig}

\newenvironment{definition}{\vspace{2 ex}{\noindent{\bf Definition}}}
        {\vspace{2 ex}}
\newenvironment{ques}{\vspace{2 ex}}{\vspace{2 ex}}


\theoremstyle{definition}

\newenvironment{Proof}{\noindent {\sc Proof.}}{$\Box$ \vspace{2 ex}}
\newtheorem{Wp}{Writing Problem}
\newtheorem{Ep}{Extra Credit Problem}
\renewcommand{\theenumi}{\alph{enumi}}

\oddsidemargin-1mm
\evensidemargin-0mm
\textwidth6.5in
\topmargin-15mm
%\headsep25pt
\textheight8.75in
\footskip27pt

\pagestyle{empty}
\begin{document}

\noindent \textit{\textbf{Math 380a, Fall 2016}} \hspace{1.3cm}
\textit{\textbf{HOMEWORK $\#$6}} \hspace{1.3cm} \textit{\textbf{Peter
Gylys-Colwell}} 

\vspace{1cm}

\begin{ques}
	\textbf{1.} We can prove this by induction:\\
		Base Case: for $n = 1$, we have $\frac{a_1}{a_1} \geq 1$ is true.\\
		For the inductive step we look at 
		$$((a_1 + \dots + a_n) + a_{n+1})((\frac 1{a_1} + \dots +\frac
		1{a_n}) + \frac 1{a_{n+1}})$$
		$$= (a_1 + \dots a_n)(\frac 1{a_1} + \dots + \frac 1{a_n}) +
		a_{n+1}(\frac 1{a_1} + \dots +\frac 1{a_n})+ (a_1 + \dots
		a_n)\frac{1}{a_{n+1}}+ 1$$
		By the inductive hypothesis
		$$\geq n^2 + a_{n+1}(\frac 1{a_1} + \dots +\frac 1{a_n})+ (a_1 + \dots
		a_n)\frac{1}{a_{n+1}}+ 1$$
		If we look term for term at 
		$$a_{n+1}(\frac 1{a_1} + \dots +\frac 1{a_n})+ (a_1 + \dots
		a_n)\frac{1}{a_{n+1}}$$
		We have 
		$$\sum_{i=1}^n \frac{a_{n+1}}{a_i} + \frac{a_i}{a_{n+1}}
		= \sum_{i=1}^n \frac{a_i^2 + a_{n+1}^2}{a_ia_{n+1}}$$
		now consider 
		$$2a_ia_{n+1} < 2a_ia_{n+1} + a_i^2 + a_{n+1}^2 = (a_i + a_{n+1})^2$$
		And by the cauchy shwartz ineq:
		$$ \leq a_i^2 + a_{n+1}^2$$
		Therefore 
		$$1 = \frac{a_i^2 + a_{n+1}^2}{a_i^2 + a_{n+1}^2} < \frac{a_i^2
		+ a_{n+1}^2}{2a_ia_{n+1}}$$
		And so 
		$$\sum_{i=1}^n \frac{a_i^2 + a_{n+1}^2}{a_ia_{n+1}} > \sum_{i=1}^n 2 = 2n$$
		So we have
		$$(a_1 + \dots + a_n + a_{n+1})(\frac 1{a_1} + \dots +\frac
		1{a_n} + \frac 1{a_{n+1}}) \geq n^2 + 2n + 1 = (n+1)^2$$
		Which is the desired result
\end{ques}

\begin{ques}
	\textbf{2.} 
		\begin{enumerate}
			\item 
				Using the two number form of the AM-GM inequality we have
				$$\frac{a_1 + a_2 + a_3 + a_4}{4} =
				\frac{\frac{a_1 + a_2}2 + \frac{a_3 + a_4}
				2}{2} \geq \sqrt{\frac{a_1 + a_2}2 \frac{a_3 +
				a_4} 2}$$
				We can use the inequality again since $ab \leq
				cd \Rightarrow \sqrt{ab} \leq \sqrt{cd}$ for
				any $a, b, c, d \geq 0$. 
				$$\geq \sqrt{\sqrt{a_1a_2} \sqrt{a_3 a_4}} =
				\sqrt[4]{a_1a_2a_3a_4}$$
			\item 
				For three numbers $a_1, a_2, a_3$ we have the
				following from the AM-GM result for four
				numbers:
				$$\frac{a_1 + a_2 + a_3 + \frac{a_1 + a_2 +
				a_3}{3}}{4} \geq \sqrt[4]{a_1a_2a_3}\sqrt[4]
				{\frac{a_1 + a_2 + a_3}{3}}$$
				We observe that
				$$\frac{a_1 + a_2 + a_3 + \frac{a_1 + a_2 +
				a_3}{3}}{4} = \frac{3a_1 + 3a_2 + 3a_3 + a_1 +
				a_2 + a_3}{12} = \frac{a_1 + a_2 + a_3}{3}$$
				Therefore if we divide $\sqrt[4] {\frac{a_1 +
				a_2 + a_3}{3}}$ on both sides of our AM-GM
				inequality we get
				$$\left(\frac{a_1 + a_2 + a_3}{3}\right)^{3/4} \geq
				\sqrt[4]{a_1a_2a_3} = \left(\sqrt[3]{a_1a_2a_3}\right)^{3/4}$$
				If we take both sides to the $4/3$ power we get
				the desired result
			\item
				We can prove the AM-GM in general with a
				modified induction. We will prove that the $n$
				case implies the $2n$ case and then prove that
				the $n$ case implies the $n-1$ case. This would
				Imply the AM-GM for all $n$ since we prove it
				for every power of two, and then prove it for
				all numbers less than a power of two. The base
				case is the $n= 2$ which was proven in class.\\
				To prove the $2n$ case, given 
				$$\{a_1, a_2 \dots a_n, a_{n+1} \dots a_{2n}\}$$
				We will define $A$ as the Arithmetic mean and
				$G$ the Geometric mean of $\{a_1 \dots a_n\}$
				and $A'$ as the Arithmetic mean and $G'$ the
				Geometric mean  of $\{a_{n+1} \dots a_{2n}\}$\\
				Using the AM-GM for the $n = 2$ case we have
				$$\frac{A + A'}{2} \geq \sqrt{A}\sqrt{A'}$$
				By the inductive hypothesis we have $G \leq A$,
				$G' \leq A'$ so
				$$\sqrt{A}\sqrt{A'} \geq \sqrt{G}\sqrt{G'}$$
				So
				$$\frac{A + A'}{2} \geq \sqrt{G}\sqrt{G'}$$
				We observe that
				$$\sqrt{G}\sqrt{G'} = \sqrt{(a_1 \dots
				a_n)^{1/n}}\sqrt{(a_{n + 1} \dots
				a_{2n})^{1/n}} = (a_1 \cdot a_2 \dots
				a_{2n})^{1/2n} = G_{2n}$$
				Where $G_{2n}$ is the Geometric mean of $\{a_1
				\dots a_{2n}\}$.\\
				We also have that
				$$\frac{A + A'}{2} = \frac{a_1 + \dots a_n +
				a_{n+1} + \dots a_{2n}}{2n} = A_{2n}$$
				where $A_{2n}$ is the Arithmetic mean of $\{a_1
				\dots a_{2n}\}$.\\
				So we have
				$$A_{2n} \geq G_{2n}$$
				Now to prove the $n-1$ case.\\
				for a given $\{a_1, \dots a_{n-1}\}$ we define
				$a_n = A_{n-1}$ where $A_{n-1}$ is the
				Arithmetic mean of the set. By our inductive
				hypothesis we have
				$$A_n \geq G_n$$
				Where $A_n$ and $G_n$ are the Arithmetic and
				Geometric means of $\{a_1 \dots a_n\}$\\
				We have that
				$$A_n = \frac{a_1 + \dots a_{n-1}}{n} +
				\frac{a_1 + \dots a_{n-1}}{n(n-1)} =
				\frac{(n-1)(a_1 + \dots a_{n-1}) + a_1 + \dots
				a_{n-1}}{n(n-1)} = A_{n-1}$$
				We also have
				$$G_n = \sqrt[n]{a_1 \dots
				a_{n-1}}\sqrt[n]{A_{n-1}} =
				G_{n-1}^{(n-1)/n}A_{n-1}^{1/n}$$
				If we substitute these equations into our
				original inequality we have
				$$A_{n-1} \geq G_{n-1}^{(n-1)/n}A_{n-1}^{1/n}$$
				dividint the $A_{n-1}^{1/n}$ on both sides yields
				$$A_{n-1}^{(n-1)/n} \geq G_{n-1}^{(n-1)/n}$$
				And so 
				$$A_{n-1} \geq G_{n-1}$$




		\end{enumerate}
\end{ques}

\begin{ques}
	\textbf{3.} 
		We can prove this by induction:\\
		Base Case ($n = 1$) is true since there is only one element in
		the set $\{\frac{a_1}{b_1}\}$\\
		Now looking at proving
		$$\frac{a_1 + \dots a_{n+1}}{b_1 + \dots b_{n+1}} \leq
		\frac{A_{n+1}}{B_{n+1}}$$
		where $A_{n+1}, B_{n+1}$ are the numerator and denominator
		respectively of the max of $\{\frac{a_1}{b_1}, \dots
		\frac{a_{n+1}}{b_{n+1}}\}$.\\ 
		We can assume as the inductive hypothesis 
		$$\frac{a_1 + \dots a_n}{b_1 + \dots b_n} \leq \frac{A_n}{B_n}$$
		where $A_n, B_n$ are the numerator and denominator respectively
		of the max of $\{\frac{a_1}{b_1}, \dots \frac{a_n}{b_n}\}$.\\
		This is the case if and only if
		$$B_n\sum_{i=1}^n a_i \leq A_n \sum_{i=1}^n b_i$$
		We also know
		$$\frac{A_n}{B_n} \leq \frac{A_{n+1}}{B_{n+1}}$$
		since $\{\frac{a_1}{b_1}, \dots \frac{a_n}{b_n}\}$ is contained
		in $\{\frac{a_1}{b_1}, \dots \frac{a_{n+1}}{b_{n+1}}\}$, so
		$$\frac{B_{n+1}}{B_n} \leq \frac{A_{n+1}}{A_{n}}$$
		so
		$$B_{n+1}\sum_{i=1}^n a_i \leq A_{n+1} \sum_{i=1}^n b_i$$
		similarly we have
		$$\frac{A_{n+1}}{B_{n+1}} \geq \frac{a_{n+1}}{b_{n+1}}
		\Rightarrow A_{n+1}b_{n+1} \geq B_{n+1}a_{n+1}$$
		so
		$$B_{n+1}\sum_{i=1}^n a_i +  B_{n+1}a_{n+1} \leq A_{n+1}
		\sum_{i=1}^n b_i + A_{n+1}b_{n+1}$$
		$$\Updownarrow$$
		$$\frac{a_1 + \dots a_{n+1}}{b_1 + \dots b_{n+1}} \leq
		\frac{A_{n+1}}{B_{n+1}}$$
		This proves that the fraction of the sums is always less than or equal to
		the max of the ratios. To prove that it is always greater than
		or equal to the min of the ratios, we just swap the $a_i$s and
		$b_i$s. Then the max of the ratios of these terms would be the
		min of the ratio of our original ratios, so we have
		$$\frac{b_1 + \dots b_{n}}{a_1 + \dots a_{n}} \leq \frac{\beta_n}{\alpha_n}$$ 
		So
		$$\frac{a_1 + \dots a_{n}}{b_1 + \dots b_{n}} \geq \frac{\alpha_n}{\beta_n}$$ 
		Where $\frac{\alpha_n}{\beta_n}$ is the min of the ratios
\end{ques}

\begin{ques}
	\textbf{4.} 
		We can prove this by induction:\\
		Base Case: for $n = 2$ we have:
		$$(1 + a_1)(1 + a_2) = 1 + a_1 + a_2 + a_1a_2$$
		And since $a_1, a_2$ are the same sign and nonzero, we know $a_1a_2 > 0$ so
		the assersion holds.\\
		Now we have 
		$$(1 + a_1)(1+ a_2) \dots (1 + a_{n+1}) = (1 + a_1) \dots (1 +
		a_n) + (1 + a_1) \dots (1 + a_n)a_{n+1}$$
		By the inductive hypothesis we know that
		$$> 1 + 2 + \dots a_n + (1 + a_1) \dots (1 + a_n)a_{n+1}$$
		Now we go through cases, if all the $a_i$ terms are positive,
		then $1 + a_i > 1$ so 
		$$(1 + a_1) \dots (1 + a_n)a_{n+1} > a_{n+1}$$
		so we have the desired result
		$$(1 + a_1)(1+ a_2) \dots (1 + a_{n+1})> 1 + 2 + \dots a_n + a_{n+1}$$
		And the other case is $a_i$ terms are negative but  $> -1$ then
		$1 > a_i + 1 > 0$. Therefore 
		$$0 < (1 + a_1) \dots (1 + a_n) < 1$$
		so 
		$$a_{n+1}(1 + a_1) \dots (1 + a_n) > a_{n+1}$$
		so we have the desired result
		$$(1 + a_1)(1+ a_2) \dots (1 + a_{n+1})> 1 + 2 + \dots a_n + a_{n+1}$$
		
\end{ques}

\begin{ques}
	\textbf{5.} 
		We can use the AM-GM inequality for $\{1, \dots n\}$.\\
		We have
		$$AM = \frac{1 + 2 + \dots n}n = \frac{n(n+1)}{2n}= \frac{n+1}{2}$$
		and
		$$GM = \sqrt[n]{1 \cdot 2 \dots n} = \sqrt[n]{n!}$$
		And since the terms of $\{1, \dots n\}$ are not equal for $n >
		1$, we know that the AM-GM is a strict inequality:
		$$\frac{n+1}{2} > \sqrt[n]{n!}$$
		$$\left(\frac{n+1}{2}\right)^n > \sqrt{n!}$$
\end{ques}




\end{document}
