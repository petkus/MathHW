%29.09.16
\documentclass[12pt]{article}
\usepackage{amsmath, amssymb, amsthm, epsfig}

\newenvironment{definition}{\vspace{2 ex}{\noindent{\bf Definition}}}
        {\vspace{2 ex}}
\newenvironment{ques}{\vspace{2 ex}}{\vspace{2 ex}}


\theoremstyle{definition}

\newenvironment{Proof}{\noindent {\sc Proof.}}{$\Box$ \vspace{2 ex}}
\newtheorem{Wp}{Writing Problem}
\newtheorem{Ep}{Extra Credit Problem}
\renewcommand{\theenumi}{\alph{enumi}}

\oddsidemargin-1mm
\evensidemargin-0mm
\textwidth6.5in
\topmargin-15mm
%\headsep25pt
\textheight8.75in
\footskip27pt

\pagestyle{empty}
\begin{document}

\noindent \textit{\textbf{Math 380a, Fall 2016}} \hspace{1.3cm}
\textit{\textbf{HOMEWORK $\#$9}} \hspace{1.3cm} \textit{\textbf{Peter
Gylys-Colwell}} 

\vspace{1cm}

\begin{ques}
	\textbf{1.} 
		\begin{enumerate}
			\item
				The probability is $1$ since

			\item
				We want to find all the walks where number of
				steps he has taken away from the cliff (we will
				call $A$) is $\geq$ the number of steps taken
				towards the cliff (we will call $B$) except for
				on his $k$th step in which case $A - B = -1$
				$$P_k = \frac{1}{2}(P_1P_{k-2} + P_3P_{k-4} +
				\dots P_{k-2}P_1)$$
		\end{enumerate}

\end{ques}

\begin{ques}
	\textbf{2.}  %donnnnnnnnnnnnnnnnnnnnnnnnnnnnnnnnnnnnnnnnne
		We can simply integrate the density of $1$ over the area of the
		regions where if the center of the coin landed, the coin would
		be inside a square, and divide by the total area of the chess
		board. $C$ has $8 \times 8$ squares, so $64$ squares,
		each $d^2$ area so $64d^2$ area total. There is a small square
		with side length $d - a$ sharing the center point of each
		square of $C$ such that if the center of the coin lands in any of these
		squares, the coin would be contained entirely in a square of
		$C$. If the center of the coin is not in one of these squares,
		it will intersect with the boundary of one of the squares of
		$C$. There are $64$ of these squares, so the total area the
		coin's center can land on is $64(d-a)^2$. So the probability is 
		$$\frac{64(d-a)^2}{64d^2} = \frac{(d-a)^2}{d^2}$$

\end{ques}

\begin{ques}
	\textbf{3.}  %donnnnnnnnnnnnnnnnnnnnnnnnnnnnnnnnnnnnnnnnne
		We have already established in class that the expected value of
		the minimum of a set of $n$ random numbers in $[0, 1]$ is
		$\frac{1}{n+1}$. If we shift all the values over by $-1$, we would
		shift the expected value by $-1$ as well and it would be the
		expected value of the min of the $n$ terms in $[-1, 0]$. And
		finally if we multiply each term by $-1$, we would multiply the
		expected value by $-1$, and we would be measureing the maximum
		value of the $n$ terms in $[0,1]$, which is what we want.
		Therefore our expected value is 
		$$-(\frac{1}{n+1} - 1) = 1 - \frac{1}{n+1} = \frac{n}{n+1}$$

\end{ques}

\begin{ques}
	\textbf{4.} 
		For a given $P$, the probablity that our rectangle is inside
		$C$ is the area of the rectangle inscribed in $C$ with one of
		the corners being $P$ and its sides parallel to the axis
		divided by the area of the whole circle. Any
		$Q$ outside of this rectangle would create a rectangle that has
		a side that extends beyond the bounds of $C$ and any $Q$ inside
		this rectangle would create a rectangle contained in the
		inscribed rectangle which is contained in $C$. Therefore to
		find the whole probability, we can integrate uniformly over the
		domain of $P$ the areas of these inscribed rectangles divided
		by the area of $C$ and divided by the arclength we are
		integrating over. Using polar coordinates we have
		$$\frac{1}{2\pi}\int_{0}^{2\pi} \frac{|2\cos(\theta)2\sin(\theta)|}{\pi} \ d
		\theta = \frac{1}{2\pi}4\int_{0}^{\pi/2}
		\frac{4\cos(\theta)\sin(\theta)}{\pi} \ d \theta$$
		Using a u-substitution for $u = \sin(\theta), du =
		\cos(\theta)$:
		$$= \frac{4}{2\pi}\frac{4\sin^2(\pi/2)}{2\pi} = \frac{4}{\pi ^ 2}$$
\end{ques}

\begin{ques}
	\textbf{EC.}  %donnnnnnnnnnnnnnnnnnnnnnnnnnnnnnnnnnnnnnnnne
		We simulate in the following way. We flip the coin twice if the
		coin pattern is $H,T$ then we say our balanced coin result
		would be $T$, if the pattern is $T,H$ then we say $H$. Any
		other pattern and we ignore it and repeat the process. We have
		that the probablity of $H, T$ is $p(1-p)$ while the probablity
		of $T,H$ is $(1-p)p$, and so the probabilities are equal,
		therefore we are simulating a balanced coin. We know that this
		process must terminate since the probablility of not
		terminating after $n$ times is $(2(1-p)p)^n = (2(p - p^2))^n <
		1^n$ and so as $n \to \infty$, $(2(1-p)p)^n \to 0$. (The reason
		$2(p - p^2) < 1$ is because the function $f(x) = x - x^2$ on
		$[0,1]$ is $<\frac{1}{2}$, we can use the derivative to find
		the critical point at $\frac{1}{2}$ to conclude this)

\end{ques}
\end{document}
