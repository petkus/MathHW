\documentclass[12pt]{article}
\usepackage{amsmath, amssymb, amsthm, epsfig}

\newenvironment{definition}{\vspace{2 ex}{\noindent{\bf Definition}}}
        {\vspace{2 ex}}

\newenvironment{ques}[1]{\textbf{Exercise #1}\vspace{1 mm}\\ }{\bigskip}

\renewcommand{\theenumi}{\alph{enumi}}

\theoremstyle{definition}

\newenvironment{Proof}{\noindent {\sc Proof.}}{$\Box$ \vspace{2 ex}}
\newtheorem{Wp}{Writing Problem}
\newtheorem{Ep}{Extra Credit Problem}

\oddsidemargin-1mm
\evensidemargin-0mm
\textwidth6.5in
\topmargin-15mm
\textheight8.75in
\footskip27pt


\renewcommand{\l}{\left }
\renewcommand{\r}{\right }

\newcommand{\R}{\mathbb R}
\newcommand{\Q}{\mathbb Q}
\newcommand{\Z}{\mathbb Z}
\newcommand{\C}{\mathbb C}
\newcommand{\N}{\mathbb N}
\newcommand{\interior}[1]{%
  {\kern0pt#1}^{\mathrm{o}}%
}

\newcommand{\s}{\sin}
\renewcommand{\c}{\cos}

\renewcommand{\t}{\theta}
\renewcommand{\a}{\alpha}

\newcommand{\norm}[1]{\left\lVert#1\right\rVert}

\newcommand{\T}{\mathfrak{T}}

\pagestyle{empty}
\begin{document}

\noindent \textit{\textbf{Math 424, Fall 2017}} \hspace{1.3cm}
\textit{\textbf{HOMEWORK $\#$5}} \hspace{1.3cm} \textit{\textbf{Peter
Gylys-Colwell}} 

\vspace{1cm}

\begin{ques}{22}
	It does follow. For any Cauchy sequence $(a_n)_n$ with the limit $a$ (if it
	exists), we have that the set $S = (a_n)_n \cup \{a\}$ is closed and
	bounded (We know that every Cauchy sequence is bounded). From our midterm
	problem we know that any subsequence of $(a_n)_n$ converges to $a$.
	Therefore in order for $S$ to be compact, $a$ must exist in $S$ and so
	exists in $M$
\end{ques}

\begin{ques}{23}
	We know that $(0,1)$ is open in $\R$ as proven in class. For any $r$, we
	have that $(1/2,r/2) \in B_r(1/2,0)$ but $(1/2,r/2) \notin (0,1) \times \{0\}$
	therefore $B_r(1/2,0) \not \subseteq (0,1) \times \{0\}$. Therefore $(0,1)
	\times \{0\}$ is not open. 
\end{ques}


\begin{ques}{28}
	(a) \ Not necessarily. Consider the map from the unit circle $f : S^1 \to
	[0, 2\pi)$ where $f(\c(\t), \s(\t)) = \t$. We have that the inverse image
	of the open set $[0,\epsilon)$ is the closed set $\{(\c(\t),\s(\t)): \t \in
	[0, \epsilon)\}$\\
	(b) \ Yes. Since $f$ has a continous inverse mapping $f^{-1}$ for any open
	set $U \subseteq M$ we have that the pullback of $f^{-1}$ of an open set is
	open. Thus $f(U) = (f^{-1})^{-1}(U)$ is open.\\
	(c) \ Yes. Since $f$ is bijective it has an inverse $f^{-1}$. For any open
	set $U$, the pullback of $f^{-1}$ of $U$ is just $f(U)$ which is open. Thus
	$f^{-1}$ is continous. So $f$ is a homeomorphism. \\
	(d) \ Not necessarily. Consider the map $f(x) = \frac 1 3 x^3 - x$. We know all
	polinomials are continous, and $f$ is clearly surjective. The 'humps' where
	the derivative is zero is $1, -1$, thus the open sets $(-1 -\epsilon, 1 +
	\epsilon)$ for small $\epsilon$ will be mapped to the closed set $[-\frac 2
	3, \frac 2 3]$
\end{ques}

% done
\begin{ques}{32}
	For any point $p \in \N$ we have that for $r = 1$, by definition the set
	$B_r(p) = \{p\}$ is open.\\
	Therefore singletone points are open in $\N$, so any set $S \subseteq \N$
	is open since
	$$S = \bigcup_{s \in S} \{s\}$$
	And we know arbitrary unions of open sets are open. Therefore we have that
	$S^{c}$ is open as well. The complement of an open set is closed so we know
	that $S$ is closed as well. Therefore every set $S \subseteq \N$ is clopen.\\
	This means that every function $f: \N \to M$ is continous since the inverse
	image of any open set $U \subseteq M$ will be open.
\end{ques}

\begin{ques}{34}
	For any closed set $L \subset N$ with $N$ closed from the inheritance
	principle we know $L = C \cap N$ for some closed set $C \subset M$.
	Intersections of closed sets are closed. Thus $L$ is closed in $M$.
	Conversly if $L$ is closed in $M$ then $L = N \cap L$ so $L$ is closed in
	$N$\\
	Similarly if $U \subset N$ is open and $N$ is open,  then from the
	inheritance principle $U = V \cap N$ where $V$ is open in $M$. Finite
	intersections of open sets are open, thus $U$ is open in $M$.
	Conversly if $L$ is open in $M$ then $L = N \cap L$ so $L$ is open in
	$N$
\end{ques}

\begin{ques}{38}
	For $d_E$:  \\
	Checking all the axioms of metrics:\\
	$\sqrt{a^2 + b^2}$ is clearly nonegative for all $a,b \in \R$\\
	$\sqrt{d_X(a_x,c_x)^2 + d_Y(a_y,c_y)^2} = 0$ iff $d_X(a_x,c_x) =
	d_Y(a_y,c_y) = 0$ iff $x = c$
	It is clear $d_E(x,y) = d_E(y,x)$\\
	For $d_E(a, c) \leq d_E(a,b) + d_E(b,c)$ we have $\sqrt{d_X(a_x,c_x)^2 +
	d_Y(a_y,c_y)^2}$\\ 
	$\leq \sqrt{(d_X(a_x,b_x) + d_X(b_x,c_x))^2 +
	(d_X(a_x,b_x) + d_X(b_x,c_x))^2} \leq \sqrt{d_X(a_x,c_x)^2 + d_Y(a_y,c_y)^2} +
	\sqrt{d_X(a_x,c_x)^2 + d_Y(a_y,c_y)^2}$ \\ 
	For $d_\text{max}$:\\

	For $d_\text{sum}$:\\

\end{ques}

\begin{ques}{52}
	(a) \ Letting 
\end{ques}

\begin{ques}{Additional Problem 1}
	
\end{ques}
\end{document}
