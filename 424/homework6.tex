\documentclass[12pt]{article}
\usepackage{amsmath, amssymb, amsthm, epsfig}

\newenvironment{definition}{\vspace{2 ex}{\noindent{\bf Definition}}}
        {\vspace{2 ex}}

\newenvironment{ques}[1]{\textbf{Exercise #1}\vspace{1 mm}\\ }{\bigskip}

\renewcommand{\theenumi}{\alph{enumi}}

\theoremstyle{definition}

\newenvironment{Proof}{\noindent {\sc Proof.}}{$\Box$ \vspace{2 ex}}
\newtheorem{Wp}{Writing Problem}
\newtheorem{Ep}{Extra Credit Problem}

\oddsidemargin-1mm
\evensidemargin-0mm
\textwidth6.5in
\topmargin-15mm
\textheight8.75in
\footskip27pt


\renewcommand{\l}{\left }
\renewcommand{\r}{\right }

\newcommand{\R}{\mathbb R}
\newcommand{\Q}{\mathbb Q}
\newcommand{\Z}{\mathbb Z}
\newcommand{\C}{\mathbb C}
\newcommand{\N}{\mathbb N}
\newcommand{\interior}[1]{%
  {\kern0pt#1}^{\mathrm{o}}%
}

\newcommand{\s}{\sin}
\renewcommand{\c}{\cos}

\renewcommand{\t}{\theta}
\renewcommand{\a}{\alpha}

\newcommand{\norm}[1]{\left\lVert#1\right\rVert}

\newcommand{\T}{\mathfrak{T}}

\newcommand{\dist}{\text{dist}}

\pagestyle{empty}
\begin{document}

\noindent \textit{\textbf{Math 424, Fall 2017}} \hspace{1.3cm}
\textit{\textbf{HOMEWORK $\#$5}} \hspace{1.3cm} \textit{\textbf{Peter
Gylys-Colwell}} 

\vspace{1cm}

\begin{ques}{41}
	Let us define the metric $d(x,y) = |x - y|$. Then $B = \{x \in \R^m : d(x,0)
	\leq 1\}$. For any convergent sequence $(a_n)_n \in B$ with limit $a$, we have
	that $a \in B$ since if $d(a,0) = 1 + \epsilon$ then for any $N > 0$ and $n >
	N$ we have that $d(a_n, a) \leq d(a_n, 0) + d(a,0) \Rightarrow d(a_n,a)  -
	d(a_n,0) \geq d(a,0)$ so $d(a_n,a) \geq \epsilon$ which cotradicts
	convergence. Thus $B$ is closed. $B$ is clearly bounded since $d(x,0) \leq
	1 \forall x \in B$. Thus since $B$ is closed, bounded, and a subset of
	$\R^m$, it is compact.
\end{ques}

\begin{ques}{42}
	The problem is that it is not necessarily true that the convergent
	subsequences in $(a_n)_n$ and $(b_n)_n$ will have the same indicies. So we
	dont necessarily have that $(a_{n_k}, b_{n_k})$ exists as a convergent
	subsequence of $(a_n, b_n)$ 
\end{ques}

\begin{ques}{43}
	For any sequences $(a_n)_n \in A$, $(b_n)_n \in B$, since $A \times B$ is
	compact we have that the sequence $(a_n,b_n)$ has a
	convergent subsequence $(a_{n_k},b_{n_k})$. We know that a sequence in $M \times
	N$ converges iff the components in $M$ and $N$ converge. Therefore
	$a_{n_k}$ and $b_{n_k}$ are convergent subsequences for $(a_n)_n, (b_n)_n$
	respectivly. Thus $A, B$ are compact.
\end{ques}

\begin{ques}{44}
	(a) \ For any convergent sequence $(m_n, y_n)_n \in G, m_n \in M,
	y_n \in \R$ where $G$ is the graph of $f$ with the limit $(m_n,y_n) \to
	(m,y)$, we have that $y_n = f(m_n)$. Thus since $f$ is continuous we have
	that $f(m) = f(y)$ and thus $(y,m) \in G$, so $G$ is closed\\
	\\
	(b) \ For any sequence $(m_n, y_n)_n \in G, m_n \in M,
	y_n \in \R$, since $M$ is compact, there exists a convergent subsequence
	$(m_{n_k})$ of $(m_n)_n$, and thus $(m_{n_k},y_{n_k}) \to
	(m,y)$, we have that $y_n = f(m_n)$. Thus since $f$ is continuous we have
	that $f(m) = f(y)$ and thus $(y,m) \in G$, so $G$ is compact\\
	\\
	(c) \ Suppose for contradiction there is a convergent sequence $m_n \in M$
	with limit $m$ where $f(m_n)$ does not converge. Thus there exists a
	$\epsilon > 0$ where we can choose a subsequence $y_{n_k} = f(m_{n_k})$
	such that $d(y_{n_k}, f(m)) > \epsilon$. However no subsequence of 
	$(m_{n_k}, y_{n_k})$ converges since if we have any convergent subsequence
	$(m_{n_{k_l}}, y_{n_{k_l}})_l$ then we have that $m_{n_{k_l}} \to m$ yet
	$y_{n_{k_l}} \to y \neq f(m)$. Therefore $G$ would not contain the limit
	point of $(m_{n_{k_l}}, y_{n_{k_l}})_l$ which would contradict $G$ being
	closed (which is implied by compactness).  Thus we contradict
	compactness.\\
	\\
	(d) \ We can define the discontinous function
	$$f(x) = \begin{cases} 
      0 & x = 0 \\
      \frac{1}{x} & x \neq 0
   \end{cases}$$
   We have the graph is the union of three closed sets in $\R^2$ the singleton
   $\{0\}$, and two curves $\{(x,y): y = \frac 1 x, x > 0\}$ and $\{(x,y): y =
   \frac 1 x, x < 0\}$ which are closed, thus the graph is closed.
\end{ques}

\begin{ques}{46}
	We have that $A \times B$ is the product of compact sets and thus compact.
	We know that the distance function $d$ is continuous, and thus $d: A \times
	B \to \R$ maps to a compact set. Thus $d(A \times B)$ is compact so it
	contains its smallest value. Thus we have that $\exists (a,b) \in A \times
	B$ with $d(a,b) \leq d(a_0,b_0)$ for all $(a_0,b_0) \in A \times B$\\
\end{ques}

\begin{ques}{53}
	This is true. For each $K_n$ choose two points points $a_n,b_n \in K_n$
	where $d(a_n,b_n) = $diam $K_n$. We have the sequence $(a_n,b_n)_n \in
	K_1^2$. Since $K_1$ is compact we know that there exists a subsequence
	$(a_{n_k}, b_{n_k})_k$ which converges to $(a,b)$. We have the limits
	for the components (since a sequence conveges iff its components converge)
	$a,b \in K$ since each $K_i$ contains the tail of the subsequences
	$a_{n_k}, b_{n_k}$ for $n_k > i$ (which has the same limit) and since
	each $K_i$ is closed, it must contain the limit thus each $K_i$
	contians $a,b$.  Now we have that $d(a_{n_k},b_{n_k})$ is a convergent
	sequence converging to $d(a,b)$ since $d$ is continuous. Since
	$d(a_{n_k},b_{n_k}) \geq \mu$ we know that its limit $d(a,b) \geq \mu$.
	Thus diam $K \geq \mu$
\end{ques}

\begin{ques}{55}
	(a) \ If $p$ is a limit, then we have a sequence $(p_n)_n \in S$ where for each
	$\epsilon > 0$ we can choose a $p_n$ where $d(p_n,p) < \epsilon$ and 
	the inf$\{d(p_n,p)\} = 0$ we have inf$\{d(p_n,p)\} \geq \dist(S,p) \geq
	0$, thus  $\dist(S,p) = 0$. Conversely if $\dist(S,p) = 0$ then for $\epsilon =
	\frac 1 n$ we can choose $p_n \in S$ such that $d(p_n,p) < \epsilon$. Thus
	we have that the sequence $(p_n)_n$ converges to $p$.\\
	\\
	(b) \ For any $\epsilon > 0$ let $\delta = \epsilon$. For any $p, q$ with
	$d(p,q) < \delta$ we have that for any $s \in S$
	$$d(q,s) \leq d(p,q) + d(p,s) \Rightarrow d(q,s) - d(p,s) \leq d(p,q) < \epsilon$$
	relabeling $q$ and $p$ yields the other inequality: $d(p,s) - d(q,s) \leq
	\epsilon$. Thus we have that $|d(p,s) - d(q,s)| < \epsilon$ for all $s \in
	S$. We have that 
	$$|\dist(p,S) - \dist(q,S)| = |\inf\{ d(p,s), s \in S\} -
	\inf\{ d(q,s), s \in S\}| \leq \inf\{|d(p,s) - d(q,s)|, s \in S\} \leq \epsilon$$
	We have the first inequality since for any $d(p,s_1) - d(q,s_2)$ we have
	that $|d(p,s_1) - d(q,s_2)| \leq |d(p,s_1) - d(q,s_1)|$. Thus we have that
	$\dist(p,S)$ is uniformly continuous.

\end{ques}

\begin{ques}{Additional Problem 1}
	Suppose for contradiction that $A/(U_1 \cup \dots U_n)$ is not empty for
	all $n$. Then we can choose a sequence $(a_n)_n$ where $a_n \in A/(U_1 \cup
	\dots U_n)$. Since $A$ is compact, there exists a subsequence $(a_{n_k})_k
	\in A$ which converges to $a$. We have that $a \in U_N$ for some $N$ since
	$A$ is a union of the $U_i$s. Since $U_N$ is open there is a $r$ such that
	$B_r(a) \subset U_N$. However this is a contradiction of convergence of
	$a_{n_k}$ since for all $n_k >N$ we have that $a_{n_k} \notin U_N$ and thus
	not in $B_r(a)$. Thus we have that $A/(U_1 \cup \dots \cup U_N) =
	\emptyset$ so $A \subset (U_1 \cup \dots \cup U_N)$.
\end{ques}

\begin{ques}{Additional Problem 2}
	We know the norm is continuous, thus $|f(x)|$ is a continuous function
	mapping to $\R$. We know the preimage of open sets are open. We can use
	these facts as follows: \\
	Suppose for contradiction $f$ was unbounded. We have that 
	$$\R = (-1,1) \cup (-2,2) \cup (-3,3) \cup \dots (-n,n) \cup \dots$$.
	Thus if we take the preimage of $f$ we get a union of open sets
	$$A = U_1 \cup U_2 \cup \dots \cup U_n \cup \dots $$
	We have that $A/(U_1 \cup U_2 \cup \dots U_N) \neq \emptyset $ for all $N$
	and thus $A$ is not a finite union of open sets which is a contradiction.
	We know that $A/(U_1 \cup U_2 \cup \dots U_N) \neq \emptyset $ since $f$ is
	unbounded there exists $a \in A$ such that $|f(a)| > N$ and thus $f(a)
	\notin (-N, N)$ which means $a$ is not in the preimages $U_1 \dots U_N$
\end{ques}
\end{document}
