\documentclass[12pt]{article}
\usepackage{amsmath, amssymb, amsthm, epsfig}

\newenvironment{definition}{\vspace{2 ex}{\noindent{\bf Definition}}}
        {\vspace{2 ex}}

\newenvironment{ques}[1]{\textbf{Exersise #1}\vspace{1 mm}\\ }{\bigskip}

\renewcommand{\theenumi}{\alph{enumi}}

\theoremstyle{definition}

\newenvironment{Proof}{\noindent {\sc Proof.}}{$\Box$ \vspace{2 ex}}
\newtheorem{Wp}{Writing Problem}
\newtheorem{Ep}{Extra Credit Problem}

\oddsidemargin-1mm
\evensidemargin-0mm
\textwidth6.5in
\topmargin-15mm
\textheight8.75in
\footskip27pt


\renewcommand{\l}{\left }
\renewcommand{\r}{\right }

\newcommand{\R}{\mathbb R}
\newcommand{\Q}{\mathbb Q}
\newcommand{\Z}{\mathbb Z}
\newcommand{\C}{\mathbb C}
\newcommand{\N}{\mathbb N}
\newcommand{\interior}[1]{%
  {\kern0pt#1}^{\mathrm{o}}%
}

\newcommand{\s}{\sin}
\renewcommand{\c}{\cos}

\renewcommand{\t}{\theta}
\renewcommand{\a}{\alpha}

\newcommand{\norm}[1]{\left\lVert#1\right\rVert}

\newcommand{\T}{\mathfrak{T}}

\pagestyle{empty}
\begin{document}

\noindent \textit{\textbf{Math 424, Fall 2017}} \hspace{1.3cm}
\textit{\textbf{HOMEWORK $\#$4}} \hspace{1.3cm} \textit{\textbf{Peter
Gylys-Colwell}} 

\vspace{1cm}

\begin{ques}{7}
	For any convergent sequence $(p_n)$, let $p$ be the limit of $(p_n)$. We
	can choose $\epsilon = 1$ and from the definition of convergent there
	exists $N$ such that $\delta(p_n,p) < 1$ for all $n > N$, the number of
	numbers $\delta(p_i, p)$s with $i < N$ is finite therefore we  can choose
	the $p_k$ with the largest $\delta(p_i,p)$. Therefore we have that $(p_n)$
	is bounded by $B = 1 + \delta(p_k, p)$ around $p$ since for any $p_i$ if $i >
	N$ then $\delta(p_i,p) < 1 < B$ and if $i \leq N$ then from how $p_k$ was
	chosen we know $\delta(p_i,p) \leq \delta(p_k,p) < B$.
\end{ques}

\begin{ques}{8}
\begin{enumerate}
	\item 
		For any $\epsilon > 0$ we have by definition there exists a limit $x$
		and a $N >0$ such that $|x - x_n| < \epsilon$ for all $n > N$. We have
		that $||x| - |x_n|| = |x| - |x_n|$ or $|x_n| - |x|$ depending on if $x
		> x_n$ or $x \leq x_n$. From the triangle ineq we have $|x| - |x_n|
		\leq |x - x_n|$ and $|x_n| - |x| \leq |x - x_n|$ thus $||x| - |x_n||
		\leq |x - x_n|$. Therefore for all $n > N$ we have $||x| - |x_n|| <
		\epsilon$ and thus $(|x_n|)$ converges to $|x|$
	\item
		If $(|x_n|)$ converges in $\R$ then $(x_n)$ converges in $\R$
	\item
		This is not true. Consider the sequence $(a_n) = (-1)^n$. We have that
		$(|a_n|) = (1) \to 1$ while for any $N >0$ we can choose $\epsilon = 1$
		and there exists $a_n, a_{n+1}$ with $n > N$ and $|a_n - a_{n+1}| >
		\epsilon$ thus $(a_n)$ is not Cauchy and so not convergent
\end{enumerate}
\end{ques}


\begin{ques}{14}
\begin{enumerate}
	\item
		If we have the isometry $f: M \to N$, for any open set $U \subseteq N$
		and any point in the preimage $x \in f^{-1}(U)$, since $U$ is open we
		know there exists $r \in \R^+$ such that $B_r(f(x)) \subset U$. I claim
		that $B_r(x) \subseteq f^{-1}(U)$ and thus $f^{-1}(U)$ is open so $f$ is
		continuous. The argument is the following:\\
		We have that for any point $p \in B_r(x)$ that $d_M(x,p) <r$ and we
		have that $d_M(x, p) = d_N(f(x), f(p)) < r$ thus 
		$$f(p) \in B_r(f(x)) \Rightarrow f(p) \in U \Rightarrow p \in f^{-1}(U)
		\Rightarrow B_r(x) \subseteq f^{-1}(U)$$
	\item
		Notice that since $f$ is bijective, $f^{-1}$ is a well defined
		function. $f^{-1}$ is also an isometry since we have that for any
		$x, y \in N$ there exists $p,q \in M$ where $f(p) = x, f(q) = y$ so
		$$d_N(x,y) = d_N(f(p),f(q)) = d_M(p, q) \Rightarrow
		d_M(f^{-1}(x),f^{-1}(y)) = d_M(p, q) = d_N(x,y)$$
		Therefore $f^{-1}$ is continuous as we have proven in (a) so $f$ is a
		homeomorphism as it fits the topological definition of a homeomorphism.
	\item
		If there exists an isometry $f: [0,1] \to [0,2]$ then since $f$ is
		surjective there exists $a,b \in [0,1]$ where $f(a) = 0, f(b) = 2$
		however we have that $\delta(a,b) \leq 1$, however
		$\delta(f(a),f(b)) = \delta(0, 2) > 1$. Therefore
		$\delta(a,b) \neq \delta(f(a),f(b))$ which contradicts $f$ being an isometry.
		
\end{enumerate}
\end{ques}

\begin{ques}{25}
	The only possible sequence of points in the singleton set $\{p\}$ is the
	constant sequence $(p_n)n: p_n = p \forall n$, which converge to $p$. Thus
	a singlton set contains all its limit points and so is closed.\\
	Every finite set of points is a finite union of singletons which are
	closed, since the finite union of closed sets is closed, the finite set of
	points is closed.
\end{ques}

\begin{ques}{26}
	If none of $U$s points are limits of its complement, then its complement
	contains all of its limit points, and thus is closed. The complement of a
	closed set is open so $U$ must be open. Conversly if $U$ is open then the
	complement of $U$ is closed so the complement contains all of its limit
	points, since $U$ and its complement are disjoint, this means that $U$ does
	not contain any of its complement's limit points.
\end{ques}

\begin{ques}{27}
	(a) \ For any point $p \in \bar S$, we have that that there exists a
	sequence $(p_n)_n$ contained in $S$ such that $(p_n)_n \to p$ since $S
	\subset T$, $(p_n)_n$ is a sequence of points in $T$ as well and thus $p$
	is a limit point in $T$, therefore $p \in \bar T$ so $\bar S \subset \bar
	T$\\
	(b) \ For any point $p \in \interior S$, we have that there exists an $r$
	such that $B_r(p) \subset S$ since $S \subset T$ we have that $B_r(p)
	\subset T$ and therefore $p \in \interior T$ so $\interior S \subset
	\interior T$
\end{ques}

\begin{ques}{19}
	$\Q$ is not homeomorphic to $\N$. \\
	We have that every subset of $\N$ is open. The reason is because for any
	point in any subset $p \in T \subset \N$ we let $r = 1$ then $B_r(p) =
	\{p\} \subset T$.\\
	Therefore for any bijection $f:\Q \to \N$ the inverse image of a singleton
	is a singlton: $f^{-1}(\{p\}) = \{q\}$ and in $\N$ $\{p\}$ is open, but
	singletons in $\Q$ are not open, so the inverse image of an open set is not
	open, therefore $f$ is not continuous so not a homeomorphism.
\end{ques}

\begin{ques}{30}
	If there exists a metric $\delta$ that defined $\T$ then by the axioms of
	metrics, we know $\delta(a,b) \neq 0$ since $a \neq b$ thus for $r =
	\delta(a,b)$, from the way topologies are defined by a metric we have that
	$B_r(a) \in \T$ however $B_r(a) = \{a\}$ since $\delta(a,b) \not < r$ so $b
	\notin B_r(a)$ thus $\{a\} \in \T$ which is a contradiction.
\end{ques}

\end{document}
