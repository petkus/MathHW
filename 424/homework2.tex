%29.09.16
\documentclass[12pt]{article}
\usepackage{amsmath, amssymb, amsthm, epsfig, mathtools}

\newenvironment{definition}{\vspace{2 ex}{\noindent{\bf Definition}}}
        {\vspace{2 ex}}

%\newenvironment{ques}{\vspace{2 ex}}
%Numbered environment 
\newcounter{ques}[section]
\newenvironment{ques}[1]{\textbf{Exersise #1} \vspace{1mm}}{\medskip}

\renewcommand{\theenumi}{\alph{enumi}}

\theoremstyle{definition}

\DeclarePairedDelimiter{\ceil}{\lceil}{\rceil}
\newenvironment{Proof}{\noindent {\sc Proof.}}{$\Box$ \vspace{2 ex}}
\newtheorem{Wp}{Writing Problem}
\newtheorem{Ep}{Extra Credit Problem}

\oddsidemargin-1mm
\evensidemargin-0mm
\textwidth6.5in
\topmargin-15mm
%\headsep25pt
\textheight8.75in
\footskip27pt

\pagestyle{empty}
\begin{document}

\noindent \textit{\textbf{Math 424, Fall 2017}} \hspace{1.3cm}
\textit{\textbf{HOMEWORK $\#$1}} \hspace{1.3cm} \textit{\textbf{Peter
Gylys-Colwell}} 

\vspace{1cm}

\begin{ques}{20}
	If $a_n \to a$ and $a_n \to b$ with $b \neq a$, let $\epsilon = \frac
	{|b - a|} 2 > 0$. By definition there exists $N$ such that for all $k
	> N$, $|a_k - a| < \epsilon$. However by triangle inequality, for any
	$k > N$ we have 
	$$|b - a| = |b - x_k + x_k - a| \leq |b - x_k| + |x_k - a|$$
	and so since $|a - x_k| < \frac{|b-a|}{2} > 0$
	$$|b-a| - \frac {|b-a|}{2} < |b - a| - |x_k - a| \leq |b - x_k|$$
	So we have $|b - x_k| > \frac{|b - a|}{2} = \epsilon$ for all $k > N$.
	Thus $x_n \not \to b$, which is a contradiction
\end{ques}

\begin{ques}{1}
	Let $x \cdot 1^* = A|A'$, $1^* = B|B'$ and $x = C|C'$. By definition 
	$$A = \{bc : b, c \geq 0, b \in B, c \in C \text{ or } a : a < 0, a \in
	B \cup C\}$$
	Let $a \in A$ if $a < 0$ then $a \in C$ since $x > 1^* > 0^*$ we have
	$S = \{a \in \mathbb Q: a < 0\} \subset B \subset C$\\
	If $a \geq 0$ then $a = bc$ for some $b \in B, c \in C$. By definition
	of $1^*$ we know $b < 1$ and so $bc < c$ and therefore $bc = a \in C$
	since $a \in \mathbb Q$ and if $a \in \mathbb Q - C = C'$ then $a < c$
	which contradicts the axiom that elements of $C'$ are greater than
	elements of $C$. Therefore we have $A \subseteq C$\\
	For any $c \in C$, if $c < 0$ we know $c \in B$ and $c \in C$ so $c \in A$.\\
	If $c \geq 0$ we have that since $C$ does not contain an upper bound,
	there must exist a $c' \in C$ such that $c < c'$ therefore $\frac c
	{c'} \in \mathbb Q$ and $ < 1$. Therefore $\frac c {c'} \in B$ and we
	have $c' \frac c {c'} = c \in A$. Therefore $C \subseteq A$. Thus we
	have equality, $C =A \Rightarrow x \cdot 1^* = x$

\end{ques}

\begin{ques}{2}
	\begin{enumerate}
		\item
			If $a_n$ is Cauchy then we know that $a_n$ is bounded,
			so $\forall n, a_n < B$ for some $B \in \mathbb R$. For a given
			$\epsilon >0$ there is a $N >0$ such that $j,k > N$
			implies $|a_j - a_k| < \frac{\epsilon}{ 2B}$. We have
			that (since $|a_j + a_k| < 2M$)
			$$|a_j^2 - a_k^2| = |a_j - a_k||a_j + a_k| <\frac{\epsilon}{ 2B} 2B$$
			And thus $|a_j^2 - a_k^2| < \epsilon$, so $a^2_n$ is Cauchy
		\item
			If $a^2_n$ is Cauchy, it is bounded. We have for
			some $C \in \mathbb R$, $C^2 > a_n^2 \geq c^2$. Given
			$\epsilon > 0$, there is a $N$ such that for any $j,k >
			N$ we have $|a_j^2 - a_k^2| < 2C\epsilon$. Again we
			have 
			$$|a_j^2 - a_k^2| = |a_j - a_k||a_j + a_k| <
			\frac{\epsilon}{2C} |a_j + a_k|$$
			and we have $|a_j + a_k| < 2C$, so 
			$$|a_j - a_k| < \epsilon$$
			Thus $a_n$ is Cauchy

	\end{enumerate}
\end{ques}

\begin{ques}{3}
	Base Case:\\
	$n = 1$, $1 + c = 1 + c$\\
	Inductive Step:\\
	assuming $n$th case we have
	$$(1 + c)^{n+1} = (1 + c)^n + c(1 + c)^n \geq 1 + nc + c(1 + c)^n \geq
	1 + nc + c = 1 + (n+1)c$$
	And thus the $n+1$ case is true
\end{ques}

\begin{ques}{4}
\begin{enumerate}
		\item
			Since $r>0$ we can write $r$ as $r = 1 + c$ with $c >
			0$. Suppose such an upper bound $x$ existed, then we
			have $x \leq 1 + \ceil{(x - 1)/c}c$. However as proven
			in problem 3, if we let $n = \ceil{(x - 1)/c} + 1$ then
			$r^n \geq 1 + nc > 1 + (n - 1)c = x$
		\item
			Since $r > 0$, we know $\frac 1 {r^n} > 0$. For
			$\epsilon > 0$, we know from part a that there exists
			$N$ such that $r^N > \frac 1 \epsilon$, and for all $n
			> N$, $r^n > \frac 1 \epsilon$ since $r^{n-N} > 1$ and
			$r^n = r^Nr^{n-N}$.
			Therefore we have $|r^n| > |\frac 1 \epsilon|$, and so
			$|\frac 1 {r^n}| < \epsilon$. Thus $\frac 1 {r^n} \to
			0$

	\end{enumerate}
\end{ques}
\begin{ques}{5}
	From the Triangle Ineq:
	$$|y| + |x - y| \geq |y + x - y| = |x|$$
	So
	$$|x - y| \geq |x| - |y|$$
	This argument works relabeling $x$ and $y$, so $|y - x| \geq |y| -
	|x|$. Depending on which is larger, we know $||x| - |y|| = |x| - |y|$
	or $|y| - |x|$, either way, we have
	$$|x- y| = |y-x| \geq ||x| - |y||$$
\end{ques}
\begin{ques}{6}
	If $x = \lambda y$, we know that $|\langle \lambda y, y \rangle | =
	|\lambda \langle y, y \rangle | = |\lambda |y|^2| = |x||y|$. \\
	If $|\langle x, y\rangle | = |x||y|$, we can define $Q(t) = \langle x +
	ty, x + ty\rangle$. By bilinear properties of the inner product we have
	$$Q(t) = \langle x + ty, x + ty\rangle = \langle x, x + ty \rangle +
	\langle ty, x + ty \rangle$$ 
	$$= \langle x, x \rangle  + t\langle x, y \rangle + t\langle y, x
	\rangle + t^2\langle y, y \rangle$$
	By assumtion that $|\langle x, y\rangle | = |x||y|$
	$$= |x|^2 + 2t|x||y| + t^2|y|^2$$
	$$= (|x| + t|y|)^2$$
	Letting $t = -\frac {|x|}{|y|}$ we have
	$$Q(-\frac {|x|}{|y|}) = \left(|x| + -\frac {|x|}{|y|}|y|\right)^2 = 0$$
	Therefore we have
	$$\langle x + -\frac {|x|}{|y|}y, x + -\frac {|x|}{|y|}y\rangle = 0$$
	Which is the case if and only if $x  -\frac {|x|}{|y|}y = 0
	\Rightarrow x = \frac {|x|}{|y|}y$


\end{ques}
\end{document}
