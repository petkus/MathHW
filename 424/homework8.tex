\documentclass[12pt]{article}
\usepackage{amsmath, amssymb, amsthm, epsfig}

\newenvironment{definition}{\vspace{2 ex}{\noindent{\bf Definition}}}
        {\vspace{2 ex}}

\newenvironment{ques}[1]{\textbf{Exercise #1}\vspace{1 mm}\\ }{\bigskip}

\renewcommand{\theenumi}{\alph{enumi}}

\theoremstyle{definition}

\newenvironment{Proof}{\noindent {\sc Proof.}}{$\Box$ \vspace{2 ex}}
\newtheorem{Wp}{Writing Problem}
\newtheorem{Ep}{Extra Credit Problem}

\oddsidemargin-1mm
\evensidemargin-0mm
\textwidth6.5in
\topmargin-15mm
\textheight8.75in
\footskip27pt


\renewcommand{\l}{\left }
\renewcommand{\r}{\right }

\newcommand{\R}{\mathbb R}
\newcommand{\Q}{\mathbb Q}
\newcommand{\Z}{\mathbb Z}
\newcommand{\C}{\mathbb C}
\newcommand{\N}{\mathbb N}
\renewcommand{\i}{\text{int} \ }
\newcommand{\interior}[1]{%
  {\kern0pt#1}^{\mathrm{o}}%
}

\newcommand{\s}{\sin}
\renewcommand{\c}{\cos}

\renewcommand{\t}{\theta}
\renewcommand{\a}{\alpha}

\newcommand{\norm}[1]{\left\lVert#1\right\rVert}

\newcommand{\T}{\mathfrak{T}}

\newcommand{\dist}{\text{dist}}

\pagestyle{empty}
\begin{document}

\noindent \textit{\textbf{Math 424, Fall 2017}} \hspace{1.3cm}
\textit{\textbf{HOMEWORK $\#$8}} \hspace{1.3cm} \textit{\textbf{Peter
Gylys-Colwell}} 

\vspace{1cm}

\begin{ques}{91}
	Given any $\epsilon > 0$, consider the covering of $N$ by $\epsilon/2$-
	neighborhoods $B = \{B_{\epsilon/2}(q): q \in N\}$ and the preimage $P =
	\{f^{-1}(S): S \in B\}$. Since $\cup_{S \in B} S = N$, we have that
	$\cup_{U \in P} U = M$. Thus $P$ covers $M$ (and $P$ is a collection open
	sets since $f$ is continuous and we are taking preimages of open sets) so
	from the lebesgue number lemma there exists $\lambda > 0$ such that for any
	$m \in M$ there is a $U \in P$ such that $B_\lambda(m) \subset U$. Thus for
	any $x,y \in N$ where $d(x,y) < \lambda$ we have that $x,y \in
	B_\lambda(x)$, thus from what we have shown there is a $m \in M$ such that
	$B_\lambda(x) \subset f^{-1}(B_{\epsilon / 2}(m))$ so $f(x), f(y) \in
	B_{\epsilon / 2}(m)$. Thus from the triangle ineq, $d_M(f(x),f(y)) <
	d_M(f(x),m) + d_M(f(y),m) \leq \epsilon$. Thus $f$ is uniformly continuous.
\end{ques}

\begin{ques}{93}
	We can consider the complements. Let $\mathcal U = \{U = M - C: C \in
	\mathcal C\}$. The finite intersection property translates to for any
	finite collection $U_1, U_2, \dots U_n \in \mathcal U$, we have that from
	Demorgans law:
	$$\bigcup_{i=1}^n U_i = \bigcup_{i=1}^n M - C_i = M - \bigcap_{i = 1}^n C_i \neq M$$
	Thus $\mathcal U$ does not contain a finite subcovering of $M$. Thus it
	must be the case that $M$ is not covered by $\mathcal U$ or we contradict
	covering compact. Thus from Demorgans law
	$$\bigcup_{U \in \mathcal U} U = \bigcup_{C \in \mathcal C} M - C = M -
	\bigcap_{C \in \mathcal C} C \neq M$$
	which is only the case if $\bigcap_{C \in \mathcal C} C \neq \emptyset$
\end{ques}

\begin{ques}{94}
	For any collection of open sets $\mathcal U$ which covers $M$, if the
	finite intersection property holds, consider the complements $\mathcal C =
	\{C = M - U: U \in \mathcal U\}$. Since $\mathcal U$ covers $M$ we have
	$$M = \bigcup_{U \in \mathcal U} U =  \bigcup_{C \in \mathcal C}M -  C = M
	- \bigcap_{C \in \mathcal C} C $$
	Thus $\bigcap_{C \in \mathcal C} C = \emptyset $. Thus $\mathcal C$ must
	not satisfy the finite intersection property so there exists $C_1, C_2,
	\dots C_n$ such that 
	$$\bigcap_{i=1}^n C_i = \emptyset \Rightarrow M = M - \bigcap_{i=1}^n C =
	\bigcup_{i=1}^n M - C_i = \bigcup_{i=1}^n U_i$$
	Thus we have a finite subcover.
\end{ques}

\begin{ques}{96}
	From the definition of dense we have that $B \subset \overline A$, thus
	$\overline B \subset \overline A$ since $\overline B$ is contained in every
	closed set which contains $B$. Since $B$ is dense in $C$ we have $C \subset
	\overline B \subset \overline A$. Thus $A$ is dense in $C$
\end{ques}

\begin{ques}{1}
	We have that for all $x$
	$$f'(x) = \lim_{t \to x} \frac{f(x) - f(t)}{x-t}$$
	From what we are givien however we have that 
	$$\frac{|f(x) - f(t)|}{|x-t|} \leq |x-t|$$
	Thus
	$$-|x-t| \leq f'(x) = \frac{f(x) - f(t)}{x-t} \leq |x-t|$$
	Thus as $t \to x$ we get $0 \leq f'(x) \le 0 \Rightarrow f'(x) = 0$ for all
	$x$. Thus $f$ is constant
\end{ques}

\begin{ques}{2}
	(a) \ Given any $\varepsilon > 0$, let $\delta =
	\l(\frac{\varepsilon}{H}\r)^{1/\a}$. We have that for any $x,t \in (a,b)$
	where $|x - t| \le \delta$, $|f(x) - f(t)| \le H|x - t|^\a \le H\delta^\a
	= \varepsilon$, thus $f$ is uniformly continuous.\\
	We can thus extend $f$ by letting $f(a) = \lim_{x \to a} f(x)$, and $f(b) =
	\lim_{x \to b} f(x)$. This extension of $f$ is also $\a$-Holder since if we
	fix $u \in (a,b)$ and consider the limits  $\lim_{x\to a} |f(x) - f(u)|
	\leq \lim_{x \to a} H|x - u|^\a$, since these functions are continuous at $a$
	and we have the inequality $|f(x) - f(u)| \le H|x-u|^\a$ for all $x < a$,
	the inequality holds at $a$. This same argument can be applied to $b$
	(as well as now $x= a, u = b$ since we showed it holds for $u = a$ and $x \in
	(a,b)$) by relabeling the terms.\\
	\\
	(b) \ This is the Lipshitz condition.\\
	\\
	(c) \ We have that for all $x$
	$$f'(x) = \lim_{t \to x} \frac{f(x) - f(t)}{x-t}$$
	From what we are givien however we have that 
	$$\frac{|f(x) - f(t)|}{|x-t|} \leq |x-t|^{\a - 1}$$
	Thus
	$$-|x-t|^{\a - 1} \leq f'(x) = \frac{f(x) - f(t)}{x-t} \leq |x-t|^{\a - 1}$$
	Thus as $t \to x$ we get $0 \leq f'(x) \le 0 \Rightarrow f'(x) = 0$ for all
	$x$. Thus $f$ is constant
\end{ques}

\begin{ques}{3}
	(a) \ Suppose for contradiction there exists $x,y \in (a,b)$ with $x < y$ and
	$f(x) \ge f(y)$. However from the mean value theorem (since $f$ is
	differentiable on $[x,y] \subset (a,b)$ the mvt holds) we have that
	$\exists \t \in (x,y)$ such that 
	$$f'(\t) = \frac{f(y) - f(x)}{y-x}$$
	However $f(y) - f(x) \le 0$ and $y-x > 0$ thus $f'(\t) \le 0$ which is a
	contradiction.\\
	\\
	(b) \ $f$ is monotone increasing by similar reasoning:\\
	Suppose for contradiction there exists $x,y \in (a,b)$ with $x < y$ and
	$f(x) > f(y)$. However from the mean value theorem (since $f$ is
	differentiable on $[x,y] \subset (a,b)$ the mvt holds) we have that
	$\exists \t \in (x,y)$ such that 
	$$f'(\t) = \frac{f(y) - f(x)}{y-x}$$
	However $f(y) - f(x) < 0$ and $y-x > 0$ thus $f'(\t) < 0$ which is a
	contradiction.\\
\end{ques}

\begin{ques}{9}
	(a) \ let us define $g(x) = x - f(x)$. Notice that $g'(x) = 1 - f'(x) > 1 -
	L > 0$ for all $x$, let $\epsilon = 1- L$. Notice that for any $x$, $g(x) =
	0  \Leftrightarrow f(x) = x$, thus if we find a unique $0$ of $g(x)$ we are
	done.\\
	For existence:\\
	Suppose $g(0) = A < 0$ (if $A = 0$ we are done). We have from the mvt
	$$\frac{g(x) - g(0)}{x} = g'(\t) > \epsilon$$
	for all $x$ and some $\t \in (0,x)$. Rearranging the inequality yields
	$$g(x) > \epsilon x + A$$
	Thus if we choose $x > A/\epsilon$ we get $g(x) > 0$ thus we have $g(x) > 0
	> g(0)$ and so from the intermedite value theorem there exists $\a \in
	(0,x)$ where $g(\a) = 0$\\
	If $g(0) = A > 0$ then now
	$$\frac{g(0) - g(x)}{-x} = g'(\t) > \epsilon$$
	and thus rearranging we get
	$$-g(x) > -x\epsilon - g(0) \Rightarrow g(x) < x\epsilon + A$$
	Thus choosing $x > -A/\epsilon$ we get $g(x) > 0$ and thus from intermedite
	value theorem we get $\exists \alpha \in (0,x)$ such that $g(\a) = 0$.\\
	Uniqueness:\\
	If there exists $\alpha \neq \beta \in \R$ where $g(\alpha) = g(\beta) = 0$
	then from the mean value theorem there exists $\t \in (\alpha, \beta)$ where
	$$g'(\t) = \frac{g(\a) - g(\beta)}{\a - \beta} = 0$$
	But this contradicts $g'(\t) > \epsilon$. Thus this is not possible
\\
	\\
	(b) \ Let $f(x) = x + e^{-x}$. We have that $x \neq x + e^{-x}$ for all $x$
	and $f'(x) = 1 - e^{-x} < 1$ for all $x$ 
\end{ques}

\begin{ques}{Additional Problem 1}
	Given any sequence $x_n \in K$ we can define the chain $A_1 \supset A_2
	\supset \dots $ of relatively closed sets in $K$ as $A_n = \overline{B_n}
	\cap K$ with $B_n = \{x_j: j \geq n\}$. It is clear $A_n \supset A_{n+1}$
	since $B_n \supset B_{n+1}$.\\
	Thus we have from assumption
	$$p \in \bigcap A_n \neq \emptyset$$
	We have that $p$ is the limit of some subsequence of $x_n$ (and thus $K$ is
	compact). We can
	construct this subsequence inductively as follows (letting $n_k =1$):\\
	We  have that $p \in \overline B_n$ for all $n$, thus for $\epsilon = 
	\frac{1}{k}$ there exists $x_{n_k} \in B_{1 + n_{k - 1}}$ so that $d(p,x_{n_k}) <
	\epsilon$. We thus have that $n_k > n_{k-1}$ since $x_{n_k} \in B_{1 +
	n_{k-1}}$ and all the indicies in $B_{1 + n_{k-1}}$ are greater than
	$n_{k-1}$ and thus we have a subsequence. Thus we have the
	subsequence $(x_{n_k})_k \to p$ since
	$d(x_{n_k}, p) \to 0$
\end{ques}
\end{document}
