%29.09.16
\documentclass[12pt]{article}
\usepackage{amsmath, amssymb, amsthm, epsfig}

\newenvironment{definition}{\vspace{2 ex}{\noindent{\bf Definition}}}
        {\vspace{2 ex}}

%\newenvironment{ques}{\vspace{2 ex}}
%Numbered environment 
\newcounter{ques}[section]
\newenvironment{ques}[1]{\textbf{Exersise #1} \vspace{1mm}}{\medskip}

\renewcommand{\theenumi}{\alph{enumi}}

\theoremstyle{definition}

\newenvironment{Proof}{\noindent {\sc Proof.}}{$\Box$ \vspace{2 ex}}
\newtheorem{Wp}{Writing Problem}
\newtheorem{Ep}{Extra Credit Problem}

\oddsidemargin-1mm
\evensidemargin-0mm
\textwidth6.5in
\topmargin-15mm
%\headsep25pt
\textheight8.75in
\footskip27pt

\pagestyle{empty}
\begin{document}

\noindent \textit{\textbf{Math 424, Fall 2017}} \hspace{1.3cm}
\textit{\textbf{HOMEWORK $\#$1}} \hspace{1.3cm} \textit{\textbf{Peter
Gylys-Colwell}} 

\vspace{1cm}

\begin{ques}{7}
	\begin{enumerate}
		\item
			For any even integer $n$ we can write it as the product
			$n = 2k$ for some $k \in \mathbb{Z}$. Therefore $n^2 =
			(2k)^2 = 4k^2$ and therefore $4$ divides $n^2$

		\item
			For any even integer $n$ we can write it as the product
			$n = 4k$ for some $k \in \mathbb{Z}$. Therefore $n^3 =
			(2k)^3 = 8k^3$ and therefore $8$ divides $n^3$

		\item
			In the prime factorization of twice and odd cube,
			$2k^3$ where $k$ odd, we know $2$ does not divide $k$
			and therefore does not divide $k^3$ and so there is
			only $2^1$ in the prime factorization of $2k^3$.
			Therefore $8$ cannot divide $2k^3$ since $8 = 2^3$ does
			not divide the powers of $2$ in the prime factorization
			of $2k^3$

		\item
			Suppose for contradiction $\sqrt[3]{2} = \frac a b$
			where $a,b$ are relatively prime. Then we have $2b^3 =
			a^3$. Since $2b^3$ is even, $a^3$ is even. The only way
			it is possible for $a^3$ to be divisible by $2$ is if
			$2$ divides $a$. Therefore $a$ must be even, which
			means $a = 2n$ for some $n \in \mathbb z$ so $a^3 =
			8n^3 = 2b^3$, So $b^3 = 4n^3$. Therefore $b^3$ is even
			which means $b$ must be even
	\end{enumerate}

\end{ques}

\begin{ques}{10}
	Let $x = A|B$, by definition we have $-x = C|D$ where $C = \{r \in
	\mathbb Q : \text{ for some } b \in B, \text{ not the smallest element
	of }B, r = -b\}$ and $D$ is the rest of $\mathbb Q$. By definition we
	have $x + (-x) = E|F$ where $E = \{r \in \mathbb Q: \text{ for some } a
	\in A \text{ and some } c \in C \text{ we have } a + c = r\}$ and $F$
	is the rest of $\mathbb Q$. Since $0^* = N|M = \{r \in \mathbb Q: r <
	0\}|\{r \in \mathbb Q: r \geq 0\}$, we wish to show $N = E \Rightarrow
	x + (-x) = 0$. For any $e \in E$ we have $e = a + c$ for some $a \in A$
	and $c \in C$. From how $C$ was defined we know $c = -b$ for some $b
	\in B$. By definition of a cut we know $a < b$, therefore (subtracting
	$b$ on both sides) we have $a - b < 0$. And so from how $N$ was defined
	we have that $e = a + (- b) \in N$, and therefore $E \subseteq N$. Now
	take any element $n \in N$. We know that $n < 0$. Let $a$ be an element of
	$A$ chosen such that $a + |n/2|$ is not in $A$. We know such an $a$
	exists since if we start with any element of $A$ and iteratively add
	$|n/2|$ we will get arbitrarily large, since $A$ is bounded from
	above by some element of $B$ there must be a iteration which is no
	longer in $A$, and so the previous iteration is our desired $a$.
	Therefore we have $a + |n/2| \in B$ and so (since $n < 0$) we have $x =
	a \in A$ and $y = a - n \in B$.  We have $x + (-y) \in E$ and $x
	+ (-y) = a - (a - n) = n$ so $n \in E$ which means $N \subseteq E$ and
	thus we have equality of the two sets. Thus $x + (-x) = 0^*$
\end{ques}

\begin{ques}{13}
	\begin{enumerate}
		\item
			If there was no $s \in S$ such that $b - \epsilon < s$
			then by definition $b - \epsilon$ would be an upper
			bound of $S$. However $b - \epsilon < b$ and thus
			contradicting $b$ being a least upper bound. Therefore
			there must exist $s \in S$ with $b - \epsilon < s$
		\item
			Yes, as I have proven in part a
		\item
			To show that $x$ is an upper bound:\\
			For any $a \in A$ with $a = A^*|B^*$ and $a \neq x$ we
			have that $A^* = \{q \in \mathbb Q: q < a$. If there
			was $b \in A^*$ with $b \notin A$ then $b < a$, $b \in
			B$ but that contradicts every element of $B$ being
			larger than every element of $A$, therefore $\forall b
			\in A^*, b \in A$ and so $A^* \subset A \Rightarrow a
			<x$.\\
			To show that $x$ is the least upper bound:\\
			If there exists $s < x$ with $s$ an upper bound of $A$.
			Let $s^* = C|D$, since $s < x$ we have $C \subset A$.
			Therefore there exists $a \in A$ with $a \notin C$.
			Since $a \notin C$, $a \in D$ and so $a > c$ for all $c
			\in C$. Letting $a^* = E|F$ we have that $E = \{q \in
			\mathbb Q: q < a\}$. Therefore $C \subseteq E$ and so
			$a \geq s$ since $s$ is an upperbound of $A$, $a = s$,
			which contradicts $A$ not containing any upperbounds
			(condition 3 of cuts). Therefore such an $s$ cannot
			exist



	\end{enumerate}
\end{ques}

\begin{ques}{1}
	\begin{enumerate}
		\item
			$$\{x \in \mathbb Q: x^2 = 2\} = \emptyset$$
		\item
			If $x \in \mathbb Q$ and $x > 0$ then $\exists \ n \in
			\mathbb N$ such that $\frac 1 n < x$

	\end{enumerate}
\end{ques}

\begin{ques}{2}
\begin{enumerate}
		\item
			Let $x = A|B$. We know by definition $B$ is nonempty
			and therefore there exists $y \in B$ with $y \in
			\mathbb Q$. If it is the case that $y = x$ we know that
			$x + 1 \in \mathbb Q$ and we know how ordering works
			with rational numbers well enough to conclude $x +1 >
			x$. Otherwise we have $y = C|D$. For any $a \in  A$ we
			know $y > a$ since $y \in B$. We have by definition $C
			= \{a \in \mathbb Q: a < y\}$ and therefore $a \in C$
			so $A \subseteq C$ and so $y > x$
		\item
			Letting $x = A|B$ we know by definition of $0$ that $x
			> 0 \Rightarrow C \subseteq A$ where $C = \{a \in \mathbb
			Q: a < 0\}$ and $\exists y \in A, y\notin C$. Even
			stronger we can say $\exists z \in A, z > 0$ since if
			$0$ was the only element in $A$ not in $C$ then $A$
			would contain an upperbound since $0$ would be $\geq a \ 
			\forall a \in A$. Therefore $z > 0$. $z$ is less than
			$x$ since $z \in A$ so the set $E$ defined by the cut
			$z = E|F$ is contained in $A$. This is because $E = \{q
			\in \mathbb Q: q < z\}$ and for any $q < z$ we have $q \in A$
			since $q \in B$ would contradict elements of $B$ being
			larger than elements of $A$. If $z = x$ then $E = A$,
			and yet $z \in A = E$ which contradicts how rational
			cuts are defined. Therefore $0 < z < x$

	\end{enumerate}
\end{ques}
\end{document}
