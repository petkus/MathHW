%29.09.16
\documentclass[12pt]{article}
\usepackage{amsmath, amssymb, amsthm, epsfig}

\newenvironment{definition}{\vspace{2 ex}{\noindent{\bf Definition}}}
        {\vspace{2 ex}}

%\newenvironment{ques}{\vspace{2 ex}}
%Numbered environment 
\newcounter{ques}[section]
\newenvironment{ques}[1]{\textbf{Exersise #1} \vspace{1mm}}{\medskip}

\renewcommand{\theenumi}{\alph{enumi}}

\theoremstyle{definition}

\newenvironment{Proof}{\noindent {\sc Proof.}}{$\Box$ \vspace{2 ex}}
\newtheorem{Wp}{Writing Problem}
\newtheorem{Ep}{Extra Credit Problem}

\oddsidemargin-1mm
\evensidemargin-0mm
\textwidth6.5in
\topmargin-15mm
%\headsep25pt
\textheight8.75in
\footskip27pt

\pagestyle{empty}
\begin{document}

\noindent \textit{\textbf{Math 424, Fall 2017}} \hspace{1.3cm}
\textit{\textbf{HOMEWORK $\#$1}} \hspace{1.3cm} \textit{\textbf{Peter
Gylys-Colwell}} 

\vspace{1cm}

\begin{ques}{7}
	\begin{enumerate}
		\item
			For any even integer $n$ we can write it as the product
			$n = 2k$ for some $k \in \mathbb{Z}$. Therefore $n^2 =
			(2k)^2 = 4k^2$ and therefore $4$ divides $n^2$

		\item
			For any even integer $n$ we can write it as the product
			$n = 4k$ for some $k \in \mathbb{Z}$. Therefore $n^3 =
			(2k)^3 = 8k^3$ and therefore $8$ divides $n^3$

		\item
			In the prime factorization of twice and odd cube,
			$2k^3$ where $k$ odd, we know $2$ does not divide $k$
			and therefore does not divide $k^3$ and so there is
			only $2^1$ in the prime factorization of $2k^3$.
			Therefore $8$ cannot divide $2k^3$ since $8 = 2^3$ does
			not divide the powers of $2$ in the prime factorization
			of $2k^3$

		\item
			Suppose for contradiction $\sqrt[3]{2} = \frac a b$
			where $a,b$ are relatively prime. Then we have $2b^3 =
			a^3$. Since $2b^3$ is even, $a^3$ is even. The only way
			it is possible for $a^3$ to be divisible by $2$ is if
			$2$ divides $a$. Therefore $a$ must be even, which
			means $a = 2n$ for some $n \in \mathbb z$ so $a^3 =
			8n^3 = 2b^3$, So $b^3 = 4n^3$. Therefore $b^3$ is even
			which means $b$ must be even
	\end{enumerate}

\end{ques}

\begin{ques}{10}
	Let $x = A|B$, by definition we have $-x = C|D$ where $C = \{r \in
	\mathbb Q : \text{ for some } b \in B, \text{ not the smallest element
	of }B, r = -b\}$ and $D$ is the rest of $\mathbb Q$. By definition we
	have $x + (-x) = E|F$ where $E = \{r \in \mathbb Q: \text{ for some } a
	\in A \text{ and some } c \in C \text{ we have } a + c = r\} $= \{r \in
	\mathbb Q: r < 0\}|\{r \in \mathbb Q: r \geq 0\}$
\end{ques}

\begin{ques}{13}
\end{ques}

\begin{ques}{1}
	\begin{enumerate}
		\item
			$$\{x \in \mathbb Q: x^2 = 2\} = \emptyset$$
		\item
			If $x \in \mathbb Q$ and $x > 0$ then $\exists \ n \in
			\mathbb N$ such that $\frac 1 n < x$

	\end{enumerate}
\end{ques}

\begin{ques}{2}
\end{ques}
\end{document}
