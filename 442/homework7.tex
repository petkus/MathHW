\documentclass[12pt]{article}
\usepackage{amsmath, amssymb, amsthm, epsfig}
\setlength\parindent{0pt}

\newenvironment{definition}{\vspace{2 ex}{\noindent{\bf Definition}}}
        {\vspace{2 ex}}

\newenvironment{ques}[1]{\textbf{#1}\vspace{1 mm}\\ }{\bigskip}

\renewcommand{\theenumi}{\alph{enumi}}

\theoremstyle{definition}

\newenvironment{Proof}{\noindent {\sc Proof.}}{$\Box$ \vspace{2 ex}}
\newtheorem{Wp}{Writing Problem}
\newtheorem{Ep}{Extra Credit Problem}

\oddsidemargin-1mm
\evensidemargin-0mm
\textwidth6.5in
\topmargin-15mm
\textheight8.75in
\footskip27pt


\renewcommand{\l}{\left }
\renewcommand{\r}{\right }

\newcommand{\R}{\mathbb R}
\newcommand{\Q}{\mathbb Q}
\newcommand{\Z}{\mathbb Z}
\newcommand{\C}{\mathbb C}
\newcommand{\N}{\mathbb N}

\renewcommand{\k}{\kappa}
\renewcommand{\t}{\tau}

\renewcommand{\det}{\text{det}}

\newcommand{\s}{\sin}
\renewcommand{\c}{\cos}
\newcommand{\id}{\text{id}}

\renewcommand{\t}{\theta}
\renewcommand{\a}{\alpha}
\renewcommand{\b}{\beta}

\newcommand{\norm}[1]{\left\lVert#1\right\rVert}

\newcommand{\T}{\mathcal{T}}

\pagestyle{empty}
\begin{document}

\noindent \textit{\textbf{Math 442, Winter 2018}} \hspace{1.3cm}
\textit{\textbf{HOMEWORK $\#$7}} \hspace{1.3cm} \textit{\textbf{Peter
Gylys-Colwell}} 

\vspace{1cm}

\begin{ques}{2-5, 1}
	For any $W = ax_u + bx_v \in T_p(S)$, we have that $I_p(W) = Ea^2 + 2Fab +
	Gb^2$ where $E,F,G$ is given as follows: \\
	(a)\
	$$E = \langle x_u, x_u \rangle = (a\c u \c v )^2 + (b \c u \s v)^2 + (c \s u)^2$$
	$$G = \langle x_v, x_v \rangle = (a\s u \s v)^2 + (b \c u \s v)^2$$
	$$F = \langle x_u, x_v \rangle = b^2\c^2 u \s^2 v - a^2 \s u \c u \s v \c v$$
	(b)\
	$$E = \langle x_u, x_u \rangle = (a\c v )^2 + (b \s v)^2 + 4u^2$$
	$$G = \langle x_v, x_v \rangle = a^2u^2\s^2 v + b^2 u^2 \c^2 v$$
	$$F = \langle x_u, x_v \rangle = b^2u\c v \s v - a^2 u \c v \s v$$
	(c)\
	$$E = \langle x_u, x_u \rangle = (a\cosh v )^2 + (b \sinh v)^2 + 4u^2$$
	$$G = \langle x_v, x_v \rangle = a^2u^2\sinh^2 v + b^2 u^2 \cosh^2 v$$
	$$F = \langle x_u, x_v \rangle = b^2u\cosh v \sinh v - a^2 u \cosh v \sinh v$$
	(d)\
	$$E = \langle x_u, x_u \rangle = (a\cosh u \c v )^2 + (b \cosh u \s v)^2 +
	(c \sinh u)^2$$
	$$G = \langle x_v, x_v \rangle = (a \sinh u \s v)^2 v + (b \sinh u \cos v)^2$$
	$$F = \langle x_u, x_v \rangle = b^2\cosh u \sinh u \c v \s v - a^2 \cosh u
	\sinh u \c v \s v$$
	$$= (b^2 - a^2)\cosh u \sinh u \c v \s v $$
\end{ques}

\begin{ques}{2-5 5}
	Parameterize the surface by 
	$$g(x,y) = (x,y,f(x,y))$$
	we have 
	$$g_x = (1, 0, f_x)$$
	$$g_y = (0, 1, f_y)$$
	$$|g_x \wedge g_y| = |(-f_x, -f_y, 1)| = \sqrt{1 + f_x^2 + f_y^2}$$
	Thus by definition of area
	$$A = \iint_Q \sqrt{1 + f_x^2 + f_y^2}$$
\end{ques}

\begin{ques}{2-5 7}
	($\Rightarrow$) For any coordinate curve with respect to $u$
	$x(u,v_0):[u_1,u_2] \to S$ where $v_0$ is fixed, we have that the length of
	the curve is
	$$\int_{u_1}^{u_2} \l|x_u(u,v_0)\r| du = \int_{u_1}^{u_2}
	\sqrt E du$$
	We have that for any choice of $v$ this length must be the same since we
	can form the quadralateral with verticies $(u_1,v_0), (u_2,v_0), (u_1,v_1),
	(u_2,v_1)$ and conclude that the arc length of the curve
	$x(u,v_0):[u_1,u_2] \to S$ is the same as the curve $x(u,v):[u_1,u_2] \to
	S$. Thus
	$$\frac d {dv} \int_{u_1}^{u_2} \sqrt E du = 0 \Rightarrow \frac d {dv} E =
	0$$
	By swapping the labeling of $u$ and $v$ this argument also concludes that
	$\frac d {du} G = 0$\\
	($\Leftarrow$) for any quadralateral with the verticies $(u_1,v_1),
	(u_2,v_1), (u_1,v_2), (u_2,v_2)$, if we have $\frac d {dv} E = 0$ then
	$$\frac d {dv} \int_{u_1}^{u_2} \sqrt E du = 0$$
	and thus the arc length from $u_1$ to $u_2$ is constant with respect to $v$.\\
	Similarly $\frac d {du} G = 0$ implies the arc length from $v_1$ to $v_2$
	is constant with respect to $u$. Thus the lengths of the curves on opposite
	sides of the quadralateral are equal
\end{ques}

\begin{ques}{2-5 8}
	We can reparametrize $\R^2$ :
	$$f(u,v) = \int \frac 1 {\sqrt E} \ du,\ g(u,v) = \int \frac 1 { \sqrt G}\
	dv$$
	(we can choose any integration constant for the
	indefinate integral)\\
	now we have the parametrization of $S$
	$$y(u,v) = x(f(u,v), g(u,v))$$
	Since $\frac d {dv} E = 0$ and $\frac d {du} G = 0$ we have $g_u = 0, f_v =
	0$ and thus by the chain rule
	$$y_u = x_u f_u + x_v g_u = x_u \frac 1 {\sqrt E}$$
	$$y_v = x_u f_v + x_v g_v = x_v \frac 1 {\sqrt G}$$
	Thus with our new parametrization
	$$E_y = \langle y_u, y_u \rangle = \frac 1 E \langle x_u, x_u \rangle = 1$$
	$$G_y = \langle y_v, y_v \rangle = \frac 1 G \langle x_v, x_v \rangle = 1$$
	we have the identity $F_y = \frac {\c \t}{|E||G|}$ and thus $F_y = \c \t$
\end{ques}

\begin{ques}{2-5 10}
	$$x_\rho = (\c \t, \s \t, 0)$$
	$$x_\t = (-\rho \s \t, \rho \c \t, 0)$$
	and thus we get
	$$E = \c^2 \t + \s^2 \t = 1$$
	$$G = \rho^2(\c^2 \t + \s^2 \t) = \rho^2$$
	$$F = \rho \c \t \s \t - \rho \c \t \s \t = 0$$
\end{ques}
\end{document}
