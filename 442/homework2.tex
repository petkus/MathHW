\documentclass[12pt]{article}
\usepackage{amsmath, amssymb, amsthm, epsfig}
\setlength\parindent{0pt}

\newenvironment{definition}{\vspace{2 ex}{\noindent{\bf Definition}}}
        {\vspace{2 ex}}

\newenvironment{ques}[1]{\textbf{#1}\vspace{1 mm}\\ }{\bigskip}

\renewcommand{\theenumi}{\alph{enumi}}

\theoremstyle{definition}

\newenvironment{Proof}{\noindent {\sc Proof.}}{$\Box$ \vspace{2 ex}}
\newtheorem{Wp}{Writing Problem}
\newtheorem{Ep}{Extra Credit Problem}

\oddsidemargin-1mm
\evensidemargin-0mm
\textwidth6.5in
\topmargin-15mm
\textheight8.75in
\footskip27pt


\renewcommand{\l}{\left }
\renewcommand{\r}{\right }

\newcommand{\R}{\mathbb R}
\newcommand{\Q}{\mathbb Q}
\newcommand{\Z}{\mathbb Z}
\newcommand{\C}{\mathbb C}
\newcommand{\N}{\mathbb N}
\renewcommand{\det}{\text{det}}

\newcommand{\s}{\sin}
\renewcommand{\c}{\cos}

\renewcommand{\t}{\theta}
\renewcommand{\a}{\alpha}

\newcommand{\norm}[1]{\left\lVert#1\right\rVert}

\newcommand{\T}{\mathcal{T}}

\pagestyle{empty}
\begin{document}

\noindent \textit{\textbf{Math 442, Winter 2018}} \hspace{1.3cm}
\textit{\textbf{HOMEWORK $\#$2}} \hspace{1.3cm} \textit{\textbf{Peter
Gylys-Colwell}} 

\vspace{1cm}

\begin{ques}{1-5, 15}
	We get $|\tau|$ from the identity $b' = \tau n$ 
	$$|b'(s)| = |\tau(s)|$$
	We get $\kappa$ as follows:
	$$t = n \wedge b = \frac{b'}{\tau} \wedge b$$
	We know that $\kappa$ is invariant of orientation, so $\kappa = |t'| =
	|(-t)'|$. We have
	$$\kappa = |(\pm t)'| = \l|\frac d {dt}\l(\frac{b'}{|\tau|} \wedge b \r) \r|$$
\end{ques}

\begin{ques}{1-5 16}
	Using the Frenet Equations and the fact $t \cdot b = 0$ we get $\kappa^2 + \tau^2$
	$$n' \cdot n' = (\kappa t + \tau b) \cdot (\kappa t + \tau b) = \kappa^2 +
	\tau^2$$
	We have
	$$n \wedge n' = n \wedge (-\tau b - \kappa t) = -\tau t + \kappa b$$
	We also have 
	$$-n'' = \kappa' t + \kappa t' + \tau' b + \tau b = (\kappa^2 + \tau^2)n +
	\kappa't + \tau'b$$
	Thus
	$$(n \wedge n')\cdot n'' = \tau \kappa' - \kappa \tau'$$
	Thus we have the function $f(s)$
	$$f(s) = \frac{(n \wedge n')\cdot n''}{\kappa^2 + \tau^2} = \frac{\tau \kappa' -
	\kappa \tau'}{\kappa^2 + \tau^2} = \frac{(\tau\kappa' -
	\kappa\tau')/\tau^2}{\frac{\kappa^2}{\tau^2} + 1} =
	\frac{\l(\frac{\kappa}{\tau}\r)'}{\frac{\kappa^2}{\tau^2} + 1}$$
	Thus integrating this function (determined entirely from $n$)
	$$\int f(s)\ ds = \arctan \frac \kappa \tau$$
	Thus taking $\tan$ on both sides gives us $\kappa / \tau$. This together
	with knowing $\kappa > 0$ and $\kappa^2 + \tau^2$ lets us determine
	$\kappa, \tau$ since it is a system of two equations and two unknown variables.
\end{ques}

\begin{ques}{1-5 17}
	(a) \ ($\Rightarrow$) If $\a$ is a helix, let $v$ be the direction vector
	such that $t \cdot v = c$ is constant. Differentiating yields
	$$\kappa n \cdot v = 0$$
	So either $\kappa = 0$ (which yields $\kappa / \tau$ is constant) or $n$ is
	perpendicular to $v$.  Thus for constants $c_1, c_2$
	$$v = c_1t + c_2 b$$
	Differentiating:
	$$0 = v' = c_1\kappa n + c_2 \tau n$$
	thus
	$$-\frac{c_2}{c_1} = \frac \kappa \tau$$
	$(\Leftarrow )$ Choosing $c_1, c_2$ so they satisfy the same equality
	$$-\frac{c_2}{c_1} = \frac \kappa \tau$$
	We let $v = c_1t + c_2 b$ and we have
	$$\kappa n \cdot v = \kappa n \cdot (c_1t + c_2 b) = 0$$
	So $t \cdot v$ is constant (since its derivative is $0$). $v$ is a constant
	vector since 
	$$v' =  c_1\kappa n + c_2 \tau n = 0$$
	thus $\a$ is a helix\\
	\\
	(b) \ $(\Rightarrow$) Using the same $v$ as in (a), we already established
	that $n \cdot v = 0$ and thus all normal lines are parallel to the plane
	generated by $v$.\\
	($\Leftarrow$) If $n \cdot v = 0$, then $\frac{d}{ds} t \cdot v = 0$ so $t
	\cdot v$ is constant.\\
	\\
	(c) \ $(\Rightarrow$) from (a) and (b) we know $t,n$ make a constant angle
	with $v$. Since $b$ is always perpendicular to $t, n$ this means that $b$
	must also make a constant angle with $v$.\\
	$(\Leftarrow$) if $b \cdot v$ is constant, then by differentiating we get
	$n \cdot v = 0$ and thus from (b) we know $\a$ is a helix
	\\
	\\
	(d) \ calculating $t$:
	$$t(s) = \l( \frac a c \sin \theta(s), \frac a c \cos \theta(s), \frac b c\r)$$
	Let $v = e_3$ we have
	$$t \cdot e_3 = \frac b c$$
	Thus $\a$ is a helix. As established above we have that $v = c_1t + c_2b$
	where $\frac{\kappa}{\tau} = \frac{-c_1}{c_2}$, we have $c_1 = v \cdot t =
	\frac b c$ and 
	$$c_2 = |v - c_1t| = |\frac a c (\sin \theta(s), \cos \theta(s), 0)| = \frac a c$$
	So $\frac \kappa \tau = \frac a b$
\end{ques}

\begin{ques}{1-7 1}
	No such curve exists since it violates the isoperimetric inequality:
	$$l^2 - 4\pi A = 36 - 4(3)\pi < 0$$
\end{ques}

\begin{ques}{1-7 2}
	Consider the circle with AB as a chord, and semicircle $s_1$ from $A$ to $B$
	with arclength $l$. There is the other semicircle on the other side of
	$s_1$ from $A$ to $B$ we will label $s_2$. For any curve $C$ from $A$ to
	$B$, we have the closed curve $C \cup s_2$. The area of this curve is
	precisely the area bounded by $C$ and $\overline{AB}$ plus the area bounded
	by $s_2$ and $\overline{AB}$: (here $A_{N,M}$ will denote the area of the
	region bounded by the curves $N,M$)
	$$A_{C,s_2} = A_{s_2,AB} + A_{C,AB}$$
	Similarly 
	$$A_{s_1,s_2} = A_{s_1,AB} + A_{s_2,AB}$$
	Notice that the arclengths of $s_1 \cup s_2$, $C \cup s_2$ are the same, So
	we can use the isoperimetric inequality for the circle $s_1 \cup s_2$ to
	conclude
	$$A_{C,s_2} \leq A_{s_1,s_2}$$
	canceling $A_{s_2,AB}$ on both sides:
	$$A_{C,AB} \leq A_{s_1,AB}$$
	So area is maximized by the semicircle $s_1$.

\end{ques}
\end{document}
