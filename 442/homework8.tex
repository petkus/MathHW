\documentclass[12pt]{article}
\usepackage{amsmath, amssymb, amsthm, epsfig}
\setlength\parindent{0pt}

\newenvironment{definition}{\vspace{2 ex}{\noindent{\bf Definition}}}
        {\vspace{2 ex}}

\newenvironment{ques}[1]{\textbf{#1}\vspace{1 mm}\\ }{\bigskip}

\renewcommand{\theenumi}{\alph{enumi}}

\theoremstyle{definition}

\newenvironment{Proof}{\noindent {\sc Proof.}}{$\Box$ \vspace{2 ex}}
\newtheorem{Wp}{Writing Problem}
\newtheorem{Ep}{Extra Credit Problem}

\oddsidemargin-1mm
\evensidemargin-0mm
\textwidth6.5in
\topmargin-15mm
\textheight8.75in
\footskip27pt


\renewcommand{\l}{\left }
\renewcommand{\r}{\right }

\newcommand{\R}{\mathbb R}
\newcommand{\Q}{\mathbb Q}
\newcommand{\Z}{\mathbb Z}
\newcommand{\C}{\mathbb C}
\newcommand{\N}{\mathbb N}

\renewcommand{\k}{\kappa}
\renewcommand{\t}{\tau}

\renewcommand{\det}{\text{det}}

\newcommand{\s}{\sin}
\renewcommand{\c}{\cos}
\newcommand{\id}{\text{id}}

\renewcommand{\t}{\theta}
\renewcommand{\a}{\alpha}
\renewcommand{\b}{\beta}

\newcommand{\norm}[1]{\left\lVert#1\right\rVert}

\newcommand{\T}{\mathcal{T}}

\pagestyle{empty}
\begin{document}

\noindent \textit{\textbf{Math 442, Winter 2018}} \hspace{1.3cm}
\textit{\textbf{HOMEWORK $\#$8}} \hspace{1.3cm} \textit{\textbf{Peter
Gylys-Colwell}} 

\vspace{1cm}

\begin{ques}{2-6, 1}
	Assume for contradiction there exists a differentiable field of unit normal vectors
	$$N:S \to \R^3$$
	Letting $x(u,v), y(s,t)$ be the parametrizations of
	$V_1,V_2$, for any $p \in V_1$ we have that with appropriate reordering of $u,v$
	$$N(p) = \frac{x_u \wedge x_v}{|x_u \wedge x_v|}$$
	on all of $V_1$ and similarly
	$$N(p) = \frac{y_u \wedge y_v}{|y_u \wedge y_v|}$$
	on all of $V_2$. However since the change of coordinate jacobian from
	$x$ to $y$ is different in sign on $W_1$ and $W_2$ and it is the case
	$$x_u \wedge x_v = (y_u \wedge y_v) \frac{\partial x}{\partial y}$$
	where $\frac{\partial x}{\partial y}$ is the jacobian of the coordinate
	change, we get the contradiction
	$$N(p) = -N(p)$$
	for either $p \in W_1$ or $p \in W_2$
\end{ques}

\begin{ques}{2-6 2}
	Letting $Y = \{Y_i\}$ be a family of coordinate neighborhoods which establish
	$S_2$ to be orientable with corresponding parametrizations $y_i$. For each
	$p \in S_1$ there is a neighborhood $V_p \subset S_1$ around $p$ such
	that $\varphi$ is a diffeomorphism on $V_p$. Let $\varphi_p$ be this
	diffeomorphism. We have that $\varphi_p(p)$ is contained in some $Y_i$
	which we will call $Y_p$ with corresponding parametrization $y_p$. We have
	the following family of coordinate functions covering $S_2$
	$$\{f_p = \varphi_p^{-1} \circ y_p: y_p^{-1}(Y_p \cap \varphi_p(V_p)) \to S_2\}$$
	it is clear this is a covering since each $p \in S_2$ is in the image of
	$\varphi_p^{-1} \circ y_p$. This covering establishes $S_2$ to be
	orientable since the change of coordinate calculation from $f_p$ to $f_q$
	$$f_q^{-1} \circ f_p = y^{-1}_q \circ \varphi \circ \varphi^{-1} \circ y_p
	= y_q^{-1} \circ y_p$$
	is the same as the change of coordinates from $y_p$ to $y_q$ which has
	positive Jacobian
\end{ques}

\begin{ques}{2-6 4}
	Let $N_\a$, $N_\b$ be the associated normal vector fields of
	two coordinate neighborhoods $\{U_\a\}, \{V_\b\}$ which satisfy the
	conditions of Def 1. Notice that
	$$F(p) = |N_\a(p) - N_\b(p)| = 
	\begin{cases}
	0 & N_\a(p) = N_\b(p)\\
	2 & N_\a(p) = -N_\b(p)
	\end{cases}$$
	Is a continuous function $F:S \to \R$. Thus since $S$ is connected, the
	image of $F$ is connected so is either entirely $0$ or $2$. Thus $N_\a =
	N_\b$ or $N_\a = -N_\b$ on all of $S$. Thus there is at most two possible
	orientations since there are at most two possible normal vectors fields (we
	know that $U_\a$ and $V_\b$ define the same orientation if and only if they
	define the same normal vector fields)
\end{ques}

\begin{ques}{2-6 5}
	(a)\ Notice that $\phi$ and $\phi^{-1}$ are local diffeomorphisms. Thus
	from problem 2 if $S_1$ is orientable then $S_2$ is orientable and
	similarly if $S_2$ is orientable then $S_1$ is orientable\\
	(b)\ Given a family of coordinate neighborhoods $\{U_\a\}$ which establish
	an orientation on $S_1$ we have that $\{\varphi (U_\a)\}$ is a family of
	coordinate neighborhood which establishes an orientation on $S_2$. The
	reason we know this family establishes an orientation is the same reasoning
	as in problem $2$, we have that the change of basis from $\varphi \circ
	x_1$ to $\varphi \circ x_2$ is the same as the change of basis from $x_1$
	to $x_2$ which has positive Jacobian\\
	For the Sphere we will use the stereographic parametrizations
	$$\varphi_1:U_1 \to S - \{(0,0,1)\}, \varphi_2:U_2 \to S - \{(0,0,-1)\}$$
	we have that the normal established by this
	parametrization has the evaluations 
	$$(\varphi_{1,u} \wedge \varphi_{1,v}) (0,0,0) = N(0,0,1) = (0,0,1)$$
	$$(\varphi_{2,u} \wedge \varphi_{2,v}) (0,0,0) =N(0,0,-1) = (0,0,-1)$$
	that the differential $J$ of the Antipodal map is $-I$ (negative the identity)
	we have that the normal at $(0,0,1)$ using the parametrizations
	$A \circ \varphi_1, A \circ \varphi_2$. We have that $(0,0,1) =
	A(\varphi_2(0,0,0))$ and thus our new normal is
	$$N'(0,0,1) = ((A \circ \varphi_2)_u \wedge (A \circ \varphi_2)_v)(0,0,0)$$
	by chain rule
	$$= ((-I  \varphi_{2,u}) \wedge (-I \varphi_{2,v}))(0,0,0)$$
	by bilinearity
	$$ = (-1)^2(\varphi_{2,u} \wedge \varphi_{2,v})(0,0,0) = (0,0,-1) \neq N(0,0,1)$$
	Thus we get a different normal and so a different orientation
\end{ques}

\begin{ques}{2-6 6}
	The notion of orientation is that $\a(t), \b(s)$ have the same
	orientation if the tangent vectors are the same:\\
	When $\a(a) = \b(b) = p$, $\a'(a)= T_\a(p) = T_\b(p) = \b'(b)$\\
	Each parametrization $\a$ induces a continuous tangent vector field $T_\a:C \to \R^3$
	where $T_\a(p) = \a'(s)$ (where $\a(s) = p$)\\
	Notice that we have a continuous function
	$$F(p) = |T_\a(p) - T_\b(p)|:C \to \R$$
	where
	$$F(p) = 
	\begin{cases}
	0 & T_\a(p) = T_\b(p)\\
	2 & T_\a(p) = -T_\b(p)
	\end{cases}$$
	Since $C$ is connected $F(C)$ is connected so $T_\a = T_\b$ or $T_\a =
	-T_\b$ on all of $C$. \\
	Thus there is at most two possible
	orientations since there are at most two possible tangent vector fields
	defined on $C$ (we know that $\a$ and $\b$ define the same orientation
	if and only if they define the same tangent vector fields)

\end{ques}
\end{document}
