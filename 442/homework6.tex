\documentclass[12pt]{article}
\usepackage{amsmath, amssymb, amsthm, epsfig}
\setlength\parindent{0pt}

\newenvironment{definition}{\vspace{2 ex}{\noindent{\bf Definition}}}
        {\vspace{2 ex}}

\newenvironment{ques}[1]{\textbf{#1}\vspace{1 mm}\\ }{\bigskip}

\renewcommand{\theenumi}{\alph{enumi}}

\theoremstyle{definition}

\newenvironment{Proof}{\noindent {\sc Proof.}}{$\Box$ \vspace{2 ex}}
\newtheorem{Wp}{Writing Problem}
\newtheorem{Ep}{Extra Credit Problem}

\oddsidemargin-1mm
\evensidemargin-0mm
\textwidth6.5in
\topmargin-15mm
\textheight8.75in
\footskip27pt


\renewcommand{\l}{\left }
\renewcommand{\r}{\right }

\newcommand{\R}{\mathbb R}
\newcommand{\Q}{\mathbb Q}
\newcommand{\Z}{\mathbb Z}
\newcommand{\C}{\mathbb C}
\newcommand{\N}{\mathbb N}

\renewcommand{\k}{\kappa}
\renewcommand{\t}{\tau}

\renewcommand{\det}{\text{det}}

\newcommand{\s}{\sin}
\renewcommand{\c}{\cos}
\newcommand{\id}{\text{id}}

\renewcommand{\t}{\theta}
\renewcommand{\a}{\alpha}
\renewcommand{\b}{\beta}

\newcommand{\norm}[1]{\left\lVert#1\right\rVert}

\newcommand{\T}{\mathcal{T}}

\pagestyle{empty}
\begin{document}

\noindent \textit{\textbf{Math 442, Winter 2018}} \hspace{1.3cm}
\textit{\textbf{HOMEWORK $\#$6}} \hspace{1.3cm} \textit{\textbf{Peter
Gylys-Colwell}} 

\vspace{1cm}

\begin{ques}{2-4, 10}
	Calculating the tangent vectors with the parametrization $x$
	$$x_s = ((1 - r\k (s) \c v)t(s) + (-r\t(s)\c v)b(s) + (r\t(s)\s
	v)n(s)) $$
	$$x_v = ((-r \s v)n(s) + (r \c v)b(s))$$
	After numberous computations we arrive at
	$$N(s,v) \cdot x_s = -(n(s)\c v + b(s) \s v) \cdot((1 - r\k (s) \c v)t(s) +
	(-r\t(s)\c v)b(s) + (r\t(s)\s v)n(s)) = 0$$
	$$N(s,v) \cdot x_v = -(n(s)\c v + b(s) \s v) \cdot((-r \s v)n(s) + (r \c
	v)b(s)) = 0$$
	Thus $N(s,v)$ is normal to $S$ since it is normal to both $x_s, x_v$
\end{ques}

\begin{ques}{2-4 18}
	By problem 17 we know that if $S$ and $P$ intersect transversally, in other
	words $P$ is not the tangent space of $S$ at $p$ then $P$ must intersect
	$P$ by a regular curve. Thus since $S$ intersects $P$ at only a point, it
	must be the case $P = T_p(S)$\\
	To prove 17, we know that $S_1$ and $S_2$ are locally the graphs of some
	smooth functions $f(x,y,z) = 0, g(x,y,z) = 0$ in a neighborhood of $p$. $0$
	is a regular value of $f,g$. $S_1 \cap S_2$ is precisely the inverse image
	of $0$ of the function $F(x,y,z) = (f(x,y,z), g(x,y,,z))$. Since the
	normals of $S_1, S_2$ are not linearly dependent $(f_x,f_y,f_z),
	(g_x,g_y,g_z)$ are linearly independent. Thus $(0,0)$ is a regular value of
	$F$ so $S_1 \cap S_2$ is a regular curve
\end{ques}

\begin{ques}{2-4 21}
	For any $p,q \in S$, since $S$ is connected and regular it is path
	connected. Thus there exists a path $\a:[0,1] \to S$ from $p$ to $q$. We
	have that
	$$\frac d {ds} f(\a(s)) = df_{\a(s)} (\a'(s)) = 0$$
	Thus $f(\a(s))$ is constant as a function from $\R \to \R^3$ and thus
	$$f(p) = f(\a(0)) = f(\a(1)) = f(q)$$
	Thus $f$ is constant
\end{ques}

\begin{ques}{2-4 24}
	For any curve $\a:(a,b) \to S_1$ at $p$ ($\a(0) = p$) by the usual chain rule
	$$\frac d {ds} ( \psi \circ \varphi \circ \a)$$
	$$= d\psi_{\varphi(p)}\ d\varphi_p \a'(0)$$
	Thus by definition of the matrix multiplied by $\a'(0)$ to get the tangent vector:
	$$d(\psi \circ \varphi) = d\psi_{\varphi(p)}\ d\varphi_p$$
\end{ques}
\end{document}
