\documentclass[12pt]{article}
\usepackage{amsmath, amssymb, amsthm, epsfig}
\setlength\parindent{0pt}

\newenvironment{definition}{\vspace{2 ex}{\noindent{\bf Definition}}}
        {\vspace{2 ex}}

\newenvironment{ques}[1]{\textbf{#1}\vspace{1 mm}\\ }{\bigskip}

\renewcommand{\theenumi}{\alph{enumi}}

\theoremstyle{definition}

\newenvironment{Proof}{\noindent {\sc Proof.}}{$\Box$ \vspace{2 ex}}
\newtheorem{Wp}{Writing Problem}
\newtheorem{Ep}{Extra Credit Problem}

\oddsidemargin-1mm
\evensidemargin-0mm
\textwidth6.5in
\topmargin-15mm
\textheight8.75in
\footskip27pt


\renewcommand{\l}{\left }
\renewcommand{\r}{\right }

\newcommand{\R}{\mathbb R}
\newcommand{\Q}{\mathbb Q}
\newcommand{\Z}{\mathbb Z}
\newcommand{\C}{\mathbb C}
\newcommand{\N}{\mathbb N}
\renewcommand{\det}{\text{det}}

\newcommand{\s}{\sin}
\renewcommand{\c}{\cos}
\newcommand{\id}{\text{id}}

\renewcommand{\t}{\theta}
\renewcommand{\a}{\alpha}

\newcommand{\norm}[1]{\left\lVert#1\right\rVert}

\newcommand{\T}{\mathcal{T}}

\pagestyle{empty}
\begin{document}

\noindent \textit{\textbf{Math 442, Winter 2018}} \hspace{1.3cm}
\textit{\textbf{HOMEWORK $\#$5}} \hspace{1.3cm} \textit{\textbf{Peter
Gylys-Colwell}} 

\vspace{1cm}

\begin{ques}{2-4, 1}
	For any curve
	$$\a:(-\epsilon, \epsilon) \to S$$
	with $\a(0) = p = (x_0,y_0,z_0)$ we have that
	$$f(\a(s)) = 0$$
	Differentiating both sides and applying the chain rule
	$$df(\a(0))\cdot \a'(0)=0$$
	Thus we have that 
	$$df(\a(0) = df_p$$
	is normal to every tangent vector and thus is the normal vector of the
	tangent plane. Thus tangent plane has equation
	$$df_p \cdot (x - p) = 0$$
\end{ques}

\begin{ques}{2-4 3}
	We can define the function $g(x,y,z) = z - f(x,y)$\\
	We have that $0$ is a regular value of $g$ since $g$ has no critical points
	$$dg = (-f_x(x,y), -f_y(x,y), 1) \neq 0$$
	Notice that 
	$$S = \{(x,y,f(x,y)): (x,y) \in \R^2\} = \{(x,y,z) \in \R^3: g(x,y,z) =
	f(x,y) - z = 0\}$$
	Thus we can apply the result of 2-4 1 to conclude the tangent surface at $p
	= (x_0,y_0,z_0)$ is given by
	$$dg_p \cdot (x - p) = 0$$
	which is the same as
	$$(-f_{x}(x_0,y_0), -f_{y}(x_0,y_0), 1) \cdot (x - x_0,y - y_0,z - z_0) = 0$$
	$$(f_{x}(x_0,y_0), f_{y}(x_0,y_0), 1) \cdot (x - x_0,y - y_0, f(x_0,y_0)) = z$$
\end{ques}

\begin{ques}{2-4 4}
	Letting $g(x,y) = xf(y/x)$ from the previous problem we know the tangent
	plane for any $p = (x_0, y_0, z_0)$
	$$(g_{x}(x_0,y_0), g_{y}(x_0,y_0), 1) \cdot (x - x_0,y - y_0, f(x_0,y_0)) = z$$
	using the chain rule yields
	$$g_x(x_0,y_0) = f(y_0/x_0) - \frac {y_0} {x_0} f'(y_0/x_0)$$
	$$g_y(x_0,y_0) = f'(y_0/x_0)$$
	$$(f(y_0/x_0) - \frac {y_0} {x_0} f'(y_0/x_0), f'(y_0/x_0), 1) \cdot (x - x_0,y
	- y_0, f(x_0,y_0)) = z$$
	Plugging in $x=0,y = 0$ yields
	$$(f(y_0/x_0) - \frac {y_0} {x_0} f'(y_0/x_0), f'(y_0/x_0), 1) \cdot (- x_0,
	- y_0, f(x_0,y_0)) = z$$
	$$= f'(y_0/x_0)(y_0-y_0) + f(y_0/x_0)(x_0 - x_0) = 0$$
	Thus the point $(0,0,0)$ is on the tangent plane
\end{ques}

\begin{ques}{2-4 8}
	By definition we know linear transormations are differentiable maps from
	$\R^3$ to $\R^3$ and thus $L|_S$ is a smooth map\\
	We know that every linear transormation can be represented as a matrix
	$$
	L(x) = 
	\begin{bmatrix}
	a_1 & b_1 & c_1\\
	a_2 & b_2 & c_2\\
	a_3 & b_3 & c_3
	\end{bmatrix}
	\begin{bmatrix}
	x_1 \\
	x_2 \\
	x_3
	\end{bmatrix}
	$$
	Notice that differentiating at any $p$ yields
	$$
	dL = 
	\begin{bmatrix}
	a_1 & b_1 & c_1\\
	a_2 & b_2 & c_2\\
	a_3 & b_3 & c_3
	\end{bmatrix}
	$$
	And thus 
	$$L(W) = dL_p(W)$$
	For any vector $W$
\end{ques}
\end{document}
