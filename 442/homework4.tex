\documentclass[12pt]{article}
\usepackage{amsmath, amssymb, amsthm, epsfig}
\setlength\parindent{0pt}

\newenvironment{definition}{\vspace{2 ex}{\noindent{\bf Definition}}}
        {\vspace{2 ex}}

\newenvironment{ques}[1]{\textbf{#1}\vspace{1 mm}\\ }{\bigskip}

\renewcommand{\theenumi}{\alph{enumi}}

\theoremstyle{definition}

\newenvironment{Proof}{\noindent {\sc Proof.}}{$\Box$ \vspace{2 ex}}
\newtheorem{Wp}{Writing Problem}
\newtheorem{Ep}{Extra Credit Problem}

\oddsidemargin-1mm
\evensidemargin-0mm
\textwidth6.5in
\topmargin-15mm
\textheight8.75in
\footskip27pt


\renewcommand{\l}{\left }
\renewcommand{\r}{\right }

\newcommand{\R}{\mathbb R}
\newcommand{\Q}{\mathbb Q}
\newcommand{\Z}{\mathbb Z}
\newcommand{\C}{\mathbb C}
\newcommand{\N}{\mathbb N}
\renewcommand{\det}{\text{det}}

\newcommand{\s}{\sin}
\renewcommand{\c}{\cos}
\newcommand{\id}{\text{id}}

\renewcommand{\t}{\theta}
\renewcommand{\a}{\alpha}

\newcommand{\norm}[1]{\left\lVert#1\right\rVert}

\newcommand{\T}{\mathcal{T}}

\pagestyle{empty}
\begin{document}

\noindent \textit{\textbf{Math 442, Winter 2018}} \hspace{1.3cm}
\textit{\textbf{HOMEWORK $\#$4}} \hspace{1.3cm} \textit{\textbf{Peter
Gylys-Colwell}} 

\vspace{1cm}

\begin{ques}{2-3, 1}
	$A$ is continuous since its components are continuous functions. Given any
	point $p \in S^2$ we can choose parametarizations $x_1: U_1 \subset \R^2
	\to S^2$, $x_2: U_2 \subset \R^2 \to S^2$ such that $p \in x_1(U_1)$, $A(p)
	\in x_2(U_2)$.\\
	From class we established there are the following parametarizations we can
	choose from:
	$$x_1,x_2 \in \{(\cos \theta \sin \varphi, \sin \t \sin \varphi, \cos
	\varphi), (\cos \theta \sin \varphi + \pi, \sin \t \sin \varphi + \pi, \cos
	\varphi + \pi),$$
	$$(\cos \theta + \pi \sin \varphi, \sin \t + \pi \sin \varphi, \cos
	\varphi) \dots\}$$
	We must show that 
	$$x_2^{-1} \circ A \circ x_1$$
	is differentiable at $p$. Notice that $A = A^{-1}$ and thus by showing $A$
	is differentiable we have shown its inverse to be differentiable. Hence
	concluding $A$ is a diffeomorphism.\\
	It is clear this composition is differentiable since each component
	$$x_2^{-1} \circ A \circ x_1(p) = (x_{2,1}^{-1}(-x_{1,1}(p_1)),
	x_{2,2}^{-1}(-x_{1,2}(p_2)), x_{2,3}^{-1}(-x_{1,3}(p_3)))$$
	Is a composition of differentiable functions and thus differentiable
	regardless of choice of $x_1, x_2$.
\end{ques}

\begin{ques}{2-3 3}
	We have the diffeomorphism $f: \R^2 \to P$ with $f(x,y) = (x,y,z^2)$. To
	establish $f$ is a diffeomorphism, notice that $f$ is also a
	parametarization. Thus the conditions of being a diffeomorphism rely on
	asking whether over any open set if
	$$f^{-1} \circ f \circ \text{id}$$
	is differntiable. This mapping is the identity mapping so clearly differntiable.
\end{ques}

\begin{ques}{2-3 4}
	The diffeomorphism $f:S^2 \to E$ is defined by 
	$$f(x,y,z) = \l(a x, b y, c z\r)$$
	We have that this is a diffeomorphism since notice that for any
	parametarization $x_1: U \subset \R^2 \to S^2$ around a point $p \in S^2$
	we have a parametarization $x_2 = f \circ x_1$ of $E$ around $f(p)$. Thus we have
	$$x_1^{-1} \circ f \circ x_2 = x_1^{-1} \circ f^{-1} \circ f \circ x_1 =
	\text{id}$$
	is differentiable
\end{ques}

\begin{ques}{2-3 6}
	For diffeomorphism $f: S_1 \to S_2$, suppose we have the parametarizations
	around a point $p \in S_1$ and $f(p) \in S_2$
	$$x_1: U \subset \R^2 \to S_1, x_2: V \subset \R^2 \to S_2$$
	$$y_1: U \subset \R^2 \to S_1, y_2: V \subset \R^2 \to S_2$$
	We need to show that if the parametarization with $x_1, x_2$ establishes
	$f$ to be differentiable then so does $y_1, y_2$\\
	We can use the change of Parametarization Theorem. Letting 
	$$W_1 = x_1(U) \cap y_1(U), W_2 = x_2(V) \cap y_2(V)$$
	We have the diffeomorphisms $h_1, h_2$
	$$h_1 = x_1^{-1} \circ y_1: y_1^{-1}(W_1) \to x_1^{-1}(W_1)$$
	$$h_2 = x_2^{-1} \circ y_2: y_2^{-1}(W_2) \to x_2^{-1}(W_2)$$
	Notice that $p \in W_1$ and $f(p) \in W_2$ and thus when we restrict to the
	open set $W_2$ such that 
	$$W_3 = x_1^{-1}(f^{-1}(W_2))\cap y^{-1}(f^{-1}(W_2)) \subset W_1$$
	$$y_2^{-1} \circ f \circ y_1|_{W_3} = h_2^{-1} \circ x_2^{-1} \circ f \circ x_1
	\circ h_1$$
	Thus we have that $y_1, y_2$ establish $f$ to be a diffeomorphism under
	this restriction since $y_2^{-1} \circ f \circ y_1|_{W_3}$ is a composition
	of the diffeomorphic functions $h_2^{-1}, x_2^{-1} \circ f \circ x_1, h_1$
\end{ques}

\begin{ques}{2-3 7}
	Checking the three conditions of an equivalence relation:\\
	Reflexivity:\\
	We have that the identity 
	$$\id : S \to S$$
	is a diffeomorphism and thus $S \sim S$\\
	Transitivity:\\
	If we have the diffeomorphisms
	$$f_1:S_1 \to S_2, f_2:S_2 \to S_3$$
	Then it is the case that 
	$$f_3 = f_2 \circ f_1 :S_1 \to S_3$$
	is a diffeomorphism. The reason for this is because for parametarizations
	$$x_1: U_1 \to S_1, x_2: U_2 \to S_2, x_3:U_3 \to S_3$$
	establishing that $f_1, f_2$ is diffeomorphic, we have that for a
	sufficientnly small domain around our point $p \in U_1$ in question
	$$x_3^{-1} \circ f_3 \circ x_1 = x_3^{-1} \circ f_2 \circ x_2 \circ
	x_2^{-1} \circ f_1 \circ x_1$$
	This is the composition of differentiable functions with differentiable
	inverses $x_2^{-1} \circ f_1 \circ x_1, x_3^{-1} \circ f_2 \circ x_2$ and
	therefore is itself a differentiable function with differentiable inverse.
	Thus $f_3$ is a diffeomorphism so $S_1 \sim S_3$\\
	Symmetry:\\
	If there exists the diffeomorphism
	$$f:S_1 \to S_2$$
	then it is the case that $f^{-1}$ is a diffeomorphism as well and thus
	$$f^{-1}:S_2 \to S_1$$
	establishes $S_2 \sim S_1$
\end{ques}
\end{document}
