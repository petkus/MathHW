\documentclass[12pt]{article}
\usepackage{amsmath, amssymb, amsthm, epsfig}
\setlength\parindent{0pt}

\newenvironment{definition}{\vspace{2 ex}{\noindent{\bf Definition}}}
        {\vspace{2 ex}}

\newenvironment{ques}[1]{\textbf{#1}\vspace{1 mm}\\ }{\bigskip}

\renewcommand{\theenumi}{\alph{enumi}}

\theoremstyle{definition}

\newenvironment{Proof}{\noindent {\sc Proof.}}{$\Box$ \vspace{2 ex}}
\newtheorem{Wp}{Writing Problem}
\newtheorem{Ep}{Extra Credit Problem}

\oddsidemargin-1mm
\evensidemargin-0mm
\textwidth6.5in
\topmargin-15mm
\textheight8.75in
\footskip27pt


\renewcommand{\l}{\left }
\renewcommand{\r}{\right }

\newcommand{\R}{\mathbb R}
\newcommand{\Q}{\mathbb Q}
\newcommand{\Z}{\mathbb Z}
\newcommand{\C}{\mathbb C}
\newcommand{\N}{\mathbb N}
\renewcommand{\det}{\text{det}}

\newcommand{\s}{\sin}
\renewcommand{\c}{\cos}

\renewcommand{\t}{\theta}
\renewcommand{\a}{\alpha}

\newcommand{\norm}[1]{\left\lVert#1\right\rVert}

\newcommand{\T}{\mathcal{T}}

\pagestyle{empty}
\begin{document}

\noindent \textit{\textbf{Math 442, Winter 2018}} \hspace{1.3cm}
\textit{\textbf{HOMEWORK $\#$3}} \hspace{1.3cm} \textit{\textbf{Peter
Gylys-Colwell}} 

\vspace{1cm}

\begin{ques}{2-2, 1}
	Checking all conditions:\\
	For any $p = (x_0,y_0,z_0)$, we have the homeomorphism $f: U \subset \R^2 \to C$
	where $C$ denotes the cylinder. We define $f$ as
	$$f(\theta,y) = (\cos(\theta), \sin(\theta), z_0 + y)$$
	And for $x_0 \neq -1, y_0 \neq 0$, $U = (-\pi, \pi) \times (-1,1)$,
	otherwise $U =  (0, 2\pi) \times (-1,1)$. This map is smooth since each
	component is differentiable. $f$ has a continuous inverse since $f$ is
	bijective and the inverse of the compenents are continuous since locally,
	the inverse is equal to $(\sin^{-1}(y), z - z_0)$ or $(\cos^{-1}(x), z -
	z_0)$. The jacobian is
	$$
	df_{(\theta,y)} = 
	\begin{bmatrix}
	-\sin(\theta) & 0\\
	\cos(\theta) & 0\\
	0 & 1
	\end{bmatrix}
	$$
	Which is linearly independent for any $(\theta, y) \in U$
\end{ques}

\begin{ques}{2-2 2}
	$C = \{(x,y,z) \in \R^3; z = 0, x^2 + y^2 \leq 1\}$ is not regular. The reason
	for this is because the point $(0,1,0) \in C$ cannot have a neighborhood
	$V$ such that there is a homeomorphism $f:U \subset \R^2 \to V \cap C$. The
	reason for this is because $V \cap C$ is homeomorphic to $C'=\{(x,y) \in \R^2;
	x^2 + y^2 \leq 1\}$ by the map $\pi_{12}(x,y,z) = (x,y)$ (which has the
	continuous inverse map $\pi_{12}^{-1}(x,y) = (x,y,0)$). We have that $C'$
	is a closed set in $\R^2$. Thus composing homeomorphisms, we would get
	$\pi_{12} \circ f : U \to C'$ is a homeomorphism from closed unit disc in
	$\R^2$ to the open unit disc in $\R^2$. We know that such a mapping is not
	possible since removing a point on the boundary of the closed unit disc
	still yields a null homotopic set, while removing any point on the unit
	open disc yields a set which is not null homotopic (we proved this in Math 441)\\
	\\
	For $D = \{(x,y,z) \in \R^3; z = 0, x^2 + y^2 < 1\}$ we have the homeomorphism 
	$$f: \{x,y) \in \R^2; x^2 + y^2 < 1\} \to D$$
	Where $f(x,y) = (x,y,0)$ which is clearly smooth with a continuous inverse
	$f^{-1}(x,y,0) = (x,y)$ and linearly independent jacobian:
	$$
	df_{(x,y)} = 
	\begin{bmatrix}
	1 & 0\\
	0& 1\\
	0 & 0
	\end{bmatrix}
	$$
\end{ques}

\begin{ques}{2-2 7}
	(a)\ Calculating the Jacobian:
	$$
	df_{(x,y,z)} = 2(x + y + z - 1)(1,1,1)
	$$
	Thus the critical points is the simplex $x + y + z = 1$. Evaluating $f$ at
	any point in the simplex yields the critical value $0$.\\
	\\
	(b)\ We have that $f(\R^3) = \R^+$. All $\R^3 \setminus \{0\}$ are regular
	points and thus from Prop 2, $f^{-1}(c)$ for $c > 0$ is regular. We have
	that $f^{-1}(0)$ is regular too since
	$$f^{-1}(0) = \{(x,y,z): x + y + z = 1\}$$
	is a plane (which we know is regular)
	\\
	\\
	(c)\ Calculating the Jacobian:
	$$
	df_{(x,y,z)} = (yz^2, xz^2, 2xyz)
	$$
	The critical points is the plane $z = 0$ union the line $y = 0, x = 0$.
	Evaluating $f$ at any critical point yields the critical value $0$.\\
	\\
	For $c \in f(\R^3) \setminus \{0\}$, $c$ is regular so $f^{-1}(c)$ yields a
	regular surface. For $c = 0$, $f^{-1}(0)$ is not regular since it is the
	union of three normal planes $x = 0, y = 0, z = 0$ which intersect at
	$(0,0,0)$. There is no well defined tangent vector at the point $(0,0,0)$

\end{ques}

\begin{ques}{2-2 8}
	We know that $dx_q$ is one to one if and only if $\frac{\partial
	x}{\partial u}$ and $\frac{\partial x}{\partial v}$ are linearly
	independent. From the definition of '$\wedge$' we know that $\frac{\partial
	x}{\partial u} \wedge \frac{\partial x}{\partial v} = 0$ if and only if the
	vectors are linearly dependent. Thus $dx_q$ is one-to-one iff $\frac{\partial
	x}{\partial u} \wedge \frac{\partial x}{\partial v} \neq 0$
\end{ques}

\begin{ques}{2-2 15}
	Letting $c$ be the speed of the points, for a given $t$ we have the positions
	$$p(t) = (0,0,ct), q(t) = (a,ct,0)$$
	Thus the line containing $p(t),q(t)$ parameterized by $s$ is described as
	$$p(t) + s(q(t) - p(t)) = (0,0,ct) + s(a,ct,-ct)$$
	So for $x,y,z$ on the line we have
	$$\frac x a = \frac{y}{ct} = \frac{z - ct}{-ct} = s$$
	$$ct x = a y = act - az$$
	Letting $t$ vary in $\R^+$, we will show this is the same set as $y(x-a) +
	zx = 0$. \\
	To show $\subseteq$, notice $y = \frac{ctx}{a}, z = ct - \frac{ctx}{a}$ so we have
	$$y(x - a) + zx = \frac{ctx}{a}(x -a) + \frac{act - ctx}{a}x = 0$$
	Thus any point $(x,y,z)$ satisfying one of the line equations satisfies the
	closed form equation.\\
	To show $\supseteq$, notice that for any fixed $x$, by choosing $s$ so that
	$as = x$, we get the line $y(as - a) + asz = 0$. Thus we can choose $t$ to
	get any $y,z$ combination satisfying the equation.\\
	This establishes a continuous mapping from the parameterization to the surface:
	$$f: \R^2 \to S$$
	$$f(t,s) = p(t) + s(q(t) - p(t)) = (sa, ct, ct - cts)$$
	Calculating the jacobian:
	$$d_f =
	\begin{bmatrix}
		0 & a\\
		sc & ct\\
		c-sc & -ct
	\end{bmatrix}
	$$
	Since $c, a \neq 0$ we have that the jacobian is surjective for all $s,c$,
	and thus $f$ is a smooth function with nonsingular jacobian. $f$ is one to
	one since $s$ uniquely determines the $x$ value of a point and then $t$
	will uniquely determine the $y,z$ values.  Thus $f$ establishes $S$ to be
	regular
\end{ques}
\end{document}
