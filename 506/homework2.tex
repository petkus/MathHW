\documentclass[12pt]{article}
\usepackage{amsmath, amssymb, amsthm, epsfig, tikz-cd, mathrsfs}
\setlength\parindent{0pt}

\newenvironment{definition}{\vspace{2 ex}{\noindent{\bf Definition}}}
        {\vspace{2 ex}}

\newenvironment{ques}[1]{\textbf{#1}\vspace{1 mm}\\ }{\bigskip}

\renewcommand{\theenumi}{\alph{enumi}}

\theoremstyle{definition}

\newenvironment{Proof}{\noindent {\sc Proof.}}{$\Box$ \vspace{2 ex}}
\newtheorem{Wp}{Writing Problem}
\newtheorem{Ep}{Extra Credit Problem}

\oddsidemargin-1mm
\evensidemargin-0mm
\textwidth6.5in
\topmargin-15mm
\textheight8.75in
\footskip27pt

\DeclareMathOperator\coker{coker}
\DeclareMathOperator\coim{coim}
\DeclareMathOperator\id{id}

\DeclareMathOperator\Hom{Hom}

\renewcommand{\l}{\left }
\renewcommand{\r}{\right }

\newcommand{\R}{\mathbb R}
\newcommand{\Q}{\mathbb Q}
\newcommand{\Z}{\mathbb Z}
\newcommand{\C}{\mathbb C}
\newcommand{\N}{\mathbb N}
\newcommand{\F}{\mathbb F}
\renewcommand{\H}{\mathbb H}
\newcommand{\ndiv}{\hspace{-4pt}\not|\hspace{2pt}}

\newcommand{\tensor}{\otimes}

\renewcommand{\t}{\theta}
\renewcommand{\a}{\alpha}
\renewcommand{\b}{\beta}
\newcommand{\vp}{\varphi}

\newcommand{\norm}[1]{\left\lVert#1\right\rVert}

\newcommand{\T}{\mathcal{T}}

\newcommand{\Tor}{\text{Tor}}
\newcommand{\Ann}{\text{Ann}}
\newcommand{\End}{\text{End}}
\newcommand{\Aut}{\text{Aut}}
\newcommand{\Gal}{\text{Gal}}
\newcommand{\orb}{\text{orb}}
\newcommand{\im}{\text{im}}
\newcommand{\Tr}{\text{Tr}}
\newcommand{\s}{\sigma}
\pagestyle{empty}
\begin{document}

\noindent \textit{\textbf{Math 506, SPRING 2018}} \hspace{1.3cm}
\textit{\textbf{HOMEWORK $\#$2}} \hspace{1.3cm} \textit{\textbf{Peter
Gylys-Colwell}} 

\vspace{1cm}

\begin{ques}{1}
	We have that
	$$k(\a) \cong k[x]/(f)$$
	We have
	$$K \tensor_k k(\a) \simeq K \tensor_k k[x]/(f) \simeq K[x]/(f)$$
	as $K$ algebras. The second equivalence comes from the general fact that
	for rings $R,S$ with $R \subset S$ and $R$-module $M$ then as $S$ algebras
	$$S \tensor_R M \simeq M_S$$
	Where $M_S$ is the module $M$ extended as an $S$ module (when such an
	extention is possible)
\end{ques}

\begin{ques}{2}
	For any $a \in A, x \in F$ ($a \neq 0$ is not a zero divisor and $x \neq 0$)\\
	We have the module homomorphism $\phi_a :A \to A$ given by 
	$$\phi_a(r) = ar$$
	is injective thus we have the exact sequence
	$$\begin{tikzcd}
	0 \arrow[r] & A \arrow[r,"\phi_a"] & A
	\end{tikzcd}$$
	which corresponds to the exact sequence
	$$\begin{tikzcd}
	0 \arrow[r] & A \otimes_A F \arrow[r,"\phi_a\otimes \id"] & A \otimes_A F
	\end{tikzcd}$$
	We have that 
	$$\phi_a\otimes \id (1 \otimes x) = a \otimes x = 1 \otimes ax$$
	Since $\phi_a\otimes \id$ is injective it must have a trivial kernel and
	thus $1 \otimes ax \neq 0 \Rightarrow ax \neq 0$
\end{ques}

\begin{ques}{3}
	($\Rightarrow$):\\
	For any ideal $I \subset A$ we have the natural embedding 
	$$0 \to I \to A$$
	which by definition of flatness induces the embedding
	$$0 \to I \otimes_A F \to  A \otimes_A F$$
	We have that $A \otimes_A F \cong F$ as $A$-modules by the isomorphism
	$$a \tensor x \to ax$$
	thus we have the desired exact sequence 
	$$0 \to I \otimes_A F \to  F$$
	($\Leftarrow$):\\
	Since the Tensor product is left adjoint it is right exact, so we must only
	show left exactness. For some exact sequence of $A$-modules
	$$0 \to X \to Y$$
	We have that $Y$ is isomorphic to a quotient of a free module
	$$Y \cong \l(\bigoplus_{i \in S} A_i\r) /Q$$
	From our hypothesis we know $F$ is $A$-flat (tensoring preserves injective
	maps into $A$). We will show $F$ is $Y$-flat (and thus conclude $F$ is
	flat) by showing that if $F$ is $M$-flat then $F$ is flat for every direct
	sum and quotient module of $M$. Since $Y$ is a direct sum and then quotient
	of $A$ this implies $F$ is $Y$-flat.\\
	\\
	Assuming $F$ is $M$-flat we have
	$$F \tensor \l(\bigoplus_{i \in S} M_i\r) = \bigoplus_{i \in S} F \tensor M_i$$
	and thus for an exact sequence
	$$0 \to N \to \bigoplus_{i \in S} M_i$$
	the injection can be factored into a direct sum of the mapping into each
	component $M_i$. From flatness
	of $F$ each component is an injective mapping when tensoring with $F$ and
	thus the mapping is still injective
	$$0 \to N \tensor F \to \bigoplus_{i \in S} F \tensor M_i$$
	\\
	Suppose now we have the quotient $Q$ where $F$ is $M$-flat
	$$0 \to I \to M \to Q \to 0$$
	Let $Q'$ be a submodule of $Q$ and $M'$ its inverse image in $M$. This
	yields the commutative diagram
	$$\begin{tikzcd}
	0 \arrow[r] & I \arrow[r] & M' \arrow[d]\arrow[r] & Q' \arrow[d]\arrow[r] & 0\\
	0 \arrow[r] & I \arrow[r] & M \arrow[r] & Q \arrow[r] & 0\\
	\end{tikzcd}$$
	Tensoring with $F$ yields
	$$\begin{tikzcd}
	& & 0 \arrow[d] & K \arrow[d] & \\
	 & F \tensor I \arrow[r]\arrow[d] & F \tensor M' \arrow[d]\arrow[r] & F \tensor Q'
	 \arrow[d]\arrow[r] & 0\\
	0 \arrow[r]& F \tensor I \arrow[d] \arrow[r] & F \tensor M \arrow[r] & F
	\tensor Q&\\
	 & 0 & &
	\end{tikzcd}$$
	where $K$ is the kernel of the mapping from $F \tensor Q' \to F \tensor Q$.
	The snake lemma yields the short exact sequence
	$$0 \to K \to 0$$
	and thus $K = 0$ so we have the exact sequence
	$$0 \to F \tensor Q' \to F \tensor Q$$
	establishing $F$ to be $Q$-flat. Thus we are done.

\end{ques}

\begin{ques}{4}
	From problem $2$ we get the implication $(\Rightarrow$).\\
	$(\Leftarrow)$ From problem $3$ we know it is sufficient to show that for
	ever ideal $\langle a \rangle \subset A$ there is an embedding 
	$$\langle a \rangle \otimes_A F \to F$$
	If we consider the kernel of the natural mapping 
	$$\langle a \rangle \otimes_A F \to F$$
	$$ar \tensor x \to arx$$
	we have $ar \tensor x \to arx = 0$ can be zero if and only if $ar = 0$ or
	$x = 0$. Thus the kernel is trivial and we have an embedding\\
	\\
	An example of a torsion free module that is not flat is the ideal
	$$I = \langle x, y\rangle \subset R = k[x,y]$$
	We have the exact sequence is not preserved
	$$0 \to I \to R$$
	$$0 \to I \tensor I \to I \tensor R$$
	Since 
	$$0 \neq x \tensor y - y \tensor x \to x \tensor y - y \tensor x$$
	and in $I \tensor R$ since $1 \in R$:
	$$x \tensor y - y \tensor x = xy \tensor 1 - xy \tensor 1 = 0$$
	% It is the case that $x \tensor y- y \tensor x$ is not $0$ in $I \tensor
	% I$ since if we examine h
\end{ques}

\begin{ques}{5}
	Given $F$ flat and the short exact sequence 
	$$0 \to N \to M \to F \to 0$$
	Given any $A$-module $L$ since the tensor is right exact we must only
	show exactness around $N \tensor L$.\\
	We have that $L$ can be written as a quotient of a flat $L'$
	with the exact sequence
	$$0 \to L'' \to L' \to L \to 0$$
	(This is since every module can be written as the quotient of a free module
	and all free modules are flat)\\
	Thus we have the commutative diagram
	$$\begin{tikzcd}
	& & & 0 \arrow[d] & \\
	& L'' \tensor N \arrow[r] \arrow[d] & L'' \tensor M \arrow[r]
	\arrow[d] & L'' \tensor F \arrow[r]\arrow[d] & 0\\
	0 \arrow[r] & L' \tensor N \arrow[d]\arrow[r] & L' \tensor M
	\arrow[r]\arrow[d] & L' \tensor F & \\
	& L \tensor N \arrow[d] \arrow[r] & L \tensor M \arrow[d] & \\
	& 0 & 0 &
	\end{tikzcd}$$
	The Snake Lemma yields the exact sequence 
	$$0 \to L \tensor N \to L \tensor M$$
	Since the cokernels of the left two morphisms are $L\tensor
	N$ and $L \tensor M$. Thus we are done
\end{ques}

\begin{ques}{6}
	Since The Tensor product is left adjoint, we know that it is right exact.
	Thus exactness will be implied by left exactness. In other words that any exact
	sequence of $A$ -modules of the form
	$$0 \to X \to Y$$
	is preserved\\
	We have the following commutative diagram 
	$$\begin{tikzcd}
	& & & 0 \arrow[d] & \\
	0 \arrow[r] & X \tensor N \arrow[r] \arrow[d] & X \tensor M \arrow[r]
	\arrow[d] & X \tensor F \arrow[r]\arrow[d] & 0\\
	0 \arrow[r] & Y \tensor N \arrow[r] & Y \tensor M \arrow[r] & Y \tensor F\\
	\end{tikzcd}$$
	From flatness of $F$ we have the third vertical map is an injection. If $N$
	is flat then the first vertical map is an injection and from the Four Lemma
	it must be the case that the map $X \tensor M \to Y \tensor M$ is an
	injection. Conversly if $M$ is flat then since the mapping $X \tensor N \to
	X \tensor M \to Y \tensor M$ is injective and is equal to the mapping $X
	\tensor N \to Y \tensor N$ composed with an injective mapping, it must be
	the case that the mapping $X \tensor N \to Y \tensor N$ is injective
\end{ques}

\end{document}
