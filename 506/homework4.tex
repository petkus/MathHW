\documentclass[12pt]{article}
\usepackage{amsmath, amssymb, amsthm, epsfig, tikz-cd, mathrsfs}
\setlength\parindent{0pt}

\newenvironment{definition}{\vspace{2 ex}{\noindent{\bf Definition}}}
        {\vspace{2 ex}}

\newenvironment{ques}[1]{\textbf{#1}\vspace{1 mm}\\ }{\bigskip}

\renewcommand{\theenumi}{\alph{enumi}}

\theoremstyle{definition}

\newenvironment{Proof}{\noindent {\sc Proof.}}{$\Box$ \vspace{2 ex}}
\newtheorem{Wp}{Writing Problem}
\newtheorem{Ep}{Extra Credit Problem}

\oddsidemargin-1mm
\evensidemargin-0mm
\textwidth6.5in
\topmargin-15mm
\textheight8.75in
\footskip27pt

\DeclareMathOperator\coker{coker}
\DeclareMathOperator\coim{coim}
\DeclareMathOperator\im{im}
\DeclareMathOperator\id{id}
\DeclareMathOperator\Ext{Ext}

\DeclareMathOperator\Hom{Hom}

\renewcommand{\l}{\left }
\renewcommand{\r}{\right }

\newcommand{\R}{\mathbb R}
\newcommand{\Q}{\mathbb Q}
\newcommand{\Z}{\mathbb Z}
\newcommand{\C}{\mathbb C}
\newcommand{\N}{\mathbb N}
\newcommand{\F}{\mathbb F}
\renewcommand{\H}{\mathbb H}
\newcommand{\ndiv}{\hspace{-4pt}\not|\hspace{2pt}}

\newcommand{\tensor}{\otimes}

\renewcommand{\t}{\theta}
\renewcommand{\a}{\alpha}
\renewcommand{\b}{\beta}
\newcommand{\vp}{\varphi}

\newcommand{\norm}[1]{\left\lVert#1\right\rVert}

\newcommand{\T}{\mathcal{T}}

\newcommand{\Tor}{\text{Tor}}
\newcommand{\Ann}{\text{Ann}}
\newcommand{\End}{\text{End}}
\newcommand{\Aut}{\text{Aut}}
\newcommand{\Gal}{\text{Gal}}
\newcommand{\orb}{\text{orb}}
\newcommand{\Tr}{\text{Tr}}
\newcommand{\s}{\sigma}
\pagestyle{empty}
\begin{document}

\noindent \textit{\textbf{Math 506, SPRING 2018}} \hspace{1.3cm}
\textit{\textbf{HOMEWORK $\#$4}} \hspace{1.3cm} \textit{\textbf{Peter
Gylys-Colwell}} 

\vspace{1cm}

\begin{ques}{1}
	($\Rightarrow$) If an $R$-module $M$ has no nontrivial essential extensions
	then we know we can embed $M$ into an injective module $I$. If we consider
	the set $S$ of submodules $N$ of $I$ where $N \cap M = 0$ we can apply
	Zorns Lemma to get a maximal module $D \in S$. To check Zorns Hypothesis:\\
	$0 \in S$ (so $S \neq \emptyset$). For any chain
	$$N_1 \subset N_2 \subset N_3 \subset \dots $$
	there is a largest element $N = \bigcup N_i$ which still has the property
	$N \cap M = \bigcup N_i \cap M = 0$\\
	We have that $I/D$ is an essential extension of $M$ as follows.\\
	For any $N \subset I/D$, if it were the case that $N \cap M = 0$ then the
	pre image of $N$ from the mapping $I \to I/D$ is a module in $S$ containing
	$D$ which contradicts maximality of $D$\\
	Therefore we can conclude $M = I/D$.  This along with the fact $M \cap D =
	0$ implies $I = M \oplus D$. We know that the summand of an injective
	module is injective and thus $M$ is injective.
	% then for any ideal $I \subset R$ and module homomorphism $\varphi: I \to
	% M$, if $\varphi$ cannot be extended to a homomorphism $R \to M$ then we can
	% define $x = \varphi (1)$ and we have the nontrivial extention of $M$, $N =
	% M \otimes Rx$. This would be a contradiction and thus every $\varphi:I \to M$
	% can be extended to $R$ so by Baer's Criterion $M$ is injective
	\\
	\\
	($\Leftarrow$) If an $R$-module $I$ is injective suppose for contradiction
	$I$ has a nontrivial essential extension $M$. We have from injectivity that
	the mapping
	$$0 \to I \to M$$
	splits. Thus there is some submodule $N \subset M$ where $M = I \oplus N$ so
	$I \cap N = 0$ which is a contradiction of $M$ essential
\end{ques}

\begin{ques}{2}
	We can use Zorns Lemma. If we consider the set $\mathcal C$ of essential
	extensions of $M$, for any chain $M_1 \subset M_2 \subset \dots $ of
	essential extensions there is a maximal element $\mathcal M = \bigcup M_i$.
	We have that $\mathcal M$ is an essential extension since if $L \subset
	\mathcal M$ then $L = \bigcup L \cap M_i$ and so one $L \cap M_i \neq 0$ so
	$0 \neq (L \cap M_i) \cap M \subseteq L \cap M$. Thus from Zorns Lemma
	there is a maximal element $E(M)$\\
	\\
	We have that if $E(M) \subset I$ has a nontrivial essential extension then
	$I$ is essential extension of $M$ which is a contradiction of maximality.
	This is the case since for any $L \subset I$, $L \cap E(M)$ is a submodule of
	$E(M)$ and thus $0 \neq (L \cap E(M)) \cap M \subseteq L \cap M$. Therefore $E(M)$
	has no nontrivial extensions and so from problem 1 is injective. \\
	$E(M)$ is minimal among the injective modules containing $M$ since for any
	injection $f:M \to I$, the natural inclusion $i:M \to E(M)$ and injectivity
	of $E(M)$ yields mapping $\phi:E(M) \to I$ such that $f = \phi \circ i = \phi|_M$.
	We have that $\ker \phi|_M = \ker \phi \cap M = 0$ since $f$ is an
	injection and thus from being an essential extension $\ker \phi = 0$ so
	$\phi$ is an injection
\end{ques}

\begin{ques}{3}
	We have the projective resolution
	$$\begin{tikzcd}
	0 \arrow[r] & \Z \arrow[r, "1 \to m"] & \Z \arrow[r, "1 \to 1"] & \Z/(m)
	\arrow[r] & 0
	\end{tikzcd}$$
	Applying the Hom functor yields
	$$\begin{tikzcd}
	0 \arrow[r] & \Hom(\Z/(m),\Z/(n)) \arrow[r, "1 \to 1"] & \Hom(\Z, \Z/(n))
	\arrow[r, "1 \to m"] & \Hom(\Z, \Z/(n)) \arrow[r] & 0
	\end{tikzcd}$$
	It is the case $\Hom(\Z, \Z/(n)) \simeq \Z/(n)$ and
	$\Hom(\Z/(m),\Z/(n)) \simeq \Z/(d)$ where $d = \gcd(m,n)$. Thus the Hom
	sequence is isomorphic to
	$$\begin{tikzcd}
	0 \arrow[r] & \Z/(d) \arrow[r, "1 \to n/d"] & \Z/(n) \arrow[r, "1 \to m"] & \Z/(n)
	\arrow[r] & 0
	\end{tikzcd}$$
	Thus we have
	$$\Ext_{\Z}^0(\Z/(m), \Z/(n)) \cong \Z/(d)$$
	$$\Ext_{\Z}^1(\Z/(m), \Z/(n)) \cong \Z/(d)$$
	$$\Ext_{\Z}^n(\Z/(m), \Z/(n)) \cong 0, \forall n \geq 2$$
\end{ques}

\begin{ques}{4}
	We have the same projective resolution
	$$\begin{tikzcd}
	0 \arrow[r] & \Z \arrow[r, "1 \to n"] & \Z \arrow[r, "1 \to 1"] & \Z/(n)
	\arrow[r] & 0
	\end{tikzcd}$$
	Applying the $\Z/(m) \tensor -$ functor yields
	$$\begin{tikzcd}
	0 \arrow[r] & \Z/(m)\tensor \Z \arrow[r, "1 \tensor n"] & \Z/(m) \tensor \Z
	\arrow[r, "1 \tensor n"] & \Z/(m) \tensor \Z/(n) \arrow[r] & 0
	\end{tikzcd}$$
	We have $\Z/(m) \tensor \Z \simeq \Z/(m)$ and $\Z/(m) \tensor \Z/(n) \simeq
	\Z/(d)$ where $d = \gcd(m,n)$\\
	Thus we have
	$$\Tor_{\Z}^0(\Z/(m), \Z/(n)) \cong \Z/(d)$$
	$$\Tor_{\Z}^1(\Z/(m), \Z/(n)) \cong \ker(1 \tensor n) \cong \Z/(d)$$
	$$\Tor_{\Z}^n(\Z/(m), \Z/(n)) \cong 0, \forall n \geq 2$$

\end{ques}

\begin{ques}{5}
	Let $R = \Z/(4)$. We have the $R$ module $\Z/(2)$. We have the
	projective resolution
	$$\begin{tikzcd}
	\cdots \arrow[r] & \Z/(4) \arrow[r, "1 \to 2"] & \Z/(4) \arrow[r, "1 \to 2"] & \Z/(4)
	\arrow[r, "1 \to 2"] & \Z/(4) \arrow[r, "1 \to 1"] & \Z/(2)\arrow[r] & 0
	\end{tikzcd}$$
	Taking the Hom Functor
	$$\begin{tikzcd}
	0 \arrow[r] & \Hom(\Z/(2),\Z/(2)) \arrow[r, "1 \to 1"] & \Hom(\Z(4), \Z/(2))
	\arrow[r, "1 \to 2"] & \Hom(\Z(4), \Z/(2)) \arrow[r, "1 \to 2"] & \cdots
	\end{tikzcd}$$
	We have that $\Hom(\Z/(2),\Z/(2)) \simeq \Hom(\Z/(4),\Z/(2)) \simeq
	\Z/(2)$. Thus we have the isomorphic sequence
	$$\begin{tikzcd}
	0 \arrow[r] & \Z/(2) \arrow[r, "1 \to 1"] & \Z/(2) \arrow[r, "0"] & \Z/(2)
	\arrow[r, "0"] & \Z/(2) \arrow[r, "0"] & \cdots
	\end{tikzcd}$$
	Thus we have that
	$$\Ext_{\Z/(4)\Z}^n(\Z/(2), \Z/(2)) \cong \Z/(2)$$
	for all $n \geq 0$
\end{ques}

\begin{ques}{6}
	Notice that the example in $5$ shows that the statment is not true
\end{ques}

\end{document}
