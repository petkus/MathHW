\documentclass[12pt]{article}
\usepackage{amsmath, amssymb, amsthm, epsfig, tikz-cd, mathrsfs}
\setlength\parindent{0pt}

\newenvironment{definition}{\vspace{2 ex}{\noindent{\bf Definition}}}
        {\vspace{2 ex}}

\newenvironment{ques}[1]{\textbf{#1}\vspace{1 mm}\\ }{\bigskip}

\renewcommand{\theenumi}{\alph{enumi}}

\theoremstyle{definition}

\newenvironment{Proof}{\noindent {\sc Proof.}}{$\Box$ \vspace{2 ex}}
\newtheorem{Wp}{Writing Problem}
\newtheorem{Ep}{Extra Credit Problem}

\oddsidemargin-1mm
\evensidemargin-0mm
\textwidth6.5in
\topmargin-15mm
\textheight8.75in
\footskip27pt

\DeclareMathOperator\coker{coker}
\DeclareMathOperator\coim{coim}
\DeclareMathOperator\im{im}
\DeclareMathOperator\id{id}
\DeclareMathOperator\Ext{Ext}

\DeclareMathOperator\Hom{Hom}

\renewcommand{\l}{\left }
\renewcommand{\r}{\right }

\newcommand{\R}{\mathbb R}
\newcommand{\Q}{\mathbb Q}
\newcommand{\Z}{\mathbb Z}
\newcommand{\C}{\mathbb C}
\newcommand{\N}{\mathbb N}
\newcommand{\F}{\mathbb F}
\renewcommand{\H}{\mathbb H}
\newcommand{\ndiv}{\hspace{-4pt}\not|\hspace{2pt}}

\newcommand{\tensor}{\otimes}

\renewcommand{\t}{\theta}
\renewcommand{\a}{\alpha}
\renewcommand{\b}{\beta}
\newcommand{\vp}{\varphi}

\newcommand{\norm}[1]{\left\lVert#1\right\rVert}

\newcommand{\T}{\mathcal{T}}

\newcommand{\Tor}{\text{Tor}}
\newcommand{\Ann}{\text{Ann}}
\newcommand{\End}{\text{End}}
\newcommand{\Aut}{\text{Aut}}
\newcommand{\Gal}{\text{Gal}}
\newcommand{\orb}{\text{orb}}
\newcommand{\Tr}{\text{Tr}}
\newcommand{\s}{\sigma}
\pagestyle{empty}
\begin{document}

\noindent \textit{\textbf{Math 506, SPRING 2018}} \hspace{1.3cm}
\textit{\textbf{HOMEWORK $\#$5}} \hspace{1.3cm} \textit{\textbf{Peter
Gylys-Colwell}} 

\vspace{1cm}

\begin{ques}{1}
	For any mapping $k[t] \to k[x,y]/(1-xy)$ the mapping is fully defined by
	the image of $t$:
	$$t \to s(x,y) \in k[x,y]/(1-xy)$$
	Notice that we can  split up $s$ by $x$ and $y$ terms $s(x,y) = f(x) +
	g(y)$ since for any $x^my^n$ term, the term reduces to a
	monomial of either $x$ or $y$ depending on which power is larger (since $xy = 1$)\\
	This mapping cannot be surjective and thus is not an isomorphism since WLOG
	if $\deg f(x) > 0$ then there is no $p(t) \in k[t]$ where $p(t) \to y$.\\
	The reason for this is the degree with respect to $x$ of the image of
	$p(t)$ will be $\deg p(t) \cdot \deg f(x)$. Thus either $p(t) \in k$ or
	$p(t)$ maps to something with $x$ (which cannot be equal to $y$)\\
	\\
	We have the degree equality above since $\deg (f(x) + g(y))^n = n
	\deg f(x)$ ($g(y)$ is just a constant with respect to $x$) and so the
	leading term of $p(t)$ will map to the leading term of the image with
	degree $\deg p(t) \cdot \deg f(x)$
\end{ques}

\begin{ques}{2}
\end{ques}

\begin{ques}{3}
	$(\supseteq):$ It is clear $\ker \phi \supseteq (z_{00}z_{11} - z_{01}z_{10})$ since 
	$$z_{00}z_{11} - z_{01}z_{10} \to x_0y_0x_1y_1 - x_0y_1x_1y_0 = 0$$
	$(\subseteq):$ for any $f(z_{00}, z_{10},z_{01},z_{11}) \in \ker \phi$ we can write
	$$f = q(z_{00}, z_{10},z_{01},z_{11})(z_{00}z_{11} - z_{01}z_{10}) +
	r(z_{00},z_{10},z_{01},z_{11})$$
	so that no terms show up in $r$ where all $z_{00}z_{01}z_{10}z_{11}$
	variables are present . It must be the case that $(z_{00}z_{11} -
	z_{01}z_{10})$ divides $r$ as follows.\\
	We know that $r \in \ker \phi$ since $f - q(z_{00}z_{11} - z_{01}z_{10})
	\in \ker \phi$. If $r$ had some nonzero term
	$$cz_{00}^{n_{00}}z_{11}^{n_{11}}z_{01}^{n_{01}}z_{10}^{n_{10}}$$
	The image would be 
	$$c(x_0y_0)^{n_{00}}(x_1y_1)^{n_{11}}(x_0y_1)^{n_{01}}(x_1y_0)^{n_{10}}$$
	Thus we have corresponding powers
	$$\begin{array}{c| c| c| c}
	x_0 & x_1 & y_0 & y_1\\
	\hline
	n_{00} + n_{01} & n_{10} + n_{11} & n_{00} + n_{10} & n_{11} + n_{01}
	\end{array}$$
	There must be another term in $r$ which maps to the same term in order to
	cancel out this term (since $r$ is in the kernel)\\
	This is equivalent to finding multiple solutions to the equation
	$$Am = x$$
	where $m = (m_{00}, m_{10}, m_{01}, m_{11})^T$, $x$ is the powers described
	above, and
	$$A = \begin{bmatrix}
	1 & 0 & 1 & 0\\
	0 & 1 & 0 & 1\\
	1 & 1 & 0 & 0\\
	0 & 0 & 1 & 1
	\end{bmatrix}$$
	Computing the nullspace of $A$ over $\R$ yields the vector space spanned by
	$v = (1,-1,-1,1)$, and thus there must be another term in $r$ with powers
	$= n + kv$ for some $k \in \Z$. Since $r$ has no terms where all the powers
	are nonzero, the only way this is possible is if the powers are of the form
	$(0, n, n, 0)$ and $(n, 0, 0, n)$ which corresponds to terms of the form
	$$cz_{00}^nz_{11}^n - cz_{01}^nz_{10}^n$$
	And thus $r$ must be divisible by $(z_{00}z_{11} -
	z_{01}z_{10})$
\end{ques}

\begin{ques}{4}
	We have that in $R = k[x,y,z,t]/I$, $zy^2 = x^2y = ztx$ and so
	$$(y^2 - xt)z = 0$$
	and thus $R$ is not an integral domain so $I$ is not prime. We have that $z
	\neq 0$ in $R$ and $y^2 - xt \neq 0$ in $R$ since when viewed as
	polynomials in $y$ the leading terms can never match since $x$ and $z$ will
	never cancel out
	$$y^2 - xt \neq f(y)(-yz + x^2) + g(y)(xy - zt)$$
	$$\forall f,g \in k[x,z,t][y]$$
\end{ques}

\begin{ques}{5}
	We know that $\Z[x_1 \dots x_n]/\mathfrak m$ is a field with some finite
	charactersitic $p$. It can be viewed as a $\Z/(p)$ algebra generated by
	$x_1, \dots x_n$. We have that each $x_i$ is integral over $\Z[x_1 \dots
	x_{i-1}]/\mathfrak m$ and thus we can conclude
	$\Z[x_1 \dots x_n]/\mathfrak m$ is a finitely generated $\Z/(p)$ module and
	thus finite. \\
	% We have $x_i$ is integral over $\Z[x_1 \dots
	% x_{i-1}]/\mathfrak m$ as follows:\\
	% If $x_i \in \mathfrak m$ we are done, otherwise $x_i$ has an inverse:
	% $$x_if(x_1 \dots x_n) = 1 + m(x_1 \dots x_n)$$
	% where $m \in \mathfrak m$
\end{ques}

\begin{ques}{6}
\end{ques}

\end{document}
